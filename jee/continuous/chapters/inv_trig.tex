\renewcommand{\theequation}{\theenumi}
\begin{enumerate}[label=\arabic*.,ref=\thesubsection.\theenumi]
\numberwithin{equation}{enumi}

\item Let a, b, c be positive real numbers. Let
\begin{align*}
\theta = \tan^{-1}\sqrt{\frac{a(a + b - c)}{bc}} + \tan^{-1} \sqrt{ \frac{b(a + b + c) }{ca}}\\ 
+ \tan^{-1}\sqrt{\frac{c(a +  b - c)}{ab}}
\end{align*}
Then $\tan \theta$ = ..............

\item The numerical value of 
\begin{align*}
\tan \{2\tan^{-1}(\frac{1}{5}) - \frac{\pi}{4}\}
\end{align*} 
is equal to .............

\item The greater of the two angles 
\begin{align*}
A = 2\tan^{-1}(2\sqrt{2} - 1)
\end{align*}
\begin{align*}
B = 3\sin^{-1}(\frac{1}{3}) + \sin^{-1}(\frac{3}{5})
\end{align*}
is.................

\textbf{MCQ's with One Correct Answer}

\item The value of $\tan^{-1}[(\cos^{-1}\frac{4}{5}) + \tan^{-1}(\frac{2}{3})]$ is
\begin{enumerate}
\item $\frac{6}{17}$
\item $\frac{7}{16}$
\item $\frac{16}{7}$
\item none of these
\end{enumerate}

\item If we consider only the principle values of the inverse trigonometric functions then the value of
$\tan(\cos^{-1}\frac{1}{5\sqrt{2}} - \sin^{-1}\frac{4}{\sqrt{17}})$ is
\begin{enumerate}
\item $\frac{\sqrt{29}}{3}$
\item $\frac{29}{3}$
\item $\frac{\sqrt{3}}{29}$
\item $\frac{3}{29}$
\end{enumerate}

\item The number of real solutions of 
\begin{align*}
\tan^{-1} \sqrt{x(x + 1)} + \sin^{-1} \sqrt{x^2 + x + 1} = \frac{\pi}{2}
\end{align*}
\begin{enumerate}
\item zero
\item one
\item two
\item infinite
\end{enumerate}

\item If 
\begin{align*}
\sin^{-1}(x - \frac{x^{2}}{2} + \frac{x^{3}}{4}......) + \cos^{-1}(x^{2} - \frac{x^{4}}{2} + \frac{x^{6}}{4}....) = \frac{\pi}{2} 
\end{align*}
for $0 < |x| < \sqrt{2}$, then x equals
\begin{enumerate}
\item 1/2
\item 1
\item -1/2
\item -1
\end{enumerate}

\item The value of x for which 
\begin{align*}
\sin(\cot^{-1}(1 + x) = \cos(\tan^{-1}x))
\end{align*}
is
\begin{enumerate}
\item 1/2
\item 1
\item 0
\item -1/2
\end{enumerate}

\item If $0 < x < 1$, then 
\begin{align*}
\sqrt{1 + x^2}[\{x\cos(\cot^{-1}x) + \sin(\cot^{-1}x)\} - 1]^{1/2} =
\end{align*}
\begin{enumerate}
\item $\frac{x}{1 + x^2}$
\item x
\item $x\sqrt{1 + x^2}$
\item $\sqrt{1 + x^2}$
\end{enumerate}

\item The value of 
\begin{align*}
\cot(\sum_{n=1}^{23} \cot^{-1}(1 + \sum_{k = 1}^{n}2k))=
\end{align*}
\begin{enumerate}
\item $\frac{23}{25}$
\item $\frac{25}{23}$
\item $\frac{23}{24}$
\item $\frac{24}{23}$
\end{enumerate}

\textbf{MCQs with One or More than One Correct}

\item The principal value of $\sin^{-1}(sin\frac{2\pi}{3})$ is
\begin{enumerate}
\item $-\frac{2\pi}{3}$
\item $\frac{2\pi}{3}$
\item $\frac{4\pi}{3}$
\item none of these
\end{enumerate}

\item If $\alpha = 3\sin^{-1}(\frac{6}{11})$ and $\beta = 3\cos^{-1}(\frac{4}{9})$, where the inverse trigonometric functions take only the principal values, then the correct option(s) is(are)
\begin{enumerate}   
\item $\cos\beta > 0$
\item $\sin\beta < 0$
\item $\cos(\alpha + \beta) > 0$
\item $\cos\alpha < 0$
\end{enumerate}

\item For non-negative integers n, let
\begin{align*}
f(n) = \frac{\sum_{k=0}^{n}\sin(\frac{k+1}{n+2}\pi)\sin(\frac{k+2}{n+2}\pi)}{\sum_{k=0}^{n}\sin^2(\frac{k+1}{n+2}\pi)}
\end{align*}
Assuming $\cos^{-1}x$ takes values [0, $\pi$], which of the following options is/are correct?
\begin{enumerate}
\item $[ \lim_{n \to \infty} f(n) = \frac{1}{2}]$
\item $f(4) = \frac{\sqrt{3}}{2}$
\item If $\alpha = \tan(\cos^{-1}{f(6)})$, then $\alpha^2 + 2\alpha - 1 = 0$
\item $\sin(7\cos^{-1}{f(5)}) = 0$
\end {enumerate}

\item Find the value of: 
\begin{align*}
\cos(2\cos^{-1}{x} + \sin^{-1}{x}) at x = \frac{1}{5}
\end{align*}
where $0 \leq \cos^{-1}{x} \leq \pi$ and $-\frac{\pi}{2} \leq \sin^{-1}{x} \leq \frac{\pi}{2}$.

\item Find all the solutions of 
\begin{align*}
4\cos^{2}x \sin x - 2\sin^{2}x = 3\sin x
\end{align*} 

\item Prove that $\cos \tan^{-1}\sin \cot^{-1}x = \sqrt{\frac{x^2 + 1}{x^2 + 2}}$

\textbf{Integer Value Correct Type:}

\item The number of real solutions of the equation
\begin{align*}
\sin^{-1} (\sum_{i = 1}^{\infty}x^{i + 1} - x\sum_{i = 1}^{\infty}(\frac{x}{2})^{i})\\
 = \frac{\pi}{2} - \cos^{-1}(\sum_{i = 1}^{\infty}(\frac{-x}{2})^{i} - \sum_{i = 1}^{\infty}(-x)^{i})
\end{align*}
lying in the interval $(\frac{-1}{2}, \frac{1}{2})$ is............

\item The value of 
\begin{align*}
\sec^{-1}\frac{1}{4}\sum_{k = 0}^{10}\sec(\frac{7\pi}{12} + \frac{k\pi}{2})\sec(\frac{7\pi}{12} + \frac{(k + 1)\pi}{2})
\end{align*}
in the interval $[\frac{-\pi}{4}, \frac{3\pi}{4}]$ equals..........

\textbf{Section-B}

\item $\cot^{-1}(\sqrt{\cos \alpha})-\tan^{-1}(\sqrt{\cos \alpha}) = x$, then $\sin x$ =
\begin{enumerate}
\item $\tan^{2}(\frac{\alpha}{2})$
\item $\cot^{2}(\frac{\alpha}{2})$
\item $\tan \alpha$
\item $\cot(\frac{\alpha}{2})$
\end{enumerate}

\item The trigonometric equation $\sin^{-1}x = 2\sin^{-1}a$ has a solution for
\begin{enumerate}
\item $|a| \geq \frac{1}{\sqrt{2}}$
\item $\frac{1}{2} < |a| < \frac{1}{\sqrt{2}}$
\item all real values of a
\item $|a| < \frac{1}{2}$
\end{enumerate}

\item If $\cos^{-1}x - \cos^{-1}\frac{y}{2} = \alpha$,  then $4x^2 - 4xy \cos \alpha + y^2$ is equal to
\begin{enumerate}
\item $2\sin 2\alpha$
\item 4
\item $4\sin^{2}\alpha$
\item $-4\sin^{2}\alpha$
\end{enumerate}

\item If $\sin^{-1}(\frac{x}{5}) + \cosec^{-1}(\frac{5}{4}) = \frac{\pi}{2}$, then the value of x is
\begin{enumerate}
\item 4
\item 5
\item 1
\item 3
\end{enumerate}

\item The value of $\cot(\cosec^{-1}(\frac{5}{3}) + \tan^{-1}(\frac{2}{3}))$ is
\begin{enumerate}
\item $\frac{6}{17}$
\item $\frac{3}{17}$
\item $\frac{4}{17}$
\item $\frac{5}{17}$
\end{enumerate}

\item If x, y, z are in A.P and $\tan^{-1}y$, $\tan^{-1}z$ are also in A.P., then
\begin{enumerate}
\item x = y = z
\item 2x = 3y = 6z
\item 6x = 3y = 2z
\item 6x = 4y = 3z
\end{enumerate}

\item Let 
\begin{align*}
\tan^{-1}y = \tan^{-1}x + \tan^{-1}(\frac{2x}{1 - x^2})
\end{align*}
where $|x| < \frac{1}{\sqrt{3}}$. Then a value of y is
\begin{enumerate}
\item $\frac{3x - x^3}{1 + 3x^2}$
\item $\frac{3x + x^3}{1 + 3x^2}$
\item $\frac{3x - x^3}{1 - 3x^2}$
\item $\frac{3x + x^3}{1 - 3x^2}$
\end{enumerate}

\item If $\cos^{-1}(\frac{2}{3x}) + \cos^{-1}(\frac{3}{4x}) = \frac{\pi}{2}(x > \frac{3}{4})$, then x is equal to
\begin{enumerate}
\item $\frac{\sqrt{145}}{12}$
\item $\frac{\sqrt{145}}{10}$
\item $\frac{\sqrt{146}}{12}$
\item $\frac{\sqrt{145}}{11}$
\end{enumerate}

\clearpage
\item Match the following:
\begin{table}[ht!]
\centering
\begin{tabular}{c c} 
\textbf{Column I} & \textbf{Column II}\\ [0.5ex] 
     A. $\sum_{n = 1}^{23} \tan^{-1}(\frac{1}{2i^{2}})$ = t, then $\tan t$ = &               (p). 1\\
     B. Sides a, b, c of a triangle ABC are in A.P.\\ $\cos \theta_1 = \frac{a}{b + c}$,
       $\cos \theta_2 = \frac{b}{a + c}$, $\cos \theta_3 = \frac{c}{a + b}$,\\ 
       then $\tan^2(\frac{\theta_1}{2}) + \tan^2(\frac{\theta_3}{2})$ = &                 (q). $\frac{\sqrt{5}}{3}$\\
     C. A line is perpendicular to x + 2y + 2z = 0 and\\ passes through (0, 1, 0) 
        Then the perpendicular\\ distance of this line from the origin is &               (r). $\frac{2}{3}$\\[1ex] 
\end{tabular}
\end{table}\\

\item Match the following:
\begin{table}[ht!]
\centering
\begin{tabular}{c c} 
 \textbf{Column I} & \textbf{Column II}\\ [0.5ex] 
 A. If a = 1 and b = 0, then (x, y)&                           (p). lies on the circle $x^2 +y^2=1$\\
 B. If a = 1 and b = 1, then (x, y)&                           (q). lies on $(x^2 - 1)(y^2 - 1) = 0$\\
 C. If a = 1 and b = 2, then (x, y)&                           (r). lies on $y = x$\\
 D. If a = 2 and b = 2, then (x, y)&                           (s). lies on $(4x^2 - 1)(y^2 - 1) = 0$\\[1ex] 
\end{tabular}
\end{table}\\

\item Match the following:
\begin{table}[ht!]
\centering
\begin{tabular}{c c} 
\textbf{Column I} & \textbf{Column II}\\ [0.5ex] 
   P. $(\frac{1}{y^2}(\frac{\cos(\tan^{-1}{y}) + y\sin(\tan^{-1}{y})}{\cot(\sin^{-1}{y})
      + \tan(\sin^{-1}{y})})^2 + y^4)^{\frac{1}{2}}$\\ takes value is           &(i) $\frac{1}{2}\sqrt{\frac{5}{3}}$\\
   Q. If $\cos x + \cos y + \cos z = 0$ =\\ $\sin x + \sin y + \sin z$ then
      possible\\ value of\\ $\cos \frac{x - y}{2}$ is                               &(ii) $\sqrt{2}$\\
   R. If $\cos(\frac{\pi}{4} - x) \cos2x + \sin x \sin 2\sec x$ =\\  
      $\cos x \sin{2x} \sec x + \cos(\frac{\pi}{4} + x)$\\ 
      then possible value of  $\sec x$ is                                       &(iii) $\frac{1}{2}$\\
   S. If $\cot(\sin^{-1}\sqrt{1 - x^2}) = \sin(\tan^{-1}{x}\sqrt{6})$,\\
      $x \neq 0$ then possible value of $\sec x$ is                             &(iv)  1\\[1ex] 
\textbf{codes:}
\begin{tabular}{ c c c c c}
      P & Q & R & S\\
  (a) 4 & 3 & 1 & 2\\
  (b) 4 & 3 & 2 & 1\\
  (c) 3 & 4 & 2 & 1\\
  (d) 3 & 4 & 1 & 2\\
\end{tabular}
\end{tabular}
\end{table}
\end{enumerate}

