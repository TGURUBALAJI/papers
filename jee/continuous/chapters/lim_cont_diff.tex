\renewcommand{\theequation}{\theenumi}
\begin{enumerate}[label=\arabic*.,ref=\thesubsection.\theenumi]
\numberwithin{equation}{enumi}

\item Let $f(x)$=\resizebox{.33 \textwidth}{!} 
{$\begin{cases}
(x-1)^2\sin \dfrac{1}{(x-1)} - |x|,& \text{if $x\ne 1$}. \\
-1, & \text{if x=1}.
\end{cases}$} \\ 
\\be real-valued function. Then find the set of points where $f(x)$ is not differentiable ?

\item Let\begin{equation*}
f(x)=\begin{cases}
\dfrac{\left(x^3+x^2-16x+20\right)}{\left(x-2\right)^2}, &\text{if $x\ne 2$}\\
k, & \text{if x=2}
\end{cases}
\end{equation*}
If $f(x)$ is continuous for all x, then find k ?

\item[~]\item A discontinuous function $y=f(x)$ satisfying $x^2+y^2=4$ is given by $f(x)=$ .......

\item[~]\item
$\lim\limits_{x \to 1}\left(1-x\right)\tan\dfrac{\pi x}{2} $= ........

\item If f(x)=\resizebox{.34 \textwidth}{!} 
{$\begin{cases}\sin x, & \text{$x \ne n\pi$}, n=0,\pm 1, \pm 2, \pm 3, .....\\ 
2, & \text{otherwise} \end{cases}$} \item[~]\item[~]and g(x)=$\begin{cases}x^2+1, & \text{$x\ne 0,2$}\\ 4,& \text{$x=0$}\\ 5,& \text{$x=2$}\end{cases}$
then \item[~]\item[~]$\lim\limits_{x \to 0}g\left[f(x)\right]$ is ...........

\item[~]\item $\lim\limits_{x \to -\infty}\left[\dfrac{x^4\sin \left(\dfrac{1}{x}\right)+x^2}{\left(1+|x|^3\right)}\right]$ = .........

\item[~]\item If $f(9)=9, f'(9)=4$, then $\lim\limits_{x \to 9} \dfrac{\sqrt{f(x)}-3}{\sqrt{x}-3}$ equals ........

\item[~]\item $ABC$ is an isosceles triangle inscribed in a circle of radius $r$. If $AB$ = $AC$ and $h$ is the altitude from $A$ to $BC$ then the triangle $ABC$ has perimeter $P$ = $\left(2\left(\sqrt{2hr-h^2}\right)+\sqrt{2hr}\right)$ and area $A$= ........ also $\lim\limits_{h \to 0}\dfrac{A}{P^3}$= ........

\item[~]\item $\lim\limits_{x \to \infty}\left(\dfrac{x+6}{x+1}\right)^{x+4}$ = .......

\item[~] \item Let $f(x)=x|x|$. The set of points where $f(x)$ is twice differentiable is .........

\item[~] \item Let $f(x)$ = $\left[x\right]\sin\left(\dfrac{\pi}{\left[x+1\right]}\right)$, where \item[~]\item[~][$\bullet$] denotes the greatest integer function. The domain of $f$ is ........ and the points of discontinuity of $f$ in the domain are .........

\item[~] \item$\lim\limits_{x \to 0}\left(\dfrac{1+5x^2}{1+3x^2}\right)^{1/x^2}$ = ...........

\item[~] \item[~]\item Let $f(x)$ be a continuous function defined for $1\leq x\leq 3$. If $f(x)$ takes rational values for all $x$ and $f(2)$ = 10, then $f(1.5)$ = ........ 

\item[~] \item If $\lim\limits_{x \to a}\left[f(x)g(x)\right]$ exists then both $\lim\limits_{x \to a}f(x)$ \item[~] \item[~]and $\lim\limits_{x \to a}g(x)$ exist. (True / False)

\item[~] \item If $f(x)=\sqrt{\dfrac{x-\sin x}{x+\cos^2x}}$, then $\lim\limits_{x \to \infty}f(x)$ is
\begin{itemize}
\begin{multicols}{2}
\item[(a)] 0 \item[~]\item[(c)] 1 \item[(b)] $\infty$ \item[~]\item[(d)] none of these
\end{multicols}
\end{itemize}

\item For a real number $y$, let $\left[y\right]$ denotes the greatest integer less than or equal to $y$ : Then \\ \\the function $f(x)$= $\dfrac{\tan\left(\pi\left[x-\pi\right]\right)}{1+\left[x\right]^2}$ is \\
\begin{itemize}
\item[(a)] discontinuous at some $x$
\item[(b)] continuous at all $x$, but the derivative $f'(x)$ does not exist for some $x$
\item[(c)] $f'(x)$ exists for all $x$, but the second derivative $f''(x)$ does not exist for some $x$
\item[(d)] $f'(x)$ exists for all $x$
\end{itemize}

\item[~] \item There exists a function $f(x)$, satisfying $f(0)=1$, $f'(0)=-1$, $f(x)>0$ for all $x$, and
\begin{itemize}
\item[(a)] $f''(x)>0$ for all $x$\\ \item[(b)] $-1<f''(x)<0$ for all $x$ \item[~]\item[(c)] $-2\leq f''(x)\leq -1$ for all $x$ \item[~] \item[(d)] $f''(x)<-2$ for all $x$
\end{itemize}

\item[~] \item If G(x)= $-\sqrt{25-x^2}$ then $\lim\limits_{x \to 1}\dfrac{G(x)-G(1)}{x-1}$ has the value
\begin{itemize}
\begin{multicols}{2}
\item[(a)] $\dfrac{1}{24}$ \item[~] \item[(c)] $-\sqrt{24}$ \item[(b)] $\dfrac{1}{5}$ \item[~] \item[(d)] none of these
\end{multicols}
\end{itemize}

\item[~] \item If $f(a)\mathbin{=}2, f'(a)\mathbin{=}1, g(a)\mathbin{=}-1, g'(a)\mathbin{=}2$, \item[~]\item[~] then the value of $\lim\limits_{x \to a}\dfrac{g(x)f(a)-g(a)f(x)}{x-a}$ is
\begin{itemize}
\begin{multicols}{2}
\item[(a)] -5 \item[~] \item[(c)] 5 \item[(b)] $\dfrac{1}{5}$ \item[~] \item[(d)] none of these
\end{multicols}
\end{itemize}

\item[~] \item The function \[f(x)=\dfrac{ln(1+ax)-ln(1-bx)}{x}\] is not defined at $x\mathbin{=}0$. The value which should be assigned to $f$ at $x=0$ so that it is continuous at $x\mathbin{=}0$, is
\begin{itemize}
\begin{multicols}{2}
\item[(a)] $a-b$ \item[~] \item[(c)] $ln a - ln b$ \item[(b)] $a+b$ \item[~] \item[(d)] none of these
\end{multicols}
\end{itemize}

\item[~] \item$\lim\limits_{n \to \infty}\left\{\dfrac{1}{1-n^2}+\dfrac{2}{1-n^2}+.....+\dfrac{n}{1-n^2}\right\}$ \item[~] \item[~]is equal to
\begin{itemize}
\begin{multicols}{2}
\item[(a)] 0 \item[~] \item[(c)] $\dfrac{1}{2}$ \item[(b)] $-\dfrac{1}{2}$ \item[~] \item[(d)] none of these
\end{multicols}
\end{itemize}

\item[~] \item If $f(x)=\begin{cases}\mathbin{=}\dfrac{\sin[x]}{[x]}, & \text{$[x]\neq 0$} \\
0, & \text{$[x]=0$} \end{cases}$\item[~] \item[~]
Where [x] denotes the greatest integer less than or equal to $x$, then $\lim\limits_{x \to 0}f(x)$ equals
\begin{itemize}
\begin{multicols}{2}
\item[(a)] 1 \item[~] \item[(c)] -1 \item[(b)] 0 \item[~] \item[(d)] none of these
\end{multicols}
\end{itemize}

\item[~] \item Let $f:R\to R$ be differentiable function and $f(1)=4$. Then the value of $\lim\limits_{x \to 1}\int\limits_4^{f(x)}\dfrac{2t}{x-1} dt$ is
\begin{itemize}
\begin{multicols}{2}
\item[(a)] $8f'(1)$ \item[~] \item[(c)] $2f'(1)$ \item[(b)] $4f'(1)$ \item[~] \item[(d)] $f'(1)$
\end{multicols}
\end{itemize}

\item[~] \item Let [$\bullet$] denote the greatest integer function and $f(x)=[\tan^2x]$, then
\begin{itemize}
\item[(a)] $\lim\limits_{x \to 0}f(x)$ does not exist \item[~]
\item[(b)] $f(x)$ is continuous at $x=0$
\item[(c)] $f(x)$ is not differentiable at $x=0$
\item[(d)] $f'(0)=1$
\end{itemize}

\item[~] \item The function $f(x)=[x]\cos\left(\dfrac{2x-1}{2}\right)\pi$, [$\bullet$] denotes the greatest integer function, is discontinuous at
\begin{itemize}
\item[(a)] All x \item[(b)] All integer points \item[(c)] No x \item[(d)] x which is not an integer
\end{itemize}

\item[~] \item $\lim\limits_{n \to \infty}\dfrac{1}{n}\sum\limits_{r=1}^{2n}\dfrac{r}{\sqrt{n^2+r^2}}$ equals \item[~] \item[~]
\begin{itemize}
\begin{multicols}{2}
\item[(a)] $1+\sqrt{5}$ \item[~] \item[(c)] $-1+\sqrt{2}$ \item[(b)] $-1+\sqrt{5}$ \item[~] \item[(d)] $1+\sqrt{2}$
\end{multicols}
\end{itemize} 

\item The function $f(x)\mathbin{=}[x^2]-[x^2]$ (where $[y]$ is the greatest integer less than or equal to $y$), is discontinuous at
\begin{itemize}
\item[(a)] all integers
\item[(b)] all integers except 0 and 1
\item[(c)] all integers except 0
\item[(d)] all integers except 1
\end{itemize}

\item The function $f(x)\mathbin{=}\left(x^2-1\right)|x^2-3x+2|+\cos\left(|x|\right)$ is NOT differentiable at
\begin{itemize}
\begin{multicols}{4}
\item[(a)] -1 \item[(b)] 0 \item[(c)] 1 \item[(d)] 2
\end{multicols}
\end{itemize}

\item[~] \item $\lim\limits_{x \to 0}\dfrac{x \tan 2x-2x \tan x}{\left(1-\cos 2x\right)^2}$ is \item[~] \item[~]
\begin{itemize}
\begin{multicols}{4}
\item[(a)] 2 \item[(b)] -2 \item[(c)] 1/2 \item[(d)] -1/2
\end{multicols}
\end{itemize}

\item[~] \item For x $\in$ R, $\lim\limits_{x \to \infty}\left(\dfrac{x-3}{x+2}\right)^x$ =  \item[~] 
\begin{itemize}
\begin{multicols}{4}
\item[(a)] $e$ \item[(b)] $e^{-1}$ \item[(c)] $e^{-5}$ \item[(d)] $e^{5}$
\end{multicols}
\end{itemize}

\item $\lim\limits_{x \to 0}\dfrac{\sin(\pi\cos^2x)}{x^2}$ equals
\begin{itemize}
\begin{multicols}{4}
\item[(a)] $-\pi$ \item[(b)] $\pi$ \item[(c)] $\pi/2$ \item[(d)] 1
\end{multicols}
\end{itemize}

\item The left-hand derivative of $f(x)$=$[x]sin(\pi x)$ at $x\mathbin{=}k$, k and integer, is
\begin{itemize}
\begin{multicols}{2}
\item[(a)] $(-1)^k(k-1)\pi$ \item[~]\item[(c)] $(-1)^kk\pi$ \item[(b)] $(-1)^{k-1}(k-1)\pi$ \item[~]\item[(d)] $(-1)^{k-1}k\pi$
\end{multicols}
\end{itemize}

\item Let $f:R\to R$ be a function defined by $f(x)=max\{x,x^3\}$. The set fo all points where $f(x)$ is NOT differentiable is
\begin{itemize}
\begin{multicols}{2}
\item[(a)] \{-1,1\} \item[~]\item[(c)] \{0,1\}\item[(b)] \{-1,0\} \item[~] \item[(d)] \{-1,0,1\}
\end{multicols}
\end{itemize}

\item Which of the following functions is differentiable at x = 0 ?
\begin{itemize}
\begin{multicols}{2}
\item[(a)] $\cos(|x|)+|x|$ \item[~]\item[(c)] $\sin(|x|)+|x|$ \item[(b)] $\cos(|x|)-|x|$ \item[~] \item[(d)] $\sin(|x|)-|x|$
\end{multicols}
\end{itemize}

\item The domain of the derivative of the function $f(x)\mathbin{=}\begin{cases} \tan^{-1}x & \text{if $|x|\leq 1$}\\\dfrac{1}{2}(|x|-1) & \text{$|x|>1$}\end{cases}$ is
\begin{itemize}
\begin{multicols}{2}
\item[(a)] $R-\{0\}$ \item[~]\item[(c)] $R-\{1\}$\item[(b)] $R-\{-1\}$ \item[~] \item[(d)] $R-\{-1,1\}$
\end{multicols}
\end{itemize}

\item The integer for which \item[~]\item[~]$\lim\limits_{x \to 0}\dfrac{(\cos x-1)(\cos x-e^x)}{x^n}$ is a finite \item[~]\item[~]non-zero number is
\begin{itemize}
\begin{multicols}{4}
\item[(a)] 1 \item[(b)] 2 \item[(c)] 3 \item[(d)] 4
\end{multicols}
\end{itemize}

\item Let $f:R \to R$ be such that $f(1)=3$ and \\$f'(1)=6$. Then $\lim\limits_{x \to 0}\left(\dfrac{f(1+x)}{f(1)}\right)^{1/x}$ eqauls \\
\begin{itemize}
\begin{multicols}{4}
\item[(a)] 1 \item[(b)] $e^{1/2}$ \item[(c)] $e^2$ \item[(d)] $e^3$
\end{multicols}
\end{itemize}

\item If $\lim\limits_{x \to 0}\dfrac{((a-n)nx-\tan x)\sin nx}{x^2}=0$, where n is nonzero real number, then a is equal to
\begin{itemize}
\begin{multicols}{2}
\item[(a)] 0 \item[~] \item[(c)] $n$\item[(b)] $\dfrac{n+1}{n}$ \item[~] \item[(d)] $n+\dfrac{1}{n}$
\end{multicols}
\end{itemize}

\item[~] \item $\lim\limits_{h \to 0}\dfrac{f(2h+2+h^2)-f(2)}{f(h-h^2+1)-f(1)}$, given that \\ \item[~]$f'(2)=6$ and $f'(1)=4$
\begin{itemize}
\begin{multicols}{2}
\item[(a)] does not exist \item[(c)] is equal to 3/2 \item[(b)] is equal to -3/2 \item[(d)] is equal to 3
\end{multicols}
\end{itemize}

\item If (x) is differentiable and strictly increasing function, then the value of \item[~] \item[~]$\lim\limits_{x \to 0}\dfrac{f(x^2)-f(x)}{f(x)-f(0)}$ is \item[~]
\begin{itemize}
\begin{multicols}{4}
\item[(a)] 1 \item[(b)] 0 \item[(c)] -1 \item[(d)] 2
\end{multicols}
\end{itemize}

\item The function given by $y=||x|-1|$ is differentiable for all real numbers except the points
\begin{itemize}
\begin{multicols}{2}
\item[(a)] \{0,1,-1\} \item[(c)] 1 \item[(b)] $\pm 1$ \item[(d)] -1
\end{multicols}
\end{itemize}

\item If $f(x)$ is continuous and differentiable function and $f(1/n)=0 \forall n \geq 1$ and $n \in$ I, then
\begin{itemize}
\item[(a)] $f(x)=0, x\in (0,1]$
\item[(b)] $f(0)=0, f'(0)=0$
\item[(c)] $f(0)=0=f'(0), x \in (0,1]$
\item[(d)] $f(0)=0$ and $f'(0)$ need not to be zero
\end{itemize}

\item The value of \\$\lim\limits{x \to 0}\left((\sin x)^{1/x}+(1+x)^{\sin x}\right)$, where $x>0$ is
\begin{itemize}
\begin{multicols}{4}
\item[(a)] 0 \item[(b)] -1 \item[(c)] 1 \item[(d)] 2
\end{multicols}
\end{itemize}

\item Let $f(x)$ be differentiable on the interval $(0,\infty)$ such that $f(1)=1$, and \\ \item[~]$\lim\limits_{t \to x}\dfrac{t^2f(x)-x^2f(t)}{t-x}=1$ for each $x > 0$.\item[~] \item[~] Then $f(x)$ is
\begin{itemize}
\begin{multicols}{2}
\item[(a)] $\dfrac{1}{3x}+\dfrac{2x^2}{3}$ \item[~]\item[(c)] $\dfrac{-1}{x}+\dfrac{2}{x^2}$ \item[(b)] $\dfrac{-1}{3x}+\dfrac{4x^2}{3}$ \item[~]\item[(d)] $\dfrac{1}{x}$
\end{multicols}
\end{itemize}

\item[~] \item $\lim\limits_{x \to \dfrac{\pi}{4}}\dfrac{\int\limits_2^{\sec x^2}f(t) dt}{x^2-\dfrac{\pi ^2}{16}}$ equals \\ 
\begin{itemize}
\begin{multicols}{2}
\item[(a)] $\dfrac{8}{\pi}f(2)$ \item[~]\item[(c)] $\dfrac{2}{\pi}f\left(\dfrac{1}{2}\right)$ \item[(b)] $\dfrac{2}{\pi}f(2)$ \item[~]\item[(d)] $4f(2)$
\end{multicols}
\end{itemize}

\item[~] \item Let $g(x)=\dfrac{(x-1)^n}{\log\cos^m(x-1)}$; $0<x<2$, \\ \\m and n are integers, $m \ne 0, n>0$, let $p$ be the left hand derivative of $|x-1|$ at x = 1. If $\lim\limits_{x \to 1}g(x)=p$, then
\begin{itemize}
\begin{multicols}{2}
\item[(a)] $n=1, m=1$ \item[~]\item[(c)] $n=2, m=2$ \item[(b)] $n=1, m=1$ \item[~]\item[(d)] $n>2, m=n$
\end{multicols}
\end{itemize} 

\item[~] \item If $\lim\limits_{x \to 0}\left[1+xln(1+b^2)\right]^{1/x}=2bsin^2\theta$, $b>0$ and $\theta \in (-\pi, \pi]$, then the value of $\theta$ is
\begin{itemize}
\begin{multicols}{4}
\item[(a)] $\pm\dfrac{\pi}{4}$ \item[(b)] $\pm\dfrac{\pi}{3}$ \item[(c)] $\pm\dfrac{\pi}{6}$ \item[(d)] $\pm\dfrac{\pi}{2}$
\end{multicols}
\end{itemize}

\item[~] \item If $\lim\limits_{x \to \infty}\left(\dfrac{x^2+x+1}{x+1}-ax-b\right)=4$, then
\begin{itemize}
\begin{multicols}{2}
\item[(a)] $a=1, b=4$ \item[(c)] $a=2, b=-3$ \item[(b)] $a=1, b=-4$ \item[(d)] $a=2, b=3$
\end{multicols}
\end{itemize} 

\item[~] \item Let $f(x)=\begin{cases}
x^2\left|\cos\dfrac{\pi}{x}\right|, & \text{$x\ne 0$}\\
0, & \text{$x=0$}
\end{cases}$, \\ \item[~]$x \in R$ then $f$ is
\begin{itemize}
\item[(a)] differentiable both at x = 0 and at x = 2
\item[(b)] differentiable at x= 0 but not differentiable at x = 2
\item[(c)] not differentiable at x = 0 but differentiable at x = 2
\item[(d)] differentiable neither at x = 0 nor at x = 2
\end{itemize}

\item Let $\alpha(a)$ and $\beta(a)$ be the roots of the equa\\ \\tion
$\left(\sqrt[3]{1+a}-1\right)x^2$+$\left(\sqrt{1+a}-1\right)x$+\\ \\$\left(\sqrt[6]{1+a}-1\right)$=0
 where a$>$-1. Then \\ \\$\lim\limits_{a \to 0^+}\alpha(a)$ and $\lim\limits_{x \to 0^+}\beta(a)$ are
\begin{itemize}
\begin{multicols}{2}
\item[(a)] $-\dfrac{5}{2}$ and 1 \item[~] \item[(c)] $-\dfrac{7}{2}$ and 2 \item[(b)] $-\dfrac{1}{2}$ and -1 \item[~]\item[(d)] $-\dfrac{9}{2}$ and 3
\end{multicols}
\end{itemize} 

\item If x+$|y|$ = 2y, then y as a function of x is
\begin{itemize}
\item[(a)] defined for all real x
\item[(b)] continuous at x = 0
\item[(c)] differentiable for all x
\item[(d)] such that $\dfrac{dy}{dx}=\dfrac{1}{3}$ for $x<0$\\
\end{itemize}

\item If $f(x)=x(\sqrt{x}-\sqrt{x+1})$, then
\begin{itemize}
\item[(a)] f(x) is continuous but not differentiable at x = 0
\item[(b)] f(x) is differentiable at x = 0
\item[(c)] f(x) is not differentiable at x = 0
\item[(d)] none of these
\end{itemize}

\item The function $f(x)=1+|\sin x|$ is
\begin{itemize}
\item[(a)] continuous nowhere
\item[(b)] continuous everywhere
\item[(c)] differentiable nowhere
\item[(d)] not differentiable at $x=0$
\item[(e)] not differentiable at infinite number of points
\end{itemize}

\item Let [x] denote the greatest integer less than or equal to x. If $f(x)=[x \sin\pi x]$, then $f(x)$ is
\begin{itemize}
\item[(a)] continuous at $x=0$ \item[(b)] continuous in (-1,0) \item[(c)] differentiable at $x=1$ \item[(d)] differentiable in (-1,1) \item[(e)] none of these
\end{itemize}

\item The set of all points where the function $f(x)=\dfrac{x}{(1+|x|)}$ is differentiable, is
\begin{itemize}
\begin{multicols}{2}
\item[(a)] $(-\infty,\infty)$ \item[(c)] $(-\infty,0)\cup(0,\infty)$ \item[(e)] None \item[(b)] $[0,\infty)$ \item[(d)] $(0,\infty)$
\end{multicols}
\end{itemize}

\item The function \\$f(x)=\begin{cases}
|x-3|, & \text{$x\geq 1$}\\
\dfrac{x^2}{4}-\dfrac{3x}{2}+\dfrac{13}{4}, & \text{$x < 1$}
\end{cases}$ is\\
\begin{itemize}
\item[(a)] continuous at $x=1$ \item[(b)] differentiable at $x=1$ \item[(c)] continuous at $x=3$ \item[(d)] differentiable at $x=3$
\end{itemize}

\item If $f(x)=\dfrac{1}{2}x-1$, then on the interval $[0,\pi]$
\begin{itemize}
\item[(a)] $\tan[f(x)]$ and $1/f(x)$ are both\\ continuous
\item[(b)] $\tan[f(x)]$ and $1/f(x)$ are both\\ discontinuous
\item[(c)] $\tan[f(x)]$ and $f^{-1}(x)$ are both\\ continuous
\item[(d)] $\tan[f(x)]$ is continuous but $1/f(x)$ is not
\end{itemize}

\item The value of $\lim\limits_{x \to 0}\dfrac{\sqrt{\dfrac{1}{2}(1-\cos 2x)}}{x}$
\begin{itemize}
\begin{multicols}{2}
\item[(a)] 1 \item[(c)] 0 \item[(b)] -1 \item[(d)] none of these
\end{multicols}
\end{itemize}

\item The following functions are continuous on $(0,\pi)$
\begin{itemize}
\item[(a)] $\tan x$ \\
\item[(b)] $\int\limits_0^xt\sin\dfrac{1}{t}dt$\\ \\
\item[(c)] $\begin{cases}
1, & \text{$0<x\leq\dfrac{3\pi}{4}$} \\
2\sin\dfrac{2}{9}x, & \text{$\dfrac{3\pi}{4}<x<\pi$}
\end{cases}$ \\ \\
\item[(d)] $\begin{cases}
x\sin x, & \text{$0<x\leq\dfrac{\pi}{2}$} \\
\dfrac{\pi}{2}\sin (\pi+x), & \text{$\dfrac{\pi}{2}<x<\pi$}
\end{cases}$
\end{itemize}

\item Let $f(x)=\begin{cases}
0, & \text{$x<0$}\\
x^2, & \text{$x\geq 0$}
\end{cases}$ then for all $x$
\begin{itemize}
\item[(a)] $f'$ is differentiable \item[(b)] $f$ is differentiable \item[(c)] $f'$ is continuous \item[(d)] $f$ is continuous
\end{itemize}

\item Let $g(x)=xf(x)$, where \\
\\$f(x)=\begin{cases}
x\sin\dfrac{1}{x}, & \text{$x\ne 0$}\\
0, & \text{$x=0$}
\end{cases}$. At $x=0$ \\
\begin{itemize}
\item[(a)] $g$ is differentiable but $g'$ is not contionuous
\item[(b)] $g$ is differentiable while $f$ is not
\item[(c)] both $f$ and $g$ are differentiable
\item[(d)] $g$ is differentiable and $g'$ is continuous \\
\end{itemize}

\item The function $f(x)$=max$\{(1-x),(1+x),2\}$, $x \in(-\infty,\infty)$ is
\begin{itemize}
\item[(a)] continuous at all points
\item[(b)] diffrentiable at all points
\item[(c)] diffrentiable at all points except at $x=1$ and $x=-1$
\item[(d)] continuous at all points except at $x=1$ and $x=-1$, where it is discontinuous\\
\end{itemize}

\item Let $h(x)$=min$\{x,x^2\}$, for every real number of $x$, then
\begin{itemize}
\item[(a)] $h$ is continuous for all $x$
\item[(b)] $h$ is differentiable for all $x$
\item[(c)] $h'(x)=1$, for all $x>1$
\item[(d)] $h$ is differentiable at two values of $x$
\end{itemize}

\item[~] \item $\lim\limits_{x \to 1}\dfrac{\sqrt{1-\cos 2(x-1)}}{x-1}$
\begin{itemize}
\item[(a)] exists and it equals $\sqrt{2}$
\item[(b)] exists and it equals $-\sqrt{2}$
\item[(c)] does not exist because $x-1 \to 0$
\item[(d)] does not exist because the left hand limit is not equal to the right hand limit\\
\end{itemize}

\item If $f(x)$=min$\{1,x^2,x^3\}$, then
\begin{itemize}
\item[(a)] $f(x)$ is continuous $\forall x \in R$
\item[(b)] $f(x)$ is continuous and differentiable everywhere
\item[(c)] $f(x)$ is not differentiable at two points
\item[(d)] $f(x)$ is not differentiable at one point
\end{itemize}

\item Let $L=\lim\limits_{x \to 0}\dfrac{a-\sqrt{a^2-x^2}-\dfrac{x^2}{4}}{x^4}, a>0$. If $l$ is finite, then
\begin{itemize}
\begin{multicols}{2}
\item[(a)] $a=2$ \item[~] \item[(c)] $L=\dfrac{1}{64}$ \item[(b)] $a=1$ \item[~] \item[(d)] $L=\dfrac{1}{32}$
\end{multicols}
\end{itemize}

\item Let $f:R\to R$ be a function such that $f(x+y)=f(x)+f(y), \forall x, y\in R$. If $f(x)$ is differentiable at $x=0$, then
\begin{itemize}
\item[(a)] $f(x)$ is differentiable only in a finite interval containing zero
\item[(b)] $f(x)$ is continuous $\forall x \in R$
\item[(c)] $f(x)$ is constant $\forall x \in R$
\item[(d)] $f(x)$ is differentiable except at finitely many points 
\end{itemize} 

\item[~]\item If $f(x)=\begin{cases}
-x-\dfrac{\pi}{2}, & \text{$x\leq -\dfrac{\pi}{2}$}\\
-\cos x, & \text{$-\dfrac{\pi}{2}<x\leq 0$}\\
x-1, & \text{$0<x\leq 1$}\\
lnx, & \text{$x>1$}
\end{cases}$, then
\begin{itemize}
\item[(a)] $f(x)$ is continuous at $x=-\dfrac{\pi}{2}$
\item[(b)] $f(x)$ is not differentiable at $x=0$
\item[(c)] $f(x)$ is differentiable at $x=1$
\item[(d)] $f(x)$ is differentiable at $x=-\dfrac{3}{2}$
\end{itemize}

\item For every integer $n$, let $a_n$ and $b_n$ be real numbers. Let function $f:IR \to IR$ be given by
$$f(x)=\begin{cases}
a_n+\sin\pi x, & \text{$x \in [2n, 2n+1]$}\\
b_n+\cos\pi x, & \text{$x \in (2n-1, 2n)$}
\end{cases}$$
for all integers $n$. If $f$ is continuous, then which of the fllowing hold(s) for all $n$ ?
\begin{itemize}
\begin{multicols}{2}
\item[(a)] $a_{n-1}-b_{n-1}=0$ \item[(c)] $a_n-b_{n+1}=1$ \item[(b)] $a_n-b_n=1$ \item[(d)] $a_{n-1}-b_n=-1$
\end{multicols}
\end{itemize}

\item For $a \in R$(the set of all real numbers), a$\neq$-1,\\
\resizebox{0.88\hsize}{!}{$\lim\limits_{x \to \infty}\dfrac{(1^a+2^a+..+n^a)}{(n+1)^{a-1}[(na+1)+(na+2)+..+(na+n)]}$}\\=$\dfrac{1}{60}$. Then $a=$
\begin{itemize}
\begin{multicols}{4}
\item[(a)] 5 \item[(b)] 7 \item[(c)] $\dfrac{-15}{2}$ \item[(d)] $\dfrac{-17}{2}$
\end{multicols}
\end{itemize}

\item Let $f:[a,b] \to [1,\infty)$ be a continuous function and let $g:R \to R$ be defined as\\
$g(x)= \begin{cases}
0, & \text{if $x<a$}\\
\int\limits_a^xf(t)dt, & \text{if $a\leq x\leq b$}\\
\int\limits_a^bf(t)dt, & \text{if $x>b$} 
\end{cases}$; then
\begin{itemize}
\item[(a)] $g(x)$ is continuous but not differentiable at a
\item[(b)] $g(x)$ is differentiable on $R$
\item[(c)] $g(x)$ is continuous but not differentiable at b
\item[(d)] $g(x)$ is continuous and differentiable at either (a) or (b) but not both
\end{itemize}

\item For every pair of continuous functions $f,g:[0,1] \to R$ such that max$\{f(x):x \in [0,1]\}$=max$\{g(x):x \in [0,1]\}$, the correct statement(s) is(are)
\begin{itemize}
\item[(a)] $(f(c))^2+3f(c)=(g(c))^2+3g(c)$ for some $c \in [0,1]$\\
\item[(b)] $(f(c))^2+f(c)=(g(c))^2+3g(c)$ for some $c \in [0,1]$\\
\item[(c)] $(f(c))^2+3f(c)=(g(c))^2+g(c)$ for some $c \in [0,1]$\\
\item[(d)] $(f(c))^2=(g(c))^2$ for some $c \in [0,1]$\\
\end{itemize}

\item Let $g:R \to R$ be a differentiable function with $g(0)=0, g'(0)=0$ and $g'(1)\neq 0$. Let \\ \\$f(x)=\begin{cases}
\dfrac{x}{|x|}g(x), & \text{$x\neq 0$}\\
0, & \text{$x=0$}
\end{cases}$ \\ and $h(x)=e^{|x|}$ for all $x \in R$. Let $(foh)(x)$ denote $f(h(x))$ and $(hof)(x)$ denote $h(f(x))$. Then which of the following is(are) true ?
\begin{itemize}
\item[(a)] $f$ is differentiable at $x=0$
\item[(b)] $h$ is differentiable at $x=0$
\item[(c)] $foh$ is differentiable at $x=0$
\item[(d)] $hof$ is differentiable at $x=0$
\end{itemize}

\item Let $a, b\in\mathbb{R}$ and $f:\mathbb{R}\to\mathbb{R}$ be defined by \\ $f(x)=a\cos (|x^3-x|)+b|x|\sin(|x^3+x|)$.\\ Then f is
\begin{itemize}
\item[(a)] differentiable at $x=0$ if $a=0$ and $b=1$
\item[(b)] differentiable at $x=1$ if $a=1$ and $b=0$
\item[(c)] NOT differentiable at $x=0$ if $a=1$ and $b=0$
\item[(d)] NOT differentiable at $x=1$ if $a=1$ and $b=1$\\
\end{itemize}

\item Let $f:\left[-\dfrac{1}{2},2\right] \to \mathbb{R}$ and $g:\left[-\dfrac{1}{2},2\right] \to \mathbb{R}$\\
\\ be functions defined by $f(x)=[x^2-3]$ and $g(x)=|x|f(x)+|4x-7|f(x)$, where [y] denotes the greatest integer less than or equal to y for y $\in R$. Then
\begin{itemize}
\item[(a)] $f$ is discontinuous exactly at three points in $\left[-\dfrac{1}{2},2\right]$\\
\item[(b)] $f$ is discontinuous exactly at four points in $\left[-\dfrac{1}{2},2\right]$\\
\item[(c)] $g$ is NOT differentiable excatly at four points in $\left(-\dfrac{1}{2},2\right)$\\
\item[(d)] $g$ is NOT differentiable exactly at five points in $\left(-\dfrac{1}{2},2\right)$\\
\end{itemize}

\item Let [x] be the greatest integer less than or equal to x. Then, at which of the following point(s) the function $f(x)$ = $x\cos(\pi(x+[x]))$ is discontinuous ?
\begin{itemize}
\begin{multicols}{2}
\item[(a)] $x=-1$ \item[(c)] $x=1$ \item[(b)] $x=0$ \item[(d)] $x=2$
\end{multicols}
\end{itemize}

\item Let \resizebox{.33 \textwidth}{!} 
{$f(x)=\dfrac{1-x(1+|1-x|)}{|1-x|}\cos\left(\dfrac{1}{1-x}\right)$} for $x\neq 1$. Then
\begin{itemize}
\item[(a)] $\lim\limits_{x \to 1^-}f(x)=0$\\
\item[(b)] $\lim\limits_{x \to 1^-}f(x)$ does not exist\\
\item[(c)] $\lim\limits_{x \to 1^+}f(x)=0$\\
\item[(d)] $\lim\limits_{x \to 1^+}f(x)$ does not exist\\
\end{itemize}

\item Let $f:\mathbb{R} \to \mathbb{R}$ and $g:\mathbb{R} \to \mathbb{R}$ be two non-constant differentiable functions. If\\
$f'(x)=\left(e^{(f(x)-g(x))}\right)g'(x)$ for all $x \in \mathbb{R}$,
\\ and $f(1)=g(2)=1$, then which of the following statement(s) is(are) TRUE ?
\begin{itemize}
\begin{multicols}{2}
\item[(a)] $f(2)<1-log_e2$ \item[~]\item[(c)] $g(1)>1-log_e2$ \item[(b)] $f(2)>1-log_e2$ \item[~]\item[(d)] $g(1)<1-log_e2$
\end{multicols}
\end{itemize}

\item Let $f:R \to R$ given by\\ $f(x)$=\resizebox{0.35\textwidth}{0.05 \textheight}{
$\begin{cases}
x^5+5x^4+10x^3+10x^2+3x+1, & \text{$x<0$}\\
x^2-x+1, & \text{$0\leq x<1$}\\
\dfrac{2}{3}x^3-4x^2+7x-\dfrac{8}{3}, &\text{$1\leq x<3$}\\
(x-2)\log_e(x-2)-x+\dfrac{10}{3}, & \text{$x\geq 3$}
\end{cases}$} \\ \\then which of the following options is/are correct ?
\begin{itemize}
\item[(a)] $f'$ has a local maximum at $x=1$
\item[(b)] $f$ is increasing on $(-\infty,0)$
\item[(c)] $f'$ is NOT differentiable at $x=1$
\item[(d)] $f$ is onto
\end{itemize}

\item Let $f:R \to R$ be a function. We say that $f$ has \\
\textbf{PROPERTY 1} if $\lim\limits_{h \to 0}\dfrac{f(h)-f(0)}{\sqrt{|h|}}$ exists and is finite, and \\
\textbf{PROPERTY 2} if $\lim\limits_{h \to 0}\dfrac{f(h)-f(0)}{h^2}$ exists and is finite\\
\begin{itemize}
\item[(a)] $f(x)=x^{2/3}$ has \textbf{PROPERTY 1}
\item[(b)] $f(x)=\sin x$ has \textbf{PROPERTY 2}
\item[(c)] $f(x)=|x|$ has \textbf{PROPERTY 1}
\item[(d)] $f(x)=x|x|$ has \textbf{PROPERTY 2}
\end{itemize} \item[~]

\item Evaluate $\lim\limits_{x \to a}\dfrac{\sqrt{a+2x}-\sqrt{3x}}{\sqrt{3a+x}-2\sqrt{x}}$, $(a\neq 0)$ \item[~] \item[~]

\item $f(x)$ is the integral of $\dfrac{2\sin x-\sin 2x}{x^3}, x\neq 0$, find $\lim\limits_{x \to 0}f'(x)$.\item[~]

\item Evaluate $\lim\limits_{h \to 0}\dfrac{(a+h)^2\sin(a+h)-a^2\sin a}{h}$ \item[~]

\item Let $f(x+y)=f(x)+f(y)$ for all $x$ and $y$. If the function $f(x)$ is continuous at $x=0$, then show that $f(x)$ is continuous at all $x$. \item[~]

\item Use the formula $\lim\limits_{x \to 0}\dfrac{a^x-1}{x}=lna$ to find $\lim\limits_{x \to 0}\dfrac{2^x-1}{(1+x)^{1/2}-1}$ \item[~]

\item Let $f(x)=\begin{cases}
1+x, & \text{$0\leq x\leq 2$}\\
3-x, & \text{$2\leq x\leq 3$}
\end{cases}$\\ 
\\Determine the form of $g(x)$ = $fIf(x)$ and hence find the points of discontinuity of $g$, if any.\\

\item Let $f(x)=\begin{cases}
\dfrac{x^2}{2}, & \text{$0\leq x<1$}\\
2x^2-3x+\dfrac{3}{2}, &\text{$1\leq x \leq 2$}
\end{cases}$ \\ 
\\Discuss the continuity of $f, f'$ and $f''$ on [0,2].\\

\item Let $f(x)=x^3-x^2+x+1$ and\\ $g(x)=\begin{cases}
$max$\{f(t); 0\leq t \leq x\}, &\text{$0\leq x\leq 1$}\\
3-x, &\text{$0\leq x \leq 2$}
\end{cases}$\\ \\Discuss the continuity and differentiability f the function $g(x)$ in the interval (0,2).\\

\item Let $f(x)$ be defined in the interval [-2,2] such that $f(x)=\begin{cases}
-1, &\text{$-2\leq x\leq 0$}\\
x-1, &\text{$0<x\leq 2$}
\end{cases}$ \\ \\and $g(x)=f(|x|)+|f(x)|$. Test the differentiability of $g(x)$ in (-2,2).\\

\item Let $f(x)$ be a continuous and $g(x)$ be a discontinuous function. Prove that $f(x)+g(x)$ is a discontinuous function.\\

\item Let $f(x)$ be a function satisfying the condition $f(-x)=f(x)$ for all real $x$. If $f'(0)$ exists, find its value.\\

\item Find the values of a and b so that the function\\ \\
$f(x)=\begin{cases}
x+a\sqrt{2}\sin x, &\text{$0\leq x\leq \pi/4$}\\
2x\cot x+b, &\text{$\pi/4\leq x\leq \pi/2$}\\
a\cos 2x-b \sin x, &\text{$\pi/2<x\leq \pi$}
\end{cases}$\\ \\is continuous for $0\leq x\leq \pi$.\\

\item Draw a graph of the function $y=[x]+|1-x|, -1\leq x\leq 3$. Determine the points, if any, where this function is not differentiable.
\item[~] 

\item Let $f(x)=\begin{cases}
\dfrac{1-\cos 4x}{x^2}, &\text{$x<0$}\\
a, &\text{$x=0$}\\
\dfrac{\sqrt{x}}{\sqrt{16+\sqrt{x}}-4}, &\text{$x>0$}
\end{cases}$\\ \\ Determine the value of $a$, if possible, so that the function is continuous at $x=0$.\\

\item A function $f:R \to R$ satisfies the equation $f(x+y)=f(x)f(y)$ for all $x,y$ in $R$ and $f(x)\neq 0$ for any $x$ in $R$. Let the function be differentiable at $x=0$ and $f'(0)=2$. Show that $f'(x)=2f(x)$ for all $x$ in $R$. hence, determine $f(x)$. \item[~]

\item Find $\lim\limits_{x \to 0}\{\tan(\pi/4+x)\}^{1/x}$. \item[~]

\item Let \\$f(x)=\begin{cases}
\{1+|\sin x|\}^{a/|\sin x|}; &\text{$\dfrac{\pi}{6}<x<0$}\\
b: &\text{$x=0$}\\
e^{\tan 2x/\tan 3x}; &\text{$0<x<\dfrac{\pi}{6}$}
\end{cases}$\\ \\Determine $a$ and $b$ such that $f(x)$ is continuous at $x=0$.\item[~]

\item Let $f\left(\dfrac{x+y}{2}\right)=\dfrac{f(x)+f(y)}{2}$ for all real \item[~] \item[~]$x$ and $y$. If $f'(0)$ exists and equal $-1$ and $f(0)=1$, find $f(2)$.\\
 
\item Determine the values of $x$ for which the following function fails to be continuous or differentiable : \\ \\
$f(x)=\begin{cases}
1-x, &\text{$x<1$}\\
(1-x)(2-x), &\text{$1\leq x\leq 2$}\\
3-x, &\text{$x>2$}
\end{cases}$ \\ \\Justify your answer.\\

\item Let $f(x),x\geq 0$, be non-negative continuous function, and let $F(x)=\int\limits_0^xf(t)dt, x\geq 0$. If for some $c>0, f(x)\leq cF(x)$ for all $x\geq 0$, then show that $f(x)=0$ for all $x\geq 0$. \item[~]

\item Let $\alpha \in R$. Prove that a function $f:R \to R$ is differentiable at $\alpha$ if and only if there is a function $g: R \to R$ which is continuous at $\alpha$ and satisfies $f(x)-f(\alpha)=g(x)(x-\alpha)$ for all $x \in R$. \item[~]

\item Let $f(x)=\begin{cases}
x+1, &\text{if $x<0$}\\
|x-1|, &\text{if $x\geq 0$}
\end{cases}$ and\\ $g(x)=\begin{cases}
x+1, &\text{if $x<0$}\\
(x-1)^2+b, &\text{if $x\geq 0$}
\end{cases}$ where $a$ and $b$ are non-negative numbers. Determine the composite function $g o f$. If $(g o f) (x)$ is continuous for all real $x$, determine the values of $a$ and $b$. Further, for these values of $a$ and $b$, is $g o f$ differentiable at $x=0$ ? Justify your answer. \item[~]

\item If a function $f:[-2a,2a] \to R$ is an odd function such that $f(x)=f(2a-x)$ for $x \in [a,2a]$ and the left hand derivative at $x=a$ is 0 then find the left hand derivative at $x=-a$.

\item $f'(0)=\lim\limits_{n \to \infty}nf\left(\dfrac{1}{n}\right)$ and $f(0)=0$. Using this find\\ \\ $\lim\limits_{n \to \infty}\left((n+1)\dfrac{2}{\pi}\cos^{-1}\left(\dfrac{1}{n}\right)-n\right)$, \\ \\$\left|cos^{-1}\dfrac{1}{n}\right| < \dfrac{\pi}{2}$

\item[~] \item if $|c|\leq\dfrac{1}{2}$ and $f(x)$ is a differentiable function at $x=0$ given by \\ \\
$f(x)=\begin{cases}
b\sin^{-1}\left(\dfrac{c+x}{2}\right), &\text{$-\dfrac{1}{2}<x<0$}\\
\dfrac{1}{2}, &\text{$x=0$}\\
\dfrac{e^{ax/2}-1}{x}, &\text{$0<x<\dfrac{1}{2}$}
\end{cases}$\\ \\
Find the value of 'a' and prove that $64b^2=4-c^2$.\\

\item If $f(x-y)=f(x)\dot{•}g(y)-f(y)\dot{•}g(x)$ and $g(x-y)=g(x)\dot{•}g(y)-f(x)\dot{•}f(y)$ for all $x, y \in R$. If right hand derivative at $x=0$ exists for $f(x)$. Find derivative of $g(x)$ at $x=0$.\\

\item Let $f:[1,\infty) \to [2,\infty)$ be a differentiable functions such that $f(1)=2$. If $6\int\limits_1^xf(t)dt=3xf(x)-x^3$ for all $x \geq 14$, then the value of $f(2)$ is ?\\

\item The largest value of non-negative integer a for which \\
$\lim\limits_{x \to 1}\left\{\dfrac{-ax+\sin(x-1)+a}{x+\sin(x-1)-1}\right\}^{\dfrac{1-x}{1-\sqrt{x}}}=\dfrac{1}{4}$ \\ \\is
\\

\item Let $f: R \to R$ and $g: R \to R$ be respectively given by $f(x)=|x|+1$ and $g(x)=x^2+1$. Define $h: R \to R$ by \\ \\
$h(x)=\begin{cases}
max \{f(x).g(x)\}, &\text{if $x\leq 0$}\\
min \{f(x).g(x)\}, &\text{if $x>0$}
\end{cases}$\\ \\
The number of points at which $h(x)$ is not differentiable is\\

\item Let $m$ and $n$ be two positive integers greater than 1. If $\lim\limits_{\alpha \to 0}\left(\dfrac{e^{\cos(\alpha^n)}-e}{\alpha^m}\right)=-\left(\dfrac{e}{2}\right)$ then \\
\\the value of $\dfrac{m}{n}$ is

\item Let $\alpha, \beta \in \mathbb{R}$ be such that $\lim\limits_{x \to 0}\dfrac{x^2\sin(\beta x)}{\alpha x-\sin x}=1$. Then $6(\alpha+\beta)$ equals \item[~]

\item $\lim\limits_{x \to 0}\dfrac{\sqrt{1-\cos 2x}}{\sqrt{2}x}$ is
\begin{itemize}
\begin{multicols}{2}
\item[(a)] 1 \item[(c)] zero \item[(b)] -1 \item[(d)] does not exist
\end{multicols}
\end{itemize}

\item $\lim\limits_{x \to \infty}\left(\dfrac{x^2+5x+3}{x^2+x+3}\right)^x$
\begin{itemize}
\begin{multicols}{4}
\item[(a)]$e^4$ \item[(b)] $e^2$ \item[(c)] $e^3$ \item[(d)] 1
\end{multicols}
\end{itemize}

\item Let $f(x)=4$ and $f'(x)=4$. Then\\ \\ $\lim\limits_{x \to 2}\dfrac{xf(2)-2f(x)}{x-2}$ is given by
\begin{itemize}
\begin{multicols}{4}
\item[(a)]2 \item[(b)] -2 \item[(c)] -4 \item[(d)] 3
\end{multicols}
\end{itemize}

\item$\lim\limits_{n \to \infty}\dfrac{1^p+2^p+3^p+...+n^p}{n^{p+1}}$
\begin{itemize}
\begin{multicols}{2}
\item[(a)] $\dfrac{1}{p+1}$ \item[~]\item[(c)] $\dfrac{1}{1-p}$ \item[(b)] $\dfrac{1}{p}-\dfrac{1}{p-1}$ \item[~] \item[(d)] $\dfrac{1}{p+2}$
\end{multicols}
\end{itemize}

\item[~] \item $\lim\limits_{x \to 0}\dfrac{\log x^n-[x]}{[x]}, n \in N$, ([x] denotes greatest integer less than or equal to x)
\begin{itemize}
\begin{multicols}{2}
\item[(a)] has value -1 \item[(c)] has value 1 \item[(b)] has value 0 \item[(d)] does not exist
\end{multicols}
\end{itemize}

\item If $f(1)=1, f'(1)=2$, then $\lim\limits_{x \to 1}\dfrac{\sqrt{f(x)}-1}{\sqrt{x}-1}$ is
\begin{itemize}
\begin{multicols}{4}
\item[(a)] 2 \item[(b)] 4 \item[(c)] 1 \item[(d)] 1/2
\end{multicols}
\end{itemize}

\item$f$ is defined in [-5,5] as\\
$f(x)=\begin{cases}
x, & \text{if $x$ is rational}\\
-x, & \text{if $x$ is irrational}
\end{cases}$. Then
\begin{itemize}
\item[(a)] $f(x)$ is continuous at every $x$, except $x=0$
\item[(b)] $f(x)$ is discontinuous at every $x$, except $x=0$
\item[(c)] $f(x)$ is continuous everywhere
\item[(d)] $f(x)$ is discontinuous everywhere\\
\end{itemize}

\item$f(x)$ and $g(x)$ are two differentiable functions on [0,2] such that $f''(x)-g''(x)=0, f'(1)=2g'(1)=4f(2)=3g(2)=9$ then $f(x)-g(x)$ at x = 3/2 is
\begin{itemize}
\begin{multicols}{4}
\item[(a)] 0 \item[(b)] 2 \item[(c)] 10 \item[(d)] 5
\end{multicols}
\end{itemize}

\item If $(x+y)=f(x).f(y) \forall x, y$ and $f(5)=2$, \\ \\$f'(0)=3$, then $f'(5)$ is
\begin{itemize}
\begin{multicols}{4}
\item[(a)] 0 \item[(b)] 1 \item[(c)] 6 \item[(d)] 2
\end{multicols}
\end{itemize}

\item \resizebox{.9\hsize}{.025\vsize}{$\lim\limits_{n \to \infty}\dfrac{1+2^4+3^4+...+n^4}{n^5}-\lim\limits_{n \to \infty}\dfrac{1+2^3+3^3+..+n^3}{n^5}$} \item[~] \item[~]
\begin{itemize}
\begin{multicols}{4}
\item[(a)] $\dfrac{1}{5}$ \item[(b)] $\dfrac{1}{30}$ \item[(c)] Zero \item[(d)] $\dfrac{1}{4}$
\end{multicols}
\end{itemize}\item[~]

\item If $\lim\limits_{x \to 0}\dfrac{\log(3+x)-\log(3-x)}{x}=k$, the value of $k$ is
\begin{itemize}
\begin{multicols}{4}
\item[(a)] $-\dfrac{2}{3}$ \item[(b)] 0 \item[(c)] $-\dfrac{1}{3}$ \item[(d)] $\dfrac{2}{3}$
\end{multicols}
\end{itemize} \item[~]

\item The value of $\lim\limits_{x \to 0}\dfrac{\int\limits_0^{x^2}\sec^2tdt}{x\sin x}$ is
\begin{itemize}
\begin{multicols}{4}
\item[(a)] 0 \item[(b)] 3 \item[(c)] 2 \item[(d)] 1
\end{multicols}
\end{itemize}

\item Let $f(a)=g(a)=k$ and their nth derivatives $f^n(a),g^n(a)$ exist and are not equal for some $n$. Further if \\ \\
$\lim\limits_{x \to a}\dfrac{f(a)g(x)-f(a)-g(a)f(x)+f(a)}{g(x)-f(x)}$=4 \\ \\ then the value of $k$ is
\begin{itemize}
\begin{multicols}{4}
\item[(a)] 0 \item[(b)] 4 \item[(c)] 2 \item[(d)] 1
\end{multicols}
\end{itemize}\item[~]

\item$\lim\limits_{x \to \dfrac{\pi}{2}}\dfrac{\left[1-\tan\left(\dfrac{x}{2}\right)\right][1-\sin x]}{\left[1+\tan\left(\dfrac{x}{2}\right)\right][\pi-2x]^3}$ is
\begin{itemize}
\begin{multicols}{4}
\item[(a)] $\infty$ \item[(b)] $\dfrac{1}{8}$ \item[(c)] 0 \item[(d)] $\dfrac{1}{32}$
\end{multicols}
\end{itemize} \item[~]

\item If $f(x)=\begin{cases}
xe^{-\left(\dfrac{1}{|x|}+\dfrac{1}{x}\right)}, &\text{$x\neq 0$}\\
0, &\text{$x=0$}
\end{cases}$ then $f(x)$ is
\begin{itemize}
\item[(a)] discontinuous everywhere
\item[(b)] continuous as well as differentiable for all $x$
\item[(c)] continuous for all $x$ but not differentiable at $x=0$
\item[(d)] neither differentiable nor continuous at $x=0$
\end{itemize}\item[~]

\item If $\lim\limits_{x \to \infty}\left(1+\dfrac{a}{x}+\dfrac{b}{x^2}\right)^{2x}=e^2$, then the \\ \\values of $a$ and $b$, are
\begin{itemize}
\begin{multicols}{2}
\item[(a)] $a=1$ and $b=2$ \item[(c)] $a \in R, b=2$ \item[(b)] $a=1, b\in R$ \item[(d)] $a\in R, b\in R$
\end{multicols}
\end{itemize}

\item Let $f(x)=\dfrac{1-\tan x}{4x-\pi}, x\neq \dfrac{\pi}{4}, x\in\left[0,\dfrac{\pi}{2}\right]$. \\ \\If $f(x)$ is continuous in $\left[0,\dfrac{\pi}{2}\right]$, then $f\left(\dfrac{\pi}{4}\right)$ is
\begin{itemize}
\begin{multicols}{4}
\item[(a)] -1 \item[(b)] $\dfrac{1}{2}$ \item[(c)] $-\dfrac{1}{2}$ \item[(d)] 1
\end{multicols}
\end{itemize}\item[~]

\item $\lim\limits_{n \to \infty}\left[\dfrac{1}{n^2}\sec^2\dfrac{1}{n^2}+\dfrac{2}{n^2}\sec^2\dfrac{4}{n^2}...+\dfrac{1}{n}\sec^21\right]$ \\ \\equals
\begin{itemize}
\begin{multicols}{2}
\item[(a)] $\dfrac{1}{2}\sec 1$ \item[~]\item[~]\item[(c)] $\tan 1$ \item[(b)] $\dfrac{1}{2}cosec 1$ \item[~]\item[~]\item[(d)] $\dfrac{1}{2}\tan 1$
\end{multicols}
\end{itemize}

\item Let $\alpha$ and $\beta$ be the distinct roots of $ax^2+bx+c=0$, then $\lim\limits_{x \to \alpha}\dfrac{1-\cos(ax^2+bx+c)}{(x-\alpha)^2}$ is equal to
\begin{itemize}
\begin{multicols}{2}
\item[(a)] $\dfrac{\alpha^2}{2}(\alpha-\beta)^2$\item[~]\item[(c)] $\dfrac{-\alpha^2}{2}(\alpha-\beta)^2$ \item[(b)] 0 \item[~]\item[(d)] $\dfrac{1}{2}(\alpha-\beta)^2$
\end{multicols}
\end{itemize}

\item Suppose $f(x)$ is differentiable at $x=1$ and $\lim\limits_{h \to 0}\dfrac{1}{h}f(1+h)=5$, then $f'(1)$ equals
\begin{itemize}
\begin{multicols}{4}
\item[(a)] 3 \item[(b)] 4 \item[(c)] 5 \item[(d)] 6
\end{multicols}
\end{itemize}

\item Let $f$ be differentiable for all $x$. If $f(1)=-2$ and $f'(x)\geq 2$ for $x \in [1,6]$, then
\begin{itemize}
\begin{multicols}{2}
\item[(a)] $f(6)\geq 8$ \item[(c)] $f(6)<5$ \item[(b)] $f(6)<8$ \item[(d)] $f(6)=5$
\end{multicols}
\end{itemize}

\item If $f$ is a real valued differentiable function satisfying $|f(x)-f(y)|\leq (x-y)^2, x, y \in R$ and $f(0)=0$, then $f(1)$ equals
\begin{itemize}
\begin{multicols}{4}
\item[(a)] -1 \item[(b)] 0 \item[(c)] 2 \item[(d)] 1
\end{multicols}
\end{itemize}

\item Let $f:R \to R$ be a function defined by$f(x)=$min$\{x+1,|x|+1\}$, Then which of the following is true ?
\begin{itemize}
\item[(a)] $f(x)$ is differentiable everywhere
\item[(b)] $f(x)$ is not differentiable at $x=0$
\item[(c)] $f(x)\geq 1$ for all $x \in R$
\item[(d)] $f(x)$ is not differentiable at $x=1$\\
\end{itemize}

\item The function $f:R/\{0\} \to R$ is given by\\
$f(x)=\dfrac{1}{x}-\dfrac{2}{e^{2x}-1}$ can be made continuous at $x=0$ by defininig $f(0)$ as
\begin{itemize}
\begin{multicols}{4}
\item[(a)] 0 \item[(b)] 1 \item[(c)] 2 \item[(d)] -1
\end{multicols}
\end{itemize} \item[~]

\item Let $f(x)=\begin{cases}
(x-1)\sin\dfrac{1}{x-1}, &\text{if $x\neq 1$}\\
0, &\text{if $x=1$}
\end{cases}$ \\ \\Then which of the following is true ?
\begin{itemize}
\item[(a)] $f$ is neither differentiable at $x=0$ nor at $x=1$
\item[(b)] $f$ is differentiable at $x=0$ and at $x=1$
\item[(c)] $f$ is differentiable at $x=0$ but not at $x=1$
\item[(d)] $f$ is differentiable at $x=1$ but not at $x=0$\\
\end{itemize}

\item Let $f:R \to R$ be a positive increasing function with $\lim\limits_{x \to \infty}\dfrac{f(3x)}{f(x)}=1$. then $\lim\limits_{x \to \infty}\dfrac{f(2x)}{f(x)}=$
\begin{itemize}
\begin{multicols}{4}
\item[(a)] $\dfrac{2}{3}$ \item[(b)] $\dfrac{3}{2}$ \item[(c)] 3 \item[(d)] 1
\end{multicols}
\end{itemize}

\item $\lim\limits_{x \to2}\left(\dfrac{\sqrt{1-\cos\{2(x-2)\}}}{x-2}\right)$
\begin{itemize}
\begin{multicols}{2}
\item[(a)] equals $\sqrt{2}$ \item[(c)] equals $\dfrac{1}{\sqrt{2}}$ \item[(b)] equals $-\sqrt{2}$ \item[(d)] does not exist
\end{multicols}
\end{itemize}

\item The values of $p$ and $q$ for which the function\\
\\
$f(x)=\begin{cases}
\dfrac{\sin(p+1)x+\sin x}{x}, &\text{$x<0$}\\
q, &\text{$x=0$}\\
\dfrac{\sqrt{x+x^2}-\sqrt{x}}{x^{3/2}}, & \text{$x>0$}
\end{cases}$ \\
\\is continuous for all $x$ in $R$, are
\begin{itemize}
\begin{multicols}{2}
\item[(a)] $p=\dfrac{5}{2}, q=\dfrac{1}{2}$ \item[~]\item[~]\item[(c)] $p=-\dfrac{3}{2}, q=\dfrac{1}{2}$ \item[(b)] $p=\dfrac{1}{2}, q=\dfrac{3}{2}$ \item[~]\item[~]\item[(d)] $p=\dfrac{1}{2}, q=-\dfrac{3}{2}$ \\
\end{multicols}
\end{itemize}

\item Let $f:R \to [0,\infty)$ be such that $\lim\limits_{x \to 5}f(x)$ exists and $\lim\limits_{x \to 5}\dfrac{(f(x))^2-9}{\sqrt{|x-5|}}=0$. Then $\lim\limits_{x \to 5}f(x)$ equals
\begin{itemize}
\begin{multicols}{4}
\item[(a)] 0 \item[(b)] 1 \item[(c)] 2 \item[(d)] 3
\end{multicols}
\end{itemize}

\item If $f:R \to R$ is a function defined by $f(x)=[x]\cos\left(\dfrac{2x-1}{2}\right)\pi$, where [x] denotes the greatest integer function, then $f$ is
\begin{itemize}
\item[(a)] continuous for every real $x$
\item[(b)] discontinuous only at $x=0$
\item[(c)] discontinuous only at non-zero integral values of $x$
\item[(d)] continuous only at $x=0$\\
\end{itemize}

\item Consider the function, $f(x)=|x-2|+|x-5|, x \in R$.\\
\textbf{Statement-1:}$f'(4)=0$\\
\textbf{Statement-2:}$f$ is continuous in [2,5], differentiable in (2,5) and $f(2)=f(5)$
\begin{itemize}
\item[(a)] Statement-1 is false, Statement-2 is true
\item[(b)] Statement-1 is true, Statement-2 is true; Statement-2 is a correct explanation for Statement-1
\item[(c)] Statement-1 is true, Statement-2 is true; Statement-2 is \textbf{not} a correct explanation for Statement-1
\item[(d)] Statement-1 is true, Statement-2 is false
\end{itemize} \item[~]

\item $\lim\limits_{x \to 0}\dfrac{(1-\cos 2x)(3+\cos x)}{x\tan 4x}$ is equal to
\begin{itemize}
\begin{multicols}{4}
\item[(a)] $-\dfrac{1}{4}$ \item[(b)] $\dfrac{1}{2}$ \item[(c)] 1 \item[(d)] 2
\end{multicols}
\end{itemize} \item[~]

\item $\lim\limits_{x \to 0}\dfrac{\sin(\pi\cos^2x)}{x^2}$ is equal to
\begin{itemize}
\begin{multicols}{4}
\item[(a)] $-\pi$ \item[(b)] $\pi$ \item[(c)] $\dfrac{\pi}{2}$ \item[(d)] 1
\end{multicols}
\end{itemize}

\item If the function,\\ \\
$g(x)=\begin{cases}
k\sqrt{x+1}, &\text{$0\leq x \leq 3$}\\
mx+2, &\text{$3<x\leq 5$}
\end{cases}$ is \\ \\differentiable, then the value of k + m is
\begin{itemize}
\begin{multicols}{4}
\item[(a)] $\dfrac{10}{3}$ \item[(b)] 4 \item[(c)] 2 \item[(d)] $\dfrac{16}{5}$
\end{multicols}
\end{itemize}

\item For $x \in R, f(x)=|\log 2-\sin x|$ and $g(x)=f(f(x))$, then
\begin{itemize}
\item[(a)] $g'(0)=-\cos(\log 2)$
\item[(b)] $g$ is differentiable at $x=0$ and $g'(0)=-\sin(\log 2)$
\item[(c)] $g$ is not differentiable at $x=0$
\item[(d)] $g'(0)=\cos(\log 2)$
\end{itemize}

\item $\lim\limits_{n \to \infty}\left(\dfrac{(n+1)(n+2)....3n}{n^{2n}}\right)^{\dfrac{1}{n}}$ is equal to
\begin{itemize}
\begin{multicols}{2}
\item[(a)] $\dfrac{9}{e^2}$ \item[~] \item[(c)] $\dfrac{18}{e^4}$ \item[(b)] $3\log 3-2$ \item[~]\item[(d)] $\dfrac{27}{e^2}$
\end{multicols}
\end{itemize}

\item Let $p=\lim\limits_{x \to 0^+}\left(1+\tan^2\sqrt{x}\right)^{\frac{1}{2x}}$ then $\log p$ is equal to
\begin{itemize}
\begin{multicols}{2}
\item[(a)] $\dfrac{1}{2}$ \item[~] \item[(c)] 2 \item[(b)] $\dfrac{1}{4}$ \item[~]\item[(d)] 1
\end{multicols}
\end{itemize}

\item $\lim\limits_{x \to \dfrac{\pi}{2}}\dfrac{\cot x-\cos x}{(\pi-2x)^3}$ equals
\begin{itemize}
\begin{multicols}{4}
\item[(a)] $\dfrac{1}{4}$ \item[(b)] $\dfrac{1}{24}$ \item[(c)] $\dfrac{1}{16}$ \item[(d)] $\dfrac{1}{8}$
\end{multicols}
\end{itemize}

\item For each $t \in R$, let [t] be the greatest integer less than or equal to t/. Then\\
\\$\lim\limits_{x \to 0^+}x\left(\left[\dfrac{1}{x}\right]+\left[\dfrac{2}{x}\right]+...+\left[\dfrac{15}{x}\right]\right)$
\begin{itemize}
\begin{multicols}{2}
\item[(a)] is equal to 15 \item[(c)] does not exist(in R) \item[(b)] is equal to 120 \item[(d)] is equal to 0
\end{multicols}
\end{itemize}

\item Let $S$ = $t \in R:f(x)=|x-\pi|(e^{|x|}-1)\sin |x|$ is not differentiable at t. Then the set S is equal to
\begin{itemize}
\begin{multicols}{2}
\item[(a)] \{0\} \item[(c)] $\{0,\pi\}$ \item[(b)] $\{\pi\}$ \item[(d)] $\phi$(an empty set)
\end{multicols}
\end{itemize}

\item $\lim\limits_{y \to 0}\dfrac{\sqrt{1+\sqrt{1+y^4}}-\sqrt{2}}{y^4}$
\begin{itemize}
\item[(a)] exists and equals $\dfrac{1}{4\sqrt{2}}$\\ \\
\item[(b)] exists and equals $\dfrac{1}{2\sqrt{2}(\sqrt{2}+1)}$\\ \\
\item[(c)] exists and equals $\dfrac{1}{2\sqrt{2}}$\\ 
\item[(d)] does not exist
\end{itemize}

\item Let $f: R \to R$ be a function defined as\\ \\
$f(x)=\begin{cases}
5, &\text{if $x\leq 1$}\\
a+bx, &\text{if $1<x<3$}\\
b+5x, &\text{if $3\leq x<5$}\\
30, &\text{if $x\geq 5$}
\end{cases}$ \\ \\Then f is
\begin{itemize}
\item[(a)] continuous if a = 5 and b = 5
\item[(b)] continuous if a = -5 and b = 10
\item[(c)] continuous if a = 0 and b = 5
\item[(d)] not continuous for any values of a and b
\end{itemize}

\item if the function f is defined on $\left(\dfrac{\pi}{6},\dfrac{\pi}{3}\right)$ by \\ \\
$f(x)=\begin{cases}
\dfrac{\sqrt{2}\cos x-1}{\cot x-1}, &\text{$x\neq \dfrac{\pi}{4}$}\\
k, &\text{$x=\dfrac{\pi}{4}$}
\end{cases}$\\ \\ is continuous, then k is equal to
\begin{itemize}
\begin{multicols}{4}
\item[(a)] 2 \item[(b)] $\dfrac{1}{2}$ \item[(c)] 1 \item[(d)] $\dfrac{1}{\sqrt{2}}$
\end{multicols}
\end{itemize}

\item Let $f(x)=15-|x-10|; x \in R$. Then the set of all values of x, at which the function, $g(x)=f(f(x))$ is not differentiable, is
\begin{itemize}
\begin{multicols}{2}
\item[(a)] \{5,10,15\} \item[(c)] \{5,10,15,20\} \item[(b)] \{10,15\} \item[(d)] \{10\}
\end{multicols}
\end{itemize}


\item In this questions there are entries in columns I and II. Each entry in column I is related to exactly one entry in column II.
\begin{align*}
\begin{tabular}{ll}
\textbf{Column I} &\textbf{Column II}\\
(A) $\sin(\pi[x])$ & (p) differentiable everywhere \\
(B) $\sin(\pi(x-[x]))$ & (q) nowhere differentiable\\
 &(r) not differentiable at 1 and -1
\end{tabular}
\end{align*}

\item In the following [x] denotes the greatest integer less than or equal to $x$.Match the functions in \textbf{Column I} with the properties in \textbf{Column II}
\begin{align*}
\begin{tabular}{ll}
\textbf{Column I} &\textbf{Column II}\\
(A) $x|x|$ & (p) continuous in (-1,1) \\
(B) $\sqrt{|x|}$ & (q) differentiable in (-1,1)\\
(C) $x+[x]$ &(r) strictly increasing in (-1,1)\\
(D) $|x-1|+|x+1|$ &(s) not differentiable at least at one point (-1,1)\\
\end{tabular}
\end{align*}

\item Let $f_1:R \to R, f_2:[0,\infty) \to R, f_3:R \to R$ and $f_4:R \to [0,\infty)$ be defined by $f_1(x)=\begin{cases}
|x|, &\text{if $x<0$}\\
e^x, &\text{if $x\geq 0$}
\end{cases}$ ;
$f_2(x)=x^2$; $f_3(x)=\begin{cases}
\sin x, &\text{if $x<0$}\\
x, &\text{if $x\geq 0$};
\end{cases}$ and 
\clearpage
\newpage$f_4(x)=\begin{cases}
f_2(f_1(x)), &\text{if $x<0$}\\
f_2(f_1(x))-1, &\text{if $x\geq 0$};
\end{cases}$
\begin{align*}
\begin{tabular}{ll}
\textbf{List-I} &\textbf{List-II}\\
(P) $f_4$ is & 1. onto but not one-one \\
(Q) $f_3$ is & 2. Neither continuous nor one-one\\
(R) $f_2of_1$ is &3. differentiable but not one-one\\
(S) $f_2$ is &4. continuous and one-one
\end{tabular}
\end{align*}
\begin{itemize}
\item[(a)] P $\to$ 3; Q $\to$ 1; R $\to$ 4; S $\to$ 2
\item[(b)] P $\to$ 1; Q $\to$ 3; R $\to$ 4; S $\to$ 2
\item[(c)] P $\to$ 3; Q $\to$ 1; R $\to$ 2; S $\to$ 4
\item[(d)] P $\to$ 1; Q $\to$ 3; R $\to$ 2; S $\to$ 4
\end{itemize} \item[~]

\item let $f_1:\mathbb{R} \to \mathbb{R}, f_2:\left(-\dfrac{\pi}{2},\dfrac{\pi}{2}\right) \to \mathbb{R}$\\ $f_3:\left(-1,e^{\dfrac{\pi}{2}-2}\right)$ and $f_4:\mathbb{R} \to \mathbb{R}$ be functions defined by
\begin{itemize}
\item[(i)] $f_1(x)=\sin\left(\sqrt{1-e^{-x^2}}\right)$
\end{itemize}
\begin{itemize}
\item[(ii)] $f_2(x)=\begin{cases}
\dfrac{|\sin x|}{tan^{-1}x}, &\text{if $x \neq 0$}\\
1, &\text{if $x=0$}
\end{cases}$, where the inverse trignometric function \\ \\$\tan^{-1}x$ assumes values in $\left(-\dfrac{\pi}{2},\dfrac{\pi}{2}\right)$\item[~]
\item[(iii)] $f_3(x)=[\sin(\log_e(x+2))]$, where for $t \in \mathbb{R}, [t]$ denotes the greatest integer less than or equal to $t$ \item[~]
\item[(iv)] $f_4(x)=\begin{cases}
x^2\sin\left(\dfrac{1}{x}\right), &\text{if $x \neq 0$}\\
0, &\text{if $x=0$}
\end{cases}$
\end{itemize}
\begin{align*}
\begin{tabular}{ll}
\textbf{List-I} &\textbf{List-II}\\
(P) The function $f_1$ is & 1. \textbf{NOT} continuous at $x=0$ \\
(Q) The function $f_2$ is & 2. Continuous at $x=0$ and \textbf{NOT} differentiable at $x=0$\\
(R) The function $f_3$ is &3. differentiable at $x=0$ and its derivative is \textbf{NOT} continuous at $x=0$\\
(S) The function $f_4$ is &4. differentiable at $x=0$ and its derivative is continuous at $x=0$
\end{tabular}
\end{align*}
\begin{itemize}
\item[(a)] P $\to$ 2; Q $\to$ 3; R $\to$ 1; S $\to$ 4
\item[(b)] P $\to$ 4; Q $\to$ 1; R $\to$ 2; S $\to$ 3
\item[(c)] P $\to$ 4; Q $\to$ 2; R $\to$ 1; S $\to$ 3
\item[(d)] P $\to$ 2; Q $\to$ 1; R $\to$ 4; S $\to$ 3
\end{itemize}
\end{enumerate}
