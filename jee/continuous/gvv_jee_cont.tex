\documentclass[journal,12pt,twocolumn]{IEEEtran}
%
\usepackage{setspace}
\usepackage{gensymb}
%\doublespacing
\singlespacing

%\usepackage{graphicx}
%\usepackage{amssymb}
%\usepackage{relsize}
\usepackage[cmex10]{amsmath}
%\usepackage{amsthm}
%\interdisplaylinepenalty=2500
%\savesymbol{iint}
%\usepackage{txfonts}
%\restoresymbol{TXF}{iint}
%\usepackage{wasysym}
\usepackage{amsthm}
\usepackage{iithtlc}
\usepackage{mathrsfs}
\usepackage{txfonts}
\usepackage{stfloats}
\usepackage{bm}
\usepackage{cite}
\usepackage{cases}
\usepackage{subfig}
%\usepackage{xtab}
\usepackage{longtable}
\usepackage{multirow}
%\usepackage{algorithm}
%\usepackage{algpseudocode}
\usepackage{enumitem}
\usepackage{mathtools}
\usepackage{tikz}
\usepackage{circuitikz}
\usepackage{verbatim}
\usepackage{tfrupee}
\usepackage[breaklinks=true]{hyperref}
%\usepackage{stmaryrd}
\usepackage{tkz-euclide} % loads  TikZ and tkz-base
\usetkzobj{all}
\usepackage{listings}
    \usepackage{color}                                            %%
    \usepackage{array}                                            %%
    \usepackage{longtable}                                        %%
    \usepackage{calc}                                             %%
    \usepackage{multirow}                                         %%
    \usepackage{hhline}                                           %%
    \usepackage{ifthen}                                           %%
  %optionally (for landscape tables embedded in another document): %%
    \usepackage{lscape}     
\usepackage{multicol}
\usepackage{chngcntr}
%\usepackage{enumerate}

%\usepackage{wasysym}
%\newcounter{MYtempeqncnt}
\DeclareMathOperator*{\Res}{Res}
%\renewcommand{\baselinestretch}{2}
\renewcommand\thesection{\arabic{section}}
\renewcommand\thesubsection{\thesection.\arabic{subsection}}
\renewcommand\thesubsubsection{\thesubsection.\arabic{subsubsection}}

\renewcommand\thesectiondis{\arabic{section}}
\renewcommand\thesubsectiondis{\thesectiondis.\arabic{subsection}}
\renewcommand\thesubsubsectiondis{\thesubsectiondis.\arabic{subsubsection}}

% correct bad hyphenation here
\hyphenation{op-tical net-works semi-conduc-tor}
\def\inputGnumericTable{}                                 %%

\lstset{
%language=C,
frame=single, 
breaklines=true,
columns=fullflexible
}
%\lstset{
%language=tex,
%frame=single, 
%breaklines=true
%}

\begin{document}
%


\newtheorem{theorem}{Theorem}[section]
\newtheorem{problem}{Problem}
\newtheorem{proposition}{Proposition}[section]
\newtheorem{lemma}{Lemma}[section]
\newtheorem{corollary}[theorem]{Corollary}
\newtheorem{example}{Example}[section]
\newtheorem{definition}[problem]{Definition}
%\newtheorem{thm}{Theorem}[section] 
%\newtheorem{defn}[thm]{Definition}
%\newtheorem{algorithm}{Algorithm}[section]
%\newtheorem{cor}{Corollary}
\newcommand{\BEQA}{\begin{eqnarray}}
\newcommand{\EEQA}{\end{eqnarray}}
\newcommand{\define}{\stackrel{\triangle}{=}}

\bibliographystyle{IEEEtran}
%\bibliographystyle{ieeetr}


\providecommand{\mbf}{\mathbf}
\providecommand{\pr}[1]{\ensuremath{\Pr\left(#1\right)}}
\providecommand{\qfunc}[1]{\ensuremath{Q\left(#1\right)}}
\providecommand{\sbrak}[1]{\ensuremath{{}\left[#1\right]}}
\providecommand{\lsbrak}[1]{\ensuremath{{}\left[#1\right.}}
\providecommand{\rsbrak}[1]{\ensuremath{{}\left.#1\right]}}
\providecommand{\brak}[1]{\ensuremath{\left(#1\right)}}
\providecommand{\lbrak}[1]{\ensuremath{\left(#1\right.}}
\providecommand{\rbrak}[1]{\ensuremath{\left.#1\right)}}
\providecommand{\cbrak}[1]{\ensuremath{\left\{#1\right\}}}
\providecommand{\lcbrak}[1]{\ensuremath{\left\{#1\right.}}
\providecommand{\rcbrak}[1]{\ensuremath{\left.#1\right\}}}
\theoremstyle{remark}
\newtheorem{rem}{Remark}
\newcommand{\sgn}{\mathop{\mathrm{sgn}}}
\providecommand{\abs}[1]{\left\vert#1\right\vert}
\providecommand{\res}[1]{\Res\displaylimits_{#1}} 
\providecommand{\norm}[1]{\left\lVert#1\right\rVert}
%\providecommand{\norm}[1]{\lVert#1\rVert}
\providecommand{\mtx}[1]{\mathbf{#1}}
\providecommand{\mean}[1]{E\left[ #1 \right]}
\providecommand{\fourier}{\overset{\mathcal{F}}{ \rightleftharpoons}}
%\providecommand{\hilbert}{\overset{\mathcal{H}}{ \rightleftharpoons}}
\providecommand{\system}{\overset{\mathcal{H}}{ \longleftrightarrow}}
	%\newcommand{\solution}[2]{\textbf{Solution:}{#1}}
\newcommand{\solution}{\noindent \textbf{Solution: }}
\newcommand{\cosec}{\,\text{cosec}\,}
\providecommand{\dec}[2]{\ensuremath{\overset{#1}{\underset{#2}{\gtrless}}}}
\newcommand{\myvec}[1]{\ensuremath{\begin{pmatrix}#1\end{pmatrix}}}
\newcommand{\mydet}[1]{\ensuremath{\begin{vmatrix}#1\end{vmatrix}}}
%\numberwithin{equation}{section}
\numberwithin{equation}{subsection}
%\numberwithin{problem}{section}
%\numberwithin{definition}{section}
\makeatletter
\@addtoreset{figure}{problem}
\makeatother

\let\StandardTheFigure\thefigure
\let\vec\mathbf
%\renewcommand{\thefigure}{\theproblem.\arabic{figure}}
\renewcommand{\thefigure}{\theproblem}
%\setlist[enumerate,1]{before=\renewcommand\theequation{\theenumi.\arabic{equation}}
%\counterwithin{equation}{enumi}


%\renewcommand{\theequation}{\arabic{subsection}.\arabic{equation}}

\def\putbox#1#2#3{\makebox[0in][l]{\makebox[#1][l]{}\raisebox{\baselineskip}[0in][0in]{\raisebox{#2}[0in][0in]{#3}}}}
     \def\rightbox#1{\makebox[0in][r]{#1}}
     \def\centbox#1{\makebox[0in]{#1}}
     \def\topbox#1{\raisebox{-\baselineskip}[0in][0in]{#1}}
     \def\midbox#1{\raisebox{-0.5\baselineskip}[0in][0in]{#1}}

\vspace{3cm}

\title{
	\logo{
Computational Approach to School Mathematics
	}
}
\author{ G V V Sharma$^{*}$% <-this % stops a space
	\thanks{*The author is with the Department
		of Electrical Engineering, Indian Institute of Technology, Hyderabad
		502285 India e-mail:  gadepall@iith.ac.in. All content in this manual is released under GNU GPL.  Free and open source.}
	
}	
%\title{
%	\logo{Matrix Analysis through Octave}{\begin{center}\includegraphics[scale=.24]{tlc}\end{center}}{}{HAMDSP}
%}


% paper title
% can use linebreaks \\ within to get better formatting as desired
%\title{Matrix Analysis through Octave}
%
%
% author names and IEEE memberships
% note positions of commas and nonbreaking spaces ( ~ ) LaTeX will not break
% a structure at a ~ so this keeps an author's name from being broken across
% two lines.
% use \thanks{} to gain access to the first footnote area
% a separate \thanks must be used for each paragraph as LaTeX2e's \thanks
% was not built to handle multiple paragraphs
%

%\author{<-this % stops a space
%\thanks{}}
%}
% note the % following the last \IEEEmembership and also \thanks - 
% these prevent an unwanted space from occurring between the last author name
% and the end of the author line. i.e., if you had this:
% 
% \author{....lastname \thanks{...} \thanks{...} }
%                     ^------------^------------^----Do not want these spaces!
%
% a space would be appended to the last name and could cause every name on that
% line to be shifted left slightly. This is one of those "LaTeX things". For
% instance, "\textbf{A} \textbf{B}" will typeset as "A B" not "AB". To get
% "AB" then you have to do: "\textbf{A}\textbf{B}"
% \thanks is no different in this regard, so shield the last } of each \thanks
% that ends a line with a % and do not let a space in before the next \thanks.
% Spaces after \IEEEmembership other than the last one are OK (and needed) as
% you are supposed to have spaces between the names. For what it is worth,
% this is a minor point as most people would not even notice if the said evil
% space somehow managed to creep in.



% The paper headers
%\markboth{Journal of \LaTeX\ Class Files,~Vol.~6, No.~1, January~2007}%
%{Shell \MakeLowercase{\textit{et al.}}: Bare Demo of IEEEtran.cls for Journals}
% The only time the second header will appear is for the odd numbered pages
% after the title page when using the twoside option.
% 
% *** Note that you probably will NOT want to include the author's ***
% *** name in the headers of peer review papers.                   ***
% You can use \ifCLASSOPTIONpeerreview for conditional compilation here if
% you desire.




% If you want to put a publisher's ID mark on the page you can do it like
% this:
%\IEEEpubid{0000--0000/00\$00.00~\copyright~2007 IEEE}
% Remember, if you use this you must call \IEEEpubidadjcol in the second
% column for its text to clear the IEEEpubid mark.



% make the title area
\maketitle

\newpage

\tableofcontents

\bigskip

\renewcommand{\thefigure}{\theenumi}
\renewcommand{\thetable}{\theenumi}
%\renewcommand{\theequation}{\theenumi}

%\begin{abstract}
%%\boldmath
%In this letter, an algorithm for evaluating the exact analytical bit error rate  (BER)  for the piecewise linear (PL) combiner for  multiple relays is presented. Previous results were available only for upto three relays. The algorithm is unique in the sense that  the actual mathematical expressions, that are prohibitively large, need not be explicitly obtained. The diversity gain due to multiple relays is shown through plots of the analytical BER, well supported by simulations. 
%
%\end{abstract}
% IEEEtran.cls defaults to using nonbold math in the Abstract.
% This preserves the distinction between vectors and scalars. However,
% if the journal you are submitting to favors bold math in the abstract,
% then you can use LaTeX's standard command \boldmath at the very start
% of the abstract to achieve this. Many IEEE journals frown on math
% in the abstract anyway.

% Note that keywords are not normally used for peerreview papers.
%\begin{IEEEkeywords}
%Cooperative diversity, decode and forward, piecewise linear
%\end{IEEEkeywords}



% For peer review papers, you can put extra information on the cover
% page as needed:
% \ifCLASSOPTIONpeerreview
% \begin{center} \bfseries EDICS Category: 3-BBND \end{center}
% \fi
%
% For peerreview papers, this IEEEtran command inserts a page break and
% creates the second title. It will be ignored for other modes.
%\IEEEpeerreviewmaketitle

\begin{abstract}
This is a problem set related to continuous maths based on JEE question papers
%.  Links to sample Python codes are available in the text.  
\end{abstract}
%Download python codes using 
%\begin{lstlisting}
%svn co https://github.com/gadepall/school/trunk/ncert/computation/codes
%\end{lstlisting}

%\section{Triangle}
%%\subsection{Construction Examples}
%%\input{./triangle/triangle_const_examples.tex}
%%\subsection{Construction Exercises}
%%\input{./triangle/triangle_const_exercises.tex} 
%\subsection{Triangle Examples}
%\input{./triangle/tri_exam.tex} 
%\subsection{Triangle Exercises}
%\input{./triangle/tri_geo.tex} 
%%
%\section{Quadrilateral}
%%\subsection{Construction Examples}
%%\input{./quad/quad_exam.tex} 
%%\subsection{Construction Exercises}
%%\input{./quad/quad_exer.tex} 
%\subsection{Quadrilateral Examples}
%\input{./quad/quad_geo_exam.tex} 
%\subsection{Quadrilateral Geometry}
%\input{./quad/quad_geo.tex} 
%%
%\section{Line}
%\subsection{Geometry: Examples}
%\input{./line/line_exam.tex}
%\subsection{Linear Inequalities: Examples}
%\input{./line/lineaeqn_exam.tex}
%\subsection{Linear Programing: Examples}
%\input{./line/lp_exam.tex}
%\subsection{Matrix Examples}
%\input{./line/matrix_exam.tex}
%\subsection{Complex Numbers}
%\input{./line/complex.tex}
%\subsection{Points and Vectors}
%\input{./line/coord_ex.tex}
%\subsection{Points on a Line}
%\input{./line/section.tex}
%\subsection{Lines and Planes}
%\input{./line/line_equation.tex}
%\subsection{Matrix Exercises}
%\input{./line/matrix_exer.tex}
%\subsection{Determinants}
%\input{./line/determinants.tex}
%\subsection{Linear Inequalities: Exercises}
%\input{./line/lineaeqn_exer.tex}
%\subsection{Linear Programing: Exercises}
%\input{./line/lp_exer.tex}
%
%%\subsection{Examples: Applications}
%%\input{./line/line_app.tex}
%\subsection{Miscellaneous}
%\input{./line/line_misc.tex}
%
%\section{Circle}
%%\subsection{Construction Examples}
%%\input{./circle/circle_examples.tex} 
%%\subsection{Construction Exercises}
%%\input{./circle/circle_exer.tex} 
%\subsection{Circle Geometry Examples}
%\input{./circle/circle_geo_exam.tex} 
%\subsection{Circle Geometry Exercises}
%\input{./circle/circle_geo.tex} 
%%\subsection{Circle Applications}
%%\input{./circle/circle_app.tex} 
%%
%\section{Conics}
%\subsection{Examples}
%\input{./conics/conics_exam.tex} 
%\subsection{Exercises}
%\input{./conics/conics_exer.tex} 
%%
%\section{Curves}
%\subsection{Examples}
%\input{./curves/curves_exam.tex} 
%\subsection{Exercises}
%\input{./curves/curves_exer.tex} 
%%\section{Miscellaneous Examples}
%%\input{./misc/misc_exam.tex} 
%\section{Miscellaneous Exercises}
%\input{./misc/misc_exer.tex} 
%%
%\section{Calculus}
%\subsection{Examples}
%\input{./calculus/calculus_exam.tex} 
%\subsection{Exercises}
%\input{./calculus/calculus_exer.tex} 
\section{Trigonometry}
\renewcommand{\theequation}{\theenumi}
\begin{enumerate}[label=\arabic*.,ref=\thesubsection.\theenumi]
\numberwithin{equation}{enumi}

 \item Suppose 
    \begin{align}
    \sin{^3x} \sin{3x} = \sum_{m=0}^{n} C_m\cos{mx}
    \end{align} is an identity in x, where $C_0 , C_1 ,......C_n$ are constants, and $C_n\neq0$ then find the value of n.
    \item Find the solution set of the system of equations 
    \begin{align}
    x + y =\frac{2\pi}{3}\\
    \cos{x} + \cos{y} = \frac{3}{2},
    \end{align} where x and y are real.
    \item Find the set of all x in the interval[0,$\pi$] for which \begin{align}
        2\sin{^2x}-3\sin{x} + 1\geq 0
    \end{align}
    \item The sides of a triangle inscribed in a given circle subtend angles $\alpha,\beta$ and $\gamma$ at the centre. Find the minimum value of the arithmetic mean of $\cos{(\alpha + \frac{\pi}{2})}, \cos{(\beta + \frac{\pi}{2})}$ and $\cos{(\gamma + \frac{\pi}{2})}.$
    \item Find the value of \\
    $\sin{\frac{\pi}{14}}\sin{\frac{3\pi}{14}}\sin{\frac{5\pi}{14}}\sin{\frac{7\pi}{14}}\sin{\frac{9\pi}{14}}\sin{\frac{11\pi}{14}}\sin{\frac{13\pi}{14}}.$
    \item If 
    \begin{align}
    K =\sin{(\frac{\pi}{18})}\sin{(\frac{5\pi}{18})}\sin{(\frac{7\pi}{18})},
    \end{align} 
    then find the numerical value of K ?
    \item If A$>$0,B$>$0 and 
    \begin{align}
    A + B = \frac{\pi}{3},
    \end{align}
    then find the maximum value of tan A tan B.
    \item Find the general value of $\theta$ satisfying the equation 
    \begin{align}
    \tan{^2\theta}  + \sec{2\theta} = 1. 
    \end{align}
    \item Find the real roots of the equation 
    \begin{align}
    \cos{^7x} + \sin{^4 x} = 1
    \end{align} 
    in the interval $(-\pi , \pi)$.
    \item If $\tan{\theta} = -\frac{4}{3},$ then find $\sin{\theta}.$ 
    \item If $\alpha+\beta+\gamma = 2\pi$ then 
    \begin{enumerate}
        \item $\tan{\frac{\alpha}{2}}+\tan{\frac{\beta}{2}}+\tan{\frac{\gamma}{2}} = \tan{\frac{\alpha}{2}}\tan{\frac{\beta}{2}}\tan{\frac{\gamma}{2}}$\\
        \item $\tan{\frac{\alpha}{2}} \tan{\frac{\beta}{2}}+\tan{\frac{\beta}{2}} \tan{\frac{\gamma}{2}}+\tan{\frac{\gamma}{2}} \tan{\frac{\alpha}{2}} = 1$\\
        \item $\tan{\frac{\alpha}{2}}+\tan{\frac{\beta}{2}}+\tan{\frac{\gamma}{2}} = -\tan{\frac{\alpha}{2}}\tan{\frac{\beta}{2}}\tan{\frac{\gamma}{2}}$\\
        \item None of these
    \end{enumerate}
    \item Given 
    \begin{align}
    A = \sin{^2 \theta} + \cos{^4 \theta}
    \end{align} then for all real values of $\theta$
    \begin{enumerate}
        \item $1\leq A \leq 2$
        \item $\frac{3}{4}\leq A \leq 1$
        \item $\frac{13}{16}\leq A \leq 1$
        \item $\frac{3}{4}\leq A \leq \frac{13}{16}$
    \end{enumerate}
   \item The equation 
   \begin{align}
       2\cos{^2\frac{x}{2}}\sin{^2 x} = x^2 + x^{-2};
       0<x<\frac{\pi}{2}
   \end{align} 
   has
   \begin{enumerate}
       \item no real solution
       \item One real solution
       \item more than the one solution
       \item none of these
   \end{enumerate}
   \item The general solution of the trigonometric equation 
   \begin{align}
      \sin{x} +\cos{x} = 1
   \end{align} is given by :
   \begin{enumerate}
       \item x = 2n$\pi; n= 0,\pm1,\pm2...$
       \item x = 2n$\pi + {\frac{\pi}{2}}; n= 0,\pm1,\pm2...$
       \item x = n$\pi + (-1)^n \frac{\pi}{4} - \frac{\pi}{4}; n= 0,\pm1,\pm2...$
       \item none of these
   \end{enumerate}
   \item The value of expression $\sqrt{3}cosec 20\degree-\sec{20\degree}$ is equal to 
   \begin{enumerate}
       \item 2
       \item $\frac{2\sin{20\degree}}{\sin{40\degree}}$
       \item 4
       \item $\frac{4\sin{20\degree}}{\sin{40\degree}}$
   \end{enumerate}
   \item The general solution of
   \begin{align}
       \sin{x}-3\sin{2x} +\sin{3x} =
       \cos{x} - 3\cos{2x} +\cos{3x}
   \end{align} is 
   \begin{enumerate}
       \item $n\pi + \frac{\pi}{8}$
       \item $\frac{n\pi}{2} + \frac{\pi}{8}$
       \item $(-1)^n \frac{n\pi}{2} + \frac{\pi}{8}$
       \item $2n\pi + \cos{^{-1}\frac{3}{2}}$
   \end{enumerate}
   \item The equation  \begin{align}
       (\cos{p} - 1) x^2 + (\cos{p}) x + \sin{p} = 0
   \end{align} In the variable x, has real roots. Then p can take any value in the interval
   \begin{enumerate}
       \item ( 0, $2\pi$)
       \item ($-\pi$ , 0) 
       \item ({$-\frac{\pi}{2}$},{$\frac{\pi}{2}$})
       \item (0 , $\pi$)
   \end{enumerate}
   \item Number of solutions of the equation 
   \begin{align}
       \tan{x} + \sec{x} = 2\cos{x}
   \end{align} 
   lying in the interval [0 , $2\pi$] is :
   \begin{enumerate}
       \item 0
       \item 1
       \item 2
       \item 3
   \end{enumerate}
   \item Let $0< x<\frac{\pi}{4}$ then ($\sec{2x} - \tan{2x}$) equals 
   \begin{enumerate}
       \item $\tan{(x - \frac{\pi}{4})}$
       \item $\tan {(\frac{\pi}{4} - x)}$
       \item $\tan{(x + \frac{\pi}{4})}$
       \item $\tan{^2 (x + \frac{\pi}{4})}$
   \end{enumerate}
   \item Let n be a positive integer such that $\sin{\frac{\pi}{2n}}+\cos{\frac{\pi}{2n}} = \frac{\sqrt n}{2}$.Then
   \begin{enumerate}
       \item $6\leq n\leq 8$
       \item $4< n\leq 8$
       \item $4\leq n\leq 8$
       \item $4< n< 8$
   \end{enumerate}
   \item If $\omega$ is an imaginary cube root of unity then the value of sin $\{(\omega^{10} + \omega^{23}) \pi -\frac{\pi}{4}\}$ is 
   \begin{enumerate}
       \item $-\frac{\sqrt3}{2}$
       \item $-\frac{1}{\sqrt2}$
       \item  $\frac{1}{\sqrt2}$
       \item  $\frac{\sqrt3}{2}$
   \end{enumerate}
   \item $3(\sin{x} -\cos{x})^4 + 6(\sin{x} +\cos{x})^2 + 4(\sin{^6x}+ \cos{^6x}) =$
   \begin{enumerate}
       \item 11
       \item 12
       \item 13
       \item 14
   \end{enumerate}
   \item The general values of $\theta$ satisfying equation 
   \begin{align}
       2\sin{^2\theta} - 3 \sin{\theta} -2 = 0
   \end{align} is 
   \begin{enumerate}
       \item $n\pi+(-1)^n\frac{\pi}{6}$
       \item $n\pi+(-1)^n\frac{\pi}{2}$
       \item $n\pi+(-1)^n\frac{5\pi}{6}$
       \item $n\pi+(-1)^n\frac{7\pi}{6}$
   \end{enumerate}
   \item $\sec{^2\theta} = \frac{4xy}{(x + y)^2}$ is true if and only if 
   \begin{enumerate}
       \item $x + y \neq 0$
       \item $x=y , x\neq 0$
       \item $x=y$
       \item $x\neq0, y\neq0$
   \end{enumerate}
   \item In a triangle PQR, $\angle{R} = \pi/2$.If $\tan({\frac{P}{2}})$ and $\tan{(\frac{Q}{2})}$ are the roots of the equation 
   \begin{align}
       ax^2 + bx + c = 0(a\neq 0)
   \end{align}
   then 
   \begin{enumerate}
       \item a+b=c
       \item b+c=a
       \item a+c=b
       \item b=c
   \end{enumerate}
   \item Let f($\theta$) =$\sin{\theta(\sin{\theta} +\sin{3\theta})}$. Then $f(\theta)$ is 
   \begin{enumerate}
       \item $\geq0$ only when $\theta\geq0$
       \item $\leq 0$ for all real $\theta$
       \item $\geq 0$ for all real $\theta$
       \item $\leq0$ only when $\theta\leq0$
   \end{enumerate}
   \item The number of distinct real roots of 
   $\begin{vmatrix}
   \sin{x} & \cos{x}  & \cos{x} \\ \cos{x} & \sin{x} & \cos{x} \\ \cos{x} & \cos{x} & \sin{x} 
   \end{vmatrix}$=0 \\
   in the interval $-\frac{\pi}{4}\leq x\leq \frac{\pi}{4}$ is 
   \begin{enumerate}
       \item 0
       \item 2
       \item 1
       \item 3
   \end{enumerate}
   \item The  maximum value of\\ $(\cos{\alpha_1}).(\cos{\alpha_2})...(\cos{\alpha_n})$, under the restrictions,
   $0\leq\alpha_1,\alpha_2,.....\alpha_n\leq{\frac{\pi}{2}}$ and $(\cot{\alpha_1}).(\cot{\alpha_2})...(\cot{\alpha_n}) = 1$
  is
   \begin{enumerate}
       \item $\frac{1}{2^{\frac{n}{2}}}$
       \item $\frac{1}{2^n}$
       \item $\frac{1}{2n}$
       \item 1
   \end{enumerate}
   \item If $\alpha+\beta =\frac{\pi}{2}$ and $\beta +\gamma =\alpha$,then $\tan{\alpha}$ equals
   \begin{enumerate}
       \item $2(\tan{\beta} + \tan{\gamma})$
       \item $\tan{\beta} + \tan{\gamma}$
       \item $\tan{\beta} +2\tan{\gamma}$
       \item 2$\tan{\beta} +\tan{\gamma}$
   \end{enumerate}
   \item The number of integral values of  k for which the equation 
   \begin{align}
       7 \cos{x}+5\sin{x}= 2k + 1
   \end{align} 
   has a solution is 
   \begin{enumerate}
       \item 4
       \item 8
       \item 10
       \item 12
   \end{enumerate}
   \item Given both $\theta$ and $\phi$ are acute angles and $\sin{\theta} =\frac{1}{2},\cos{\phi} =\frac{1}{3},$ then the value of $\theta + \phi$ belongs to 
   \begin{enumerate}
       \item $(\frac{\pi}{3},\frac{\pi}{2}]$
       \item $(\frac{\pi}{2},\frac{2\pi}{3})$
       \item $(\frac{2\pi}{3},\frac{5\pi}{6}]$
       \item $(\frac{5\pi}{6},\pi]$
   \end{enumerate}
   \item $\cos{(\alpha-\beta)} = 1$ and $\cos{(\alpha + \beta)} =\frac{1}{e}$ where $\alpha,\beta \epsilon [-\pi , \pi]$. Pairs of $\alpha,\beta$which satisfy both the equations is/are
   \begin{enumerate}
       \item 0
       \item 1
       \item 2
       \item 4
   \end{enumerate}
   \item The values of $\theta \epsilon (0 , 2\pi)$ for which $2\sin{^2\theta} - 5 \sin{\theta} + 2 > 0$, are
   \begin{enumerate}
       \item $(0 ,\frac{\pi}{6}) \cup (\frac{5\pi}{6},2\pi)$
       \item $(\frac{\pi}{8},\frac{5\pi}{6})$
       \item $(0 ,\frac{\pi}{8}) \cup (\frac{\pi}{6},\frac{5\pi}{6})$
       \item $(\frac{41\pi}{48},\pi)$
   \end{enumerate}
   \item Let $\theta \epsilon (0 ,\frac{\pi}{4})$ and $t_1 = (\tan{\theta})^{\tan{\theta}},t_2 = (\tan{\theta})^{\cot{\theta}},t_3 = (\cot{\theta})^{\tan{\theta}}$ and $t_4 = (\cot{\theta})^{\cot{\theta}}$, then
   \begin{enumerate}
       \item $t_1> t_2> t_3 > t_4$
       \item $t_4> t_3> t_1 > t_2$
       \item $t_3> t_1> t_2 > t_4$
       \item $t_2> t_3> t_1 > t_4$
   \end{enumerate}
   \item The number of solutions of the pair of equations 
   \begin{align}
       2\sin{^2\theta} - \cos{2\theta} = 0\\
       2\cos{^2\theta} - 3\sin{\theta} = 0
   \end{align}
   in the interval [0,$2\pi$] is
   \begin{enumerate}
       \item zero
       \item one
       \item two
       \item four
   \end{enumerate}
   \item For $x\epsilon (0 ,\pi)$, the equation 
   \begin{align}
       \sin{x} + 2\sin{2x}-\sin{3x} =3
   \end{align} has
   \begin{enumerate}
       \item infinitely many solutions
       \item three solutions
       \item one solution
       \item no solution
   \end{enumerate}
   \item Let $S = \{x\epsilon(-\pi , \pi) : x \neq 0, \pm{\frac{\pi}{2}}\}$. The sum of all distinct solutions of the  equation 
   \begin{align}
       \sqrt3\sec{x} + \mathrm{cosec}{x} +2(\tan{x}- \cot{x}) =0  
   \end{align}
   in the set S is equal to
   \begin{enumerate}
       \item $-\frac{7\pi}{9}$
       \item $-\frac{2\pi}{9}$
       \item 0
       \item $\frac{5\pi}{9}$
   \end{enumerate}
   \item The value of\\ $\sum_{k=1}^{13}\frac{1}{\sin{(\frac{\pi}{4}+\frac{(k-1)\pi}{6})}\sin{(\frac{\pi}{4}+\frac{k\pi}{6}})}$is equal to
   \begin{enumerate}
       \item $3 - \sqrt3$
       \item 2($3 - \sqrt3$)
       \item 2($\sqrt3 - 1$)
       \item 2($2 - \sqrt3$)
   \end{enumerate}
    \item $(1+\cos{\frac{\pi}{8}})(1+\cos{\frac{3\pi}{8}})(1+\cos{\frac{5\pi}{8}})(1+\cos{\frac{7\pi}{8}})$ is equal to 
    \begin{enumerate}
        \item $\frac{1}{2}$
        \item $\cos{(\frac{\pi}{8})}$
        \item $\frac{1}{8}$
        \item $\frac{1+\sqrt2}{2\sqrt2}$
    \end{enumerate}
    \item The expression $3[\sin{^4(\frac{3\pi}{2}-\alpha)}+\sin{^4(3\pi+\alpha]}-2[\sin{^6(\frac{\pi}{2}+\alpha)}+\sin{^6(5\pi-\alpha)]}$ is equal to 
    \begin{enumerate}
        \item 0
        \item 1
        \item 3
        \item $\sin{4\alpha} + \cos{6\alpha}$
        \item none of these
    \end{enumerate}
    \item The number of all possible triplets $(a_1,a_2,a_3)$ such that \begin{align}
        a_1 + a_2\cos{(2x)} +a_3\sin{^2(x)} =0
    \end{align}
    for all x is 
    \begin{enumerate}
        \item zero
        \item one
        \item three
        \item infinite
        \item none
    \end{enumerate}
    \item The values of $\theta$ lying between $\theta = 0$ and $\theta =\pi/2$ and satisfying the equation
    \begin{align}
        \begin{vmatrix}
        1+\sin{^2\theta} & \cos{^2\theta} & 4\sin{4\theta} \\ \sin{^2\theta }& 1+\cos{^2\theta} & 4\sin{4\theta} \\ \sin{^2\theta} & \cos{^2\theta} & 1+4\sin{4\theta}
        \end{vmatrix}=0 
    \end{align}are
    \begin{enumerate}
        \item $\frac{7\pi}{24}$
        \item $\frac{5\pi}{24}$
        \item $\frac{11\pi}{24}$
        \item $\frac{\pi}{24}$
    \end{enumerate}
    \item Let 
    \begin{align}
    2\sin{^2x}+3\sin{x}-2>0\\x^2-x-2 < 0
    \end{align}(x is measured in radians). Then x lies in the interval
    \begin{enumerate}
        \item$(\frac{\pi}{6},\frac{5\pi}{6})$
        \item$(-1,\frac{5\pi}{6})$
        \item (-1 , 2)
        \item$(\frac{\pi}{6},2)$
    \end{enumerate}
    \item The minimum value of the expression $\sin{\alpha} + \sin{\beta} + \sin{\gamma}$, where$\alpha,\beta,\gamma$are real numbers satisfying $\alpha+\beta+\gamma=\pi$ is
    \begin{enumerate}
        \item Positive
        \item zero
        \item negative
        \item -3
    \end{enumerate}
    \item The number of values of x in the interval $[0,\pi]$ satisfying the equation 
    \begin{align}
        3\sin{^2x} -7\sin{x}+2 =0
    \end{align}
    is 
    \begin{enumerate}
        \item 0
        \item 5
        \item 6
        \item 10
    \end{enumerate}
    \item Which of the following number(s) is/are/rational?
    \begin{enumerate}
        \item $\sin{15\degree}$
        \item $\cos{15\degree}$
        \item $\sin{15\degree}\cos{15\degree}$
        \item $\sin{15\degree}\cos{75\degree}$
    \end{enumerate}
    \item For a positive integer n, let $f_n(\theta) =\tan{ (\frac{\theta}{2})}(1 +\sec{\theta})(1+\sec{2\theta})(1+\sec{4\theta})......(1+\sec{2^n\theta}).$Then
    \begin{enumerate}
        \item $f_2(\frac{\pi}{16})= 1$
        \item $f_3(\frac{\pi}{32})= 1$
        \item $f_4(\frac{\pi}{64})= 1$
        \item $f_5(\frac{\pi}{128})= 1$
    \end{enumerate}
    \item If $\frac{\sin{^4x}}{2} + \frac{\cos{^4x}}{3} =\frac{1}{5},$ then
    \begin{enumerate}
        \item $\tan{^2x}=\frac{2}{3}$
        \item $\frac{\sin{^8x}}{8}+\frac{\cos{^8x}}{27}=\frac{1}{125}$
        \item $\tan{^2x}=\frac{1}{3}$
        \item $\frac{\sin{^8x}}{8}+\frac{\cos{^8x}}{27}=\frac{2}{125}$
    \end{enumerate}
    \item For $0<\theta<\frac{\pi}{2}.$ the solution(s) of 
    $\sum_{m=1}^{6}\mathrm{cosec}(\theta+\frac{(m-1)\pi}{4})\mathrm{cosec}(\theta+\frac{m\pi}{4}) = 4\sqrt{2}$ is(are)
    \begin{enumerate}
        \item $\frac{\pi}{4}$
        \item $\frac{\pi}{6}$
        \item $\frac{\pi}{12}$
        \item $\frac{5\pi}{12}$
    \end{enumerate}
    \item Let $\theta,\varphi \epsilon[0,2\pi]$ be such that $2 \cos{\theta}(1-\sin{\varphi}) = \sin{^2\theta}(\tan{\frac{\theta}{2}}+\cot{\frac{\theta}{2}})\cos{\varphi -1} , \tan{(2\pi-\theta)} > 0$ and $-1<\sin{\theta} < -\frac{\sqrt3}{2}$ , then $\varphi$  can not satisfy
    \begin{enumerate}
        \item $0< \varphi < \frac{\pi}{2}$
        \item $\frac{\pi}{2}< \varphi < \frac{4\pi}{3}$
        \item $\frac{4\pi}{3}< \varphi < \frac{3\pi}{2}$
        \item $\frac{3\pi}{2}< \varphi < 2\pi$
    \end{enumerate}
    \item The number of points in$(-\infty,\infty)$, for which \begin{align}
        x^2-x\sin{x} -\cos{x} =0
    \end{align}
    is
    \begin{enumerate}
        \item 6
        \item 4
        \item 2
        \item 0
    \end{enumerate}
    \item Let 
    \begin{align}
    f(x) = x \sin{\pi x}, x> 0
    \end{align}Then for all natural numbers n, $f'(x)$ vanishes at
    \begin{enumerate}
        \item A unique point in the interval (n,n+$\frac{1}{2}$)
        \item A unique point in the interval (n+$\frac{1}{2}$, n+1)
        \item A unique point in the interval (n,n+1)
        \item Two points in the interval (n,n+1)
    \end{enumerate}
    \item Let $\alpha$ and $\beta$ be non-zero real numbers such that $2(\cos{\beta}-\cos{\alpha}) + \cos{\alpha}\cos{\beta}= 1$.Then which of the following is/are true?
    \begin{enumerate}
        \item $\tan{(\frac{\alpha}{2})}+\sqrt3\tan{(\frac{\beta}{2})} = 0$
        \item $\sqrt3\tan{(\frac{\alpha}{2})}+\tan{(\frac{\beta}{2})} = 0$
        \item $\tan{(\frac{\alpha}{2})}-\sqrt3\tan{(\frac{\beta}{2})} = 0$
        \item $\sqrt3\tan{(\frac{\alpha}{2})}-\tan{(\frac{\beta}{2})} = 0$
    \end{enumerate}
    \item If $\tan{\alpha} = \frac{m}{m+1}$ and $\tan{\beta} =\frac{1}{2m+1}$, find the possible values of $(\alpha+\beta).$
    \item (a) Draw the graph of 
    \begin{align}
    y= \frac{1}{\sqrt2}(\sin{x} +\cos{x})
    \end{align} 
    from $x=-\frac{\pi}{2}$ to $x=\frac{\pi}{2}$\\
    (b)  If $\cos{(\alpha+\beta)} = \frac{4}{5}$, $\sin{(\alpha-\beta)} = \frac{5}{13}$ and $\alpha,\beta$ lies between 0 and $\frac{\pi}{4}$, find $\tan{2\alpha}$
    \item Given $\alpha+\beta-\gamma =\pi$, prove that 
    \begin{align}
    \sin{^2\alpha}+\sin{^2\beta}-\sin{^2\gamma} = 2\sin{\alpha}\sin{\beta} \cos\gamma
    \end{align}
    \item Given A= \{ x: $\frac{\pi}{6} \leq x\leq \frac{\pi}{3}$\} and \begin{align}
        f(x) = \cos{x} - x(1+x);
    \end{align}
    find f(A)
    \item  For all $\theta$ in $[0,\pi/2]$ show that, 
    \begin{align}
    \cos{(\sin{\theta})} \geq \sin{(\cos{\theta})}
    \end{align}.
    \item Without using tables, Prove that $(\sin{12\degree})(\sin{48\degree})(\sin{54\degree}) = \frac{1}{8}$
    \item Show that $16 \cos{(\frac{2\pi}{15})}\cos{(\frac{4\pi}{15})}\cos{(\frac{8\pi}{15})}\cos{(\frac{16\pi}{15})} = 1$
    \item Find all the solution of $4\cos{^2x}\sin{x} -2\sin{^2x} = 3\sin{x}$
    \item Find the values of x$\epsilon(-\pi,\pi)$ which satisfy the equation 
    \begin{align}
        8^{(1+\abs{\cos{x \vert}} + \vert cos^2x \vert +\vert cos^3x \vert+ .....)} = 4^3
    \end{align}
    \item Prove that $\tan{\alpha} + 2 \tan{2\alpha} + 4
    \tan{4\alpha} + 8\cot{8\alpha} = \cot{\alpha}$
    \item ABC is a triangle such that 
    $\sin{(2A+B)} = \sin{(C-A)} = -\sin{(B+2c)} = \frac{1}{2}$ If A,B and C are in arithmetic progression, determine the values of A, B and C.
    \item if exp\{$(\sin{^2x} + \sin{^4x}+ \sin{^6x}+ ......\infty$) In 2 \} satisifies the equation 
    \begin{align}
        x^2 - 9x + 8 =0
    \end{align}, find the value of $\frac{\cos{x}}{\cos{x}+\sin{x}}$, $0< x<\frac{\pi}{2}.$\\
    \item Show that the value of $\frac{\tan{x}}{\tan{3x}}$, wherever defined never lies between $\frac{1}{3}$ and 3.
    \item Determine the smallest positive value of x(in degrees) for which $\tan{(x+100\degree)} = \tan{(x+50\degree)}\tan{(x)}\tan{(x-50\degree)}.$
    \item Find the smallest positive number p for which the equation $\cos{(p\sin{x})} = \sin{(p\cos{x})}$ has a solution  $x\epsilon [0 , 2\pi]$
    \item Find all values of $\theta$ in the interval $(-\frac{\pi}{2},\frac{\pi}{2})$ satisfying the equation \begin{align}
        (1-\tan{\theta})(1+\tan{\theta})\sec{^2\theta} + 2^{\tan{^2\theta}} = 0
    \end{align}
    \item Prove that the values of the function $\frac{\sin{x} \cos{3x}}{\sin{3x}\cos{x}}$ do not lie between $\frac{1}{3}$ and 3 for any real x.
    \item Prove that $\sum_{k=1}^{n-1}(n-k) cos {\frac{2k\pi}{n}} = -\frac{n}{2}$, where $n\geq3$ is an integer
    \item If any triangle ABC, Prove that 
    $\cot{\frac{A}{2}}+\cot{\frac{B}{2}}+\cot{\frac{C}{2}}=\cot{\frac{A}{2}}\cot{\frac{B}{2}}\cot{\frac{c}{2}}$
    \item Find the range of values of t for which $2 \sin{t} = {\frac{1-2x+5x^2}{3x^2-2x-1}}, t \epsilon [{-\frac{\pi}{2}},{\frac{\pi}{2}}].$\\           

This section contains 1 paragraph, Based on each paragraph,there are 2 questions. Each question has four options (A),(B),(C) and (D) ONLY ONE of these four options is correct.
{\center\textbf{PARAGRAPH 1}}\\
Let O be the origin, and $\vec {OX},\vec {OY},\vec {OZ}$ be three unit vectors in the directions of the sides $\vec {QR},\vec {RP},\vec {PQ}$ respectively, of a triangle PQR

    \item $\abs{ \vec {OX} \times \vec {OY}}$ =
    \begin{enumerate}
        \item $\sin{(P+Q)}$
        \item $\sin{2R}$
        \item $\sin{(P+R)}$
        \item $\sin{(Q+R)}$
    \end{enumerate}
    \item If the triangle PQR varies, then the minimum value of $\cos{(P+Q)}+\cos{(Q+R)}+\cos{(R+P)}$ is \begin{enumerate}
        \item $-\frac{5}{3}$
        \item $-\frac{3}{2}$
        \item $ \frac{3}{2}$
        \item $ \frac{5}{3}$
    \end{enumerate}
    \item The number of all possible values of $\theta$ where $0<\theta<\pi$,for which the system of equations\\\\
    $(y+z) \cos 3\theta = (xyz)\sin 3 \theta$\\\\
    $x \sin 3\theta =  {\frac{2\cos 3 \theta}{y}}+{\frac{2 \sin 3\theta}{z}}$\\\\
    $(xyz)\sin 3\theta = (y+2z)\cos 3\theta + y\sin 3\theta$\\
    have a solution ($x_0 , y_0 , z_0$) with $y_0 z_0 \neq 0$,is
    \item The number of values of $\theta$ in the interval, $({-\frac{\pi}{2}},{\frac{\pi}{2}})$ such that $\theta \neq {\frac{n \pi}{5}}$ for n = 0 , $\pm1$,$\pm2$ and $\tan\theta = \cot 5\theta$ as well as $\sin 2\theta = \cos 4\theta$ is 
    \item The maximum value of the expression ${\frac{1}{\sin^2\theta + 3 \sin\theta \cos\theta + 5\cos^2 \theta}}$ is 
    \item Two parallel chords of a circle of radius 2 are at a distance $\sqrt3 + 1$ apart. If the chords subtend at the center, angles of ${\frac{\pi}{k}}$ and ${\frac{2\pi}{k}}$, where $k > 0$, then the value of [k] is\\
    \textbf{Note}: [k] denotes the largest integer less than or equal to k.
    \item The positive integer value of $n>3$ satisfying the equation \\
    ${\frac{1}{\sin({\frac{\pi}{n}})}}= {\frac{1}{\sin({\frac{2\pi}{n}})}} + {\frac{1}{\sin({\frac{3\pi}{n}})}}$ is 
    \item The number of distinct solutions of the equation ${\frac{5}{4}}\cos^2 2x + \cos^4 x + \sin^4 x + \cos^6 x + \sin^6 x = 2$ in the interval $[0 , 2\pi]$ is
    \item Let a,b,c be three non-zero real numbers such that the equation : $\sqrt 3 a \cos x+ 2b \sin x = c , x\epsilon[{-\frac{\pi}{2}},{\frac{\pi}{2}}]$ has two distinct real roots $\alpha$ and $\beta$ with $\alpha + \beta = {\frac{\pi}{3}}$. Then, the value of ${\frac{b}{a}}$ is 
    \item The period of $\sin^2 \theta $ is 
    \begin{enumerate}
        \item $\pi^2$
        \item $\pi$
        \item $2\pi$
        \item $\pi/2$
    \end{enumerate}
    \item The number of solution of $\tan x + \sec x = 2\cos x $ in $[0,2\pi)$ is 
    \begin{enumerate}
        \item 2
        \item 3
        \item 0
        \item 1
    \end{enumerate}
    \item Which one is not periodic 
    \begin{enumerate}
        \item $\abs{\sin3x} + \sin^2 x$
        \item $\cos\sqrt x + \cos^2 x$
        \item $\cos 4x + \tan^2 x$
        \item $\cos 2x + \sin x$
    \end{enumerate}
    \item Let $\alpha, \beta$ be such that $\pi<\alpha-\beta< 3\pi$. If $\sin\alpha + \sin \beta = {-\frac{21}{65}}$ and $\cos \alpha + \cos\beta = {-\frac{27}{65}}$, then the value of $\cos{\frac{\alpha-\beta}{2}}$
    \begin{enumerate}
        \item ${-\frac{6}{65}}$
        \item ${\frac{3}{\sqrt{130}}}$
        \item ${\frac{6}{65}}$
        \item ${-\frac{3}{\sqrt{130}}}$
    \end{enumerate}
    \item If u = $\sqrt{a^2\cos^2\theta +b^2\sin^2\theta}$+$\sqrt{a^2\sin^2\theta +b^2\cos^2\theta}$then the difference between the maximum and minimum values of $u^2$ is given by
    \begin{enumerate}
        \item $(a - b) ^2$
        \item $2\sqrt{a^2 + b^2}$
        \item $(a + b) ^2$
        \item $2 (a^2+b^2)$
    \end{enumerate}
    \item A line makes the same angle $\theta$, with each of the x and z axis. If the angle $\beta$, which it makes with y-axis, is such that $\sin^2\beta = 3\sin^2\theta$, then $\cos^2\theta$ equals
    \begin{enumerate}
        \item ${\frac{2}{5}}$
        \item ${\frac{1}{5}}$
        \item ${\frac{3}{5}}$
        \item ${\frac{2}{3}}$
    \end{enumerate}
    \item The number of values of x in the interval $[0, 3\pi]$ satisfying the equation 
    \begin{align}
    2\sin{^2x} + 5\sin{x} -3 =0
    \end{align} is
    \begin{enumerate}
        \item 4
        \item 6
        \item 1
        \item 2
    \end{enumerate}
    \item If $0< x<\pi$ and $\cos x+\sin x={\frac{1}{2}} $, then $\tan x$ is 
    \begin{enumerate}
        \item ${\frac{(1-\sqrt7)}{4}}$
        \item ${\frac{(4-\sqrt7)}{3}}$
        \item ${-\frac{(4+\sqrt7)}{3}}$
        \item ${\frac{(1+\sqrt7)}{4}}$
    \end{enumerate}
    \item Let A and B denote the statements\\
    A : $\cos\alpha+\cos\beta+\cos\gamma = 0$\\
    B : $\sin\alpha+\sin\beta+\sin\gamma = 0$\\
    If $\cos(\beta-\gamma)+\cos(\gamma-\alpha)+\cos(\alpha-\beta) = {-\frac{3}{2}}$,then :
    \begin{enumerate}
        \item A is false and B is true
        \item Both A and B are true
        \item both A and B are false
        \item A is true and B is false
    \end{enumerate}
    \item Let $\cos(\alpha+\beta) = {\frac{4}{5}}$ and $\sin(\alpha-\beta) = {\frac{5}{13}}$, where $0\leq\alpha, \beta\leq{\frac{\pi}{4}}$, Then $\tan 2\alpha = $
    \begin{enumerate}
        \item ${\frac{56}{33}}$
        \item ${\frac{19}{12}}$
        \item ${\frac{20}{7}}$
        \item ${\frac{25}{16}}$
    \end{enumerate}
    \item If A = $\sin^2x+\cos^4 x$ , then for all real x:
    \begin{enumerate}
        \item ${\frac{13}{16}\leq A\leq 1}$
        \item $1\leq A\leq 2$
        \item $\frac{3}{4}\leq A\leq \frac{13}{16} $
        \item ${\frac{3}{4}\leq A\leq 1}$
    \end{enumerate}
    \item In a $\triangle PQR$, If $3 \sin P + 4 \cos Q = 6$ and 4 $\sin Q + 3 \cos P = 1$, then the angle R is equal to :
    \begin{enumerate}
        \item ${\frac{5\pi}{6}}$
        \item ${\frac{\pi}{6}}$
        \item ${\frac{\pi}{4}}$
        \item ${\frac{3\pi}{4}}$
    \end{enumerate}
    \item ABCD is a trapezium such that AB and CD are parallel and $BC\perp CD$. If $\angle ADB= \theta$, BC=p and CD=q, then AB is equal to :
    \begin{enumerate}
        \item ${\frac{(p^2+q^2)\sin\theta}{p\cos\theta + q\sin \theta}}$\\
        \item ${\frac{p^2 + q^2\cos \theta}{p\cos \theta + q\sin \theta}}$\\
        \item ${\frac{p^2 + q^2}{p^2\cos \theta + q^2\sin \theta}}$\\
        \item ${\frac{(p^2 + q^2)\sin \theta}{(p \cos \theta + q \sin \theta)^2}}$\\
    \end{enumerate}
    \item The expression $\frac{\tan{A}}{1- \cot{A}}+\frac{\cot{A}}{1-\tan{A}}$ can be written as:
    \begin{enumerate}
        \item $\sin A \cos A + 1$
        \item $\sec{A} cosec A+ 1$
        \item $\tan{A} + \cot{A}$
        \item $\sec{A} + cosec A$
    \end{enumerate}
    \item Let $f_k(x) = {\frac{1}{k}}(\sin^k x+ \cos^k x)$ where $x\epsilon R$ and $k\geq 1$. Then $f_4(x) - f_6(x)$ equals
    \begin{enumerate}
        \item {$\frac{1}{4}$}
        \item {$\frac{1}{12}$}
        \item {$\frac{1}{6}$}
        \item {$\frac{1}{3}$}
    \end{enumerate}
    \item If $0\leq x < 2\pi$, then the number of real values of x, which satisfy the equation $\cos x+\cos 2x+ \cos 3x+\cos 4x =0$ is:
    \begin{enumerate}
    \item 7
    \item 9
    \item 3
    \item 5
    \end{enumerate}
    \item If $5(\tan^2x-\cos^2 x) = 2\cos2x + 9$,then the value of cos4x is :
    \begin{enumerate}
        \item {$-\frac{7}{9}$}
        \item {$-\frac{3}{5}$}
        \item {$\frac{1}{3}$}
        \item {$\frac{2}{9}$}
    \end{enumerate}
    \item If sum of all the solutions of the equation 8 $\cos x.(\cos({\frac{\pi}{6}}+ x)(\cos({\frac{\pi}{6}}- x)-{\frac{1}{2}}) -1 $ in $[0 , \pi]$ is $k\pi$. then k is equal to :
    \begin{enumerate}
        \item {$\frac{13}{9}$}
        \item {$\frac{8}{9}$}
        \item {$\frac{20}{9}$}
        \item {$\frac{2}{3}$}
    \end{enumerate}
    \item For any $\theta \epsilon ({\frac{\pi}{4}},{\frac{\pi}{2}})$ the expression $3(\sin\theta - \cos\theta)^4 + 6(\sin\theta + \cos \theta)^2 + 4 \sin^2\theta$ equals:
    \begin{enumerate}
        \item $13- 4\cos^2\theta+6\sin^2\theta\cos^2\theta$
        \item $13- 4\cos^6\theta$
        \item $13- 4\cos^2\theta+6\cos^4\theta$
        \item $13- 4\cos^4\theta+2\sin^2\theta\cos^2\theta$
    \end{enumerate}
    \item The value of $\cos^2 10\degree- \cos10\degree\cos 50\degree + \cos^2 50\degree$ is:
    \begin{enumerate}
        \item ${\frac{3}{4}}+ \cos 20\degree$
        \item ${\frac{3}{4}}$
        \item ${\frac{3}{2}}(1+ \cos 20\degree)$
        \item ${\frac{3}{2}}$
    \end{enumerate}
    \item Let S=\{$\theta \epsilon [-2\pi,2\pi]: 2\cos^\theta+3\sin\theta =0$\}. Then the sum of the elements of S is 
    \begin{enumerate}
        \item ${\frac{13\pi}{6}}$
        \item ${\frac{5\pi}{3}}$
        \item 2
        \item 1
    \end{enumerate}
    {\Large\textbf{Match the Following}}\\\\
{\textbf{DIRECTIONS (Q.1):}}
\begin{textit}
{Each question contains statements given in two columns, which have to be matched. The statements in Column-I are labelled A, B , C and D, while the statements in Column-II are labelled p,q,r,s and t. Any given statement in Column-I can have correct matching with ONE OR MORE statement(s) in Column-II. The appropriate bubbles corresponding to the answers to theses questions have to be darkened as illustrated in the following example:\\
If the correct matches are A-p, s and t; B-q and r; C-p and q; D -s then the correct darkening of bubbles will look like the given} \end{textit}
\line(1,0){250}
\begin{enumerate}
    \item In this question there are entries in columns 1 and 2. Each entry in column 1 is related to exactly one entry in column 2. Write the correct letter from column 2 against the entry number in column 1 in your answer book.\\
{\Large{$\frac{\sin3\alpha}{\cos2\alpha}$}} is\\\\
\begin{tabular}{llll}
\textbf{Column-I} &   \enspace   &   \textbf{Column-II}\\
(A) Positive &   \enspace   &   (p)$({\frac{13\pi}{48}},{\frac{14\pi}{48}})$\\
&&&\\
(B) Negative    &   \enspace   & (q)$({\frac{14\pi}{48}},{\frac{18\pi}{48}})$\\
&&&\\
    &\enspace   &   (r)$({\frac{18\pi}{48}},{\frac{23\pi}{48}})$\\
&&&\\
  &\enspace  &   (s)$(0,{\frac{\pi}{2}})$\\
&&&\\
\end{tabular}
\item Let\\
f(x)=$\sin{(\pi \cos{x})}$ and g(x)=$\cos{(2\pi \sin{x})}$be two functions defined for $x>0$. Define the following sets whose elements are written i n the increasing order.\\\\
X =\{$x : f(x) = 0$\},Y =\{$x : f'(x) = 0$\} \\\\
Z =\{$x : g(x) = 0$\},W =\{$x : g'(x) = 0$\}\\\\
List-I contains the sets X,Y,Z and W. List-II contains some information regarding these sets.\\
\begin{tabular}{llll}
\textbf{Column-I} &   \enspace   &   \textbf{Column-II}\\
(A)X &   \enspace   &   (p)$\supseteq\{{\frac{\pi}{2}},{\frac{3\pi}{2}},4\pi,7\pi\}$\\
&&&\\
(B)Y    &   \enspace   & (q)an arithmetic progression\\
&&&\\
(C)Z    &\enspace   &   (r)NOT an arithmetic progression\\
&&&\\
(D)W &\enspace   &   (s)$\supseteq\{{\frac{\pi}{6}},{\frac{7\pi}{6}},{\frac{13\pi}{6}}\}$\\
&&&\\
    &\enspace   &   (t)$\supseteq\{{\frac{\pi}{3}},{\frac{2\pi}{3}},\pi\}$\\&&&\\
    &\enspace   &   (u)$\supseteq\{{\frac{\pi}{6}},{\frac{3\pi}{4}}\}$\\
\end{tabular}
Which of the following is the only CORRECT combination?
\begin{enumerate}
    \item (IV),(P),(R),(S)
    \item (III),(P),(Q),(U)
    \item (III),(R),(U)
    \item (IV),(Q),(T)
\end{enumerate}
\item Let $f(x) = \sin(\pi\cos x)$ and $g(x) = \cos(2\pi\sin x)$ be two functions defined for $x>0$.Define the following sets whose elements are written in the increasing order\\\\
X =\{$x : f(x) = 0$\},Y =\{$x : f'(x) = 0$\} \\\\
Z =\{$x : g(x) = 0$\},W =\{$x : g'(x) = 0$\}\\\\
List-I contains the sets X,Y,Z and W. List-II contains some information regarding these sets.\\
\begin{tabular}{llll}
\textbf{Column-I} &   \enspace   &   \textbf{Column-II}\\
(A)X &   \enspace   &   (p)$\supseteq\{{\frac{\pi}{2}},{\frac{3\pi}{2}},4\pi,7\pi\}$\\
&&&\\
(B)Y    &   \enspace   & (q)an arithmetic progression\\
&&&\\
(C)Z    &\enspace   &   (r)NOT an arithmetic progression\\
&&&\\
(D)W &\enspace   &   (s)$\supseteq\{{\frac{\pi}{6}},{\frac{7\pi}{6}},{\frac{13\pi}{6}}\}$\\
&&&\\
    &\enspace   &   (t)$\supseteq\{{\frac{\pi}{3}},{\frac{2\pi}{3}},\pi\}$\\&&&\\
    &\enspace  &   (u)$\supseteq\{{\frac{\pi}{6}},{\frac{3\pi}{4}}\}$\\
\end{tabular}
Which of the following is the only CORRECT combination?\\
\begin{enumerate}
    \item (I),(Q),(U)
    \item (I),(P),(R)
    \item (II),(R),(S)
    \item (II),(Q),(T)
\end{enumerate}
\end{enumerate}
    \end{enumerate}
%\end{document}
    
 
\section{Inverse Trigonometric Functions}
\renewcommand{\theequation}{\theenumi}
\begin{enumerate}[label=\arabic*.,ref=\thesubsection.\theenumi]
\numberwithin{equation}{enumi}

\item Let a, b, c be positive real numbers. Let
\begin{align*}
\theta = \tan^{-1}\sqrt{\frac{a(a + b - c)}{bc}} + \tan^{-1} \sqrt{ \frac{b(a + b + c) }{ca}}\\ 
+ \tan^{-1}\sqrt{\frac{c(a +  b - c)}{ab}}
\end{align*}
Then $\tan \theta$ = ..............

\item The numerical value of 
\begin{align*}
\tan \{2\tan^{-1}(\frac{1}{5}) - \frac{\pi}{4}\}
\end{align*} 
is equal to .............

\item The greater of the two angles 
\begin{align*}
A = 2\tan^{-1}(2\sqrt{2} - 1)
\end{align*}
\begin{align*}
B = 3\sin^{-1}(\frac{1}{3}) + \sin^{-1}(\frac{3}{5})
\end{align*}
is.................

\textbf{MCQ's with One Correct Answer}

\item The value of $\tan^{-1}[(\cos^{-1}\frac{4}{5}) + \tan^{-1}(\frac{2}{3})]$ is
\begin{enumerate}
\item $\frac{6}{17}$
\item $\frac{7}{16}$
\item $\frac{16}{7}$
\item none of these
\end{enumerate}

\item If we consider only the principle values of the inverse trigonometric functions then the value of
$\tan(\cos^{-1}\frac{1}{5\sqrt{2}} - \sin^{-1}\frac{4}{\sqrt{17}})$ is
\begin{enumerate}
\item $\frac{\sqrt{29}}{3}$
\item $\frac{29}{3}$
\item $\frac{\sqrt{3}}{29}$
\item $\frac{3}{29}$
\end{enumerate}

\item The number of real solutions of 
\begin{align*}
\tan^{-1} \sqrt{x(x + 1)} + \sin^{-1} \sqrt{x^2 + x + 1} = \frac{\pi}{2}
\end{align*}
\begin{enumerate}
\item zero
\item one
\item two
\item infinite
\end{enumerate}

\item If 
\begin{align*}
\sin^{-1}(x - \frac{x^{2}}{2} + \frac{x^{3}}{4}......) + \cos^{-1}(x^{2} - \frac{x^{4}}{2} + \frac{x^{6}}{4}....) = \frac{\pi}{2} 
\end{align*}
for $0 < |x| < \sqrt{2}$, then x equals
\begin{enumerate}
\item 1/2
\item 1
\item -1/2
\item -1
\end{enumerate}

\item The value of x for which 
\begin{align*}
\sin(\cot^{-1}(1 + x) = \cos(\tan^{-1}x))
\end{align*}
is
\begin{enumerate}
\item 1/2
\item 1
\item 0
\item -1/2
\end{enumerate}

\item If $0 < x < 1$, then 
\begin{align*}
\sqrt{1 + x^2}[\{x\cos(\cot^{-1}x) + \sin(\cot^{-1}x)\} - 1]^{1/2} =
\end{align*}
\begin{enumerate}
\item $\frac{x}{1 + x^2}$
\item x
\item $x\sqrt{1 + x^2}$
\item $\sqrt{1 + x^2}$
\end{enumerate}

\item The value of 
\begin{align*}
\cot(\sum_{n=1}^{23} \cot^{-1}(1 + \sum_{k = 1}^{n}2k))=
\end{align*}
\begin{enumerate}
\item $\frac{23}{25}$
\item $\frac{25}{23}$
\item $\frac{23}{24}$
\item $\frac{24}{23}$
\end{enumerate}

\textbf{MCQs with One or More than One Correct}

\item The principal value of $\sin^{-1}(sin\frac{2\pi}{3})$ is
\begin{enumerate}
\item $-\frac{2\pi}{3}$
\item $\frac{2\pi}{3}$
\item $\frac{4\pi}{3}$
\item none of these
\end{enumerate}

\item If $\alpha = 3\sin^{-1}(\frac{6}{11})$ and $\beta = 3\cos^{-1}(\frac{4}{9})$, where the inverse trigonometric functions take only the principal values, then the correct option(s) is(are)
\begin{enumerate}   
\item $\cos\beta > 0$
\item $\sin\beta < 0$
\item $\cos(\alpha + \beta) > 0$
\item $\cos\alpha < 0$
\end{enumerate}

\item For non-negative integers n, let
\begin{align*}
f(n) = \frac{\sum_{k=0}^{n}\sin(\frac{k+1}{n+2}\pi)\sin(\frac{k+2}{n+2}\pi)}{\sum_{k=0}^{n}\sin^2(\frac{k+1}{n+2}\pi)}
\end{align*}
Assuming $\cos^{-1}x$ takes values [0, $\pi$], which of the following options is/are correct?
\begin{enumerate}
\item $[ \lim_{n \to \infty} f(n) = \frac{1}{2}]$
\item $f(4) = \frac{\sqrt{3}}{2}$
\item If $\alpha = \tan(\cos^{-1}{f(6)})$, then $\alpha^2 + 2\alpha - 1 = 0$
\item $\sin(7\cos^{-1}{f(5)}) = 0$
\end {enumerate}

\item Find the value of: 
\begin{align*}
\cos(2\cos^{-1}{x} + \sin^{-1}{x}) at x = \frac{1}{5}
\end{align*}
where $0 \leq \cos^{-1}{x} \leq \pi$ and $-\frac{\pi}{2} \leq \sin^{-1}{x} \leq \frac{\pi}{2}$.

\item Find all the solutions of 
\begin{align*}
4\cos^{2}x \sin x - 2\sin^{2}x = 3\sin x
\end{align*} 

\item Prove that $\cos \tan^{-1}\sin \cot^{-1}x = \sqrt{\frac{x^2 + 1}{x^2 + 2}}$

\textbf{Integer Value Correct Type:}

\item The number of real solutions of the equation
\begin{align*}
\sin^{-1} (\sum_{i = 1}^{\infty}x^{i + 1} - x\sum_{i = 1}^{\infty}(\frac{x}{2})^{i})\\
 = \frac{\pi}{2} - \cos^{-1}(\sum_{i = 1}^{\infty}(\frac{-x}{2})^{i} - \sum_{i = 1}^{\infty}(-x)^{i})
\end{align*}
lying in the interval $(\frac{-1}{2}, \frac{1}{2})$ is............

\item The value of 
\begin{align*}
\sec^{-1}\frac{1}{4}\sum_{k = 0}^{10}\sec(\frac{7\pi}{12} + \frac{k\pi}{2})\sec(\frac{7\pi}{12} + \frac{(k + 1)\pi}{2})
\end{align*}
in the interval $[\frac{-\pi}{4}, \frac{3\pi}{4}]$ equals..........

\textbf{Section-B}

\item $\cot^{-1}(\sqrt{\cos \alpha})-\tan^{-1}(\sqrt{\cos \alpha}) = x$, then $\sin x$ =
\begin{enumerate}
\item $\tan^{2}(\frac{\alpha}{2})$
\item $\cot^{2}(\frac{\alpha}{2})$
\item $\tan \alpha$
\item $\cot(\frac{\alpha}{2})$
\end{enumerate}

\item The trigonometric equation $\sin^{-1}x = 2\sin^{-1}a$ has a solution for
\begin{enumerate}
\item $|a| \geq \frac{1}{\sqrt{2}}$
\item $\frac{1}{2} < |a| < \frac{1}{\sqrt{2}}$
\item all real values of a
\item $|a| < \frac{1}{2}$
\end{enumerate}

\item If $\cos^{-1}x - \cos^{-1}\frac{y}{2} = \alpha$,  then $4x^2 - 4xy \cos \alpha + y^2$ is equal to
\begin{enumerate}
\item $2\sin 2\alpha$
\item 4
\item $4\sin^{2}\alpha$
\item $-4\sin^{2}\alpha$
\end{enumerate}

\item If $\sin^{-1}(\frac{x}{5}) + \cosec^{-1}(\frac{5}{4}) = \frac{\pi}{2}$, then the value of x is
\begin{enumerate}
\item 4
\item 5
\item 1
\item 3
\end{enumerate}

\item The value of $\cot(\cosec^{-1}(\frac{5}{3}) + \tan^{-1}(\frac{2}{3}))$ is
\begin{enumerate}
\item $\frac{6}{17}$
\item $\frac{3}{17}$
\item $\frac{4}{17}$
\item $\frac{5}{17}$
\end{enumerate}

\item If x, y, z are in A.P and $\tan^{-1}y$, $\tan^{-1}z$ are also in A.P., then
\begin{enumerate}
\item x = y = z
\item 2x = 3y = 6z
\item 6x = 3y = 2z
\item 6x = 4y = 3z
\end{enumerate}

\item Let 
\begin{align*}
\tan^{-1}y = \tan^{-1}x + \tan^{-1}(\frac{2x}{1 - x^2})
\end{align*}
where $|x| < \frac{1}{\sqrt{3}}$. Then a value of y is
\begin{enumerate}
\item $\frac{3x - x^3}{1 + 3x^2}$
\item $\frac{3x + x^3}{1 + 3x^2}$
\item $\frac{3x - x^3}{1 - 3x^2}$
\item $\frac{3x + x^3}{1 - 3x^2}$
\end{enumerate}

\item If $\cos^{-1}(\frac{2}{3x}) + \cos^{-1}(\frac{3}{4x}) = \frac{\pi}{2}(x > \frac{3}{4})$, then x is equal to
\begin{enumerate}
\item $\frac{\sqrt{145}}{12}$
\item $\frac{\sqrt{145}}{10}$
\item $\frac{\sqrt{146}}{12}$
\item $\frac{\sqrt{145}}{11}$
\end{enumerate}

\clearpage
\item Match the following:
\begin{table}[ht!]
\centering
\begin{tabular}{c c} 
\textbf{Column I} & \textbf{Column II}\\ [0.5ex] 
     A. $\sum_{n = 1}^{23} \tan^{-1}(\frac{1}{2i^{2}})$ = t, then $\tan t$ = &               (p). 1\\
     B. Sides a, b, c of a triangle ABC are in A.P.\\ $\cos \theta_1 = \frac{a}{b + c}$,
       $\cos \theta_2 = \frac{b}{a + c}$, $\cos \theta_3 = \frac{c}{a + b}$,\\ 
       then $\tan^2(\frac{\theta_1}{2}) + \tan^2(\frac{\theta_3}{2})$ = &                 (q). $\frac{\sqrt{5}}{3}$\\
     C. A line is perpendicular to x + 2y + 2z = 0 and\\ passes through (0, 1, 0) 
        Then the perpendicular\\ distance of this line from the origin is &               (r). $\frac{2}{3}$\\[1ex] 
\end{tabular}
\end{table}\\

\item Match the following:
\begin{table}[ht!]
\centering
\begin{tabular}{c c} 
 \textbf{Column I} & \textbf{Column II}\\ [0.5ex] 
 A. If a = 1 and b = 0, then (x, y)&                           (p). lies on the circle $x^2 +y^2=1$\\
 B. If a = 1 and b = 1, then (x, y)&                           (q). lies on $(x^2 - 1)(y^2 - 1) = 0$\\
 C. If a = 1 and b = 2, then (x, y)&                           (r). lies on $y = x$\\
 D. If a = 2 and b = 2, then (x, y)&                           (s). lies on $(4x^2 - 1)(y^2 - 1) = 0$\\[1ex] 
\end{tabular}
\end{table}\\

\item Match the following:
\begin{table}[ht!]
\centering
\begin{tabular}{c c} 
\textbf{Column I} & \textbf{Column II}\\ [0.5ex] 
   P. $(\frac{1}{y^2}(\frac{\cos(\tan^{-1}{y}) + y\sin(\tan^{-1}{y})}{\cot(\sin^{-1}{y})
      + \tan(\sin^{-1}{y})})^2 + y^4)^{\frac{1}{2}}$\\ takes value is           &(i) $\frac{1}{2}\sqrt{\frac{5}{3}}$\\
   Q. If $\cos x + \cos y + \cos z = 0$ =\\ $\sin x + \sin y + \sin z$ then
      possible\\ value of\\ $\cos \frac{x - y}{2}$ is                               &(ii) $\sqrt{2}$\\
   R. If $\cos(\frac{\pi}{4} - x) \cos2x + \sin x \sin 2\sec x$ =\\  
      $\cos x \sin{2x} \sec x + \cos(\frac{\pi}{4} + x)$\\ 
      then possible value of  $\sec x$ is                                       &(iii) $\frac{1}{2}$\\
   S. If $\cot(\sin^{-1}\sqrt{1 - x^2}) = \sin(\tan^{-1}{x}\sqrt{6})$,\\
      $x \neq 0$ then possible value of $\sec x$ is                             &(iv)  1\\[1ex] 
\textbf{codes:}
\begin{tabular}{ c c c c c}
      P & Q & R & S\\
  (a) 4 & 3 & 1 & 2\\
  (b) 4 & 3 & 2 & 1\\
  (c) 3 & 4 & 2 & 1\\
  (d) 3 & 4 & 1 & 2\\
\end{tabular}
\end{tabular}
\end{table}
\end{enumerate}

 
\section{Functions}
\renewcommand{\theequation}{\theenumi}
\begin{enumerate}[label=\arabic*.,ref=\thesubsection.\theenumi]
\numberwithin{equation}{enumi}

\item The values of $f(x)$ = 3 $\sin\Bigg(\sqrt{\frac{\pi^2}{16}-x^2}\Bigg)$ lie in the interval.......

\item For the function f(x) = $\frac{x}{1+e^{1/x}}$, x $\neq$ 0 and f(x) = 0, x = 0
the derivative from the right, $f^{'}(0+)$= ........, and the derivative from the left, $f^{'}(0-)$= ..........

\item The domain of the funtion f(x)=$\sin^{-1}\Big(log_2\frac{x^2}{2}\Big)$ is given by .......

\item Let A be a set of n distinct elements. Then the total number of distinct functions from A to A is.......and out of these.......are onto functions.

\item If 
\begin{align*}
f(x) = \sin ln\Big(\frac{\sqrt{4 - x^{2}}}{1-x}\Big),
\end{align*}
then domain of f(x) is.... and its range is......

\item There are exactly two distinct linear functions.......,and......which map [-1,1] onto [0,2].

\item If f is an even function defined on the interval (-5,5), then four real values of x satisfying the equation f(x)=$f(\frac{x+1}{x+2})$ are.........., ..........., ........., and......

\item If
\begin{align*}
 f(x)= \sin^{2}x + \sin^{2} \Big(x+\frac{\pi}{3}\Big) + cosxcos\Big(x+\frac{\pi}{3}\Big)
\end{align*}
and g$\big(\frac{5}{4}\big)=1$, then (gof)(x)= ........

\item If f(x)=$(a-x^{n})^{1/n}$ where $a > 0$ and n is a positive integer, then f[f(x)]=x.

\item The function f(x)=$\frac{x^{2}+4x+30}{x^{2}-8x+18}$ is not one-to-one.
                                          
\item If $f_{1}(x)$ and $f_{2}(x)$ are defined on domains $D_{1}$ and $D_{2}$ respectively, then $f_{1}(x)+ f_{2}(x)$  is defined on $D_{1} \cup D_{2}$.

\item Let R be the set of real numbers. If f:R $\to$ R is a function defined by f(x)=$x^{2}$, then f is :
\begin{enumerate}
\item Injective but not surjective
\item Surjective but not injective
\item Bijective
\item None of these.
\end{enumerate}

\item The entire graphs of the equation y = $x^{2}+kx-x+9$ is stirctly above the x-axis if and only if
\begin{enumerate}
\item $k < 7$
\item $-5 < k < 7$
\item $k > -5$
\item None of these.
\end{enumerate}

\item Let f(x)= $\begin{vmatrix} x-1 \end{vmatrix}$. Then
\begin{enumerate}
\item $f(x^{2})=(f(x))^{2}$
\item f(x+y)=f(x)+f(y)
\item $f(\begin{vmatrix} x \end{vmatrix}) = \begin{vmatrix} f(x) \end{vmatrix}$
\item None of these
\end{enumerate}

\item If x satisfies $\begin{vmatrix} x-1 \end{vmatrix} + \begin{vmatrix} x-2 \end {vmatrix} + \begin{vmatrix} x-3 \end{vmatrix} \geq 6$,then
\begin{enumerate}
\item $0 \leq x \leq 4$
\item $x \leq -2$ or $x \geq 4$
\item $x \leq 0$ or $x \geq 4$
\item None of these
\end{enumerate}

\item If f(x)=$\cos(ln x)$, then f(x)f(y) - $\frac{1}{2}\Big[f\big(\frac{x}{y}\big)+f(xy)\Big]$  has the value
\begin{enumerate}
\item -1
\item $\frac{1}{2}$
\item -2
\item none of these
\end{enumerate}

\item The domain of definition of the function y=$\frac{1}{log_{10}(1-x)} + \sqrt{x+2}$ is
\begin{enumerate}
\item (-3,2) excluding -2.5
\item $[0,1]$ excluding 0.5
\item $[-2,1)$ excluding 0
\item none of these
\end{enumerate}

\item Which of the following functions is periodic?
\begin{enumerate}
\item f(x)=x-[x] where [x] denotes the largest integer less than or equal to the real number x
\item f(x)=$\sin\frac{1}{x}$ for $x \neq 0$, f(0)=0
\item f(x)=x$\cos x$
\item none of these
\end{enumerate}

\item Let f(x)=$\sin x$ and g(x) = ln $\begin{vmatrix} x \end{vmatrix}$. If the ranges of the composition functions $fog$ and $gof$ are $R_{1}$ and $R_{2}$ respectively, then
\begin{enumerate}
\item $R_1 = \{u: -1 \leq u < 1\}, R_2 = \{v: -\infty < v < 0\}$
\item $R_1 = \{u: -\infty < u < 0\}, R_2 = \{v: -1 \leq v \leq 0\}$
\item $R_1 = \{u: -1 < u < 1\}, R_2 = \{v: -\infty < v < 0\}$
\item $R_1 = \{u: -1 \leq u \leq 1\}, R_2 = \{v: -\infty < v \leq 0\}$
\end{enumerate}

\item Let $f(x)=(x+1)^{2}-1$, x $\geq -1$. Then the set $\{x: f(x)= f^{-1}(x)\}$ is
\begin{enumerate}
\item $\{0,-1, \frac{-3+i\sqrt{3}}{2}, \frac{-3-i\sqrt{3}}{2}\}$
\item $\{0,1,-1\}$
\item $\{0,-1\}$
\item empty
\end{enumerate}

\item The function f(x) = $\begin{vmatrix} px-q \end{vmatrix}$ + r$\begin{vmatrix} x \end{vmatrix}$,
 $x \in (-\infty,\infty)$ where p $>$ 0, q $>$ 0, r $>$ 0 assumes its minimum value only on one point if
\begin{enumerate}
\item p $\neq$ q
\item r $\neq$ q
\item r $\neq$ p
\item p = q = r
\end{enumerate}

\item Let $f(x)$ be defined for all $x > 0$ and be continuos. Let $f(x)$ satisfy $f(\frac{x}{y})$ = $f(x)-f(y)$ for all x,y and $f(e)=1$. Then
\begin{enumerate}
\item $f(x)$ is bounded
\item $f(\frac{1}{x}) \to 0$ as $x \to 0$
\item $xf(x)\to 1$ as $x \to 0$
\item $f(x)=ln x$
\end{enumerate}

\item If the function $f:[1,\infty) \rightarrow [1,\infty)$ is defined by $f(x)=2^{x(x-1)}$, then $f^{-1}(x)$ is
\begin{enumerate}
\item $\frac{1}{2}^{x(x-1)}$
\item $\frac{1}{2}(1+\sqrt{1+4log_2x})$
\item $\frac{1}{2}(1-\sqrt{1+4log_2x})$
\item not defined
\end{enumerate}

\item Let $f: R\rightarrow$ R be any function. Define g: R $\rightarrow$ R by $g(x)=\begin{vmatrix} f(x) \end{vmatrix}$ for all x. Then g is
\begin{enumerate}
\item onto if f is onto
\item one-one if f is one-one
\item continuos if f is continuos
\item differentiable if f is differentiable
\end{enumerate} 

\item The domain of definition of the function $f(x)$ given by the equation $2^{x} + 2^{y} = 2$ is 
\begin{enumerate}
\item $0< x \leq 1$
\item $0 \leq x \leq 1$
\item $-\infty < x \leq 0$
\item $-\infty < x < 1$
\end{enumerate}

\item Let $g(x) = 1+x-[x]$ and \[f(x)=\begin{cases} 
      -1, & x < 0\\
       0, & x = 0.\\
       1, & x > 0 
   \end{cases}\] then for all x, $f(g(x))$ is equal to
\begin{enumerate}
\item x
\item 1
\item $f(x)$
\item $g(x)$
\end{enumerate} 

\item If $f:[1,\infty) \rightarrow [2,\infty)$ is given by $f(x) = x+\frac{1}{x}$ then $f^{-1}(x)$ equals
\begin{enumerate}
\item $(x+\sqrt{x^2-4})/2$
\item $x/(1+x^2)$
\item $(x-\sqrt{x^2-4})/2$
\item $1+\sqrt{x^2-4}$
\end{enumerate}

\item The domain of definition of $f(x)=\frac{log_2(x+3)}{x^2+3x+2}$ is
\begin{enumerate}
\item $R\backslash\{-1,-2 \}$
\item $(-2,\infty)$
\item R$\backslash\{-1,-2,-3 \}$
\item $(-3,\infty)\backslash\{-1,-2 \}$
\end{enumerate} 

\item Let E=$\{1,2,3,4 \}$ and F=$\{1,2 \}$. Then the number of onto functions from E to F is
\begin{enumerate}
\item 14
\item 16
\item 12
\item 8
\end{enumerate}

\item Let $f(x)=\frac{\alpha x}{x+1}$,x $\neq$ -1. Then, for what value of $\alpha$ is $f(f(x))=x$?
\begin{enumerate}
\item $\sqrt{2}$
\item $-\sqrt{2}$
\item 1
\item -1
\end{enumerate} 

\item Suppose $f(x)=(x+1)^2$ for x $\geq$ -1. If $g(x)$ is the function whose graph is the reflection of the  graph of $f(x)$ with respect to the line y=x then $g(x)$ equals
\begin{enumerate}
\item $-\sqrt{x}-1, x \geq 0$
\item $\frac{1}{(x+1)^{2}}, x > -1$
\item $\sqrt{x+1}, x\geq-1$
\item $\sqrt{x}-1, x\geq0$
\end{enumerate}

\item Let function $f: R \rightarrow$ R be defined by $f(x) = 2x+\sin x$ for x $\in$ R, then $f$ is
\begin{enumerate}
\item one-to-one and onto
\item one-to-one but NOT onto
\item onto but NOT one-to-one
\item neither one-to-one nor onto
\end{enumerate}

\item If $f: [0,\infty) \rightarrow [0,\infty)$, and $f(x)=\frac{x}{1+x}$ then $f$ is
\begin{enumerate}
\item one-one and onto
\item one-one but not onto
\item onto but not one-one
\item neither one-one nor onto
\end{enumerate}

\item Domain of the definition of the function $f(x)=\sqrt{\sin^{-1}(2x)+\frac{\pi}{6}}$ for real valued $x$ ,is
\begin{enumerate}
\item $[-\frac{1}{4},\frac{1}{2}]$
\item $[-\frac{1}{2},\frac{1}{2}]$
\item $[-\frac{1}{2},\frac{1}{9}]$
\item $[-\frac{1}{4},\frac{1}{4}]$
\end{enumerate}

\item Range of the function $f(x)=\frac{x^2+x+2}{x^2+x+1}$;$x\epsilon$R is
\begin{enumerate}
\item $(1,\infty)$
\item $(1,\frac{11}{7}]$
\item $(1,\frac{7}{3}]$
\item $(1,\frac{7}{5}]$
\end{enumerate}

\item If $f(x)=x^{2}+2bx+2c^{2}$ and $g(x)=-x^{2}-2cx+b^{2}$ such that min $f(x) > max g(x)$, then the relation between b and c, is
\begin{enumerate}
\item no real value of b \& c
\item $0<c<b\sqrt{2}$
\item $\begin{vmatrix} c \end{vmatrix} < \begin{vmatrix} b\end{vmatrix} \sqrt{2}$
\item $\begin{vmatrix} c \end{vmatrix} > \begin{vmatrix} b\end{vmatrix} \sqrt{2}$
\end{enumerate}

\item If $f(x)=\sin x+\cos x$, $g(x)=x^{2}-1$, then $g(f(x))$ is invertible in the domain
\begin{enumerate}
\item $[0,\frac{\pi}{2}]$
\item $[-\frac{\pi}{4},\frac{\pi}{4}]$
\item $[-\frac{\pi}{2},\frac{\pi}{2}]$
\item $[0,\pi]$
\end{enumerate}

\item If the functions $f(x)$ and $g(x)$ are defined on R$\rightarrow$R such that \[f(x)=\begin{cases} 
       0, & x \in $rational$\\
       x, & x \in $irrational$\\
   \end{cases}\]; \[g(x)=\begin{cases} 
       0, & x \in $irrational$\\
       x, & x \in $rational$\\
   \end{cases}\] then $(f-g)(x)$ is
\begin{enumerate}
\item one-one \& onto
\item neither one-one nor onto
\item one-one but not onto
\item onto but not one-one
\end{enumerate}

\item X and Y are two sets and $f: X \rightarrow Y$. If $\{ f(c)=y; c \subset X, y \subset Y \}$ and  
$\{ f^{-1}(d) = x; d \subset Y, x \subset X \}$, then the true statement is
\begin{enumerate}
\item $f(f^{-1}(b))$ = b
\item $f^{-1}(f(a))$ = a
\item $f(f^{-1}(b))$ = b,b $\subset$ y
\item $f(f^{-1}(a))$ = a,a $\subset$ x
\end{enumerate} 

\item If $F(x) = \Big(f\Big(\frac{x}{2}\Big)\Big)^2+\Big(g\Big(\frac{x}{2}\Big)\Big)^2$ where $f^{"}(x)= -f(x)$ and $g(x)=f^{'}(x)$ and given that F(5)=5, then F(10) is equal to
\begin{enumerate}
\item 5
\item 10
\item 0
\item 15
\end{enumerate}

\item Let $f(x)=\frac{x}{(1+x^n)^{1/n}}$ for n $\geq$ 2 and $g(x)=
\underbrace{(fofo......of)}_{\text{f occurs n times}}(x)$. Then $\int x^{n-2}g(x)dx$ equals.
\begin{enumerate}
\item $\frac{1}{n(n-1)}(1+nx^n)^{1-\frac{1}{n}}+K$
\item $\frac{1}{(n-1)}(1+nx^n)^{1-\frac{1}{n}}+K$
\item $\frac{1}{n(n+1)}(1+nx^n)^{1+\frac{1}{n}}+K$
\item $\frac{1}{(n+1)}(1+nx^n)^{1+\frac{1}{n}}+K$
\end{enumerate}

\item Let f, g and h be real-valued functions defined on the interval $[0,1]$ by $f(x)=e^{x{^2}}+e^{{-x}^2}$, $g(x)=xe^{x{^2}}+e^{{-x}^2}$ and $h(x)=x^2e^{x{^2}}+e^{{-x}^2}$. If a, b and c denote, respectively, the absolute maximum of f,g and h on $[0,1]$, then
\begin{enumerate}
\item a = b and c $\neq$ b
\item a = c and a $\neq$ b
\item a $\neq$ b and c $\neq$ b
\item a = b = c
\end{enumerate}

\item Let $f(x)=x^2$ and $g(x)=\sin x$ for all $x \in$ R. Then the set of all x satisfying $(fogogof)(x)=(gogof)(x)$, where $(fog)(x)=f(g(x))$, is
\begin{enumerate}
\item $\pm \sqrt{n\pi}$, n $\in \{0,1,2.....\}$
\item $\pm \sqrt{n\pi}$, n $\in \{1,2.....\}$
\item $\frac{\pi}{2}+2n\pi$, n $\in \{.....-2,-1,0,1,2.....\}$
\item $2n \pi$, n $\in \{.....-2,-1,0,1,2.....\}$
\end{enumerate}

\item The function $f:[0,3] \rightarrow [1,29]$, defined by $f(x)=2x^{3}-15x^{2}+36x+1$, is
\begin{enumerate}
\item one-one and onto
\item onto but not one-one
\item one-one but not onto
\item neither one-one nor onto
\end{enumerate}

\item If y=$f(x)=\frac{x+2}{x-1}$ then
\begin{enumerate}
\item $x=f(y)$
\item $f(1)$=3
\item y increases with $x$ for $x<1$
\item $f$ is a rational function of $x$
\end{enumerate}

\item Let $g(x)$ be a function defined on $[-1,1]$. If the area of the equilateral triangle with two of its vertices at (0,0) and $[x,g(x)]$ is $\frac{\sqrt{3}}{4}$, then the function $g(x)$ is
\begin{enumerate}
\item $g(x)=\pm\sqrt{1-x^2}$
\item $g(x)=\sqrt{1-x^2}$
\item $g(x)=-\sqrt{1-x^2}$
\item $g(x)=\sqrt{1+x^2}$
\end{enumerate}

\item If f(x)=$\cos [\pi^2]x + \cos [-\pi^2]x$, where [x] stands for the greatest integer function, then
\begin{enumerate}
\item $f(\frac{\pi}{2})=-1$
\item $f(\pi)=1$
\item $f(-\pi)=0$
\item $f(\frac{\pi}{4})=1$
\end{enumerate}

\item If $f(x)$=3x-5, then $f^{-1}(x)$
\begin{enumerate}
\item is given by $\frac{1}{3x-5}$
\item is given by $\frac{x+5}{3}$
\item does not exist because $f$ is not one-one
\item does not exist because $f$ is not onto.
\end{enumerate}

\item If $g(f(x))=|\sin x|$ and $f(g(x))=(\sin \sqrt{x})^2$, then
\begin{enumerate}
\item $f(x)= \sin^2x$,$g(x)=\sqrt{x}$
\item $f(x)=\sin x$,$g(x)= \begin{vmatrix} x \end{vmatrix}$
\item $f(x)=x^2$, $g(x)=\sin \sqrt{x}$
\item $f$ and g cannot be determined.
\end{enumerate}

\item Let $f$:(0,1) $\rightarrow$ R be defined by $f(x)=\frac{b-x}{1-bx}$, where b is a constant such that $0<b<1$. Then 
\begin{enumerate}
\item $f$ is not invertible on (0,1)
\item $f\neq f^{-1}$ on (0,1) and $f^{'}(b)=\frac{1}{f^{'}(0)}$
\item $f=f^{-1}$ on (0,1) and $f^{'}(b)=\frac{1}{f^{'}(0)}$
\item $f^{-1}$ is differentiable (0,1)
\end{enumerate}

\item Let $f$:(-1,1) $\rightarrow$ IR be such that $f(\cos 4\Theta)=\frac{2}{2-\sec^2\Theta}$ for $\Theta\in\Big(0,\frac{\pi}{4}\Big) \cup \Big(\frac{\pi}{4},\frac{\pi}{2}\Big)$. Then the values of $f(\frac{1}{3})$ is 
\begin{enumerate}
\item 1-$\sqrt{\frac{3}{2}}$
\item 1+$\sqrt{\frac{3}{2}}$
\item 1-$\sqrt{\frac{2}{3}}$
\item 1+$\sqrt{\frac{2}{3}}$
\end{enumerate}

\item The funciton 
$f(x)$ = 2$\begin{vmatrix} x \end{vmatrix}$
 + $\begin{vmatrix} x+2 \end{vmatrix}$
 - $\begin{vmatrix}\begin{vmatrix} x+2 \end{vmatrix} 
 - 2\begin{vmatrix} x \end{vmatrix}\end{vmatrix}$ has a local minimum or a local maximum at x=
\begin{enumerate}
\item -2
\item $\frac{-2}{3}$
\item 2
\item $\frac{2}{3}$
\end{enumerate}

\item Let $f: (-\frac{\pi}{2},\frac{\pi}{2})\rightarrow R$ be given by $f(x)=(log(\sec x+\tan x))^3$. Then
\begin{enumerate}
\item $f(x)$ is an odd function 
\item $f(x)$ is one-one function
\item $f(x)$ is an onto function
\item $f(x)$ is an even function
\end{enumerate}

\item Let a$\in$R and let $f$: R $\rightarrow$ R be given by $f(x)=x^5-5x+a$. Then
\begin{enumerate}
\item $f(x)$ has three real roots if a$>$4
\item $f(x)$ has only real root if a$>$4
\item $f(x)$ has three real roots if a$<$-4
\item $f(x)$ has three real roots if -4$<$a$<$4
\end{enumerate}

\item Let $f(x)=\sin\Big(\frac{\pi}{6}\sin\Big(\frac{\pi}{2}\sin x\Big)\Big)$ for all $x\in$R and $g(x)=\frac{\pi}{2}\sin x$ for all x$\in$R. Let $(fog)(x)$ denote $f(g(x))$ and $(gof)(x)$ denote $g(f(x)).$ Then which of the following is true?
\begin{enumerate}
\item Range of $f$ is $[-\frac{1}{2},\frac{1}{2}]$
\item Range of fog is $[-\frac{1}{2},\frac{1}{2}]$
\item $\lim_{x \to 0}\frac{f(x)}{g(x)}=\frac{\pi}{6}$
\item There is an $x\in$R such that $(gof)(x)=1$
\end{enumerate}

\item Find the domain and range of the function $f(x)=\frac{x^2}{1+x^2}.$ Is the function one-to-one?

\item Draw the graph of $y = \begin{vmatrix} x \end{vmatrix}^{1/2}$ for $-1 \leq x \leq 1$.

\item If $f(x)=x^{9}-6x^{8}-2x^{7}+12x^{6}+x^{4}-7x^{3}+6x^{2}+x-3$, find $f(6)$.

\item Consider the following relations in the set of real numbers R. $R = \{(x,y);x \in R, y\in R, x^2+y^2 \leq 25\}$\\
$R^{'} = \{(x,y):x \in R, y \in R, y \geq \frac{4}{9}x^2\}$. Find the domain and the range of $R \cap R^{'}$. Is the relation $R \cap R^{'}$ a function?

\item Let A and B be two sets each with a finite number of elements. Assume that there is an injective mapping from A to B and that there is an injective mapping from B to A. Prove that there is a bijective mapping from A to B.

\item Let $f$ be a one-one function with domain $\{x,y,z\}$ and range $\{1,2,3\}$. It is given that exactly one of the following statements is true and the remaining two are false $f(x)=1$, $f(y) \neq 1$,$f(z) \neq 2$ determine $f^{-1}(1)$. 
 
\item Let R be the set of real numbers and $f$: R $\rightarrow$ R be such that for all x and y in R$\begin{vmatrix} f(x)-f(y) \end{vmatrix} \leq \begin{vmatrix} x-y \end{vmatrix}^3$. Prove that $f(x)$ is a constant.

\item Find the natural number $'a'$ for which $\sum_{k=1}^{n} f(a+k) = 16(2^n-1)$, where the function $'f'$ satisfies the relation $f(x+y)=f(x)f(y)$ for all natural numbers x,y and further $f(1)=2$.

\item Let $\{x\}$ and $[x]$ denotes the fractional and integral part of a real number $x$ respectively. Solve 4$\{x\}$ = $x+[x]$.

\item A function $f: IR \rightarrow IR$, where IR is the set of real numbers, defined by $f(x)=\frac{\alpha x^{2}+6x-8}{\alpha+6x-8x^{2}}$. Find the interval of values of $\alpha$ for which f is onto. Is the function one-to-one for $\alpha = 3$? Justify your answer.

\item Let $f(x) = Ax^{2}+Bx+c$ where A,B,C are real numbers. Prove that if $f(x)$ is an integer whenever x is an integer, then the numbers 2A,A+B and C are all integers. Conversely, prove that if the numbers 2A,A+B and C are all integers then $f(x)$ is an integer whenever $x$ is an integer.

\item Let $f: [0,4\pi] \rightarrow [0,\pi]$ be defined by $f(x)=\cos^{-1}(\cos x)$. The number of points $x \in [0,4\pi]$ satisfying the equation $f(x)=\frac{10-x}{10}$ is

\item The value of $((log_29)^{2})^{\frac{1}{log_2(log_29)}} \times (\sqrt{7})^{\frac{1}{log_47}}$ is ......

\item Let X be a set with exactly 5 elements and Y be a set with exactly 7 elements. If $\alpha$ is the number of one-one functions from X to Y and $\beta$ is the number of onto functions from Y to X, then the value of $\frac{1}{5!}(\beta-\alpha)$ is ......

\item The domain of $\sin^{-1}[log_3(x/3)]$ is
\begin{enumerate}
\item $[1,9]$
\item $[-1,9]$
\item $[-9,1]$
\item $[-9,-1]$
\end{enumerate}

\item The function $f(x)=log(x+\sqrt{x^{2}+1})$, is
\begin{enumerate}
\item neither an even nor an odd function
\item an even function
\item an odd function
\item a periodic function.
\end{enumerate}

\item Domain of definition of the function $f(x)=\frac{3}{4-x^{2}}+log_{10}(x^{3}-x)$, is
\begin{enumerate}
\item $(-1,0) \cup (1,2) \cup (2,\infty)$
\item (a,2)
\item $(-1,0) \cup (a,2)$
\item $(1,2) \cup (2,\infty)$.
\end{enumerate}

\item If $f: R \rightarrow R$ satisfies $f(x+y)=f(x)+f(y)$, for all $x$, y $\in$ R and f(1)=7, then 
$\sum_{r=1}^{n} f(r)$ is
\begin{enumerate}
\item $\frac{7n(n+1)}{2}$
\item $\frac{7n}{2}$
\item $\frac{7(n+1)}{2}$
\item $7n+(n+1)$
\end{enumerate}

\item A function $f$ from the set of natural numbers to integers defined by \[f(n)=\begin{cases} 
       \frac{n-1}{2}, & $when n is odd$\\
       -\frac{n}{2}, & $when n is even$\\
   \end{cases}\] is
\begin{enumerate}
\item neither one-one nor onto
\item one-one but not onto
\item onto but not one-one
\item one-one and onto both.
\end{enumerate}
    
\item The range of the function $f(x)= ^{7-x}P_{x-3}$ is
\begin{enumerate}
\item $\{1,2,3,4,5\}$
\item $\{1,2,3,4,5,6\}$
\item $\{1,2,3,4\}$
\item $\{1,2,3,\}$
\end{enumerate}

\item Let $f: R \rightarrow S$, defined by $f(x)=\sin x-\sqrt{3}\cos x+1$, is onto, then the interval of S is
\begin{enumerate}
\item $[-1,3]$
\item $[-1,1]$
\item $[0,1]$
\item $[0,3]$
\end{enumerate}

\item The graph of the function $y=f(x)$ is symmetrical about the line x=2, then
\begin{enumerate}
\item $f(x)=-f(-x)$
\item $f(2+x)=f(2-x)$
\item $f(x)=f(-x)$
\item $f(x+2)=f(x-2)$
\end{enumerate}

\item The domain of the function $f(x)=\frac{\sin^{-1}(x-3)}{\sqrt{9-x^{2}}}$ is
\begin{enumerate}
\item $[1,2]$
\item $[2,3)$
\item $[1,2]$
\item $[2,3]$
\end{enumerate}

\item Let $f: (-1,1) \rightarrow B$, be a function defined by $f(x)=tan^{-1}\frac{2x}{1-x^2}$, then $f$ is both one-one and onto when B is the interval
\begin{enumerate}
\item $(0,\frac{\pi}{2})$
\item $[0,\frac{\pi}{2})$
\item $[-\frac{\pi}{2},\frac{\pi}{2})$
\item $(-\frac{\pi}{2},\frac{\pi}{2})$
\end{enumerate}

\item A function is matched below against an interval where it is supposed to be increasing. Which of the following pairs is incorrectly matched?
\begin{table}[h!]
\centering
\begin{tabular}{c c} 
 Interval & Function\\ [0.5ex] 
 (a). $(-\infty,\infty)$ & $x^3-3x^2+3x+3$\\ 
 (b). $[2,\infty)$ & $2x^3-3x^2-12x+6$\\
 (c). $(-\infty,\frac{1}{3}]$ & $3x^2-2x+1$\\
 (d). $(-\infty,-4)$ & $x^3+6x^2+6$\\[1ex] 
\end{tabular}
\end{table}

\item A real valued function $f(x)$ satisfies the functional equation 
\begin{align*}
f(x-y)=f(x)f(y)-f(a-x)f(a+y)
\end{align*} 
where a is a given constant and $f(0)=1, f(2a-x)$ is equal to 
\begin{enumerate}
\item $-f(x)$
\item $f(x)$
\item $f(a)+f(a-x)$
\item $f(-x)$
\end{enumerate}

\item The Largest interval lying in $(\frac{-\pi}{2},\frac{\pi}{2})$ for which the function, 
\begin{align*}
f(x)=4^{-x^2}+cos^{-1}(\frac{x}{2}-1)+log(\cos x))
\end{align*},
is defined, is
\begin{enumerate}
\item $[-\frac{\pi}{4},\frac{\pi}{2})$
\item $[0,\frac{\pi}{2})$
\item $[0,\pi]$
\item $(-\frac{\pi}{2},\frac{\pi}{2})$
\end{enumerate} 

\item Let $f: N \rightarrow Y$ be a function defined as $f(x)=4x+3$ where
\begin{align*}
Y=\{ y \in N:y=4x+3 for some x \in N\}
\end{align*}
\begin{enumerate}
\item $g(y)=\frac{3y+4}{3}$
\item $g(y)=4+\frac{y+3}{4}$
\item $g(y)=\frac{y+3}{4}$
\item $g(y)=\frac{y-3}{4}$
\end{enumerate}

\item Let $f(x)=(x+1)^2-1$,x $\geq$ -1\\
Statement-1 : The set $\lbrace x:f(x)=f^{-1}(x)=\lbrace0,-1\rbrace\rbrace$\\
Statement-2 : $f$ is a bijection
\begin{enumerate}
\item Statement-1 is true,Statement-2 is true. Statement-2 is not a correct explanation for Statement-1.
\item Statement-1 is true, Statement-2 is false.
\item Statement-1 is false,Statement-2 is true.
\item Statement-1 is true, Statement-2 is true. Statement-2 is a correct explanation for Statement-1.
\end{enumerate}

\item For real x, let $f(x)=x^{3}+5x+1$, then 
\begin{enumerate}
\item $f$ is onto R but not one-one
\item $f$ is one-one and onto R
\item $f$ is neither one-one nor onto R
\item $f$ is one-one but not onto R
\end{enumerate}

\item The domain of the function $f(x)=\frac{1}{\sqrt{\begin{vmatrix} x \end{vmatrix} - x}}$ is
\begin{enumerate}
\item $(0,\infty)$
\item $(-\infty,0)$
\item $(-\infty,\infty)-\{0\}$
\item $(-\infty,\infty)$
\end{enumerate}

\item For $x \in R-\{0,1\}$, let $f_{1}(x)=\frac{1}{x}, f_{2}(x)=1-x$ and $f_{3}(x)=\frac{1}{1-x}$ be three given functions. If a function, J(x) satisfies $(f_{2}oJof_{1})(x)=f_{3}(x)$ then J(x) is equal to:
\begin{enumerate}
\item $f_3(x)$
\item $f_3(x)$
\item $f_2(x)$
\item $f_1(x)$
\end{enumerate} 

\item If the fractional part of the number $\frac{2^{403}}{15}$ is $\frac{k}{15}$, then k is equal to:
\begin{enumerate}
\item 6
\item 8
\item 4
\item 14
\end{enumerate}

\item If the function $f:R-\{1,-1\}$. A defined by f(x)=$\frac{x^2}{1-x^2}$, is surjective, then A is equal to:
\begin{enumerate}
\item R-$\{-1\}$
\item $[0,\infty)$
\item R-$[-1,0)$
\item R-$(-1,0)$
\end{enumerate}

\item Let
\begin{align*}
\sum_{k=1}^{10} f(a+k)=16(2^{10}-1),
\end{align*} 
where the function f satisifies f(x+y)=f(x)f(y) for all natural numbers x,y and f(a) is = 2. Then the natural number $'a'$ is:
\begin{enumerate}
\item 2
\item 16
\item 4
\item 3
\end{enumerate}

\textbf{Match the following}

\item Let the function defined in Colum 1 have domain $(-\frac{\pi}{2},\frac{\pi}{2})$ and range $(-\infty,\infty)$
\begin{table}[h!]
\centering
\begin{tabular}{c c} 
 Column I & Column II\\ [0.5ex] 
 (A) 1+2x & (p) onto but not one-one\\ 
 (B) $\tan x$ & (q) one-one but not onto\\
     & (r) one-one and onto\\
     & (s) neither one-one nor onto\\[1ex] 
\end{tabular}
\end{table}

\item Let $f(x)=\frac{x^2-6x+5}{x^2-5x+6}$\\
Match of expressions/statements in Column I with expressions/statements in Column II and indicate your answer by darkening the appropriate bubbles in the 4$\times$4 matrix given in the ORS.
\begin{table}[h!]
\centering
\begin{tabular}{c c} 
 Column I & Column II\\ [0.5ex] 
 (A) If $-1 < x < 1$, then $f(x)$ satisfies & (p) $0 < f(x) < 1$\\ 
 (B) If $1 < x < 2$, then $f(x)$ satisfies & (q) $f(x) < 0$\\
 (C) If $3 < x < 5$, then $f(x)$ satisfies   & (r) $f(x) > 0$\\
 (D) If $x > 5$, then $f(x)$ satisfies  & (s) $f(x) < 1$\\[1ex] 
\end{tabular}
\end{table}
 

\item Let $E_1 = \{ x \in R: x \neq 1 and \frac{x}{x-1} > 0\}$ and $E_2 = \{ x \in E_1:\sin^{-1}\Big(log_e(\frac{x}{x-1})\Big) is a real number\}$. (Here, the inverse trigonometric function $\sin^{-1}x$ assumes values in $[-\frac{\pi}{2},\frac{\pi}{2}]$).\\
Let $f: E_1 \rightarrow R$ be the function defined by $f(x)=log_e(\frac{x}{x-1})$ and g: $E_2 \rightarrow$ R be the function defined by $g(x)=sin^{-1}(log_e(\frac{x}{x-1}))$.
\begin{table}[h!]
\centering
\begin{tabular}{c c} 
 LIST-I & LIST-II\\ [0.5ex] 
  P. The range of $f$ is &         1. $(-\infty,\frac{1}{1-e}]\cup[\frac{e}{e-1},\infty)$\\ 
  Q. The range of g contains &     2. (0,1)\\
  R. The domain of $f$ contains &  3. $[-\frac{1}{2},\frac{1}{2}]$\\
  S. The domain of g is &          4. $(-\infty,0)\cup(0,\infty)$\\
                                 & 5. $(-\infty,\frac{e}{e-1}]$\\
                                 & 6. $(-\infty,0)\cup(\frac{1}{2},\frac{e}{e-1}]$\\[1ex] 
\end{tabular}
\end{table}\\
The correct option is:
\begin{enumerate}
\item $P \rightarrow 4; Q \rightarrow 2; R \rightarrow 1; S \rightarrow 1$
\item $P \rightarrow 4; Q \rightarrow 2; R \rightarrow 1; S \rightarrow 6$ 
\item $P \rightarrow 3; Q \rightarrow 3; R \rightarrow 6; S \rightarrow 5$
\item $P \rightarrow 4; Q \rightarrow 3; R \rightarrow 6; S \rightarrow 5$
\end{enumerate} 
\end{enumerate} 

 


\section{Quadratic Equations and Inequations}
\renewcommand{\theequation}{\theenumi}
\begin{enumerate}[label=\arabic*.,ref=\thesubsection.\theenumi]
\numberwithin{equation}{enumi}
\item The coefficient of $x^{99}$ in the polynomial
$(x-1)(x-2)$.............$(x-100)$ is........

\item If $2+i\sqrt{3}$ is a root of the equation
 $x^{2} +px+q=0$, where p and q are real, then
$(p,q)=(.................... , ....................)$

\item If the product of the roots of the equation
$x^{2}-3kx+2e^{21nk}-1=0$ is $7$, then the roots are real for
$k=.....................$

\item If the quadratic equations $x^{2} +ax+b=0$ and $x^{2}+bx+a=0$ $(a\neq b)$ have a common root then, the numerical value of $a+b$ is....................

\item The solution of equation
$log_7 log_5(\sqrt{x+5}+\sqrt{x})=0$ is .......

\item If $x<0,y<0,x+y+\frac{x} {y}=\frac{1}{2}$ and $(x+y)\frac{x}{y}=-\frac{1}{2}$, then $x=.............$and $y=.................$

\item Let n and k be positive such that $n\geq{\frac{k(k+1)}{2}}$.\\ The number of solutions $(x_1, x_2,.....,x_k), \\x_1\geq{1}, x_2\geq{2},...,x_k\geq{k}$, all integers, satisfying $x_1+x_2+...+x_k=n$, is...........

\item The sum of all the real roots of the equation\\$\begin{vmatrix} x-2 \end{vmatrix}^{2} + \begin{vmatrix} x-1 \end{vmatrix}-2 = 0$ is................

\item For every integer $n>1$, the inequality\\$(n!)^{\frac{1}{n}}<\frac{n+1}{2}$ holds.

\item The equation $2x^{2}+3x+1=0$ has an irrational root.

\item If $a<b<c<d$, then the roots of the equation\\
$(x-a)(x-c)+2(x-b)(x-d)=0$ are real and distinct.

\item If $n_1, n_2,......n_p$ are p positive integers, whose sum is an even number, then the number of odd integers among them is odd.

\item If $P(x)=ax^{2}+bx+c$ and $Q(x)=-ax^{2}+dx+c$, where $ac\neq0$, then $P(x)Q(x)=0$ has at least two real roots.

\item If x and y are positive real numbers and m, n are any positive integers, then $\frac{x^{n}y^{m}}{(1+x^{2n})(1+y^{2m})}>\frac{1}{4}$

\item If l,m,n are real, l $\neq$ m, then the roots by the equation $(l-m)x^{2}-5(l+m)x-2(l-m)=0$ are:
\begin{enumerate}
\item Real and equal
\item Complex
\item Real and unequal
\item None of these
\end{enumerate}
 
\item The equation $x+2y+2z=1$ and $2x+4y+4z=9$ have
\begin{enumerate}
\item Only one solution 
\item Only two solutions
\item Infinite number of solutions
\item None of these
\end{enumerate}
 
\item If x,y and z are real and different and $u=x^{2}+4y^{2}+9z^{2}-6yz-3zx-2xy$ then u is always.
\begin{enumerate}
\item non negative
\item zero
\item non positive
\item none of these
\end{enumerate}

\item Let $a>0,b>0$ and $c>0$. Then the roots of the equation $ax^{2}+bx+c=0$
\begin{enumerate}
\item are real and negative
\item have negative real parts
\item both (a) and(b)
\item none of these
\end{enumerate}

\item Both the roots of the equation $(x-b)(x-c)+(x-a)(x-c)+(x-a)(x-b)=0$ are always
\begin{enumerate}
\item positive
\item real 
\item negative
\item none of these
\end{enumerate}

\item The least value of the expression $2 log_{10}x-log_{x}(0.01)$, for $x>1$, is
\begin{enumerate}
\item $10$
\item $2$ 
\item $-0.01$
\item none of these
\end{enumerate}

\item If $(x^{2}+px+1)$ is a factor of $(ax^{3}+bx+c)$, then
\begin{enumerate}
\item $a^{2}+c^{2}=-ab$
\item $a^{2}-c^{2}=-ab$ 
\item $a^{2}-c^{2}=ab$
\item none of these
\end{enumerate}

\item The number of real solutions of the equation $|x|^{2}-3|x|+2=0$ is
\begin{enumerate}
\item $4$
\item $1$ 
\item $3$
\item $2$ 
\end{enumerate}

\item Two towns A and B are 60 km apart. A school is to be built to serve $150$ students in town A and $50$ students in town B. If the total distance to be travelled by all $200$ students is to be as small as possible, then the school should be built at
\begin{enumerate}
\item town B
\item $45$ km from town A
\item town A
\item $45$km from town B
\end{enumerate}

\item If p,q,r are any real numbers, then
\begin{enumerate}
\item max(p,q)$<$max(p,q,r)
\item min(p,q)$=\frac{1}{2}(p+q-|p-q|)$ 
\item max(p,q)$<$min(p,q,r)
\item none of these 
\end{enumerate}

\item The largest interval for which\\ $x^{12}-x^{9}+x^{4}-x+1>0$ is 
\begin{enumerate}
\item $-4<x\leq0$
\item $0<x<1$ 
\item $-100<x<100$
\item $-\infty<x<\infty$ 
\end{enumerate}

\item The equation $x-\frac{2}{x-1}=1-\frac{2}{x-1}$ has
\begin{enumerate}
\item no root
\item one root 
\item two equal roots
\item infinitely many roots
\end{enumerate}

\item If $a^{2}+b^{2}+c^{2}=1$, then $ab+bc+ca$ lies in the interval
\begin{enumerate}
\item $[\frac{1}{2},2]$
\item $[-1,2]$ 
\item $[-\frac{1}{2},1]$
\item $[-1,\frac{1}{2}]$
\end{enumerate}

\item If $log_{0.3}(x-1)<log_{0.09}(x-1)$, then x lies in the interval
\begin{enumerate}
\item $(2,\infty)$
\item $(1,2)$ 
\item $(-2,-1)$
\item none of these 
\end{enumerate}

\item If $\alpha$ and $\beta$ are the roots of $x^{2}+px+q=0 $ and $ \alpha^{4},\beta^{4}$ are the roots of $x^{2}-rx+s=0$, then the equation $x^{2}-4qx+2q^{2}-r=0$ has always
\begin{enumerate}
\item two real roots
\item two positive roots
\item two negative roots
\item one positive and one negative root 
\end{enumerate}
*Question has more than one correct option.

\item Let a,b,c be real numbers, $a\neq0$. If $\alpha$ is a root of $a^{2}x^{2}+bx+c=0$. $\beta$ is the root of $a^{2}x^{2}-bx-c=0$ and $0<\alpha<\beta$, then the equation $a^{2}x^{2}+2bx+2c=0$ has a root $\gamma$ that always satisfies
\begin{enumerate}
\item $\gamma = \frac{\alpha+\beta}{2}$
\item $\gamma = \alpha+\frac{\beta}{2}$ 
\item $\gamma = \alpha$
\item $\alpha<\gamma<\beta$
\end{enumerate}

%\item The number of solutions of the equation $\sin(e)^{x}=5^{x}+5^{-x}$ is
%\begin{enumerate}
%\item $0$
%\item $1$ 
%\item $2$
%\item Infinitely many
%\end{enumerate}

\item Let $\alpha, \beta$ be the roots of the equation $(x-a)(x-b)=c, c\neq0$. Then the roots of the equation $(x-\alpha)(x-\beta)+c=0$ are
\begin{enumerate}
\item a,c
\item b,c 
\item a,b
\item $a+c,b+c$
\end{enumerate}

\item The number of points of intersection of two curves $y=2 sinx$ and $y=5x^{2}+2x+3$ is
\begin{enumerate}
\item $0$
\item $1$ 
\item $2$
\item $\infty$ 
\end{enumerate}

\item If p,q,r are $+$ve and are in A.P., the roots of quadratic equation $px^{2}+qx+r=0$ are all real for
\begin{enumerate}
\item $|\frac{r}{p}-7|\geq{4\sqrt{3}}$
\item $|\frac{p}{r}-7|\geq{4\sqrt{3}}$ 
\item all p nd r
\item no p and r 
\end{enumerate}

\item Let p,q$\in\{1,2,3,4\} $. The number of equations of the form $px^{2}+qx+1=0$ having real roots is
\begin{enumerate}
\item $15$
\item $9$ 
\item $7$
\item $8$ 
\end{enumerate}

\item If the roots of the equation\\ $x^{2}-2ax+a^{2}+a-3=0$ are real and less than 3, then
\begin{enumerate}
\item $a<2$
\item $2\leq{a}\leq{3}$ 
\item $3<a\leq{4}$
\item $a>4$ 
\end{enumerate}

\item If $\alpha$ and $\beta$ $(\alpha<\beta)$ are the roots of the equation $x^{2}+bx+c=0$, where $c<0<b$, then
\begin{enumerate}
\item $0<\alpha<\beta$
\item $\alpha<0<\beta<|\alpha|$ 
\item $\alpha<\beta<0$
\item $\alpha<0<|\alpha|<\beta$
\end{enumerate}

\item If a,b,c,d are positive real numbers such that $a+b+c+d=2$, then $M=(a+b)(c+d)$ satisfies the relation 
\begin{enumerate}
\item $0\leq{M}\leq1$
\item $1\leq{M}\leq2$ 
\item $2\leq{M}\leq3$
\item $3\leq{M}\leq4$
\end{enumerate}

\item If $b>a$, then the equation $(x-a)(x-b)-1=0$ has 
\begin{enumerate}
\item both roots in (a,b)
\item both roots in$(-\infty,a)$ 
\item both roots in $(b,+\infty)$
\item one root in $(-\infty,a)$ and the other in $(b,+\infty)$
\end{enumerate}

\item For the equation $3x^{2}+px+3=0, p>0$, if one of the root is square of the other, then p is equal to
\begin{enumerate}
\item $1/3$
\item $1$ 
\item $3$
\item $2/3$
\end{enumerate}

%\item If $a_1, a_2,.....,a_n$ are positive real numbers whose product is a fixed number c, then the minimum value of $a_1+a_2+.......+a_{n-1}+2a_n$ is
%\begin{enumerate}
%\item $n(2c)^{\frac{1}{n}}$
%\item $(n+1)c^\frac{1}{n}$ 
%\item $2nc^\frac{1}{n}$
%\item $(n+1)(2c)^\frac{1}{n}$
%\end{enumerate}

\item The set of all real numbers x for which $x^{2}-|x+2|+x>0$, is
\begin{enumerate}
\item $(-\infty,-2)\cup(2,\infty)$
\item $(-\infty,-\sqrt2)\cup(\sqrt2,\infty)$ 
\item $(-\infty,-1)\cup(1,\infty)$
\item $(\sqrt2,\infty)$ 
\end{enumerate}

\item If $\alpha \in(0,\frac{\pi}{2})$ then$\sqrt{x^{2}+x}+\frac{tan^{2}\alpha}{\sqrt{x^{2}+x}}$ is always greater than or equal to 
\begin{enumerate}
\item $2tan\alpha$
\item $1$ 
\item $2$
\item $sec^{2}\alpha$ 
\end{enumerate}

\item For all 'x',$x^{2}+2ax+10-3a>0$, then the interval in which 'a' lies is 
\begin{enumerate}
\item $a<-5$
\item $-5<a<2$ 
\item $a>5$
\item $2<a<5$
\end{enumerate}

\item If one root is square of the other root of the equation $x^{2}+px+q=0$, then the relation between p and q is
\begin{enumerate}
\item $p^{3}-q(3p-1)+q^{2}=0$
\item $p^{3}-q(3p+1)+q^{2}=0$
\item $p^{3}+q(3p-1)+q^{2}=0$
\item $p^{3}+q(3p+1)+q^{2}=0$ 
\end{enumerate}

\item Let a,b,c be the sides of a triangle where ${a}\neq{b}\neq{c}$ and $\lambda \in$R. If the roots of the equation $x^{2}+2(a+b+c)x+3\lambda(ab+bc+ca)=0$ are real, then
\begin{enumerate}
\item $\lambda<\frac{4}{3}$
\item $\lambda>\frac{5}{3}$
\item $\lambda\in(\frac{1}{3},\frac{5}{3})$ 
\item $\lambda\in(\frac{4}{3},\frac{5}{3})$
\end{enumerate}

\item Let $\alpha, \beta$ be the roots of the equation $x^{2}-px+r=0$ and $\frac{\alpha}{2},2\beta$ be the roots of the equation $x^{2}-qx+r=0$. Then the value of r is
\begin{enumerate}
\item $\frac{2}{9}(p-q)(2q-p)$
\item $\frac{2}{9}(q-p)(2p-q)$
\item $\frac{2}{9}(q-2p)(2q-p)$
\item $\frac{2}{9}(2p-q)(2q-p)$ 
\end{enumerate}

\item Let p and q be real numbers such that $p\neq0$, $p^{3}\neq{q}$ and $p^{3}\neq-q$. If $\alpha$ and $\beta$ are non zero complex numbers satisfying $\alpha+\beta=-p$ and $\alpha^{3}+\beta^{3}=q$, then the quadratic equation having $\frac{\alpha}{\beta}$ and $\frac{\beta}{\alpha}$ as its roots is 
\begin{enumerate}
\item $(p^{3}+q)x^{2}-(p^{3}+2q)x+(p^{3}+q)=0$
\item $(p^{3}+q)x^{2}-(p^{3}-2q)x+(p^{3}+q)=0$ 
\item $(p^{3}-q)x^{2}-(5p^{3}-2q)x+(p^{3}-q)=0$
\item $(p^{3}-q)x^{2}-(5p^{3}+2q)x+(p^{3}-q)=0$ 
\end{enumerate}

\item Let $(x_0,y_0)$ be the solution of the following equations $(2x)^{ln2}=(3y)^{ln3}$ and $3^{lnx}=2^{lny}$. Then $x_0$ is
\begin{enumerate}
\item $\frac{1}{6}$
\item $\frac{1}{3}$
\item $\frac{1}{2}$
\item $6$ 
\end{enumerate}

\item Let $\alpha$ and $\beta$ be the roots of $x^{2}-6x-2=0$, with $\alpha>\beta$. If $a_n=\alpha^{n}-\beta^{n}$ for $n\geq1$, then the value of $\frac{a_{10}-2a_{8}}{2a_{9}}$ is
\begin{enumerate}
\item $1$
\item $2$ 
\item $3$
\item $4$ 
\end{enumerate}

\item A value of b for which the equations\\
$x^{2}+bx-1=0$\\
$x^{2}+x+b=0$\\
have one root in common is 
\begin{enumerate}
\item $-\sqrt2$
\item $-i\sqrt3$ 
\item $i\sqrt5$
\item $\sqrt2$ 
\end{enumerate}

\item The quadratic equation $p(x)=0$ with real coefficients has purely imaginary roots. Then the equation p(p(x))=0 has
\begin{enumerate}
\item one purely imaginary root
\item all real roots
\item two real and two purely imaginary roots
\item neither real nor purely imaginary roots
\end{enumerate}

\item Let $-\frac{\pi}{6}<\theta<-\frac{\pi}{12}$. Suppose $\alpha_1$ and $\beta_1$ are the roots of the equation $x^{2}-2x (sec\alpha) +1=0$ and $\alpha_2$ and $\beta_2$ are the roots of the equation $x^{2}-2x tan\theta-1=0$. If $\alpha_1>\beta_1$ and $\alpha_2>\beta_2$, then $\alpha_1+\beta_2$ equals 
\begin{enumerate}
\item $2(sec\theta-tan\theta)$
\item $2sec\theta$ 
\item $-2tan\theta$
\item $0$ 
\end{enumerate}

\item For real x, the function$\frac{(x-a)(x-b)}{x-c}$ will assume all real values provided
\begin{enumerate}
\item $a>b>c$
\item $a<b<c$ 
\item $a>c>b$
\item $a<c<b$ 
\end{enumerate}

\item If S is the set of all real x such that$\frac{2x-1}{2x^{2}+3x^{2}+x}$ is positive, then S contains 
\begin{enumerate}
\item $[-\infty,-\frac{3}{2}]$
\item $[-\frac{3}{2},-\frac{1}{4}]$
\item $[-\frac{1}{4},\frac{1}{2}]$
\item $[\frac{1}{2},3]$
\end{enumerate}

\item If a,b and c are distinct positive numbers, then the expression (b+c-a)(c+a-b)(a+b-c)-abc is
\begin{enumerate}
\item positive
\item negative 
\item non-positive
\item non-negative
\item none of these 
\end{enumerate}

\item If a,b,c,d and p are distinct real numbers such that\\$(a^{2}+b^{2})p^{2}-2(ab+bc+cd)p+(b^{2}+c^{2}+d^{2})\leq0$ then a,b,c,d
\begin{enumerate}
\item are in A.P.
\item are in G.P. 
\item are in H.P.
\item satisfy ab$=$cd
\item none of these 
\end{enumerate}

\item The equation $x^{3/4(log_{2}x)^{2}+log_2x^{-5/4}}=\sqrt2$ has
\begin{enumerate}
\item at least one real solution
\item exactly three solutions
\item exactly one irrational solution
\item complex roots
\end{enumerate}

\item The product of n positive numbers is unity. Then their sum is
\begin{enumerate}
\item a positive integer
\item divisible by n
\item equal to $n+\frac{1}{n}$
\item never less than n
\end{enumerate}

\item Number of divisor of the form $4n+2(n\geq0)$ of the integer $240$ is
\begin{enumerate}
\item $4$
\item $8$ 
\item $10$
\item $3$
\end{enumerate}

\item If $3^{x}=4^{x-1}$, then $x=$
\begin{enumerate}
\item $\frac{2log_32}{2log_32-1}$
\item $\frac{2}{2-2log_23}$
\item $\frac{1}{1-log_43}$
\item $\frac{2log_23}{2log_23-1}$
\end{enumerate}

\item Let S be the set of all non-zero real numbers $\alpha$ such that the quadratic equation $\alpha{x}^{2}-x+\alpha=0$ has two distinct real roots $x_1$ and $x_2$ satisfying the inequality $|x_1-x_2|<1$. Which of the following intervals is (are) a subset (s) of S?
\begin{enumerate}
\item $(-\frac{1}{2},-\frac{1}{\sqrt5})$
\item $(-\frac{1}{\sqrt5},0)$
\item $(0,\frac{1}{\sqrt5})$
\item $(\frac{1}{\sqrt5},\frac{1}{2})$
\end{enumerate}

\item Solve for x: $4^x-3^{x-\frac{1}{2}}=3^{x+\frac{1}{2}}-2^{2x-1}$

\item If (m,n)$=\frac{(1-x^m)(1-x^{m-1}).......(1-x^{m-n+1})}{(1-x)(1-x^2).........(1-x^n)}$ where m and n are positive integers$(n\leq{m})$, show that $(m,n+1)=(m-1,n+1)+x^{m-n-1}(m-1,n)$.

\item Solve for x: $\sqrt{x+1}-\sqrt{x-1}=1$.

\item Solve the following equation for x:
\begin{align} 
2log_{x} a+log_{ax} a+3log_{a^2{x}} a=0, a>0.
\end{align}
\item Show that the square of $\frac{\sqrt{26-15\sqrt{3}}}{5\sqrt{2}-\sqrt{38+5\sqrt{3}}}$,is a rational number.

\item Sketch the solution set of the following system of inequalities: 
\begin{align}
x^{2}+y^{2}-2x\geq0;3x-y-12\leq0;y-x\leq0;y\geq0.
\end{align}

\item Find all integers x for which $(5x-1)<(x+1)^2<(7x-3)$

\item If $\alpha,\beta$ are the roots of $x^2+px+q=0$ and $\gamma,\delta$ are the roots of $x^2+rx+s=0$, evalute$(\alpha-\gamma)(\alpha-\delta)(\beta-\gamma)(\beta-\delta)$ in terms of p,q,,r and s. Deduce the condition that the equations have a common root.

\item Given $n^4<10^n$ for fixed positive integer $n\geq2$ prove that $(n+1)^4<10^{n+1}$

\item Let y=$\sqrt{\frac{(x+1)(x-3)}{(x-2)}}$ Find all the real values of x for which y takes real values.

\item For what values of m, does the system of equations $3x+my=m$,$2x-5y=20$ has solution satisfying the condition $x>0,y>0$.

\item Find the solution set of the system $x+2y+z=1;$ $2x-3y-w=2;$ $x\geq0;y\geq0;z\geq0;w\geq0$.

\item Show that the equation$e^{sinx}-e^{-sinx}-4=0$ has no real solution.

\item mm squares of equal size are arranged to form a rectangle of dimension m by n, where m and n are natural numbers. Two squares will be called 'neighbours' if they have exactly one common side. A natural number is written in each square such that the number written in any square is the arithmetic mean of the numbers written in its neighbouring squares. Show that this is possible only if all the numbers used are equal.

\item If one root of the quadratic equation $ax^2+bx+c=0$ is equal to the n-th power of the other, then show that $(ac)^{\frac{1}{n+1}}+(a^nc)^{\frac{1}{n+1}}+b=0$

\item Find all real values of x which satisfy $x^2-3x+2>0$ and $x^2-2x-4\leq0$.

\item Solve for x;  $(5+2\sqrt{6})^{x^2-3}+(5-2\sqrt6)^{x^2-3}=10$.

\item For $a\leq0$, determine all real roots of the equation$x^2-2a|x-a|-3a^2=0$

\item Find the set of all x for which $\frac{2x}{(2x^2+5x+2)}>\frac{1}{(x+1)}$.

\item Solve ${x^2+4x+3}+2x+5=0$

\item Let a,b,c be real. If $ax^2+bx+c=0$ has two real roots $\alpha$ and $\beta$, where $\alpha<-1$ and $\beta>1$, then show that $1+\frac{c}{a}+|\frac{b}{a}|<0$.

\item Let S be a square of unit area. Consider any quadrilateral which has one vertex on each side of S. If a,b,c and d denote the lengths of the sides of the quadrilateral, prove that $2\leq{a}^2+b^2+c^2+d^2\leq4$.

\item If $\alpha,\beta$ are the roots of $ax^2+bx+c=0, (a\neq0)$ and $\alpha+\delta,\beta+\delta$ are the roots of $Ax^2+Bx+C=0, (A\neq0)$ for some constant $\delta$, then prove that $\frac{b^2-4ac}{a^2}=\frac{B^2-4AC}{A^2}$.

\item Let a,b,c be real numbers with $a\neq0$ and let $\alpha,\beta$ be the roots of the equation $ax^2+bx+c=0$. Express the roots of $a^3x^2+abcx+c^3=0$ in terms of $\alpha,\beta$.

\item If $x^2+(a-b)x+(1-a-b)=0$ where a,b$\in$R then find the values of a for which equation has unequal real roots for all values of b.

\item If a,b,c are positive real numbers. Then prove that $(a+1)^7(b+1)^7(c+1)^7>7^7a^4b^4c^4$.

\item Let a and b be the roots of the equation $x^2-10ax-11b=0$ are c,d then the value of $a+b+c+d$, when $a\neq{b}\neq{c}\neq{d}$, is.

\item Let p,q be integers and let $\alpha,\beta$ be the roots of the equation, $x^2-x-1=0$, where $\alpha\neq\beta$. For n$=0,1,2$,.......,let $a_n=p\alpha^n+q\beta^n$.
FACT: If a and b are rational numbers and \\$a+b\sqrt5=0$, then a$=0=$b.s

\item $a_{12} =$
\begin{enumerate}
\item $a_{11}-a_{10}$
\item $a_{11}+a_{10}$ 
\item $2a_{11}+a_{10}$
\item $a_{11}+2a_{10}$
\end{enumerate}

\item If $a_4 =28$, then $p+2q=\in$
\begin{enumerate}
\item $21$
\item $14$ 
\item $7$
\item $12$ 
\end{enumerate}

\item Let a,b,c,p,q be real numbers. Suppose $\alpha,\beta$ are the roots of the equation $x^2+2px+q=0$ and $\alpha, \frac{1}{\beta}$ are the roots of the equation $ax^2 +2bx+c=0$, where $\beta^2{\notin\lbrace-1,0,1\rbrace}$\\
         STATEMENT-1:$(p^2-q)(b^2-ac)\geq0$
          and\\
         STATEMENT-2:$b\neq{pa}$ or $c\neq{qa}$
\begin{enumerate}
\item STATEMENT-1 is True, STATEMENT-2 is True;STATEMENT-2 is a correct explanation for STATEMENT-1
\item STATEMENT-1 is True,STATEMENT-2 is True;STATEMENT-2 is NOT a correct explanation for STATEMENT-1
\item STATEMENT-1 is True,STATEMENT-2 is False
\item STATEMENT-1 is False,STATEMENT-2 is True
\end{enumerate}

\item Let (x,y,z) be points with integer coordinates satisfying the system of homogeneous equations:
$3x-y-z=0$,$-3x+z=0$,$-3x+2y+z=0$ Then the number of such points for which $x^2+y^2+z^2\leq100$ is

\item The smallest value of k, for which both the roots of the equation $x^2-8kx+16(k^2-k+1)=0$ are real, distinct and have values at least $4$ is

\item The minimum value of the sum of real numbers $a^{-5}, a^{-4}, 3a^{-3}, 1, a^8$ and $a^{10}$ where $a>0$ is

\item The number of distinct real roots of $x^4-4x^3+12x^2+x-1=0$ is

\item If $\alpha\neq\beta$ but $\alpha^2=5\alpha-3$ and $\beta^2=5\beta-3$ then the equation having $\alpha/\beta$ and $\beta/\alpha$ as its roots is 
\begin{enumerate}
\item $3x^2-19x+3=0$
\item $3x^2+19x-3=0$ 
\item $3x^2-19x-3=0$
\item $x^2-5x+3=0$ 
\end{enumerate}

\item Difference between the corresponding roots of $x^2+ax+b=0$ and $x^2+bx+a=0$ is same and $a\neq{b}$, then
\begin{enumerate}
\item $a+b+4=0$
\item $a+b-4=0$ 
\item $a-b-4=0$
\item $a-b+4=0$
\end{enumerate}

\item Product of real roots of the equation $t^2x^2+|x|+9=0$
\begin{enumerate}
\item is always positive
\item is always negative 
\item does not exist
\item none of these 
\end{enumerate}

\item If p and q are the roots of the equation $x^2+px+q=0$, then
\begin{enumerate}
\item $p=1, q=-2$
\item $p=0, q=1$ 
\item $p=-2, q=0$
\item $p=-2, q=1$ 
\end{enumerate}

\item If a,b,c are distinct +ve real numbers and $a^2+b^2+c^2=1$ then $ab+bc+ca$ is
\begin{enumerate}
\item less than 1
\item equal to 1 
\item greater than 1
\item any real number 
\end{enumerate}

\item If the sum of the roots of the quadratic equation $ax^2+bx+c=0$ is equal to the sum of the squares of their reciprocals, then $\frac{a}{c},\frac{b}{a}$ and $\frac{c}{b}$ are in
\begin{enumerate}
\item Arithmetic-Geometric Progression
\item Arithmetic Progression 
\item Geometric Progression
\item Harmonic Progression
\end{enumerate}

\item The value of 'a' for which one root of the quadratic equation $(a^2-5a+3)x^2+(3a-1)x+2=0$ is twice as large as the other is
\begin{enumerate}
\item $-\frac{1}{3}$
\item $\frac{2}{3}$
\item $-\frac{2}{3}$
\item $\frac{1}{3}$
\end{enumerate}

\item The number of real solutions of the equation $x^2-3|x|+2=0$ is
\begin{enumerate}
\item $3$
\item $2$ 
\item $4$
\item $1$
\end{enumerate}

\item The real number x when added to its inverse gives the minimum value of the sum at x equal to
\begin{enumerate}
\item $-2$
\item $2$ 
\item $1$
\item $-1$ 
\end{enumerate}

\item Let two numbers have arithmetic mean $9$ and geometric mean $4$. Then these numbers are the roots of the quadratic equation
\begin{enumerate}
\item $x^2-18x-16=0$
\item $x^2-18x+16=0$ 
\item $x^2+18x-16=0$
\item $x^2+18x+16=0$ 
\end{enumerate}
 
\item If (1-p) is a root of quadratic equation $x^2+px+(1-p)=0$ then its root are
\begin{enumerate}
\item $-1,2$
\item $-1,1$
\item $0,-1$
\item $0,1$
\end{enumerate}

\item If one root of the equation $x^2+px+12=0$ is $4$, while the equation $x^2+px+q=0$ has equal roots, then the value of 'q' is
\begin{enumerate}
\item $4$
\item $12$ 
\item $3$
\item $\frac{49}{4}$ 
\end{enumerate}

\item In a triangle PQR,$\angle{R}=\frac{\pi}{2}$. If tan$(\frac{P}{2})$ and -tan$(\frac{Q}{2})$ are roots of $ax^2+bx+c=0, a\neq0$ then
\begin{enumerate}
\item $a=b+c$
\item $c=a+b$
\item $b=c$
\item $b=a+c$ 
\end{enumerate}

\item If both the roots of the quadratic equation $x^2-2kx+k^2+k-5=0$ are less than $5$, then k lies in the interval
\begin{enumerate}
\item $(5,6]$
\item $(6,\infty)$
\item $(-\infty, 4)$
\item $[4,5]$ 
\end{enumerate}

\item If the roots of the quadratic equation $x^2+px+q=0$ are tan$30\degree$ and tan${15}\degree$, respectively, then the value of 2+q-p is
\begin{enumerate}
\item $2$
\item $3$
\item $0$
\item $1$ 
\end{enumerate}

\item All the values of m for which both roots of the equation $x^2-2mx+m^2-1=0$ are greater than $-2$ but less than $4$, lies in the interval
\begin{enumerate}
\item $-2<m<0$
\item $m>3$
\item $-1<m<3$
\item $1<m<4$ 
\end{enumerate}

\item If x is real , the maximum value of $\frac{3x^2+9x+17}{3x^2+9x+7}$ is
\begin{enumerate}
\item $\frac{1}{4}$
\item $41$
\item $1$
\item $\frac{17}{7}$ 
\end{enumerate}

\item If the difference between the roots of the equation $x^2+ax+1=0$ is less than $\sqrt5$, then the set of possible values of a is
\begin{enumerate}
\item $(3,\infty)$
\item $(-\infty,-3)$
\item $(-3,3)$
\item $(-3,\infty)$
\end{enumerate}

\item Statement-1: For every natural number $n\geq{2}$,
$\frac{1}{\sqrt1}+\frac{1}{\sqrt2}+.............+\frac{1}{\sqrt{n}}>\sqrt{n}$ 
Statement-2: For every natural number $n\geq{2}$,
$\sqrt{n(n+1)}<n+1$
\begin{enumerate}
\item Statement-1 is false, Statement-2 is true
\item Statement-1 is true, Statement-2 is true; Statement-2 is a correct explanation for Statement-1
\item Statement-1 is true, Statement-2 is true; Statement-2 is not a correct explanation for Statement-1
\item Statement-1 is true, Statement-2 is false 
\end{enumerate}

\item The quadratic equation $x^2-6x+a=0$ and $x^2-cx+6=0$ have one root in common. The other roots of the first and second equations are integers in the ratio 4:3. Then the common root is
\begin{enumerate}
\item $1$
\item $4$
\item $3$
\item $2$ 
\end{enumerate}

\item If the roots of the equation $bx^2+cx+a=0$ be imaginary, then for all real values of x, the expression $3b^2x^2+6bcx+2c^2$ is:
\begin{enumerate}
\item less than $4ab$
\item greater than $-4ab$
\item less than $-4ab$
\item greater than $4ab$
\end{enumerate}

\item If $|z-\frac{4}{z}|=2$, then the maximum value of $|Z|$ is equal to: 
\begin{enumerate}
\item $\sqrt{5}+1$
\item $2$
\item $2+\sqrt{2}$
\item $\sqrt{3}+1$ 
\end{enumerate}

\item If $\alpha$ and $\beta$ are the roots of the equation $x^2-x+1=0$, then $\alpha^{2009}+\beta^{2009}=$
\begin{enumerate}
\item $-1$
\item $1$
\item $2$
\item $-2$ 
\end{enumerate}

\item The equation $e^{sinx}-e^{-sinx}-4=0$ has:
\begin{enumerate}
\item infinite number of real roots
\item no real roots
\item exactly one real root
\item exactly four real roots
\end{enumerate}

\item The real number $k$ for which the equation, $2x^3+3x+k=0$ has two distinct real roots in [0,1]
\begin{enumerate}
\item lies between $1$ and $2$
\item lies between $2$ and $3$
\item lies between $-1$ and $0$
\item does not exist 
\end{enumerate}

\item The number of values of $k$, for which the system of equations:\\
$(k+1)x+8y=4k$\\
$kx+(k+3)y=3k-1$
\begin{enumerate}
\item infinite
\item $1$
\item $2$
\item $3$
\end{enumerate}

\item If the equations $x^2+2x+3=0$ and $ax^2+bx+c=0$, a,b,c$\in$R, have a common root, then a:b:c is
\begin{enumerate}
\item $1:2:3$
\item $3:2:1$
\item $1:3:2$
\item $3:1:2$ 
\end{enumerate}

\item Is a$\in$R and the equation $-3(x-[x])^2+2(x-[x])+a^2=0$ 
(where[x] denotes the greatest integer $\leq{x})$ has no integral solution, then all possible values of a lie in the interval:
\begin{enumerate}
\item $(-2,-1)$
\item $(-\infty,-2)\cup(2,\infty)$
\item $(-1,0)\cup(0,1)$
\item $(1,2)$
\end{enumerate}

\item Let $\alpha$ and $\beta$ be the roots of the equation $px^2+qx+r=0,p\neq0$. If $p,q,r$ are in A.P. and $\frac{1}{\alpha}+\frac{1}{\beta}=4$, then the value of $|\alpha-\beta|$ is:
\begin{enumerate}
\item $\frac{\sqrt{34}}{9}$
\item $\frac{2\sqrt{13}}{9}$
\item $\frac{\sqrt{61}}{9}$
\item $\frac{2\sqrt{17}}{9}$
\end{enumerate}

\item Let $\alpha$ and $\beta$ be the roots of equation $x^2-6x-2=0$. If $a_n=\alpha^n-\beta^n$, for $n\geq1$, then the value of $\frac{a_{10}-2a_8}{2a_9}$ is equal to:
\begin{enumerate}
\item $3$
\item $-3$
\item $6$
\item $-6$ 
\end{enumerate}

\item The sum of all real values of x satisfying the equation $(x^2-5x+5)^{x^2+4x-60}=1$ is:
\begin{enumerate}
\item $6$
\item $5$
\item $3$
\item $-4$ 
\end{enumerate}

\item If $\alpha,\beta\in$ C are the distinct roots, of the equation $x^2-x+1=0$, then $\alpha^{101}+\beta^{107}$ is equal to:
\begin{enumerate}
\item $0$
\item $1$
\item $2$
\item $-1$ 
\end{enumerate}

\item Let p,q $\in$R. If $2-\sqrt{3}$ is a root of the quadratic equation, $x^2+px+q=0$, then:
\begin{enumerate}
\item $p^2-4q+12 = 0$
\item $q^2-4p-16 = 0$
\item $q^2+4p+14 = 0$
\item $p^2-4q-12 = 0$
\end{enumerate}
\end{enumerate}
 
\section{Indefinite integrals}
\renewcommand{\theequation}{\theenumi}
\begin{enumerate}[label=\arabic*.,ref=\thesubsection.\theenumi]
\numberwithin{equation}{enumi}

\item If $\int_{}\frac{4e^{x} + 6e^{-x}}{9e^{x} - 4e^{-x}}dx = Ax + Blog(9e^{2x} - 4) + C$, then A =.........., B =.........., and C =............

\textbf{ MCQs with One Correct Answer:}
\item The value of the integral $\int_{}\frac{\cos^{3}x + \cos^{5}x}{\sin^{2}x + \sin^{4}x}dx$ is
\begin{enumerate}
\item $\sin x - 6\tan^{-1}(\sin x) + C$
\item $\sin x - 2(\sin x)^{-1} + C$
\item $\sin x - 2(\sin x)^{-1} - 6\tan^{-1}(\sin x) + C$
\item $\sin x - 2(\sin x)^{-1} + 5\tan^{-1}(\sin x) + C$
\end{enumerate}

\item If $\int_{\sin x}^{1}t^{2}f(t)dt = 1 - \sin x$, then $f(\frac{1}{\sqrt{3}})$ is
\begin{enumerate}
\item $\frac{1}{3}$
\item $\frac{1}{\sqrt{3}}$
\item $3$
\item $\sqrt{3}$
\end{enumerate}

\item $\int_{}\frac{x^2 - 1}{x^3\sqrt{2x^4 - 2x^2 + 1}}dx$ =
\begin{enumerate}
\item $\frac{\sqrt{2x^4 - 2x^2 + 1}}{x^2} + C$
\item $\frac{\sqrt{2x^4 - 2x^2 + 1}}{x^3} + C$
\item $\frac{\sqrt{2x^4 - 2x^2 + 1}}{x} + C$
\item $\frac{\sqrt{2x^4 - 2x^2 + 1}}{2x^2} + C$
\end{enumerate}

\item Let I = $\int_{}\frac{e^x}{e^{4x} + e^{2x} + 1}dx$, J = $\int_{}\frac{e^{-x}}{e^{-4x} + e^{-2x} + 1}dx$. Then for an arbitary constant C, then the value of J - I equals
\begin{enumerate}
\item $\frac{1}{2}log(\frac{e^{4x} - e^{2x} + 1}{e^{4x} + e^{2x} + 1}) + C$
\item $\frac{1}{2}log(\frac{e^{2x} + e^{x} + 1}{e^{2x} - e^{x} + 1}) + C$
\item $\frac{1}{2}log(\frac{e^{2x} - e^{x} + 1}{e^{2x} + e^{x} + 1}) + C$
\item $\frac{1}{2}log(\frac{e^{4x} + e^{2x} + 1}{e^{4x} - e^{2x} + 1}) + C$
\end{enumerate}

\item The integral $\int_{}\frac{\sec^{2}x}{(\sec x + \tan x)^{9/2}}dx$ equals(for some arbitary constant K)
\begin{enumerate}
\item $-\frac{1}{(\sec x + \tan x)^{11/2}}\{\frac{1}{11} - \frac{1}{7}(\sec x + \tan x)^2\} + K$
\item $\frac{1}{(\sec x + \tan x)^{11/2}}\{\frac{1}{11} - \frac{1}{7}(\sec x + \tan x)^2\} + K$
\item $-\frac{1}{(\sec x + \tan x)^{11/2}}\{\frac{1}{11} + \frac{1}{7}(\sec x + \tan x)^2\} + K$
\item $\frac{1}{(\sec x + \tan x)^{11/2}}\{\frac{1}{11} + \frac{1}{7}(\sec x + \tan x)^2\} + K$
\end{enumerate}

\textbf{Subjective Problems:}

\item Evaluate 
\begin{align*}
\int_{}\frac{\sin x}{\sin x - \cos x}dx
\end{align*}

\item Evaluate 
\begin{align*}
\int_{}\frac{x^2}{(a + bx)^2}
\end{align*}

\item Evaluate 
\begin{align*}
\int_{}(e^{log x} + \sin x)\cos x dx
\end{align*}

\item Evaluate: 
\begin{align*}
\int_{}\frac{(x - 1)e^x}{(x + 1)^3}dx
\end{align*}

\item Evaluate the following 
\begin{align*}
\int_{}\frac{dx}{x^2(x^4 + 1)^3/4}
\end{align*}

\item Evaluate the following 
\begin{align*}
\int_{}\sqrt{\frac{1 - \sqrt{x}}{1 + \sqrt{x}}}dx
\end{align*}

\item Evaluate: 
\begin{align*}
\int_{}[\frac{(\cos2x)^1/2}{\sin x}]dx
\end{align*}

\item Evaluate 
\begin{align*}
\int_{}(\sqrt{\tan x} + \sqrt{\cot x})dx
\end{align*}

\item Find the indefinite integral
\begin{align*}
\int_{}(\frac{1}{x^{1/3} + 4^{1/4}} + \frac{ln(1 + x^{1/6})}{x^{1/3} + x^{1/2}})dx
\end{align*}

\item Find the indefinite integral
\begin{align*}
\int_{}\cos 2\theta ln(\frac{\cos \theta + \sin \theta}{\cos \theta - \sin \theta})d\theta
\end{align*}

\item Evaluate 
\begin{align*}
\int_{}\frac{(x + 1)}{x(1 + xe^x)^2}dx
\end{align*}

\item Integrate 
\begin{align*}
\frac{x^3 + 3x + 2}{(x^2 + 1)^2(x + 1)}dx
\end{align*}

\item Evaluate
\begin{align*}
\int_{}\sin^{-1}(\frac{2x + 2}{\sqrt{4x^2 + 8x + 13}})dx.
\end{align*}

\item For any natural number m, evaluate
\begin{align*}
\int_{}(x^{3m} + x^{2m} + x^{m})(2x^{2m} + 3x^m + 6)^{l/m}dx, x > 0
\end{align*}

\textbf{Assertion and Reason Type Questions:}

\item Let F(x) be an indefinite integral of $\sin^{2}x$.\\
\textbf{STATEMENT-1:} The function F(x) satisfies $F(x + \pi) = F(x)$ for all real x.\\
\textbf{STATEMENT-2:} $\sin^{2}(x + \pi) = \sin^{2}x$ for all real x.
\begin{enumerate}
\item Statement-1 is True, Statement-2 is True, Statement-2 is a correct explanation for Statement-1.
\item Statement-1 is True, Statement-2 is True, Statement-2 is NOT a correct explanation for Statement-1.
\item Statement-1 is True, Statement-2 is False.
\item Statement-1 is False, Statement-2 is True.
\end{enumerate}

\textbf{Section - B:}

\item If $\int_{}\frac{\sin x}{\sin(x - \alpha)}dx = Ax + Blog\sin(x - alpha) + C$, then the value of (A, B) is
\begin{enumerate}
\item $(-\cos\alpha, \sin\alpha)$
\item $(\cos\alpha, \sin\alpha)$
\item $(-\sin\alpha, \cos\alpha)$
\item $(\sin\alpha, -\cos\alpha)$
\end{enumerate}

\item $\int_{}\frac{dx}{\cos x - \sin x}$ is equal to
\begin{enumerate}
\item $\frac{1}{\sqrt{2}}log|\tan(\frac{x}{2}) + \frac{3\pi}{8}| + C$
\item $\frac{1}{\sqrt{2}}log|\cot(\frac{x}{2})| + C$
\item $\frac{1}{\sqrt{2}}log|\tan(\frac{x}{2}) - \frac{3\pi}{8}| + C$
\item $\frac{1}{\sqrt{2}}log|\tan(\frac{x}{2}) - \frac{\pi}{8}| + C$
\end{enumerate}

\item $\int_{}\{\frac{(logx - 1)}{1 + (logx)^2}\}^2dx$ is equal to
\begin{enumerate}
\item $\frac{logx}{(logx)^2 + 1} + C$
\item $\frac{x}{x^2 + 1} + C$
\item $\frac{xe^x}{1 + x^2} + C$
\item $\frac{x}{(logx)^2 + 1} + C$
\end{enumerate} 

\item $\int_{}\frac{dx}{\cos x + \sqrt{3}\sin x}$ equals
\begin{enumerate}
\item $log \tan(\frac{x}{2} + \frac{\pi}{12}) + C$
\item $log \tan(\frac{x}{2} - \frac{\pi}{12}) + C$
\item $\frac{1}{2}log \tan(\frac{x}{2} + \frac{\pi}{12}) + C$
\item $\frac{1}{2}log \tan(\frac{x}{2} - \frac{\pi}{12}) + C$
\end{enumerate}

\item The value of $\sqrt{2}\int_{}\frac{\sin xdx}{\sin(x - \frac{\pi}{4})}$ is
\begin{enumerate}
\item $x+log|\cos(x - \frac{\pi}{4})| + C$
\item $x-log|\sin(x - \frac{\pi}{4})| + C$
\item $x+log|\sin(x - \frac{\pi}{4})| + C$
\item $x-log|\cos(x - \frac{\pi}{4})| + C$
\end{enumerate}

\item If the $\int_{}\frac{5\tan x}{\tan x - 2}dx = x + aln|\sin x - 2\cos x| + k$, then a is equal to:
\begin{enumerate}
\item -1
\item -2
\item 1
\item 2
\end{enumerate}

\item If $\int_{}f(x)dx = \psi(x)$, then $\int_{}x^5f(x^3)dx$ is equal to
\begin{enumerate}
\item $\frac{1}{3}[x^3\psi(x^3) - \int_{}x^2\psi(x^3)dx] + C$
\item $\frac{1}{3}[x^3\psi(x^3) - 3\int_{}x^3\psi(x^3)dx] + C$
\item $\frac{1}{3}[x^3\psi(x^3) - \int_{}x^2\psi(x^3)dx] + C$
\item $\frac{1}{3}[x^3\psi(x^3) - \int_{}x^3\psi(x^3)dx] + C$
\end{enumerate}

\item The integral $\int_{}(1 + x -\frac{1}{x})e^{x + \frac{1}{x}}dx$ is equal to
\begin{enumerate}
\item $(x + 1)e^{x + \frac{1}{x}} + C$
\item $(-x)e^{x + \frac{1}{x}} + C$
\item $(x - 1)e^{x + \frac{1}{x}} + C$
\item $(x)e^{x + \frac{1}{x}} + C$
\end{enumerate}

\item The integral $\int_{}\frac{dx}{x^2(x^4 + 1)^3/4}$ equals
\begin{enumerate}
\item $-(x^4 + 1)^{1/4} + C$
\item $-(\frac{x^4 + 1}{x^4})^{1/4} + C$
\item $(\frac{x^4 + 1}{x^4}) + C$
\item $(x^4 + 1)^{1/4} + C$
\end{enumerate}

\item The integral $\int_{}\frac{2x^{12} + 5x^9}{(x^5 + x^3 + 1)^3}dx$ is equal to:
\begin{enumerate}
\item $\frac{x^5}{2(x^5 + x^3 + 1)^2} + C$
\item $\frac{-x^{10}}{2(x^5 + x^3 + 1)^2} + C$
\item $\frac{-x^5}{(x^5 + x^3 + 1)^2} + C$
\item $\frac{x^{10}}{2(x^5 + x^3 + 1)^2} + C$
\end{enumerate}

\item Let $I_n = \int_{}\tan xdx,(n > 1)$. $I_4 + I_6 = a\tan^{5}x + bx^5 + C$, where C is constant of integration, then the ordered pair (a, b) is equal to:
\begin{enumerate}
\item $(-\frac{1}{5}, 0)$
\item $(-\frac{1}{5}, 1)$
\item $(\frac{1}{5}, 0)$
\item $(\frac{1}{5}, -1)$
\end{enumerate}

\item The integral
\begin{align*}
\frac{\sin^{2}x \cos^{2}x}{(\sin^{5}x + \cos^{3}x\sin^{3}x\cos^{2} + \cos^{5}x)^2}dx
\end{align*}
is equal to
\begin{enumerate}
\item $\frac{-1}{3(1 + \tan^{3}x)} + C$
\item $\frac{1}{1 + \cot^{3}x} + C$
\item $\frac{-1}{1 + \cot^{3}x} + C$
\item $\frac{1}{3(1 + \tan^{3}x)} + C$
\end{enumerate}

\item For $x^2 \neq n\pi + 1, n \in N$(the set of natural numbers), the integral
\begin{align*}
\int_{}x \sqrt{\frac{2\sin(x^2 - 1) - \sin 2(x^2 - 1)}{2\sin(x^2 - 1) + \sin 2(x^2 - 1)}}dx
\end{align*}
is equal to
\begin{enumerate}
\item $log_e|\frac{1}{2}\sec^{2}(x^2 - 1)| + C$
\item $\frac{1}{2}log_e|\sec(x^2 - 1)| + C$
\item $\frac{1}{2}log_e|\sec^{2}(\frac{x^2 - 1}{2})| + C$
\item $log_e|\sec(\frac{x^2 - 1}{2})| + C$
\end{enumerate}

\item The integral $\int_{}\sec^{2/3}x\cosec^{4/3}xdx$ is equal to
\begin{enumerate}
\item $-3\tan^{-1/3}x + C$
\item $-\frac{3}{4}\tan^{-4/3}x + C$
\item $-3\cot^{-1/3}x + C$
\item $3\tan^{-1/3}x + C$
\end{enumerate}
\end{enumerate}








 
\section{Differential Equations}
\renewcommand{\theequation}{\theenumi}
\begin{enumerate}[label=\arabic*.,ref=\thesubsection.\theenumi]
\numberwithin{equation}{enumi}

\item A solution of the following diffrential equation is
\begin{align*}
\left(\frac{dy}{dx}\right)^{2} - x\frac{dy}{dx} + y = 0
\end{align*}
\begin{enumerate}
\item y = 2
\item y = 2x
\item y = 2x - 4
\item $y = 2x^2 - 4$
\end{enumerate}

\item If $x^2 + y^2 = 1$, then
\begin{enumerate}
\item $yy'' - 2(y')^{2} + 1 = 0$
\item $yy'' + (y')^{2} + 1 = 0$
\item $yy'' + (y')^{2} - 1 = 0$
\item $yy'' + 2(y')^{2} + 1 = 0$
\end{enumerate}

\item If $y(t)$ is a solution of
\begin{align*}
(1 + t)\frac{dy}{dt} - ty = 1, y(0) = -1
\end{align*}
then $y(1)$ is equal to
\begin{enumerate}
\item -1/2
\item e + 1/2
\item e - 1/2
\item 1/2
\end{enumerate}

\item If $y = f(x)$ and $\frac{2 + \sin x}{y + 1}\left(\frac{dy}{dx}\right) = \cos x$
y(0) = -1, then $y\left(\frac{\pi}{2}\right)$ equals
\begin{enumerate}
\item 1/3
\item 2/3
\item -1/3
\item 1
\end{enumerate}

\item If $y = f(x)$ and it is followsthe relation $x\cos y + y\cos x = \pi$ then $y''(0)$ = 
\begin{enumerate}
\item 1
\item -1
\item $\pi - 1$
\item $\pi$
\end{enumerate}

\item The solution of primitive integral equation $(x^2 + y^2)dy = xydx$ is $y = y(x)$. If $y(1) = 1$ and $(x_0 = e)$, then $x_0$ is equal to
\begin{enumerate}
\item $\sqrt{2(e^2 - 1)}$
\item $\sqrt{2(e^2 + 1)}$
\item $\sqrt{3}e$
\item $\sqrt{\frac{e^2 + 1}{2}}$
\end{enumerate}

\item For the primitive integral equation $ydx + y^2dy = x$; $x \in R$, $y > 0$, $y = y(x)$, $y(1) = 1$, then $y(-3)$ is
\begin{enumerate}
\item 3
\item 2
\item 1
\item 5
\end{enumerate}

\item The differential equation 
\begin{align*}
\frac{dy}{dx} = \frac{\sqrt{1 - y^2}}{y}
\end{align*}
determines a family of circles with
\begin{enumerate}
\item variable radii and a fixed centre at (0, 1)
\item variable radii and a fixed centre at (0, -1)
\item fixed radius 1 and variables centres along the x-axis
\item fixed radius 1 and variables centres along the y-axis
\end{enumerate}

\item The function $y = f(x)$ is the solution of the differential equation
\begin{align*}
\frac{dy}{dx} = \frac{xy}{x^2 - 1} = \frac{x^4 + 2x}{\sqrt{1 - x^2}}
\end{align*}
in (1, -1) satisfying f(0) = 0. Then 
\begin{align*}
\int_{-\sqrt{3}/2}^{\sqrt{3}/2}f(x)dx
\end{align*}
is
\begin{enumerate}
\item $\frac{\pi}{3} - \frac{\sqrt{3}}{2}$
\item $\frac{\pi}{3} - \frac{\sqrt{3}}{4}$
\item $\frac{\pi}{6} - \frac{\sqrt{3}}{4}$
\item $\frac{\pi}{6} - \frac{\sqrt{3}}{2}$ 
\end{enumerate}

\item If $y = y(x)$ satisfies the differential equation
\begin{align*}
8\sqrt{x}\left(\sqrt{9 + \sqrt{x}}\right)dy = \left(\sqrt{4 + \sqrt{9 + \sqrt{x}}}\right)^{-1}dx, x > 0
\end{align*}
and $y(0) = \sqrt{7}$, then $y(256)$ = 
\begin{enumerate}
\item 3
\item 9
\item 16
\item 80
\end{enumerate}

\textbf{MCQs with One or More than One Correct}

\item The order of the differential equation whose general solution is given by
\begin{align*}
y = (C_1 + C_2)\cos(x + C_3) - C_4e^x + C_5
\end{align*}
where $C_1, C_2, C_3, C_4, C_5$ are arbitrary constants, is
\begin{enumerate}
\item 5
\item 4
\item 3
\item 2
\end{enumerate}

\item The differential equation representing the family of curves
\begin{align*}
y = 2c\left(x + \sqrt{x}\right)
\end{align*}
where c is a positive parameter, is of
\begin{enumerate}
\item order 1
\item order 2
\item degree 3
\item degree 4
\end{enumerate}

\item A curve $y = f(x)$ passes through (1, 1) and at P(x, y), tangent cuts the x-axis and y-axis at A and B respectively such that BP : AP = 3 : 1. then
\begin{enumerate}
\item equation of curve is $xy' - 3y = 0$
\item normal at (1, 1) is $x + 3y = 4$
\item curve passes through (2, 1/8)
\item equation of curve is $xy' + 3y = 0$
\end{enumerate}

\item If $y(x)$ satisfies the differential equation 
\begin{align*}
y' = y\tan x = 2x\sec x 
\end{align*}
and $y(0) = 0$, then
\begin{enumerate}
\item $y\left(\frac{\pi}{4}\right) = \frac{\pi^2}{8\sqrt{2}}$
\item $y'\left(\frac{\pi}{4}\right) = \frac{\pi^2}{18}$
\item $y\left(\frac{\pi}{3}\right) = \frac{\pi^2}{9}$
\item $y'\left(\frac{\pi}{3}\right) =  y'\left(\frac{4\pi}{3}\right)+ \frac{2\pi^2}{3\sqrt{3}}$
\end{enumerate}

\item A curve passes through the point $\left(1, \frac{\pi}{6}\right)$. Let the slope of the curve at each point (x, y) be
\begin{align*}
\frac{y}{x} + \sec\left(\frac{y}{x}\right), x > 0
\end{align*}
Then the equation of the curve is
\begin{enumerate}
\item $\sin\left(\frac{y}{x}\right) = log x + \frac{1}{2}$
\item $\cosec\left(\frac{y}{x}\right) = log x + \frac{1}{2}$
\item $\sec\left(\frac{2y}{x}\right) = log x + \frac{1}{2}$
\item $\cos\left(\frac{2y}{x}\right) = log x + \frac{1}{2}$
\end{enumerate}

\item Let $f(x)$ be a solution of the differential equation
\begin{align*}
(1 + e^x)y' + ye^x = 1
\end{align*}
If $y(0) = 2$, then which of the following statementis (are) true?
\begin{enumerate}
\item $y(-4) = 0$
\item $y(-2) = 0$
\item $y(x)$ has a critical point in the interval (-1, 0)
\item $y(x)$ has no critical point in the interval (-1, 0)
\end{enumerate}

\item Consider the family of all circles whose centres lie on the straight line y = x. If this family of circle is represented by the differential equation $Py'' + Qy' + 1 = 0$, where P, Q are functions of x, y and $y'$, then which of the following statement is(are) true?
\begin{enumerate}
\item P = y + x
\item P = y - x
\item $P + Q = 1 - x + y + y' + (y')^{2}$
\item $P - Q = x + y - y' - (y')^{2}$
\end{enumerate}

\item Let $f: (0, \infty) \to R$ be a differentiable function such that
\begin{align*}
f'(x) = 2 - \frac{f(x)}{x}
\end{align*}
for all $x \in (0, \infty)$ and $f(1) \neq 1$. Then 
\begin{enumerate}
\item $\lim_{x \to 0^{+}}f'\left(\frac{1}{x}\right) = 1$
\item $\lim_{x \to 0^{+}}xf'\left(\frac{1}{x}\right) = 2$
\item $\lim_{x \to 0^{+}}x^2f'(x) = 0$
\item $|f(x)| \leq 2$ for all $x \in (0, 2)$
\end{enumerate}

\item A solution curve of the following differential equation
\begin{align*}
(x^2 + xy + 4x + 2y + 4)\frac{dy}{dx} - y^2 = 0, x > 0
\end{align*}
passes through the point (1, 3). Then the solution curve
\begin{enumerate}
\item intersects y = x + 2 exactly at one point
\item intersects y = x + 2 exactly at two points
\item intersects $y = (x + 2)^2$
\item does NOT intersect $y = (x + 2)^2$
\end{enumerate}

\item Let $f: [0, \infty) \to R$ be a continuous function such that
\begin{align*}
f(x) = 1 - 2x + \int_{0}^{x}e^{x - t}f(t)dt
\end{align*}
for all $x \in [0, \infty)$. Then which of the following statement(s) is(are) true?
\begin{enumerate}
\item The curve $y = f(x)$ passes through the point (1, 2)
\item The curve $y = f(x)$ passes through the point (2, -1)
\item The area of the region
\begin{align*}
\left\lbrace(x, y) \in [0, 1] \times R: f(x) \leq y \leq \sqrt{1 - x^2}\right\rbrace
\end{align*}
is $\frac{\pi - 2}{4}$
\item The area of the region
\begin{align*}
\left\lbrace(x, y) \in [0, 1] \times R: f(x) \leq y \leq \sqrt{1 - x^2}\right\rbrace
\end{align*}
is $\frac{\pi - 1}{4}$
\end{enumerate}

\item Let $\Gamma$ denotes a curve $y = f(x)$ which is in the first quadrant and let the point (1, 0) lie on it. Let the tangent to $\Gamma$ at a point P intersects the y-axis at $Y_p$. If $PY_p$ has length 1 for each point P on $\Gamma$, then which of the following option(s) is(are) correct?
\begin{enumerate}
\item $y = -log_e\left(\frac{1 + \sqrt{1 - x^2}}{x}\right) + \sqrt{1 - x^2}$
\item $xy' - \sqrt{1 - x^2} = 0$
\item $y = log_e\left(\frac{1 + \sqrt{1 - x^2}}{x}\right) - \sqrt{1 - x^2}$
\item $xy' + \sqrt{1 - x^2} = 0$
\end{enumerate} 

\textbf{Subjective Problems}

\item If $(a + bx)e^{y/x} = x$, then prove that
\begin{align*}
x^3\frac{d^{2}y}{dx^{2}} = \left(x\frac{dy}{dx} - y\right)
\end{align*}

\item A normal is drawn at apoint P(x, y) of a curve. It meets the x-axis at Q. If PQ is of constant length k, then show that the differential equation describing such curve is
\begin{align*}
y \frac{dy}{dx} = \pm \sqrt{k^2 - y^2}
\end{align*}
Find the equation of such curve passing through (0, k).

\item Let $y = f(x)$ be a curve passing through (1, 1) such that the triangle formed by the coordinate axes and the tangent at any point of the curve lies in the first quadrant and has area 2. Form the differential equation and determine all such possible curves.

\item Determine the equation of the curve passing through the origin, in the form $y = f(x)$, which satisfies the differential equation
\begin{align*}
\frac{dy}{dx} = \sin(10x+ 6y)
\end{align*}

\item Let $u(x)$ and $v(x)$ satisfy the differential equation
\begin{align*}
\frac{du}{dx} + p(x)u = f(x)
\end{align*}
\begin{align*}
\frac{dv}{dx} + p(x)v = g(x)
\end{align*}
where $p(x)$, $f(x)$ and $g(x)$ are continuous functions. If $u(x_1) > v(x_1)$ for some $x_1$ and $f(x) > g(x)$ for all $x > x_1$, prove that any point (x, y) where $x > x_1$ does not satisfy the equation $y = u(x)$ and $y = v(x)$ 


\item A curve passing through the point (1, 1) has the property that the perpendicular distance of the origin from the normal at any point P of the curve is equal to the distance of P from the x-axis. Determine the equation of the curve.

\item A country has a food deficit of 10 percentage. Its population grows continuously at a rate of 3 percentage per year. Its annual food production every year is 4 percentage more than that of the last year. Assuming that the average food requirement per person remains constant, prove that the country will become self-sufficient in food after n years, where n is the smallest integer bigger than or equal to $\frac{ln 10 - ln 9}{ln(1.04) - 0.03}$.

\item A hemispherical tank of radius 2 metres is initially full of water and has an outlet of $12cm^{2}$ cross-sectional area at the bottom. The outlet is opened at some instant. The flow through the outlet is according to the law 
\begin{align*}
v(t) = 0.6\sqrt{2gh(t)}
\end{align*}
where $v(t)$ and $h(t)$ are respectively the velocity of the flow through the outlet and height of the water level above the outlet at time t, and g is the accelration due to gravity. Find the time it takes to empty the tank.

\item A right circular cone with radius R and height H contains a liquid which evaporates at a rate proportional to its surface area in contact with air(proportional constant = $k > 0$). Find the time after which the cone is empty.

\item A curve $'C'$ passes through (2, 0) and the slope at (x, y) as 
\begin{align*}
\frac{(x + 1)^2 + (y - 3)}{x + 1}
\end{align*}
Find the equation of the curve. Find the area bounded by the curve and x-axis in fourth quadrant.

\item If length of tangent at any point on the curve $y = f(x)$ intercepted berween the point and the x-axis os of length 1. Find the equation of the curve.

\textbf{Section-B}

\item The order and degree of the differential equation
\begin{align*}
\left(1 + 3\frac{dy}{dx}\right)^{2/3} = 4\frac{d^{3}y}{dx^{3}}
\end{align*}
 are
\begin{enumerate}
\item $\left(1, \frac{2}{3}\right)$
\item (3, 1)
\item (3, 3)
\item (1, 2) 
\end{enumerate}

\item The solution of the equation
\begin{align*}
\frac{d^{2}y}{dx^{2}} = 2e^{-x}
\end{align*}
\begin{enumerate}
\item $\frac{e^{-2x}}{4}$
\item $\frac{e^{-2x}}{4} + cx + d$
\item $\frac{1}{4}e^{-2x} + cx^{2} + d$
\item $\frac{1}{4}e^{-4x} + cx + d$
\end{enumerate}

\item The degree and order of the differential equation of the family of all parabolas whose axis is x-axis, are respectively.
\begin{enumerate}
\item 2, 3
\item 2, 1
\item 1, 2
\item 3, 2
\end{enumerate}

\item The solution of the differential equation is
\begin{align*}
(1 + y^{2}) + (x - e^{\tan^{-1}y})\frac{dy}{dx} = 0
\end{align*}
\begin{enumerate}
\item $xe^{2\tan^{-1}y} = e^{\tan^{-1}y} + k$
\item $(x - 2) = ke^{2\tan^{-1}y}$
\item $2xe^{2\tan^{-1}y} = e^{2\tan^{-1}y} + k$
\item $xe^{\tan^{-1}y} = e^{\tan^{-1}y} + k$
\end{enumerate}

\item The differential equation for the family of circle
\begin{align}
x^2 + y^2 -2ay = 0
\end{align}
where a is an arbitraty constant is
\begin{enumerate}
\item $(x^2 + y^2)y' = 2xy$
\item $2(x^2 + y^2)y' = xy$
\item $(x^2 - y^2)y' = 2xy$
\item $2(x^2 - y^2)y' = xy$
\end{enumerate}

\item Solution of the differential equation is
\begin{align*}
ydx + (x + x^2y)dy = 0
\end{align*}
\begin{enumerate}
\item $logy = Cx$
\item $-\frac{1}{xy} + log y = C$
\item $\frac{1}{xy} + logy = C$
\item $-\frac{1}{xy} = C$
\end{enumerate}

\item The differential equation representing the family of curves
\begin{align*}
y^2 = 2c(x + \sqrt{c}), c > 0 
\end{align*}
where c is a parameter, is of order and degree follows:
\begin{enumerate}
\item order 1, degree 2
\item order 1, degree 1
\item order 1, degree 3
\item order 2, degree 2
\end{enumerate}

\item If 
\begin{align*}
x\frac{dy}{dx} = y(logy - logx + 1)
\end{align*}
then the solution of the differential equation is
\begin{enumerate}
\item $ylog\left(\frac{x}{y}\right) = cx$
\item $xlog\left(\frac{y}{x}\right) = cy$
\item $log\left(\frac{y}{x}\right) = cx$
\item $log\left(\frac{x}{y}\right) = cy$
\end{enumerate} 

\item The diffrential equation whose solution is
\begin{align}
Ax^2 + By^2 = 1
\end{align}
where A and B are arbitrary constants is of
\begin{enumerate}
\item second order and second degree
\item first order and second degree
\item first order and first degree
\item second order and first degree
\end{enumerate}

\item The differential equation of all circles passing through the origin and having their centres on the x-axis is
\begin{enumerate}
\item $y^2 = x^2 + 2xy\frac{dy}{dx}$
\item $y^2 = x^2 - 2xy\frac{dy}{dx}$
\item $x^2 = y^2 + xy\frac{dy}{dx}$
\item $x^2 = y^2 + 3xy\frac{dy}{dx}$
\end{enumerate}

\item The solution of the diffrential equation
\begin{align*}
\frac{dy}{dx} = \frac{x + y}{x}
\end{align*}
satisfying the condition $y(1) = 1$ is
\begin{enumerate}
\item $y = lnx + x$
\item $y = xlnx + x^2$
\item $y = xe^{x - 1}$
\item $y = xlnx + x$
\end{enumerate}

\item The differential equation which represents the family of curves $y = c_1e^{c_2x}$ where $c_1$ and $c_2$ are arbitrary constants, is
\begin{enumerate}
\item $y'' = y'y$
\item $yy'' = y'$
\item $yy'' = (y')^2$
\item $y' = y^2$
\end{enumerate} 

\item Solution of the differential equation is
\begin{align*}
\cos xdy = y(\sin x - y)dx, 0 < x < \frac{\pi}{2}
\end{align*}
\begin{enumerate}
\item $y\sec x = \tan x + c$
\item $y\tan x = \sec x + c$
\item $\tan x = (\sec x + c)y$
\item $\sec x = (\tan x + c)y$
\end{enumerate}

\item If 
\begin{align*}
\frac{dy}{dx} = y + 3 > 0, y(0) = 2
\end{align*}
then $yln(2)$ is equal to
\begin{enumerate}
\item 5
\item 13
\item -2
\item 7
\end{enumerate}

\item Let I be the purchase value of an equipment ans $V(t)$ be the value after it has been used for t years. The value 
$V(t)$ depreciates at a rate of given by differential equation 
\begin{align*}
\frac{dV(t)}{dt} = -k(T - t), k > 0
\end{align*}
where k is constant and T is the total life in years of the eqipment. Then the scrap value $V(T)$ of the equipment is
\begin{enumerate}
\item $I - \frac{kT^2}{2}$
\item $I - \frac{k(T - t)^2}{2}$
\item $e^{-kT}$
\item $T^2 - \frac{1}{k}$
\end{enumerate} 

\item The population $p(t)$ at time t of a certain mouse species satisfies the differential equation
\begin{align*}
\frac{dp(t)}{dt} = 0.5p(t) - 450
\end{align*}
If $p(0) = 850$, then the time at which the population becomes zero is
\begin{enumerate}
\item $2ln18$
\item $ln9$
\item $\frac{1}{2}ln18$
\item $ln18$
\end{enumerate}

\item At present, afirm is manufacturing 2000 items. It is estimated that the rate of chnage of production P w.r.t. additional number of workers x is given by
\begin{align*}
\frac{dP}{dx} = 100  - 12\sqrt{x}
\end{align*}
If the firm employs 25 more workers, then the new value of production of items is
\begin{enumerate}
\item 2500
\item 3000
\item 3500
\item 4500
\end{enumerate}

\item Let the population of rabbits surviving at time t be governed by the differential equation
\begin{align*}
\frac{dp(t)}{dt} = \frac{1}{2}p(t) - 200.
\end{align*}
If $p(0) = 100$, then $p(t)$ equals
\begin{enumerate}
\item $600 - 500e^{t/2}$
\item $400 - 300e^{-t/2}$
\item $400 - 300e^{t/2}$
\item $300 - 200e^{-t/2}$
\end{enumerate}

\item Let $y(x)$ be the solution of the differential equation
\begin{align*}
(xlogx)\frac{dy}{dx} + y = 2x logx, (x \geq 1)
\end{align*}
Then $y(e)$ is equal to
\begin{enumerate}
\item 2
\item 2e
\item e
\item 0
\end{enumerate}

\item If a curve $y = f(x)$ passes through the point (1, -1) and satisfies the differential equation 
\begin{align*}
y(1 + xy)dx = xdy
\end{align*}
then, $f\left(-\frac{1}{2}\right)$ is equal to
\begin{enumerate}
\item $\frac{2}{5}$
\item $\frac{4}{5}$
\item $-\frac{2}{5}$
\item $-\frac{4}{5}$
\end{enumerate}

\item If
\begin{align*}
(2 + \sin x)\frac{dy}{dx} + (y + 1)\cos x = 0, y(0) = 1
\end{align*}
then $y\left(\frac{\pi}{2}\right)$ is equal to
\begin{enumerate}
\item 4/3
\item 1/3
\item -2/3
\item -1/3
\end{enumerate}

\item Let $y - y(x)$ be the solution of the differential equation
\begin{align*}
\sin x\frac{dy}{dx} + y\cos x = 4x, x \in (0, \pi)
\end{align*}
If $y\left(\frac{\pi}{2}\right) = 0$, then $y\left(\frac{\pi}{6}\right)$ is equal to
\begin{enumerate}
\item $\frac{-8}{9\sqrt{3}}\pi^{2}$
\item $\frac{-8}{9}\pi^{2}$
\item $\frac{-4}{9}\pi^{2}$
\item $\frac{4}{9\sqrt{3}}\pi^{2}$
\end{enumerate}

\item If $y = y(x)$ is the solution of the differential equation 
\begin{align*}
x\frac{dy}{dx} + 2y = x^2
\end{align*}
satisfying $y(a) = 1$, then $y\left(\frac{1}{2}\right)$ is equal to
\begin{enumerate}
\item $\frac{7}{64}$
\item $\frac{1}{4}$
\item $\frac{49}{16}$
\item $\frac{13}{16}$
\end{enumerate}

\item The solution of the differential equation 
\begin{align*}
x\frac{dy}{dx} + 2y = x^2 (x \neq 0)
\end{align*}
with $y(1) = 1$ is
\begin{enumerate}
\item $y = \frac{4}{5}x^3 + \frac{1}{5x^2}$
\item $y = \frac{1}{5}x^3 + \frac{1}{5x^2}$
\item $y = \frac{1}{4}x^2 + \frac{3}{4x^2}$
\item $y = \frac{3}{4}x^2 + \frac{1}{4x^2}$
\end{enumerate}

\textbf{Assertion and Reason Type Questions}

\item Let a solution $y = y(x)$ of the differential equation
\begin{align*}
x\sqrt{x^2 - 1}dy - y\sqrt{y^2 - 1}dx = 0
\end{align*}
satisfy $y(2) = \frac{2}{\sqrt{3}}$.

\textbf{Statement-1:}

\begin{align*}
y(x) = \sec\left(\sec^{-1}x - \frac{\pi}{6}\right)
\end{align*}

\textbf{Statement-2:} 

\begin{align*}
y(x): \frac{1}{y} = \frac{2\sqrt{3}}{x} - \sqrt{1 - \frac{1}{x^2}}
\end{align*}
\begin{enumerate}
\item Statement-1 is true, Statement-2 is true, Statement-2 is a correct explanation for Statement-2
\item Statement-1 is true, Statement-2 is true, Statement-2 is not a correct explanation for Statement-2
\item Statement-1 is true, Statement-2 is false
\item Statement-1 is false, Statement-2 is true
\end{enumerate}

\textbf{Integer Value Correct Type}

\item Let
\begin{align*}
y'(x) + y(x)g'(x) = g(x), g'(x), y(0) = 0, x \in R 
\end{align*}
where $f'(x)$ denotes $\frac{df(x)}{dx}$ and $g(x)$ is a given non-constant differentiable function on R with $g(0) = g(2) = 0$. Then the value of $g(2)$ is

\item Let $f: R \to R$ be a differentiable function with $f(0) = 0$. If $y = f(x)$ satisfies the differential equation
\begin{align*}
\frac{dy}{dx} = 2(2 + 5y)(5y - 2)
\end{align*}  
then the value of $\lim_{x \to \infty}f(x)$ is..........

\item Let $f: R \to R$ be a differentiable function with $f(0) = 1$ and satisfies the equation 
\begin{align*}
f(x + y) = f(x)f'(y) + f'(x)f(y), y \in R 
\end{align*}
Then, the value of $log_e(f(4))$ is..............

\clearpage

\textbf{Match the Following Questions:}

\item Match the following
\begin{table}[ht!]
\centering
\begin{tabular}{c c} 
 \textbf{Column I} & \textbf{Column II}\\ [0.5ex] 
 (A) Interval contained in the\\ domain of
     non-zero solutions\\ of the differential equation\\
     $(x - 3)^2 + y' + y = 0$                                     &(p) $\left(-\frac{\pi}{2}, \frac{\pi}{2}\right)$\\ 
 (B) Interval containing the\\ value of the integral\\
     $\int_{1}^{5}(x-1)(x-2)(x-3)(x-4)$\\ $(x-5)dx$                    &(q) $\left(0, \frac{\pi}{2}\right)$\\
 (C) Interval in which at least\\ one of the points of
     local\\ maximum of $\cos^{2} + \sin x$ lies                    &(r) $\left(\frac{\pi}{8}, \frac{5\pi}{4}\right)$\\                                                                     
 (D) The Interval in which\\ $\tan^{-1}(\sin x + \cos x)$ is            &(s) $\left(0, \frac{\pi}{8}\right)$\\[1ex] 
\end{tabular}
\end{table}
\end{enumerate}

 
\section{Application of Derivatives}
\renewcommand{\theequation}{\theenumi}
\begin{enumerate}[label=\arabic*.,ref=\thesubsection.\theenumi]
\numberwithin{equation}{enumi}

\item The largest of $\cos(ln \theta)$ and $\ln(\cos \theta)$ if\\ $e^{\frac{\pi}{2}} < \theta < \frac{\pi}{2}$ is ..........

\item The function 
\begin{align*} 
y = 2x^2 - ln|x|
\end{align*}
is monotonically increasing for values of $x( \neq 0)$ satisfying the inequalities ...... and monotonically decreasing for values of x satisfying the inequalities .....

\item The set of all x for which ln(1 + x) $ \leq $ x is equal to ......

\item Let P be a variable point on the ellipse
\begin{align} 
\frac{x^2}{a^2} + \frac{y^2}{b^2} = 1
\end{align} 
with the foci $F_1$ and $F_2$. If A is the area of the triangle P$F_1$ $F_2$ then the maximum value of A is......

\item Let C be the curve 
\begin{align}
y^2 - 3xy + 2 = 0
\end{align}
If H is the set of points on the curve C where the tangent is horizontal and V is the set of the point on the curve C where the tangent is vertical then H = ....... and V = ....... 

\textbf{True/False:}

\item If x - r is the factor of the polynomial 
\begin{align*}
f(x)= a_nx^4 + .....+ a_0,
\end{align*} 
repeated m times ($1 < m < n$), then r is a root of $f'(x) = 0$ repeated m times.

\item For $0 < a < x$, the minimum value of the function $log_a x + log_x$ a is 2.

\textbf{MCQs with One Correct Answer:}

\item If a + b + c = 0,Then the quadratic equation 
\begin{align}
3ax^2 + 2bx + c = 0
\end{align} 
has 
\begin{enumerate}
\item at least one root in [0 , 1]
\item one root in [2,3] and the other in [-2,-1]
\item imaginary roots
\item none of these
\end{enumerate}

\item AB is a diameter of a circle and C is any point on the circumference of the circle.Then 
\begin{enumerate}
\item The area of $\Delta$ABC is maximum when it is isosceles
\item The area of $\Delta$ABC is maximum when it is isosceles
\item The perimeter of $\Delta$ABC is maximum when it is isosceles
\item none of these 
\end{enumerate}

\item The normal to the curve 
\begin{align*} 
x = a(\cos \theta + \theta \sin \theta)
\end{align*}
\begin{align*} 
y = a(\sin \theta - \theta \cos \theta) 
\end{align*} at any point $'\theta'$ is such that.
\begin{enumerate}
\item it makes a constant angle with x-axis
\item it passes through the origin 
\item it is at a constant distance from the origin 
\item none of these
\end{enumerate}

\item If 
\begin{align}
y = a lnx + bx^2 + x
\end{align} 
has its extremum values at x = -1 and x = 2 then 
\begin{enumerate}
\item a = 2, b = -1
\item a = 2, b = $\frac{-1}{2}$
\item a = -2, b = $\frac{1}{2}$
\item none of these
\end{enumerate}

\item Which one of the following curves cut the parabola 
\begin{align} 
y^2 = 4ax
\end{align} 
at right angles?
\begin{enumerate}
\item $x^2+ y^2 = a^2$
\item $y = e^{\frac{x}{2a}}$
\item y = ax
\item $x^2 = 4ay$
\end{enumerate}

\item The function defined by 
\begin{align*} 
f(x) = (x + 2)e^{-x} 
\end{align*} 
is
\begin{enumerate}
\item decreasing for all x
\item decreasing in ($-\infty$, -1) and increasing in (-1, $\infty$) 
\item increasing for all x
\item decreasing in (-1, $\infty$) and increasing in ($-\infty$, -1) 
\end{enumerate}

\item The function 
\begin{align*}
f(x) = \frac{ln(\pi + x)}{ln(e + x)}
\end{align*}
\begin{enumerate}
\item increasing on $(0, \infty,)$ 
\item decreasing on $(0, \infty)$
\item increasing on $(0, \frac{\pi}{e})$, decreasing on $(\frac{\pi}{e}, \infty)$
\item decreasing on $(0, \frac{\pi}{e}$), increasing on $(\frac{\pi}{e}, \infty)$
\end{enumerate}

\item On the interval [0, 1] the function $x^{25}(1 - x)^{75}$ takes its maximum value at the point
\begin{enumerate}
\item 0
\item $\frac{1}{4}$
\item $\frac{1}{2}$
\item $\frac{1}{3}$
\end{enumerate}

\item The slope of the tangent to a curve  
$y = f(x)$ at $[x, f(x)]$ is 2x + 1. If the curve passes through the point (1, 2), then the area bounded by the curve, the x-axis and line x = 1 is
\begin{enumerate}
\item $\frac{5}{6}$
\item $\frac{6}{5}$
\item $\frac{1}{6}$
\item 6
\end{enumerate}

\item If 
\begin{align*} 
f(x) = \frac{x}{\sin x} 
and 
\end{align*} 
\begin{align*}
g(x) = \frac{x}{\tan x}
\end{align*} 
where $0 < x \leq 1$, then in this interval
\begin{enumerate}
\item both $f(x)$ and $g(x)$ are increasing functions
\item both $f(x)$ and $g(x)$ are decreasing functions
\item $f(x)$ is an increasing function
\item $g(x)$ is an increasing function
\end{enumerate}

\item The function 
\begin{align*} 
f(x) = sin^4 x + cos^4x 
\end{align*} 
increasing if
\begin{enumerate}
\item $0 < x < \frac{\pi}{8}$
\item $\frac{\pi}{4} < x < \frac{3\pi}{8}$
\item $\frac{3\pi}{8} < x < \frac{5\pi}{8}$
\item $\frac{5\pi}{8} < x < \frac{3\pi}{4}$
\end{enumerate}

\item Consider the following statements in S and R

\textbf{S:} Both sin x and cos x are decreasing functions in the interval $(\frac{\pi}{2}, \pi)$

\textbf{R:} The differentiable function decrease in an interval (a, b), then its derivative also decreases in (a,b).

Which of the following is true
\begin{enumerate}
\item Both S and R are wrong
\item Both S and R are correct but R is not the correct explanation for S
\item S is correct and R is correct explanation for S 
\item S is Correct and R is wrong
\end{enumerate}

\item Let 
\begin{align*}
f(x) = \int e^x (x - 1)(x - 2)dx
\end{align*}
Then f decrease in the interval
\begin{enumerate}
\item $(-\infty, -2)$
\item $(-2, -1)$
\item (1, 2)
\item $(2, \infty)$
\end{enumerate}

\item If the normal ti the curve $y = f(x)$ at the point (3, 4) makes an angle $\frac{3\pi}{4}$ with the positive 
x-axis, then $f'(3)$ =
\begin{enumerate}
\item -1
\item $\frac{-3}{4}$
\item $\frac{4}{3}$
\item 1
\end{enumerate}

\item Let
\begin{align*} 
f(x) =
\left\lbrace\begin{cases} 
      |x|  &  0 < |x| \leq 2 \\
      1  & x = 0  \\
\end{cases}\right\rbrace 
\end{align*}
then at x = 0, f has
\begin{enumerate}
\item a local maximum
\item no local maximum
\item a local minimum
\item no extremum
\end{enumerate}

\item For all $x \in (0, 1)$
\begin{enumerate}
\item $e^x < 1 + x$
\item $log_e(1 + x) < x$
\item $\sin x > x$
\item $log_ex > x$
\end{enumerate}

\item If $f(x) = xe^{x(1 - x)}$ then f(x) is
\begin{enumerate}
\item increasing on [$\frac{-1}{2}, 1$]
\item decreasing on R
\item increasing on R
\item decreasing on [$\frac{-1}{2}, 1$]
\end{enumerate}

\item the triangle formed by the tangents of the curve 
\begin{align*} 
f(x) = x^2 + bx - b
\end{align*}
at the point (1, 1) and the coordinate axes, lies in the first quadrant. If its area is 2, Then the value of b is
\begin{enumerate}
\item -1
\item 3
\item -3
\item 1
\end{enumerate}

\item Let 
\begin{align} 
f(x) = (1 + b^2)x^2 + 2bx + 1
\end{align} 
and let m(b) be the minimum value of f(x). As b varies, the range of m(b) is
\begin{enumerate}
\item $[0, 1]$
\item $(0, \frac{1}{2}]$
\item $[\frac{1}{2} ,1]$
\item $(0, 1]$
\end{enumerate}

\item The length of the longest interval in which the function $3\sin x - 4\sin^3x$ is increasing is
\begin{enumerate}
\item $\frac{\pi}{3}$
\item $\frac{\pi}{2}$
\item $\frac{3\pi}{2}$
\item ${\pi}$
\end{enumerate}

\item The points on the curve 
\begin{align} 
y^3 + 3x^2 = 12y
\end{align} 
where the tangent is vertical, is 
\begin{enumerate}
\item $(\pm\frac{4}{\sqrt{3}}, -2)$
\item $(\pm\sqrt{\frac{11}{3}}, 1)$
\item (0, 0)
\item $(\pm\frac{4}{\sqrt{3}}, 2)$
\end{enumerate}

\item In [0, 1] Lagranges Mean Value theorem is NOT applicable to
\begin{enumerate}
\item 
f(x) =
\[\begin{cases} 
      \frac{1}{2} - x &  x < \frac{1}{2} \\
      (\frac{1}{2} - x)^2 & x \geq \frac{1}{2} \\
   \end{cases}\] 
\item 
f(x) =
\[\begin{cases} 
      \frac{\sin x}{x} &  x \neq 0 \\
      1 & x = 0\\
   \end{cases}\] 
\item $f(x) = x|x|$
\item $f(x) = |x|$
\end{enumerate}

\item Tangent is drawn to ellipse
\begin{align*}
\frac{x^2}{27} + y^2 = 1 at (3\sqrt{3}\cos \theta, \sin \theta)(where \theta \in (0, \frac{\pi}{2}))
\end{align*}
Then the value of $\theta$ such that sum of intercepts on axes made by this tangent is minimum is
\begin{enumerate}
\item $\frac{\pi}{3}$
\item $\frac{\pi}{6}$
\item $\frac{\pi}{8}$
\item $\frac{\pi}{4}$
\end{enumerate}

\item If 
\begin{align*}
f(x) = x^3 + bx^2 + cx + d 
\end{align*} 
and $0 < b^2 < c$, then in ($-\infty, \infty$)
\begin{enumerate}
\item $f(x)$ is strictly increasing function
\item $f(x)$ has local maxima
\item $f(x)$ is a strictly decreasing function 
\item $f(x)$ is bounded
\end{enumerate}

\item If 
\begin{align*}
f(x) = x^{\alpha} log x
\end{align*}
and $f(0) = 0$ then the value of $\alpha$ for which Rolle's theorem can be applied in [0, 1] is
\begin{enumerate}
\item -2
\item -1
\item 0
\item $\frac{1}{2}$
\end{enumerate}

\item If $P(x)$ is a polynomial of degree less than or equal to 2 then S is the set of all such polynomials so that p(0) = 0, P(1) = 1 and  $P'(x) > 0 \forall x \in [0,1]$ then 
\begin{enumerate}
\item S = $\Phi$
\item $S = ax + (1 - a)x^2 \forall a\in(0, 2)$
\item $S = ax + (1 - a)x^2 \forall a\in(0, \infty)$
\item $S = ax + (1 - a)x^2 \forall a\in(0, 1)$
\end{enumerate}
 
\item The tangent to the curve $y = e^x$ drawn at the point $(c, e^c)$ intersects the line joining the points 
$(c - 1, e^{c - 1})$ and $(c + 1, e^{c + 1})$ 
\begin{enumerate}
\item on the left of x = c
\item on the right of x = c
\item at no point 
\item at all points
\end{enumerate}

\item Consider the two curves
\begin{align*}
C_1: y^2 = 4x
\end{align*}
\begin{align*}
C_2: x^2 + y^2 - 6x + 1 = 0 
\end{align*}
then,
\begin{enumerate}
\item $C_1$ and $C_2$ touch only each other at one point
\item $C_1$ and $C_2$ touch each other exactly at two point
\item $C_1$ and $C_2$ intersect at exactly two points
\item $C_1$ and $C_2$ neither intersect nor touch each other
\end{enumerate}

\item The total number of local minima and local maxima of the function 
f(x)=
\[ \begin{cases} 
      (2 + x)^3 &  -3 < x \leq -1 \\
      x^{\frac{2}{3}}& -1 < x < 2\\
   \end{cases}
\]
is 
\begin{enumerate}
\item 0
\item 1
\item 2
\item 3
\end{enumerate}

\item Let the function g: $(-\infty, \infty) \to (\frac{-\pi}{2}, \frac{\pi}{2})$ be given by 
\begin{align*}
g(u) = 2\tan^{-1}(e^u) - \frac{\pi}{2}
\end{align*}
 Then g is
\begin{enumerate}
\item even and it is strictly increasing in (0,$\infty$)
\item odd and is strictly decreasing in ($-\infty,\infty$)
\item odd and is strictly increasing in ($-\infty,\infty$)
\item neither even nor odd but it is strictly increasing in ($-\infty,\infty$)  
\end{enumerate}

\item The least value of $a \in R$ for which $4\alpha x^2 + \frac{1}{x} \leq 1$, for all $x > 0$ is
\begin{enumerate}
\item $\frac{1}{64}$
\item $\frac{1}{32}$
\item $\frac{1}{27}$
\item $\frac{1}{25}$
\end{enumerate}

\item If $f: R \to R$ is a twice differentiable function such that $f''(x) > 0$ for all $x \in R$, and 
$f(\frac{1}{2}) = \frac{1}{2}$, $f(1) = 1$, then
\begin{enumerate}
\item $f'(1) \leq 0$
\item $0 < f'(1) \leq \frac{1}{2}$
\item $\frac{1}{2} < f'(1) \leq 1$
\item $f'(1) > 1$
\end{enumerate}

\textbf{MCQ's with One or More than One Correct Answer:}

\item Let 
\begin{align*}
P(x) = a_0 + a_1x^2 + a_2x^4.....a_nx^{2n}
\end{align*}
 be a polynomial equation in real variable x with 
$0 < a_0 < a_1 < a_2 < ..... < a_n$.The function $P(x)$ has
\begin{enumerate}
\item neither a maximum nor a minimum 
\item only one maximum 
\item only one minimum 
\item only one minimum and one maximum
\item none of these
\end{enumerate}

\item If the line ax + by + c = 0 is a normal to the curve xy = 1 then 
\begin{enumerate}
\item $a > 0, b > 0$
\item $a > 0, b < 0$
\item $a < 0, b > 0$
\item $a < 0, b < 0$
\item none of these
\end{enumerate}

\item The smallest positive root of the equation, $\tan x - x = 0$ lies in
\begin{enumerate}
\item $(0, \frac{\pi}{2})$
\item $(\frac{\pi}{2}, \pi)$
\item $(\pi, \frac{3\pi}{2})$
\item $(\frac{3\pi}{2}, 2\pi)$
\end{enumerate}

\item Let $f$ and $g$ be the increasing and decreasing functions respectively from $[0, \infty)$ to $[0, \infty)$. Let $h(x) = f(g(x))$. If $h(0) = 0$, then $h(x) - h(1)$ is
\begin{enumerate}
\item always zero
\item always negative
\item always positive
\item strictly increasing
\item None of these
\end{enumerate}

\item If 
\begin{align*}
f(x)=
\left\lbrace
\begin{array}{ll} 
      3x^2 + 12x - 1 &  -1 \leq x \leq 2\\
      37 - x & 2 < x \leq 3\\
\end{array}
\right\rbrace
\end{align*}
then:
\begin{enumerate}
\item $f(x)$ is increasing on [-1, 2]
\item $f(x)$ is continues on [-1, 3]
\item $f(2)$ does not exist
\item $f(x)$ has the maximum value at x = 2
\end{enumerate}

\item If 
\begin{align*}
h(x) = f(x) - (f(x))^2 + (f(x))^3
\end{align*}
for every real number x. Then 
\begin{enumerate}
\item h is increasing whenever f is increasing
\item h is increasing whenever f is decreasing
\item h is decreasing whenever f is decreasing
\item nothing can be said in general
\end{enumerate}

\item If 
\begin{align*}
f(x) = \frac{x^2 - 1}{x^2 + 1}
\end{align*}
for every real number then x then the minimum value of f
\begin{enumerate}
\item does not exist because f is unbounded 
\item is not attained even though f is bounded 
\item is equal to 1
\item is equal to -1
\end{enumerate}

\item The number of values of x where function 
\begin{align*}
f(x) = \cos x + \cos(\sqrt{2}x)    
\end{align*}
attains its maximum is 
\begin{enumerate}
\item 0
\item 1
\item 2
\item infinite
\end{enumerate}

\item The function 
\begin{align*}
f(x) = \int_{-1}^{x} t(e^t - 1)(t - 1)(t - 2)^{3}(t - 3)^{5}dt
\end{align*}
has a local minimum at x =
\begin{enumerate}
\item 0
\item 1
\item 2
\item 3
\end{enumerate}

\item $f(x)$ is a cubic polynomial with $f(2) = 18$ and $f(1) = -1$. Also $f(x)$ has local maxima at x = -1 and $f'(x)$ has local minima at x = 0, then
\begin{enumerate}
\item the distance between (-1, 2) and $(af(a))$, where x = a is the point of local minima is $2\sqrt{2}$
\item $f(x)$ is increasing for $x \in [1, 2\sqrt{5}]$
\item $f(x)$ has local minima at x = 1
\item the value of $f(0) = 15$
\end{enumerate}

\item Let 
f(x)=
\[ \begin{cases} 
      e^x &  1 < x \leq 1\\
      2 - e^{x - 1} & 1 < x \leq 2\\
      x - e & 2 < x \leq 3
   \end{cases}
\] 
and 
$g(x) = \int_{0}^{x}f(t)dt$, $x \in [1, 3]$ then $g(x)$ has 
\begin{enumerate}
\item local maxima at x = 1 + ln2 and local minima at x = e
\item local maxima at x = 1 and local minima at x = 2
\item no local maxima
\item no local minima
\end{enumerate}

\item For the function 
\begin{align*} 
f(x) = x \cos \frac{1}{x}, x \geq 1,
\end{align*}
\begin{enumerate}
\item for atleast one x in the interval 
\begin{align*}
[1, \infty), f(x + 2) - f(x) < 2
\end{align*}
\item $\lim_{x \to \infty} f'(x) = 1$
\item for all x in the interval 
\begin{align*}
[1, \infty), f(x + 2) - f(x) > 2
\end{align*}
\item $f'(x)$ is strictly decreasing for the interval $[1, \infty)$
\end{enumerate}

\item If 
\begin{align*}
f(x) = \int_{0}^{x} e^{x^2}(t - 2)(t - 3)dt
\end{align*} 
for all $x \in  (0, \infty)$, then 
\begin{enumerate}
\item f has local maxima at x = 2
\item f is decreasing on (2, 3)
\item there exist some $c \in (0, \infty)$, such that $f'(c) = 0$
\item f has a local minimum at x = 3
\end{enumerate}

\item A rectangular sheet of fixed perimeter with sides having length in the ratio 8 : 15 is converted into an open rectangular box by folding after removing squares of equal area from all four corners. If the total area of removed squares is 100, the resulting box has maximum volume. Then the lengths of the sides of the rectangular sheet are
\begin{enumerate}
\item 24
\item 32
\item 45
\item 60
\end{enumerate}

\item Let $f: (0, \infty) \to R$ be given by 
\begin{align*}
f(x)\int_{\frac{1}{x}}^{x} e^{-(t + \frac{1}{t})\frac{dt}{t}}
\end{align*}
then 
\begin{enumerate}
\item $f(x)$ is monotonically increasing on $[1, \infty)$
\item $f(x)$ is monotonically decreasing on(0, 1)
\item $f(x) + f(\frac{1}{x}) = 0$ for all $x \in (0, \infty)$
\item $f(2^x)$ is an odd function of x on R
\end{enumerate}

\item Let $f, g: [-1, 2] \to R$ be continuous functions which are twice differentiable on the interval (-1, 2). Let the values of f and g at the point -1, 0 and 2 be as given in the following table

\begin{tabular}{|c| c| c| c|} 
 \hline
  & x = -1 & x = 0 & x = 2 \\ 
 \hline
 f(x) & 3 & 6 & 0 \\ 
 \hline
 g(x) & 0 & 1 & -1 \\
 \hline
\end{tabular}\\

in each of the intervals (-1, 0) and (0, 2) the function $(f-3g)''$ never vanishes.Then the correct statement (s) is 
(are )
\begin{enumerate}
\item $f'(x) - 3g'(x) = 0$ has exactly three solutions in (-1, 0) $\cup$ (0, 2)
\item $f'(x) - 3g'(x) = 0$ has exactly one solutions in (-1, 0)
\item $f'(x) - 3g'(x) = 0$ has exactly one solutions in (0, 2)
\item $f'(x) - 3g'(x) = 0$ has exactly two solutions in (-1 ,0), exactly two solutions in (0, 2)
\end{enumerate}

\item Let $f: R \to R$ is a differentiable functions such that $f'(x) > 2f(x)$ for all $x \in R$ and $f(0) = 1$, then 
\begin{enumerate}
\item $f(x)$ is increasing in (0, $\infty$)
\item $f(x)$ is decreasing in (0, $\infty$)
\item $f(x) >  e^{2x}$ in (0, $\infty$)
\item $f'(x) > e^{2x}$ in (0, $\infty$)
\end{enumerate}

\item If  
$f(x) = 
\begin{vmatrix}
\cos(2x) & \cos(2x) & \sin (2x) \\ 
-\cos x & \cos x & -\sin x  \\
\sin x & \sin x& \cos x 
\end{vmatrix}$
Then 
\begin{enumerate}
\item $f'(x) = 0$ at exactly three points in ($-\pi, \pi$)
\item $f'(x) = 0$ at more than three points in ($-\pi, \pi$)
\item $f(x)$ attains its maximum at x = 0
\item $f(x)$ attains its minimum at x = 0
\end{enumerate}

\item Defines collections $\{E_1, E_2, E_3,.....\}$ of ellipses and $\{R_1, R_2, R_3,....\}$ of rectangles as follows:
\begin{align}
E_1: 
\frac{x^2}{9} + \frac{y^2}{4} = 1
\end{align}
$R_1$: Rectangle of largest area, with parallel sides to the axes inscribed in $E_1$\\
\begin{align}
E_n: Ellipse \frac{x^2}{a_n^2} + \frac{y^2}{b_n^2} = 1
\end{align}
of largest area inscribed in $R_{n - 1}$, $n > 1$;
$R_n$: Rectangle of largest area with sides parallel to the axes inscribed in $E_n$, $n > 1$.\\
Then which of the following options are correct?
\begin{enumerate}
\item The eccentricities of $E_18$ and $E_19$ are NOT equal
\item The length of latus rectum of $E_9$ is $\frac{1}{6}$
\item $\sum_{n = 1}^N$ (area of $R_n$) $<$ 24 for each positive integer N
\item The distance of a focus from the centre in $E_9$ is $\frac{\sqrt{5}}{32}$
\end{enumerate}

\item Let $f: R \to R$ be given by 
\begin{align*}
f(x) = (x - 1)(x - 2)(x - 5)
\end{align*}
 Define 
\begin{align*}
F(x) = \int_{0}^{x} f(t)dt, x > 0
\end{align*} 
Then when of the following options is/are correct?
\begin{enumerate}
\item F has a local maximum at x = 2
\item F has a local minimum at x = 1
\item F has two local maximum and one local minimum (0, $\infty$)
\item F(x) 0 for all x $\in$ (0, 5) 
\end{enumerate}

\item Let 
\begin{align*}
f(x) = \frac{\sin \pi x}{x^2}, x > 0
\end{align*}
Let $x_1 < x_2 < x_3...... < x_n < $.... be all the points of local maximum of f and $y_1 < y_2 < y_3 < $....$ < y_n <$....be all the points of local minimum of f. Then which of the following options is/are correct?
\begin{enumerate}
\item $x_{n + 1} - x_n > 2$
\item $x_n \in (2n, 2n + \frac{1}{2})$ for every n
\item $|x_n - y_n| > 1$ for every n
\item $x_1 < y_1$
\end{enumerate}

\textbf{E.Subjective Problems}

\item Prove that minimum value of
\begin{align*} 
\frac{(a + x)(b + x)}{(c + x)}, a, b > c, x > -c
\end{align*} 
is
\begin{align*} 
(\sqrt{a - c} + \sqrt{b - c})^2
\end{align*}
 
\item Let x and y be two real variables such that $x > 0$ and xy = 1. Find minimum value of x + y.

\item For all x in [0, 1], Let the second derivative $f"(x)$ of a function $f(x)$ exist and satisfy $|f"(x)| < 1$. If f(0) = f(1) Then show that $|f'(x)|<0$ for all x in [0, 1].

\item Use the function $f(x) = x^\frac{1}{x}$, $x > 0$ to determine the biggest of the two numbers $e^{\pi}$ and $\pi^e$.

\item If $f(x)$ and $g(x)$ are differentiable function for $0 \leq x \leq 1$ such that $f(0)=2$, $g(0) = 0$, $f(1) = 6$, $g(1) = 2$, then show that there exist c satisfying $0 < c < 1$ and $f'(c) = 2g'(c)$.

\item Find the shortest distance of the points(0, c) from the parabola $y = x^2$ where $0 \leq c \leq 5$.

\item If $ax^2 + \frac{b}{x} \geq c$ for all positive x where $a > 0$ and $b > 0$ show that $27ab^2 \geq 4c^3$.

\item Show that $1 + xln(x + \sqrt{x^2 + 1}) \geq \sqrt{1 + x^2}$ for all $x \geq 0$.

\item Find the coordinates of the points on the curve $y = \frac{x}{1 + x^2}$ where the tangent to the curve has the greatest slope.

\item Find all the tangents to the curve $y = \cos(x + y)$, $-2\pi \leq x \leq 2\pi$ that are parallel to the line 
x + 2y = 0.

\item Let $f(x) = \sin^3 x + \lambda \sin^2 x$, $\frac{-\pi}{2} < x < \frac{\pi}{2}$ find the intervals in which 
$\lambda$ should lie in order that $f(x)$ has exactly one minumum and one maximum.

\item Find the point on the curve 
\begin{align*}
4x^2 + a^2y^2 = 4a^2, 4 < a^2 < 8 
\end{align*}
that is farthest from the point (0, -2).

\item Investigate maxima and minima the function
\begin{align*}
f(x) = \int_{1}^{x}[2(t - 1)(t - 2)^2 + 3(t - 1)^2(t - 2)^2]dt
\end{align*}

\item Find all maxima and minima of the function $y = x(x - 1)^2, 0 \leq x \leq 2$ also determine the area bounded by the curve $y = x(x - 1)^2$ the y-axis and the line y = 2.

\item Show that $2\sin x + \tan x \geq 3x$ where $0 \leq x < \frac{\pi}{2}$.

\item A point P is given on the circumference of a circle of radius r. Chord QR is parallel to the tangent at P. Determine the maximum possible area of the triangle PQR.

\item A window of perimeter P is in the form of rectangle surrounded by a semicircle. The semi-cicular portion is fitted with coloured glass while the rectangular portion is fitted with the clear glass transmits three times as much light per square meter as the colour glass does. What is the ratio for the sides of rectangle so that the window transmits the maximum light?

\item A cubic $f(x)$ vanishes at x = 2 and has relative minimum and maximum at x = -1 and $x = \frac{1}{3}$ if 
\begin{align*}
\int_{-1}^{1}fdx = \frac{14}{3}
\end{align*}
find the cubic f(x).

\item What normal to the curve $y = x^2$ forms the shortest chord?

\item Find the equation of normal to the curve 
\begin{align*}
y = (1 + x)^y + \sin^{-1}(\sin^2 x)
\end{align*} 
at x=0. 

\item Let 
f(x) = 
\[ \begin{cases} 
      -x^3 + \frac{(b^3 - b^2 + b - 1)}{(b^2 + 3b + 2)} &  0 \leq x < 1\\
      2x - 3 & 1 \leq x \leq 3\\
      \end{cases}
\]
Find all possible real values of b such that $f(x)$ has the smallest value at x = 1.

\item The curve 
\begin{align}
y = ax^3 + bx^2 + cx + 5
\end{align} 
touches the x-axis at P(-2, 0) and cuts the y axis at a point Q where its gradient is 3 . Find a, b, c

\item The circle 
\begin{align}
x^2 + y^2 = 1
\end{align} 
cuts the x-axis at Pand Q another circle with centre at Q and variable radius intersects the first circle at R above the x-axis and the line segment PQ at S Find the maximum area of the triangle QSR.

\item Let (h, k) be a fixed point where $h > 0, k > 0$. A stright line passing through this point cuts the positive direction of the coordinate axis at points P and Q. Find the minimum area of triangle OPQ, O being the origin.

\item A curve $y = f(x)$ passes through the point p(1, 1). The normal to the curve at P is $a(y - 1) + (x - 1) = 0$. If the slope of the tangent at any point on the curve is proportional to the ordinate of the point, determine th equation of the curve also obtain the area bounded by the y-axis the curve and the normal to the curve at P.

\item Determine the points of maxima and minima of the function 
\begin{align*} 
f(x) = \frac{1}{8} ln x - bx + x^2
\end{align*}
x $>$ 0 where $b \geq 0$ is a constant.

\item Let 
f(x)=
\[ \begin{cases} 
      xe^{ax} &  x \leq 0 \\
      x + ax^2 - x^3& x > 0\\
      \end{cases}
\] 
where a is positive constant. Find  the interval in which $f'(x)$ is increasing.



\clearpage
\onecolumn

\textbf{Match the Following Questions:}

\item In this questions there are entries in column I and column II. Each entry in column I ia related to exactly one entry in column II.Write the correct letter from column II againest the entry number in column I  in your answer book.Let the functions defined in column I have domain $(-\frac{\pi}{2},\frac{\pi}{2})$
\begin{table}[ht!]
\centering
\begin{tabular}{c c} 
 \textbf{Column I} & \textbf{Column II}\\ [0.5ex] 
 (A) X + $\sin X$                                            &(p) increasing\\
 (B) $\sec x$                                                &(q) decreasing\\
                                                            &(r) neither increasing nor decreasing\\
\end{tabular}
\end{table}\\
By appropriately matching the matching the information given in the three columns of the following table.
Let f(x)=x+$log_ex-xlog_ex$ $x \in (0,\infty)$
\begin{enumerate}
    \item Column 1 contains information about zeros of f(x),f'(x) and f"(x).
    \item Column 2 contains information about the limiting behaviour of f(x),f'(x) and f"(x) at infity
    \item Column 3 contains information about the increasing/decreasing nature of f(x) and f'(x).
\end{enumerate}

\begin{center}
\begin{tabular}{llll}
Column-1 & Column-2 & Column-3\\
(I) f(x)=0 for some $x \in (1, e^2)$   &(i) $lim_{x \to \infty}f(x)=0$         &(P) f is increasing on (0, 1)\\
&&&\\
(II)f'(x) for $x \in (1, e)$           &(ii)$lim_{x \to \infty}f(x)=-\infty$   &(Q) f is increasing in $(e, e^2)$\\
&&&\\
(III)f'(x) for $x \in(0, 1)$           &(iii)$lim_{x \to \infty}f'(x)=-\infty$ &(R) f' is increasing in (0, 1)\\
&&&\\
(IV)f"(x) for $x \in (1, e)$           &(iv)$lim_{x \to \infty}f"(x)=0$        &(S) f' is decreasing in $(e, e^2)$\\
&&&\\
\end{tabular}
\end{center}

\item Which of the following option is the only correct combination
\begin{enumerate}
    \item (I)(i)(P)                \item (II)(ii)(Q)
    \item (III)(iii)(R)            \item (IV)(iv)(S)
\end{enumerate}
\item Which of the following option is the only correct combination
\begin{enumerate}
    \item (I)(ii)(R)
    \item (II)(iii)(S)
    \item (III)(iv)(P)
    \item (IV)(i)(S)
\end{enumerate}
\item Which of the following option is the only incorrect combination
\begin{enumerate}
    \item (I)(iii)(P)
    \item (II)(iv)(Q)
    \item (III)(i)(R)
    \item (II)(iii)(P)
\end{enumerate}

\clearpage
\twocolumn

\textbf{Comprehension Based Questions:}

\textbf{PASSAGE-1}

If a continuous function f defined on the real line R,assume positive and negative values in R then the equation f(x)=0 has a root in R.For example ,if it is known that a continuous function f on R is positive at some point and its minimum value is negative then the equation f(x)=0 has a root in R.consider f(x)=$ke^x-x$ for all real x where k is a real constant.

\item The line y = x meets $y = ke^x$ for $k \leq 0$ at
\begin{enumerate}
\item no point
\item one point
\item two points
\item more than two points
\end{enumerate}

\item The positive value of k for $ke^x - x = 0$ has only one root is
\begin{enumerate}
\item $\frac{1}{e}$
\item 1
\item e
\item $log_e2$
\end{enumerate}

\item for $k > 0$ the set of all values of k for which $ke^x - x = 0$ has two distinct roots
\begin{enumerate}
\item (0, $\frac{1}{e}$)
\item ($\frac{1}{e}$, 1)
\item ($\frac{1}{e}$, $\infty$)
\item (0, 1)
\end{enumerate}

\textbf{PASSAGE-2}

Let f(x) = $(1 + x)^2 \sin^2 x + x^2$ for all $x\ in IR$ and let g(x) = $\int_{1}^{x} (\frac{2(t - 1)}{t + 1} - lnt)$ f(t)dt for all $x \in (1,\infty)$

\item Consider the following statements:

\textbf{P:} There exist some x $\in$ R such that f(x) + 2x = $2(1 + x^2)$

\textbf{Q:} There exist some $x \in R$ such that 2f(x) + 1 = 2x(1 + x)
Then 
\begin{enumerate}
\item Both P and Q are true
\item P is true and Q is false
\item P is false and Q is true
\item Both P and Q are false
\end{enumerate}

\item Which of the following is true?
\begin{enumerate}
\item g is increasing on (0, $\infty$)
\item g is decreasing on (1, $\infty$)
\item g is increasing on (1, 2) and decreasing on(2, $\infty$)
\item g is decreasing on (1, 2) and increasing on (2, $\infty$)
\end{enumerate}

\textbf{PASSAGE-3}

Let f:[0, 1] $\to$ R be a function suppose the function f is twice differenciable, f(0) = f(1) = 0 and satisfies $f''(x) - 2f'(x) + f(x) \geq e^x$, x $\in$ [0, 1].

\item Which of the following is true for $0 < x < 1$?
\begin{enumerate}
\item $0 < f(x) < \infty$
\item $-\frac{1}{2} < f(x) < \frac{1}{2}$
\item $-\frac{1}{4} < f(x) < 1$
\item $-\infty < f(x) < 0$
\end{enumerate}

\item If function $e^{-x}$ f(x) assumes its minimum in the interval [0, 1] at x = $\frac{1}{4}$ which of the following is true?
\begin{enumerate}
\item $f'(x) < f(x), \frac{1}{4} < x < \frac{3}{4}$
\item $f'(x) > f(x),  0 < x < \frac{1}{4}$
\item $f'(x) < f(x),  0 < x < \frac{1}{4}$
\item $f'(x) < f(x),  \frac{3}{4} < x < 1$
\end{enumerate}

\textbf{Integer Value Correct Type:}

\item The maximum value of the function $f(x) = 2x^3 - 15x^2 + 36x - 48$ on the set A = \{$|x|x^2 + 20 \leq 9x$\}

\item Let p(x) be the polynomial of degree 4 having extremum at x = 1, 2 and $\lim_{x \to 0}(1 + \frac{p(x)}{x^2}) = 2$. then the value of p(2) is

\item Let f be a real valued differential function on R such that f(1) = 1. If the y-intercept of the tangent at any point P(x, y) on the curve y = f(x) is equal to the cube of the abscissa of P then find the value of f(-3).

\item Let f be a function defined on R such that $f'(x) = 2010(x - 2009)(x - 2010)^2(x - 2011)^3(x - 2012)^4$ forall 
$x \in R$. If g is a function defined on R with values in the interval (0, $\infty$) such that f(x) = ln(g(x)),for all 
$x \in R$. Then the number of points at which g has a local maximum is

\item Let $f: IR \to IR$ be defined as f(x) = $|x| + |x^2 - 1|$. The total number of points at which f attains either local maximum or local minimum is 

\item Let p(x) be a real polynomials of least degree which has a local maximum at x = 1 and local minimum at x = 3. If p(1) = 6 and p(3) = 2, Then $p'(0)$ is
 
\item A vertical line passing through the point(h, 0) intersects the ellipse $\frac{x^2}{4} + \frac{y^2}{3} = 1$ at the points P and Q. Let the tangent to the ellipse at P and Q meet at the point R. If $\Delta$(h) = area of the triangle PQR,$\Delta_1$ = $\max_{\frac{1}{2}\leq h \leq 1}$ $\Delta$(h) and $\Delta_2$ = $\min_{\frac{1}{2}\leq h \leq 1}$ $\Delta$(h), then $\frac{8}{\sqrt{5}}\Delta_1$ - $8\Delta_2$ =

\item The slope of the tangent to the curve $(y - x^5)^2 = x(1 + x^2)^2$ at the point (1, 3) is

\item A cylindrical container is to be made from certain solid material with the following constraints. It has a fixed inner volume of V $mm^3$ has a 2mm thick solid wall and is open at the top. The bottom of the container is a solid circular disc of thickness 2mm and is of radius equal to the outer radius of the container. If the volume of the material used to make the container is minimum when the inner radius of the container is 10mm, then the values of 
$\frac{v}{250\pi}$ is

\item Let $-1 \leq p \leq 1$. Show that the equation $4x^3 - 3x - p = 0$ has a unique root in the interval 
$[\frac{1}{2}, 1]$ and identify it.

\item Find a point on the curve $x^2 + 2y^2 = 6$ whose distance from the line x + y = 7, is minimum.

\item Using the relation $2(1 - \cos x) < x^2$, $x \neq  0$ or otherwise prove that $\sin(\tan x)\geq x \forall x \in [0, \frac{\pi}{4}]$.

\item If the function $f: [0, 4] \to R$ is differentiable then show that 
\begin{enumerate}
\item for $a, b \in(0, 4)$, $(f(4))^2 - (f(0))^2 = 8 f'(a)f(b)$
\item $\int_{0}^{4} f(t)dt = 2[\alpha f(\alpha^2) + \beta f(\beta^2)] \forall 0< \alpha, \beta < 2$
\end{enumerate}

\item If $p(1) = 0$ and $\frac{dP(x)}{dx} > P(x)$ for all x $\geq$ 1 then prove that $P(x) > 0$, for all x $>$ 1.

\item Using Rolle's theorem prove that there is at least one root for $(45^\frac{1}{100},46)$ of polynomial
\begin{align*}
P(x) = 51x^{101} - 2323(x)^{100} - 45x + 1035
\end{align*}

\item Prove that for $x \in [0, \frac{\pi}{2}]$, $\sin x + 2x \geq \frac{3x(x + 1)}{\pi}$ explain the identity if any used in the proof.

\item $|f(x_1) - f(x_2)| < (x_1 - x_2)^2$ for $x_1, x_2 \in R$. Find the equation of the tangent to the curve $y = f(x)$ at the point (1, 2).

\item If p(x) be the polynomial of degree 3 satisfying p(-1) = 10, p(1) = -6 and p(x) has maxima at x = -1
and $p'(x)$ has minima at x = 1. Find the distance between the local maxima and local minima of the curve.

\item For a twice differenciable function f(x), g(x) is defined as g(x) = $(f'(x)^2 + f''(x))$ f(x) on [a, e] If for $a < b < c < d < e$, f(a) = 0, f(b) = 2, f(c) = -1, f(d) = 2, f(e) = 0 then find the minimum number of zeros of g(x).

\item Let a + b = 4, where a $<$ 2 and let $g(x)$ be a differentiable function. If $\frac{dg}{dx} > 0$ for all x Prove that $\int_{0}^{a} g(x) dx + \int_{0}^{b}$ g(x) dx increases as (b - a) increasing.

\item Suppose $f(x)$ is a function statisfying th following conditions 
\begin{enumerate}
\item f(0) = 2, f(1) = 1
\item f has a minimum value at x = $\frac{5}{2}$ and 
\item for all x
\end{enumerate}
\[
f'(X) =
\begin{vmatrix}
2ax & 2ax - 1 & 2ax + b + 1  \\ 
b & b + 1 & -1  \\
2(ax + b) & 2ax + 2b + 1 & 2ax + b 
\end{vmatrix}
\]
where a, b be are some constants. Determine the constants a, b and the function f(x).

\item A  curve C has the property that if the tangent drawn at any point P on C the co-ordinate axes at A and B then P is the mid point of AB. The curve passes through the point (1, 1). Determine the equation of the curve.

\item Suppose
\begin{align*} 
p(x) = a_0 + a_1x + ..... + a_nx^n
\end{align*} 
If $|p(x)|\leq |e^{x - 1} - 1|$ for all x $\geq$ 0. Prove that 
\begin{align*}
|a_1 + 2a_2 + ....+na_n| \leq 1.
\end{align*}

\textbf{Section-B:}


\item The maximum distance from origin of a point on the curve 
\begin{align*}
x = a\sin t - b\sin\frac{at}{b}
\end{align*}
\begin{align*}
y = a\cos t - b\cos \frac{at}{b}, a, b > 0
\end{align*}
\begin{enumerate}
\item a - b
\item a + b
\item $\sqrt{a^2 + b^2}$
\item $\sqrt{a^2 - b^2}$
\end{enumerate}

\item If 2a + 3b + 6c = 0,($a, b, c \in R$) then the quadratic equation $ax^2 + bx + c = 0$ has 
\begin{enumerate}
\item at least one root in [0, 1]
\item at least one root in [2, 3]
\item at least one root in [4, 5]
\item none of these
\end{enumerate}

\item If the function $f(x) = 2x^3 - 9ax^2 + 12a^2 + 1$, where $a > 0$ attains its maximum and  minimum at p and q respectively such that $p^2 = q$ then a equals
\begin{enumerate}
\item $\frac{1}{2}$
\item 3
\item 1
\item 2
\end{enumerate}

\item A point on the parabola $y^2 = 18x$ at which the ordinate increase at twice the rate of the abscissa is \begin{enumerate}
\item ($\frac{9}{8}, \frac{9}{2}$)
\item (2, -4)
\item ($-\frac{9}{8}, \frac{9}{2}$)
\item (2, 4)
\end{enumerate}
    
\item A function y = f(x) has a second order derivative $f''(x) = 6(1 - x)$. If its graph passes through the point (2, 1) and at that point the tangent to the graph is y = 3x - 5 then the function is
\begin{enumerate}
\item $(x + 1)^2$
\item $(x - 1)^3$
\item $(x + 1)^3$
\item $(x - 1)^2$
\end{enumerate}

\item The normal to the curve x = $(1 + \cos \theta)$, y = $a\sin \theta$ at $\theta$ always passes through the fixed point 
\begin{enumerate}
\item (a, a)
\item (0, a)
\item (0, 0)
\item (a, 0)
\end{enumerate}

\item If 2a + 3b + 6c = 0 then at least one root of the equation $ax^2 + bx + c = 0$ lies in the interval
\begin{enumerate}
\item (1, 3)
\item (1, 2)
\item (2, 3)
\item (0, 1)
\end{enumerate}

\item Area of the greatest rectangle that can be inscribed in the ellipse $\frac{x^2}{a^2} + \frac{y^2}{b^2} = 1$ is 
\begin{enumerate}
\item 2ab
\item ab
\item $\sqrt{ab}$
\item $\frac{a}{b}$
\end{enumerate}

\item The normal to the curve x = a$(\cos \theta + \theta \sin \theta)$, y = a($\sin \theta - \theta \cos \theta$) at any point $'\theta'$ is such that
\begin{enumerate}
\item it passes through the origin 
\item it makes an angle $\frac{\pi}{2} + \theta$ with the x-axis
\item it passes through (a$\frac{\pi}{2}$, -a)
\item it is at constant distance from the origin
\end{enumerate}

\item A spherical iron ball 10 cm in radius is coated with a layer of ice of uniform thickness of ice is 5cm then the rate at which the thickness of ice decrease is 
\begin{enumerate}
\item $\frac{1}{36\pi} cm/min$
\item $\frac{1}{18\pi} cm/min$
\item $\frac{1}{54\pi} cm/min$
\item $\frac{5}{6\pi} cm/min$
\end{enumerate}

\item if the equation $a_nx^n + a_{n - 1}x^{n - 1}.......+a_1x$ = 0, $a_1 \neq 0,n\geq 2$, has a positive root 
x = $\alpha$ then the equation $na_n x^{n - 1} + (n - 1)a_{n - 1}x^{n - 2}+.......+ a_1$ = 0 has a positive root which is 
\begin{enumerate}
\item greater than $\alpha$
\item smaller than $\alpha$
\item greater than or equal to $\alpha$
\item equal to $\alpha$
\end{enumerate}

\item The function f(x) = $\frac{x}{2} + \frac{2}{x}$ has a local minimum at
\begin{enumerate}
\item x = 2
\item x = -2
\item x = 0
\item x = 1
\end{enumerate}

\item A triangular park is enclosed on two sides by a fence and on the third side by a straight river bank.The two sides having fence are of same length x,the maximum area enclosed by the park is 
\begin{enumerate}
\item $\frac{3}{2}x^2$
\item $\sqrt{\frac{x^3}{8}}$
\item $\frac{1}{2}$
\item $\pi x^2$
\end{enumerate}

\item A value of C for which conclusion of Mean Value Theorem holds for the function f(x) = $\log_e^x$ on the interval [1, 3] is 
\begin{enumerate}
\item $\log_3 e$
\item $\log_e 3$
\item $2\log_3 e$
\item $\frac{1}{2}\log_3 e$
\end{enumerate}

\item The function f(x) = $\tan^{-1}(\sin x + \cos x)$ is an increasing function in 
\begin{enumerate}
\item (0, $\frac{\pi}{2}$)
\item ($-\frac{\pi}{2}, \frac{\pi}{2}$)
\item ($\frac{\pi}{4}, \frac{\pi}{2}$)
\item ($-\frac{\pi}{2}, \frac{\pi}{4}$)
\end{enumerate}

\item If p and q are positive real numbers such that $p^2 + q^2 = 1$ then the maximum value of (p + q) is
\begin{enumerate}
\item $\frac{1}{2}$
\item $\frac{1}{\sqrt{2}}$
\item $\sqrt{2}$
\item 2
\end{enumerate}

\item Suppose the cubic $x^3 - px + q$ has three distinct real roots where $p > 0$ and $q > 0$ Then which one of the following holds?
\begin{enumerate}
\item the cubic has minimum at $\sqrt{\frac{p}{3}}$ and maximum at $-\sqrt{\frac{p}{3}}$
\item the cubic has minimum at $-\sqrt{\frac{p}{3}}$ and maximum at $\sqrt{\frac{p}{3}}$
\item the cubic has minimum at both  $\sqrt{\frac{p}{3}}$ and $-\sqrt{\frac{p}{3}}$
\item the cubic has maximum at $\sqrt{\frac{p}{3}}$ and  $-\sqrt{\frac{p}{3}}$
\end{enumerate}

\item How many real solutions does the equation $x^7 + 14x^5 + 16x^3 + 30x - 560 = 0$ have?
\begin{enumerate}
\item 7
\item 1
\item 3
\item 5
\end{enumerate}

\item Let f(x) = x$|x|$ and g(x) = $\sin x$.
\begin{enumerate}
\item Statement 1: gof is differentiable at x = 0 and its derivative is continuous at that point.
\item Statement 2: gof is twice differentiable at x = 0.
\end{enumerate}
\begin{enumerate}
\item statement-1 is true,statement-2 is true statement-2 is not correct explination for statement-1
\item statement-1 is true,statement-2 is false 
\item statement-1 is false and statement-2 is true 
\item statement-1 is true,statement-2 is true statement-2 is correct explination of statement-1
\end{enumerate}

\item Given P(x) = $x^4 + ax^3 + bx^2 + cx + d$ such that x = 0 is the only real root of $P'(x)=0$. If P(-1) $<$ P(1), Then in the interval [-1, 1]:
\begin{enumerate}
\item P(-1) is not a minimum but P(1) is the maximum of P
\item P(-1) is the minimum but P(1) is  not the maximum of P
\item Neither P(-1) is a minimum nor P(1) is the maximum of P
\item P(-1) is a minimum but P(1) is the maximum of P
\end{enumerate}

\item The equation of the tangent to the curve $y = x + \frac{4}{x^2}$ that is parallel to the x-axis is 
\begin{enumerate}
\item y = 1
\item y = 2
\item y = 3
\item y = 0
\end{enumerate}

\item Let $f: R \to R$ be defined by f(x) = 
\[ \begin{cases} 
      k - 2x &  if x \leq -1 \\
      2x + 3 & if x > -1\\
      \end{cases}
\] 
if f has a local minimum at x = -1 then a possible value of k is 
\begin{enumerate}
\item 0
\item $\frac{-1}{2}$
\item -1
\item 1
\end{enumerate}

\item Let $f: R \to R$ be a continuous function defined by f(x) = $\frac{1}{e^x + 2e^{-x}}$
\begin{enumerate}
\item Statement-1: f(c) = $\frac{1}{3}$ for some $c \in R$
\item Statement -2: $0 < f(x) \leq \frac{1}{2\sqrt{2}}$ for all $x \in R$
\end{enumerate}
\begin{enumerate}
\item statement-1 is true,statement-2 is true statement-2 is not correct explination for statement-1
\item statement-1 is true,statement-2 is false 
\item statement-1 is false and statement-2 is true 
\item statement-1 is true,statement-2 is true statement-2 is correct explination of statement-1
\end{enumerate}

\item The shortest distance between line y - x = 1 and curve x = $y^2$ is 
\begin{enumerate}
\item $\frac{3\sqrt{2}}{8}$
\item $\frac{8}{3\sqrt{2}}$
\item $\frac{4}{\sqrt{3}}$
\item $\frac{\sqrt{3}}{4}$
\end{enumerate}

\item For $x \in (0, \frac{5\pi}{2})$ define f(x) = $\int_{0}^{x}\sqrt{t} \sin t$dt then f has 
\begin{enumerate}
\item local minimum at $\pi and 2\pi$
\item local minimum at $\pi$ and local maximum at $2\pi$
\item local maximum at $\pi$ and local minimum at $2\pi$
\item local maximum at $\pi$ and $2\pi$
\end{enumerate}

\item A spherical balloon filled with 4500$\pi$ cubic meters of helium gas. If a leak in balloon causes the gas to escape at the rate of $72\pi$ cubic meters per minute then then rate at which the radius of balloon  decreases 49 minutes after the leakage began is:
\begin{enumerate}
\item $\frac{9}{7}$
\item $\frac{7}{9}$
\item $\frac{2}{9}$
\item $\frac{9}{2}$
\end{enumerate}

\item Let a, b $\in$ R be such that the function f given by f(x) = ln$|x|$ + b$x^2$ + ax, $x \neq 0$ has extreme values at x = -1 and at x = 2.
\begin{enumerate}
\item Statement-1: f has local maximum at x = -1 and at x = 2
\item Statement-2: a = $\frac{1}{2}$ and $b = -\frac{1}{4}$
\end{enumerate}
\begin{enumerate}
\item Statement-1 is false,Statement-2 is true
\item Statement-1 is true,statement-2 is true statement-2 is a correct explanation of statement -1
\item Statement-1 is true,statement-2 is true statement-2 is not a correct explanation of statement -1
\item Statement-1 is true and statement-2 is false
\end{enumerate}

\item A line is drawn through the point [1, 2] to meet the coordinates axes at P and Q such that it forms a triangle OPQ where O is the origin. If the area of the triangle OPQ is least then the slope of the line PQ is:
\begin{enumerate}
\item $\frac{-1}{4}$
\item -4
\item -2
\item $\frac{-1}{2}$
\end{enumerate}

\item The intercepts on the axis made by tangents to the curve y = $\int_0^x |t|$ dt, $x \in R$ which are parallel to the line y = 2x are equal to 
\begin{enumerate}
\item $\pm$1
\item $\pm$2
\item $\pm$3
\item $\pm$4
\end{enumerate}

\item If f and g are differentiable functions in [0, 1] satisfying f(0) = 2 = g(1), g(0) = 0 and f(1) = 6, Then for some c $\in$ [0, 1]
\begin{enumerate}
\item $f'(c) = g'(c)$
\item $f'(c) = 2g'(c)$
\item $2f'(c) = g'(c)$
\item $2f'(c) = 3g'(c)$
\end{enumerate}

\item Let f(x) be the polynomial of degree four having extreme values at x = 1 and x = 2. If $\lim_{x \to 0}$
[1 + $\frac{f(x)}{x^2}$] = 3, then f(2) is equal to:
\begin{enumerate}
\item 0
\item 4
\item -8
\item -4
\end{enumerate}

\item Consider:
\begin{align*}
f(x) = \tan^{-1}(\sqrt{\frac{1 + \sin x}{1 - \sin x}}) x \in (0, \frac{\pi}{2})
\end{align*} 
A normal to y = f(x) at x = $\frac{p}{6}$ also passes through the point:
\begin{enumerate}
\item ($\frac{\pi}{6}$, 0)
\item ($\frac{\pi}{4}$, 0)
\item (0, 0)
\item (0, $\frac{2\pi}{3}$)
\end{enumerate}

\item A wire of length 2 units is cut into two parts which are bent respectively to form a square of side = x units and a circle os radius = r units. If sum of the areas of the squares and the circle so formed is minimum then,
\begin{enumerate}
\item x = 2r
\item 2x = r
\item 2x = $(\pi + 4)$r
\item $(4 - \pi)$x = $\pi r$
\end{enumerate}

\item The function $f: R \to [\frac{-1}{2}, \frac{1}{2}]$ defined as f(x) = $\frac{x}{1 + x^2}$ is
\begin{enumerate}
\item neither injective nor surjective 
\item invertible
\item injective but not surjective 
\item surjective but not injective
\end{enumerate}

\item The Normal to the curve y(x - 2)(x - 3) = x + 6 at the point where the curve intersects the y-axis passes through the point:
\begin{enumerate}
\item ($\frac{1}{2}, \frac{1}{3}$)
\item ($-\frac{1}{2}, -\frac{1}{2}$)
\item ($\frac{1}{2}, \frac{1}{2}$)
\item ($\frac{1}{2}, -\frac{1}{3}$)
\end{enumerate}

\item Twenty meter of wire is available for fencing off a flower bed in the form of circular sector Then the maximum area of flower bed is:
\begin{enumerate}
\item 30
\item 12.5
\item 10
\item 25
\end{enumerate}

\item The eccentricity of an ellipse whose centre is at the origin is $\frac{1}{2}$ If one of its directices is x = -4 then the equation of normal to it at (1, $\frac{3}{2}$) is ;
\begin{enumerate}
\item x + 2y = 4
\item 2y - x = 2
\item 4x - 2y = 1
\item 4x + 2y = 7
\end{enumerate}

\item Let f(x) = $x^2 + \frac{1}{x^2}$ and g(x) = x - $\frac{1}{x}$, $x \in R-\{-1, 0, 1\}$. If h(x) = $\frac{f(x)}{g(x)}$ then local minimum value of h(x) is:
\begin{enumerate}
\item -3
\item -2$\sqrt{2}$
\item 2$\sqrt{2}$
\item 3
\end{enumerate}

\item If the curves $y^2 = 6x$, $9x^2 + by^2$ = 16 intersects each other at right angles then the value of b is:
\begin{enumerate}
\item $\frac{1}{2}$
\item 4
\item $\frac{9}{2}$
\item 6
\end{enumerate}

\item The maximum volume of the right circular cone having slant height 3m is
\begin{enumerate}
\item $6\pi$
\item $3\sqrt{3}\pi$
\item $\frac{4}{3}\pi$
\item $2\sqrt{3}\pi$
\end{enumerate}

\item If q denotes the acute angle between the curves, y = $10 - x^2$ and y = 2 + $x^2$ at a point of their intersection then $|\tan \theta|$ is equal to 
\begin{enumerate}
\item $\frac{4}{9}$
\item $\frac{8}{15}$
\item $\frac{7}{17}$
\item $\frac{8}{17}$
\end{enumerate}

\item If f(x) is a non-zero polynomial of degree four having local extreme points at x = -1, 0, 1 then the set 
S = \{x R: f(x) = f(0)\} contains exactly
\begin{enumerate}
\item four irrational numbers.
\item four rational numbers. 
\item two irrational and two rational number.
\item two irrational and one rational number.
\end{enumerate}

\item If the tangent to the curve y = x3 + ax - b at the point (1, -5) is perpendicular to the line -x + y + 4 = 0, then which one of the following points lie on the curve?
\begin{enumerate}
\item (-2, 1)
\item (-2, 2)
\item (2, -1)
\item (2, -2)
\end{enumerate}

\item Let S be the set of all values of x for which the tangent to the curves y = f(x) = $x^3 - x^2 - 2x$ at (x, y) is parallel to the line segment joining the points (1, f(a)) and (-1, f(-1)) then S is equal to: 
\begin{enumerate}
\item \{$\frac{1}{3}$, 1\}
\item \{$-\frac{1}{3}$, -1\}
\item \{$\frac{1}{3}$, -1\}
\item \{$-\frac{1}{3}$, 1\}
\end{enumerate}

\end{enumerate}

 
\section{Differentiation}
\renewcommand{\theequation}{\theenumi}
\begin{enumerate}[label=\arabic*.,ref=\thesubsection.\theenumi]
\numberwithin{equation}{enumi}

	\item If y=f$\left(\dfrac{2x-1}{x^{2}+1}\right)$ and f'(x) = $\sin x^{2}$, then $\dfrac{dy}{dx}$ = ........
	
	\item $f_r(x)$,$g_r(x)$, $h_r(x)$, r = 1,2,3 are polynomials in x such that $f_r(a)$ = $g_r(a)$=$h_r(a)$ r = 1,2,3 and \\
	\begin{equation*}
	F(x) = 
   \begin{vmatrix} 
   f_{1}(x) & f_{2}(x) & f_{3}(x)  \\
   g_{1}(x) & g_{2}(x) & g_{3}(x)  \\
   h_{1}(x) & h_{2}(x) & h_{3}(x)  \\
   \end{vmatrix} 
\end{equation*}
 Then F'(x) at x = a is .............
	\item If f(x) = $\log_x$(ln x), then $f'(x)$ at x = e is ............\\
	
	\item The derevative of $sec^{-1}\left(\dfrac{1}{2x^2-1}\right)$ with respect to $\sqrt{1-x^2}$ at x = $\dfrac{1}{2}$ is ...........\\
	\item If f(x) = $|x-2|$ and g(x) = f[f(x)], then g'(x) = ............. for x$>$20
	\item If x$e^{xy}$ = y + $\sin^{2}$x, then at x=0, $\dfrac{dy}{dx}$ = ...........\\
	\item The derivavtive of an even function is always an odd function.\\
	\item If $y^{2}$ = P(x), a polynomial of degree 3, \\
	then 2$\dfrac{d}{dx}\left(y^3\dfrac{d^2y}{dx^2}\right)$, equals .....
	\begin{itemize}
	\begin{multicols}{2}
	\item [(a)]P'''(x)+P'(x)
	\item [(b)]P''(x)P'''(x)
	\item [(c)]P(x)P'''(x)
	\item [(d)]constant
	
	\end{multicols}
	\end{itemize}
	\item Let f(x) be a quadratic expression which is positive for all the real values of x. If g(x) = f(x)+f'(x)+f''(x), then for any real x,
	
	 \begin{itemize}
	\begin{multicols}{2}
	\item [(a)]g(x)$<$0
	\item [(b)]g(x)$>$0
	\item [(c)]g(x) = 0
	\item [(d)]g(x)$\geq$0
	
	\end{multicols}
	\end{itemize}
	\item If y = $\sin x^{\tan x}$ then $\dfrac{dy}{dx}$ is equals to\\
	(a) $\sin x^{\tan x}$(1+$\sec^2x\log\sin(x)$\\
	(b) $\tan x(\sin x)^{\tan x-1}$.$\cos x$\\
	(c) $\sin x^{\tan x}\sec^2x\log\sin x$\\
	(d) $\tan x(\sin x)^{\tan x-1}$\\
	\item $x^2 + y^2$ = 1
	\begin{itemize}
	\begin{multicols}{2}
	\item [(a)]yy"-2$y'^2$+1 = 0
	\item [(b)]yy"+$y'^2$+1 = 0
	\item [(c)]yy"+$y'^2$-1 = 0
	\item [(d)]yy"+2$y'^2$+1=0
	
	\end{multicols}
	\end{itemize}
	\item Let f:(0 $\infty$) $\to$ R and F(x) = $\int\limits_0^x$ f(t)dt. If F($x^2$) = $x^2$(1+x), then f(4) equals
	\begin{itemize}
	\begin{multicols}{2}
	\item [(a)]$\dfrac{5}{4}$\\
	\item [(b)]7\\
	\item [(c)]4\\
	\item [(d)]0\\
	
	\end{multicols}
	\end{itemize}
	\item If y is a function of x and $\log(x+y)$-2xy = 0, then the value of y'(0) is equals to
	\begin{itemize}
	\begin{multicols}{2}
	\item [(a)]1
	\item [(b)]-1
	\item [(c)]2
	\item [(d)]0
	
	\end{multicols}
	\end{itemize}
	\item If f(x) is a twice diffferentiable function and given that \\
	f(1) = 1;f(2) = 4,f(3) = 9, then\\
	(a) f"(x)=2 for $\forall$ x $\in$ (1,3)\\
	(b) f"(x)=f'(x) = 5 for some x $\in$ (2,3)\\
	(c) f"(x)=3 for $\forall$ x $\in$ (1,3)\\
	(d) f"(x) =2 for some x $\in$ (1,3)\\
	\\
	\item $\dfrac{d^2x}{dy^2}$ equals\\
	\begin{itemize}
	\begin{multicols}{2}
	\item [(a)]$\left(\dfrac{d^2y}{dx^2}\right)^{-1}$
	\item [(b)]-$\left(\dfrac{d^2y}{dx^2}\right)^{-1}$ $\left(\dfrac{dy}{dx}\right)^{-3}$
	\item [(c)]$\left(\dfrac{d^2y}{dx^2}\right)$ $\left(\dfrac{dy}{dx}\right)^{-2}$
	\item [(d)]-$\left(\dfrac{d^2y}{dx^2}\right)$  $\left(\dfrac{dy}{dx}\right)^{-3}$
	
	\end{multicols}
	\end{itemize}
	\item Let g(x)=$\log$f(x) where f(x) is twice differentiable positive function on (0, $\infty$) such that f(x+1) = xf(x). Then, for N=1,2,3,.......\\
	g"$\left(N+\dfrac{1}{2}\right)$-g"$\left(\dfrac{1}{2}\right)$ = \\
	g"$\left(N+\dfrac{1}{2}\right)$ - g"($\dfrac{1}{2}$ = \\
	\\
	(a) -4$\left\{1+\dfrac{1}{9}+\dfrac{1}{25}+......+\dfrac{1}{(2N-1)^2}\right\}$\\
	\\
	(b) 4$\left\{1+\dfrac{1}{9}+\dfrac{1}{25}+......+\dfrac{1}{(2N-1)^2}\right\}$\\
	\\
	(c) -4$\left\{1+\dfrac{1}{9}+\dfrac{1}{25}+......+\dfrac{1}{(2N+1)^2}\right\}$\\
	\\
	(d) 4$\left\{1+\dfrac{1}{9}+\dfrac{1}{25}+......+\dfrac{1}{(2N+1)^2}\right\}$\\
	\\
	\item Let f:[0, 2] $\to$ R be a function which is continuous on [0,2] and is differntiable on(0,2) with f(0) = 1. Let\\
	F(x) = $\int\limits_0^{x^2}$f($\sqrt{t}$)dt for x $\in$ [0,2]. If F'(x) = f'(x) for all x $\in$ (0,2), then F(2) equals
	\begin{itemize}
	\begin{multicols}{2}
	\item [{a}]$e^2$ - 1
	\item [(b)]$e^4$ - 1
	\item [(c)]e - 1
	\item [(d)]$e^4$
	
	\end{multicols}
	\end{itemize}
	\item Let f:R $\to$ R, g:R $\to$ R and h : R $\to$ R be differentiable functions such that f(x)=$x^2$+3x+2, g(f(x))=x and h(g(g(x)))=x for all x $\in$ R. Then
	\begin{itemize}
	\begin{multicols}{2}
	\item [(a)]g'(2) = $\dfrac{1}{15}$\\
	\item [(b)]h'(1) = 666\\
	\item [(c)]h(0) = 16\\
	\item [(d)]h(g(3)) = 36\\
	
	\end{multicols}
	\end{itemize}
	\item For every twice differentiable function\\
	 f:R $\to$ [-2,2] with $(f(0))^2$ + $(f(0))^2$ = 85, which of the following statement(s) is True?\\
	 \\
	(a) There exist r,s $\in$ R, where r$<$s, such that f is on the open interval (r,s)\\
	(b) There exists $x_0\to$(-4,0) such that $|f'(x_0)|\leq$ 1 \\
	(c) $\displaystyle{\lim_{x \to \infty}}$ = 1\\
	(d) There exists $\alpha$ $\to$ (-4, 4) such that f($\alpha$)+f'($\alpha$) = 0 and f'($\alpha$) $\neq$ 0\\
	\\
	\item For any positive integer n, define $f_n$:(0, $\infty$)$\to$R as $f_n$(x) = $\sum\limits_{j = 1}^n$ $\tan^{-1}$ $\left(\dfrac{1}{1+(x+j)(x+j-1)}\right)$ for all x$\in$ (0,$\infty$).\\
	Here, the inverse trignometric function $\tan^{-1}$(x) assumes values in $\left(-\dfrac{\pi}{2}, \dfrac{\pi}{2}\right)$\\
	Then, which of the following statements are True?\\
	(a) $\sum\limits_{j=1}^5 \tan^2 f_j(0))$ = 55\\
	(b) $\sum\limits_{j=1}^{10} (1+f'_j(0))\sec^2(f'_j(0))$ = 10\\
	(c) For any fixed positive integer n,$\displaystyle{\lim_{x \to \infty}}$ $\tan (f_n(x))$ = $\dfrac{1}{n}$\\
	(d)  For any fixed positive integer n,$\displaystyle{\lim_{x \to \infty}}$ $\sec^2 (f_n(x))$ = 1 
	\item Let f:(0,$\pi$)$\to$R be a twice differentiable \\
	\\
	function such that $\displaystyle{\lim_{t \to \infty}}$ $\dfrac{f(x)\sin t-f(t)\sin x}{t-x}$ = \\
	\\
	$\sin^2x$ for all x$\in(0,\pi)$. If $\dfrac{\pi}{6}$ = -$\dfrac{\pi}{12}$, then which of the following statement(s)  are True?\\
	\\
	\item [(a)] f$\left(\dfrac{\pi}{4}\right)$ = $\dfrac{\pi}{4\sqrt{2}}$\\
	
	\item [(b)]f(x)$<\dfrac{x^4}{6}-x^2$ for all x$\in(0,\pi)$\\
	\item [(c)]There exist $\alpha \in (0,\pi)$ such that f'($\alpha$) = 0\\
	\item [(d)]f"$\left(\dfrac{\pi}{2}\right)$+f$\left(\dfrac{\pi}{2}\right)$ = 0\\
	\item Find the derivative of $\sin(x^2+1)$ with respect to x from first principle.\\
	\item Find the derivative of\\
	$$
	F(x)=
	\begin{cases}
	\dfrac{x-1}{2x^2-7x+5},   & \text{when $x \neq 1$}\\
	\\
	-\dfrac{1}{3},   &\text{when $x = 1$}
	\end{cases}
	$$
	\\
	\item Given y = $\dfrac{5x}{3\sqrt{(1-x)^2}}$ + $\cos^2$(2x+1); Find $\dfrac{dy}{dx}$.
	\item y = $e^{x\sin x^3}$ + ($\tan x)^x$. Find  $\dfrac{dy}{dx}$\\
	\item Let f be a twice differentiable function such that\\
	f"(x) = -f(x) and f'(x) = g(x), h(x) = [$f(x)]^2$ + [$g(x)]^2$
	find h(10) if h(5) =11\\
	\item If $\alpha$ be a repeated root of a quadratic equation f(x) = 0 and A(x), B(x) and C(x) be polynomials of degree 3,4 and 5 respectively, then show that \begin{equation*}
   \begin{vmatrix} 
   A(x) & B(x) & C(x)  \\
   A(\alpha) & B(\alpha) & C(\alpha)  \\
   A'(\alpha) & B'(\alpha) & C'(\alpha)  \\
   \end{vmatrix} 
\end{equation*}\\
	is divisible by f(x), where prime denotes the derivatives.\\
	\item If x = $\sec \theta$ - $\cos \theta$ and y= $\sec^n \theta$ - $\cos^n \theta$, then show that $(x^2 + 4)\left(\dfrac{dy}{dx}\right)^2$ = $n^2(y^2 + 4)$.\\
	\item Find $\dfrac{dy}{dx}$ at x = -1, when\\
	\\
	$(\sin y)^{\sin\left(\frac{\pi}{2} x \right)}$ + $\dfrac{\sqrt{3}}{2}\sec^{-1}(2x)$ + $2^x \tan(ln(x+2))$ = 0\\
	\\
	\item  y = $\dfrac{ax^2}{(x-a)(x-b)(x-c)}$ + $\dfrac{bx}{(x-b)(x-c)}$ + $\dfrac{c}{(x-c)}$+1, prove that\\
	\\
	 $\dfrac{y'}{y}$ = $\dfrac{1}{x}\left(\dfrac{a}{a-x}+\dfrac{b}{b-x}+\dfrac{c}{c-x}\right)$.\\
	\item Let f(x) = 2+$\cos x$ for all real x.
	\\
	\\
	STATEMENT-1: For each real t, there exist a point c in [t,t+$\pi$] such that f'(c) = 0 because\\
	STATEMENT-2 f(t) = f(t+2$\pi$) for each real t.\\
	\\
	(a) Statement-1 is True, Statement-2 is True; Statement-2 is a correct explanation of Statement-1\\
	(b) Statement-1 is True, Statement-2 is True; Statement-2 is NOT a correct explanation of Statement-1\\
	(c) Statement-1 is True, Statement-2 is False\\
	(d) Statement-1 is False, Statement-2 is True.\\
	\item Let f and g be real valued functions defined on interval (-1,1) such that g"(x) is continuous, g(0) $\neq$ 0. g'(0) = 0, g" $\neq$ 0, and f(x) = g(x)$\sin x$\\
	\\
	STATEMENT-1: $\displaystyle{\lim_{x \to 0}}$ [g(x)cot x - g(0)cosec x] = f"(0) and\\
	STATEMENT-2: f'(0) = g(0)\\
	\\
	(a) Statement-1 is True, Statement-2 is True; Statement-2 is a correct explanation of Statement-1\\
	(b) Statement-1 is True, Statement-2 is True; Statement-2 is NOT a correct explanation of Statement-1\\
	(c) Statement-1 is True, Statement-2 is False\\
	(d) Statement-1 is False, Statement-2 is True.\\
	\\
	\item If the function f(x) =$x^3 + e^{\dfrac{x}{2}}$ and g(x) =$f^{-1}$(x), then the value of g'(1) is\\
	\\
	\item Let f($\theta$) = sin$\left(\tan^{-1}\left(\dfrac{\sin \theta}{\sqrt{\cos2\theta}}\right)\right)$, where\\
	 $-\dfrac{\pi}{4}<\theta<\dfrac{\pi}{4}$, Then the value of $\dfrac{d}{d(\tan \theta)}$(f($\theta$)) is 
	\item If y = (x+$\sqrt{1+x^2})^n$, then (1+$x^2$)$\dfrac{d^2y}{dx^2}$+x$\dfrac{dy}{dx}$ is\\
	(a) $n^2y$\\
	(b) -$n^2y$\\
	(c) -y\\
	(d) $2x^2y$\\
	\item If f(y) =$e^y$, g(y) = y; y$>$0 and\\
	 \\
	 F(t) = $\int\limits_0^t$ f(t-y)g(y)dy, then\\
	 (a) F(t) = t$e^{-t}$\\
	 (b) F(t) = 1-t$e^{-t}$(1+t)\\
	 (c) F(t) = $e^{t}$-(1+t)\\
	 (a) F(t) = t$e^{t}$\\
	\\
	\item f(x) = $x^n$, then the value of\\
	\\
	f(1)-$\dfrac{f'(1)}{1!} + \dfrac{f"(1)}{2!} + \dfrac{f"'}{3!}+.........\dfrac{(-1)^n f^n((1)}{n!}$ is\\
	\begin{itemize}
	\begin{multicols}{4}
	\item [(a)]1
	\item [(b)]$2^n$
	\item [(c)]$2^n$-1
	\item [(d)]0
	
	\end{multicols}
	\end{itemize}
	\item Let f(x) be a polynomial function of second degree. If f(1) = f(-1) and a,b,c are in A.P then f'(a),f'(b),f'(c) are in\\
	
	(a)Arthemetic-Geometric progression\\
	(b)A.P\\
	(c)G.P\\
	(d)H.P\\
	\\
	\item  If $e^{{y+e}^y+e^{y+...\infty}}$, x$>$0, then $\dfrac{dy}{dx}$ \\
	\begin{itemize}
	\begin{multicols}{4}
	\item [(a)]$\dfrac{1+x}{x}$
	\item [(b)]$\dfrac{1}{x}$
	\item [(c)]$\dfrac{1-x}{x}$
	\item [(d)]$\dfrac{x}{1+x}$
	
	\end{multicols}
	\end{itemize}
	\item The value of a for which sum of the squares of the roots of the equation $x^2$-(a-2)x-a-1 = 0 assume the least value is\\
	\begin{itemize}
	\begin{multicols}{4}
	\item [(a)]1
	\item [(b)]0
	\item [(c)]3
	\item [(d)]2
	
	\end{multicols}
	\end{itemize}
	\item If the roots of the equation $x^2$-bx+c = 0 be two consecutive integers, then $b^2$-4ac equals\\
	\begin{itemize}
	\begin{multicols}{4}
	\item [(a)]-2
	\item [(b)]3
	\item [(c)]2
	\item [(d)]1
	
	\end{multicols}
	\end{itemize}
	
	\item let f:R$\to$R be a differentiable function having f(2) = 6, f'(2) = $\dfrac{1}{48}$ Then $\displaystyle{\lim_{x \to f(x)}}$ $\int_6^{f(x)}\dfrac{4t^3}{x-2}$dt equals \begin{itemize}
	\begin{multicols}{4}
	\item [(a)]24
	\item [(b)]36
	\item [(c)]12
	\item [(d)]18
	
	\end{multicols}
	\end{itemize}
	\item The set of points where f(x) =$\dfrac{x}{1+|x|}$ is differentiable is\begin{itemize}
	\begin{multicols}{2}
	\item [(a)]$(-\infty,0) \cup (0,\infty)$
	\item [(b)]$(-\infty,-1) \cup (-1,\infty)$
	\item [(c)]$(-\infty,\infty)$
	\item [(d)]$(0,\infty)$
	
	\end{multicols}
	\end{itemize}
	\item If $x^m. y^n$ = $(x+y)^{m+n}$, then $\dfrac{dy}{dx}$ is
	\begin{itemize}
	\begin{multicols}{4}
	\item [(a)]$\dfrac{y}{x}$
	\item [(b)]$\dfrac{x+y}{xy}$
	\item [(c)]xy
	\item [(d)]$\dfrac{x}{y}$
	
	\end{multicols}
	\end{itemize}
	\item Let y be an implicit function of x defined by $x^{2x}-2x^xcot y-1$ = 0. Then y'(1) equals
	 \begin{itemize}
	\begin{multicols}{4}
	\item [(a)]1
	\item [(b)]log 2
	\item [(c)]-log 2
	\item [(d)]-1
	
	\end{multicols}
	\end{itemize}
	\item Let f:(-1,1)$\to$R be a differentiable function with f(0) = -1 and f'(0) = 1. Let g(x)=$[f(2f(x)+2))]^2$ Then g'(0) = 
	 \begin{itemize}
	\begin{multicols}{4}
	\item [(a)]-4
	\item [(b)]0
	\item [(c)]-2
	\item [(d)]4
	
	\end{multicols}
	\end{itemize}
	\item $\dfrac{d^2x}{dx^y}$ equals:
	 \begin{itemize}
	\begin{multicols}{2}
	\item [(a)]-$\left(\dfrac{d^2y}{dx^2}\right)^{-1} $ $\left(\dfrac{dy}{dx}\right)^{-3}$\\
	\item [(b)]$\left(\dfrac{d^2y}{dx^2}\right) $ $\left(\dfrac{dy}{dx}\right)^{-2}$\\
	\item [(c)]-$\left(\dfrac{d^2y}{dx^2}\right) $  $\left(\dfrac{dy}{dx}\right)^{-3}$\\
	\item [(d)]$\left(\dfrac{d^2y}{dx^2}\right)^{-1} $	\\
	\end{multicols}
	\end{itemize}
	\item If y= sec($\tan^{-1}x)$, then $\dfrac{dy}{dx}$ at x = 1 is equals to:
	\begin{itemize}
	\begin{multicols}{4}
	\item [(a)]$\dfrac{1}{\sqrt{2}}$
	\item [(b)]$\dfrac{1}{2}$
	\item [(c)]1
	\item [(d)]$\sqrt{2}$
	
	\end{multicols}
	\end{itemize}
	\item If g is the inverse of a function f and\\
	 $f^{-1}(x)$ = $\dfrac{1}{1+x^5}$ then g'(x) is equals to:
	\begin{itemize}
	\begin{multicols}{2}
	\item [(a)]$\dfrac{1}{1+\left(g(x)\right)^5}$\\
	\item [(b)]1+$\left(g(x)\right)^5$\\
	\item [(c)]1+$x^5$\\
	\item [(d)]$5x^4$\\
	
	\end{multicols}
	\end{itemize}
	\item If x = -1 and x = 2 are extreme points of f(x) =$\alpha log|x|+\beta x^2 +x$ then
	\begin{itemize}
	\begin{multicols}{2}
	\item [(a)]$\alpha = 2$, $\beta = -\dfrac{1}{2}$\\
	\item [(b)]$\alpha = 2$, $\beta = \dfrac{1}{2}$\\
	\item [(c)]$\alpha = -6$, $\beta = \dfrac{1}{2}$\\
	\item [(d)]$\alpha = -6$, $\beta = -\dfrac{1}{2}$\\
	
	\end{multicols}
	\end{itemize}
	\item If for x$\in$ $\left(0,\dfrac{1}{4}\right)$, the derivative of $tan^{-1}\left(\dfrac{6x\sqrt{x}}{1-9x^3}\right)$ is $\sqrt{x}$.g(x), then g(x) equals:
	\begin{itemize}
	\begin{multicols}{2}
	\item [(a)]$\dfrac{3}{1+9x^3}$\\
	\item [(b)]$\dfrac{9}{1+9x^3}$\\
	\item [(c)]$\dfrac{3x\sqrt{x}}{1-9x^3}$\\
	\item [(d)]$\dfrac{3x}{1-9x^3}$\\
	
	\end{multicols}
	\end{itemize}
	
\end{enumerate}
 
\section{Limits, Continuity and Differentiability}
\renewcommand{\theequation}{\theenumi}
\begin{enumerate}[label=\arabic*.,ref=\thesubsection.\theenumi]
\numberwithin{equation}{enumi}

\item Let $f(x)$=\resizebox{.33 \textwidth}{!} 
{$\begin{cases}
(x-1)^2\sin \dfrac{1}{(x-1)} - |x|,& \text{if $x\ne 1$}. \\
-1, & \text{if x=1}.
\end{cases}$} \\ 
\\be real-valued function. Then find the set of points where $f(x)$ is not differentiable ?

\item Let\begin{equation*}
f(x)=\begin{cases}
\dfrac{\left(x^3+x^2-16x+20\right)}{\left(x-2\right)^2}, &\text{if $x\ne 2$}\\
k, & \text{if x=2}
\end{cases}
\end{equation*}
If $f(x)$ is continuous for all x, then find k ?

\item[~]\item A discontinuous function $y=f(x)$ satisfying $x^2+y^2=4$ is given by $f(x)=$ .......

\item[~]\item
$\lim\limits_{x \to 1}\left(1-x\right)\tan\dfrac{\pi x}{2} $= ........

\item If f(x)=\resizebox{.34 \textwidth}{!} 
{$\begin{cases}\sin x, & \text{$x \ne n\pi$}, n=0,\pm 1, \pm 2, \pm 3, .....\\ 
2, & \text{otherwise} \end{cases}$} \item[~]\item[~]and g(x)=$\begin{cases}x^2+1, & \text{$x\ne 0,2$}\\ 4,& \text{$x=0$}\\ 5,& \text{$x=2$}\end{cases}$
then \item[~]\item[~]$\lim\limits_{x \to 0}g\left[f(x)\right]$ is ...........

\item[~]\item $\lim\limits_{x \to -\infty}\left[\dfrac{x^4\sin \left(\dfrac{1}{x}\right)+x^2}{\left(1+|x|^3\right)}\right]$ = .........

\item[~]\item If $f(9)=9, f'(9)=4$, then $\lim\limits_{x \to 9} \dfrac{\sqrt{f(x)}-3}{\sqrt{x}-3}$ equals ........

\item[~]\item $ABC$ is an isosceles triangle inscribed in a circle of radius $r$. If $AB$ = $AC$ and $h$ is the altitude from $A$ to $BC$ then the triangle $ABC$ has perimeter $P$ = $\left(2\left(\sqrt{2hr-h^2}\right)+\sqrt{2hr}\right)$ and area $A$= ........ also $\lim\limits_{h \to 0}\dfrac{A}{P^3}$= ........

\item[~]\item $\lim\limits_{x \to \infty}\left(\dfrac{x+6}{x+1}\right)^{x+4}$ = .......

\item[~] \item Let $f(x)=x|x|$. The set of points where $f(x)$ is twice differentiable is .........

\item[~] \item Let $f(x)$ = $\left[x\right]\sin\left(\dfrac{\pi}{\left[x+1\right]}\right)$, where \item[~]\item[~][$\bullet$] denotes the greatest integer function. The domain of $f$ is ........ and the points of discontinuity of $f$ in the domain are .........

\item[~] \item$\lim\limits_{x \to 0}\left(\dfrac{1+5x^2}{1+3x^2}\right)^{1/x^2}$ = ...........

\item[~] \item[~]\item Let $f(x)$ be a continuous function defined for $1\leq x\leq 3$. If $f(x)$ takes rational values for all $x$ and $f(2)$ = 10, then $f(1.5)$ = ........ 

\item[~] \item If $\lim\limits_{x \to a}\left[f(x)g(x)\right]$ exists then both $\lim\limits_{x \to a}f(x)$ \item[~] \item[~]and $\lim\limits_{x \to a}g(x)$ exist. (True / False)

\item[~] \item If $f(x)=\sqrt{\dfrac{x-\sin x}{x+\cos^2x}}$, then $\lim\limits_{x \to \infty}f(x)$ is
\begin{itemize}
\begin{multicols}{2}
\item[(a)] 0 \item[~]\item[(c)] 1 \item[(b)] $\infty$ \item[~]\item[(d)] none of these
\end{multicols}
\end{itemize}

\item For a real number $y$, let $\left[y\right]$ denotes the greatest integer less than or equal to $y$ : Then \\ \\the function $f(x)$= $\dfrac{\tan\left(\pi\left[x-\pi\right]\right)}{1+\left[x\right]^2}$ is \\
\begin{itemize}
\item[(a)] discontinuous at some $x$
\item[(b)] continuous at all $x$, but the derivative $f'(x)$ does not exist for some $x$
\item[(c)] $f'(x)$ exists for all $x$, but the second derivative $f''(x)$ does not exist for some $x$
\item[(d)] $f'(x)$ exists for all $x$
\end{itemize}

\item[~] \item There exists a function $f(x)$, satisfying $f(0)=1$, $f'(0)=-1$, $f(x)>0$ for all $x$, and
\begin{itemize}
\item[(a)] $f''(x)>0$ for all $x$\\ \item[(b)] $-1<f''(x)<0$ for all $x$ \item[~]\item[(c)] $-2\leq f''(x)\leq -1$ for all $x$ \item[~] \item[(d)] $f''(x)<-2$ for all $x$
\end{itemize}

\item[~] \item If G(x)= $-\sqrt{25-x^2}$ then $\lim\limits_{x \to 1}\dfrac{G(x)-G(1)}{x-1}$ has the value
\begin{itemize}
\begin{multicols}{2}
\item[(a)] $\dfrac{1}{24}$ \item[~] \item[(c)] $-\sqrt{24}$ \item[(b)] $\dfrac{1}{5}$ \item[~] \item[(d)] none of these
\end{multicols}
\end{itemize}

\item[~] \item If $f(a)\mathbin{=}2, f'(a)\mathbin{=}1, g(a)\mathbin{=}-1, g'(a)\mathbin{=}2$, \item[~]\item[~] then the value of $\lim\limits_{x \to a}\dfrac{g(x)f(a)-g(a)f(x)}{x-a}$ is
\begin{itemize}
\begin{multicols}{2}
\item[(a)] -5 \item[~] \item[(c)] 5 \item[(b)] $\dfrac{1}{5}$ \item[~] \item[(d)] none of these
\end{multicols}
\end{itemize}

\item[~] \item The function \[f(x)=\dfrac{ln(1+ax)-ln(1-bx)}{x}\] is not defined at $x\mathbin{=}0$. The value which should be assigned to $f$ at $x=0$ so that it is continuous at $x\mathbin{=}0$, is
\begin{itemize}
\begin{multicols}{2}
\item[(a)] $a-b$ \item[~] \item[(c)] $ln a - ln b$ \item[(b)] $a+b$ \item[~] \item[(d)] none of these
\end{multicols}
\end{itemize}

\item[~] \item$\lim\limits_{n \to \infty}\left\{\dfrac{1}{1-n^2}+\dfrac{2}{1-n^2}+.....+\dfrac{n}{1-n^2}\right\}$ \item[~] \item[~]is equal to
\begin{itemize}
\begin{multicols}{2}
\item[(a)] 0 \item[~] \item[(c)] $\dfrac{1}{2}$ \item[(b)] $-\dfrac{1}{2}$ \item[~] \item[(d)] none of these
\end{multicols}
\end{itemize}

\item[~] \item If $f(x)=\begin{cases}\mathbin{=}\dfrac{\sin[x]}{[x]}, & \text{$[x]\neq 0$} \\
0, & \text{$[x]=0$} \end{cases}$\item[~] \item[~]
Where [x] denotes the greatest integer less than or equal to $x$, then $\lim\limits_{x \to 0}f(x)$ equals
\begin{itemize}
\begin{multicols}{2}
\item[(a)] 1 \item[~] \item[(c)] -1 \item[(b)] 0 \item[~] \item[(d)] none of these
\end{multicols}
\end{itemize}

\item[~] \item Let $f:R\to R$ be differentiable function and $f(1)=4$. Then the value of $\lim\limits_{x \to 1}\int\limits_4^{f(x)}\dfrac{2t}{x-1} dt$ is
\begin{itemize}
\begin{multicols}{2}
\item[(a)] $8f'(1)$ \item[~] \item[(c)] $2f'(1)$ \item[(b)] $4f'(1)$ \item[~] \item[(d)] $f'(1)$
\end{multicols}
\end{itemize}

\item[~] \item Let [$\bullet$] denote the greatest integer function and $f(x)=[\tan^2x]$, then
\begin{itemize}
\item[(a)] $\lim\limits_{x \to 0}f(x)$ does not exist \item[~]
\item[(b)] $f(x)$ is continuous at $x=0$
\item[(c)] $f(x)$ is not differentiable at $x=0$
\item[(d)] $f'(0)=1$
\end{itemize}

\item[~] \item The function $f(x)=[x]\cos\left(\dfrac{2x-1}{2}\right)\pi$, [$\bullet$] denotes the greatest integer function, is discontinuous at
\begin{itemize}
\item[(a)] All x \item[(b)] All integer points \item[(c)] No x \item[(d)] x which is not an integer
\end{itemize}

\item[~] \item $\lim\limits_{n \to \infty}\dfrac{1}{n}\sum\limits_{r=1}^{2n}\dfrac{r}{\sqrt{n^2+r^2}}$ equals \item[~] \item[~]
\begin{itemize}
\begin{multicols}{2}
\item[(a)] $1+\sqrt{5}$ \item[~] \item[(c)] $-1+\sqrt{2}$ \item[(b)] $-1+\sqrt{5}$ \item[~] \item[(d)] $1+\sqrt{2}$
\end{multicols}
\end{itemize} 

\item The function $f(x)\mathbin{=}[x^2]-[x^2]$ (where $[y]$ is the greatest integer less than or equal to $y$), is discontinuous at
\begin{itemize}
\item[(a)] all integers
\item[(b)] all integers except 0 and 1
\item[(c)] all integers except 0
\item[(d)] all integers except 1
\end{itemize}

\item The function $f(x)\mathbin{=}\left(x^2-1\right)|x^2-3x+2|+\cos\left(|x|\right)$ is NOT differentiable at
\begin{itemize}
\begin{multicols}{4}
\item[(a)] -1 \item[(b)] 0 \item[(c)] 1 \item[(d)] 2
\end{multicols}
\end{itemize}

\item[~] \item $\lim\limits_{x \to 0}\dfrac{x \tan 2x-2x \tan x}{\left(1-\cos 2x\right)^2}$ is \item[~] \item[~]
\begin{itemize}
\begin{multicols}{4}
\item[(a)] 2 \item[(b)] -2 \item[(c)] 1/2 \item[(d)] -1/2
\end{multicols}
\end{itemize}

\item[~] \item For x $\in$ R, $\lim\limits_{x \to \infty}\left(\dfrac{x-3}{x+2}\right)^x$ =  \item[~] 
\begin{itemize}
\begin{multicols}{4}
\item[(a)] $e$ \item[(b)] $e^{-1}$ \item[(c)] $e^{-5}$ \item[(d)] $e^{5}$
\end{multicols}
\end{itemize}

\item $\lim\limits_{x \to 0}\dfrac{\sin(\pi\cos^2x)}{x^2}$ equals
\begin{itemize}
\begin{multicols}{4}
\item[(a)] $-\pi$ \item[(b)] $\pi$ \item[(c)] $\pi/2$ \item[(d)] 1
\end{multicols}
\end{itemize}

\item The left-hand derivative of $f(x)$=$[x]sin(\pi x)$ at $x\mathbin{=}k$, k and integer, is
\begin{itemize}
\begin{multicols}{2}
\item[(a)] $(-1)^k(k-1)\pi$ \item[~]\item[(c)] $(-1)^kk\pi$ \item[(b)] $(-1)^{k-1}(k-1)\pi$ \item[~]\item[(d)] $(-1)^{k-1}k\pi$
\end{multicols}
\end{itemize}

\item Let $f:R\to R$ be a function defined by $f(x)=max\{x,x^3\}$. The set fo all points where $f(x)$ is NOT differentiable is
\begin{itemize}
\begin{multicols}{2}
\item[(a)] \{-1,1\} \item[~]\item[(c)] \{0,1\}\item[(b)] \{-1,0\} \item[~] \item[(d)] \{-1,0,1\}
\end{multicols}
\end{itemize}

\item Which of the following functions is differentiable at x = 0 ?
\begin{itemize}
\begin{multicols}{2}
\item[(a)] $\cos(|x|)+|x|$ \item[~]\item[(c)] $\sin(|x|)+|x|$ \item[(b)] $\cos(|x|)-|x|$ \item[~] \item[(d)] $\sin(|x|)-|x|$
\end{multicols}
\end{itemize}

\item The domain of the derivative of the function $f(x)\mathbin{=}\begin{cases} \tan^{-1}x & \text{if $|x|\leq 1$}\\\dfrac{1}{2}(|x|-1) & \text{$|x|>1$}\end{cases}$ is
\begin{itemize}
\begin{multicols}{2}
\item[(a)] $R-\{0\}$ \item[~]\item[(c)] $R-\{1\}$\item[(b)] $R-\{-1\}$ \item[~] \item[(d)] $R-\{-1,1\}$
\end{multicols}
\end{itemize}

\item The integer for which \item[~]\item[~]$\lim\limits_{x \to 0}\dfrac{(\cos x-1)(\cos x-e^x)}{x^n}$ is a finite \item[~]\item[~]non-zero number is
\begin{itemize}
\begin{multicols}{4}
\item[(a)] 1 \item[(b)] 2 \item[(c)] 3 \item[(d)] 4
\end{multicols}
\end{itemize}

\item Let $f:R \to R$ be such that $f(1)=3$ and \\$f'(1)=6$. Then $\lim\limits_{x \to 0}\left(\dfrac{f(1+x)}{f(1)}\right)^{1/x}$ eqauls \\
\begin{itemize}
\begin{multicols}{4}
\item[(a)] 1 \item[(b)] $e^{1/2}$ \item[(c)] $e^2$ \item[(d)] $e^3$
\end{multicols}
\end{itemize}

\item If $\lim\limits_{x \to 0}\dfrac{((a-n)nx-\tan x)\sin nx}{x^2}=0$, where n is nonzero real number, then a is equal to
\begin{itemize}
\begin{multicols}{2}
\item[(a)] 0 \item[~] \item[(c)] $n$\item[(b)] $\dfrac{n+1}{n}$ \item[~] \item[(d)] $n+\dfrac{1}{n}$
\end{multicols}
\end{itemize}

\item[~] \item $\lim\limits_{h \to 0}\dfrac{f(2h+2+h^2)-f(2)}{f(h-h^2+1)-f(1)}$, given that \\ \item[~]$f'(2)=6$ and $f'(1)=4$
\begin{itemize}
\begin{multicols}{2}
\item[(a)] does not exist \item[(c)] is equal to 3/2 \item[(b)] is equal to -3/2 \item[(d)] is equal to 3
\end{multicols}
\end{itemize}

\item If (x) is differentiable and strictly increasing function, then the value of \item[~] \item[~]$\lim\limits_{x \to 0}\dfrac{f(x^2)-f(x)}{f(x)-f(0)}$ is \item[~]
\begin{itemize}
\begin{multicols}{4}
\item[(a)] 1 \item[(b)] 0 \item[(c)] -1 \item[(d)] 2
\end{multicols}
\end{itemize}

\item The function given by $y=||x|-1|$ is differentiable for all real numbers except the points
\begin{itemize}
\begin{multicols}{2}
\item[(a)] \{0,1,-1\} \item[(c)] 1 \item[(b)] $\pm 1$ \item[(d)] -1
\end{multicols}
\end{itemize}

\item If $f(x)$ is continuous and differentiable function and $f(1/n)=0 \forall n \geq 1$ and $n \in$ I, then
\begin{itemize}
\item[(a)] $f(x)=0, x\in (0,1]$
\item[(b)] $f(0)=0, f'(0)=0$
\item[(c)] $f(0)=0=f'(0), x \in (0,1]$
\item[(d)] $f(0)=0$ and $f'(0)$ need not to be zero
\end{itemize}

\item The value of \\$\lim\limits{x \to 0}\left((\sin x)^{1/x}+(1+x)^{\sin x}\right)$, where $x>0$ is
\begin{itemize}
\begin{multicols}{4}
\item[(a)] 0 \item[(b)] -1 \item[(c)] 1 \item[(d)] 2
\end{multicols}
\end{itemize}

\item Let $f(x)$ be differentiable on the interval $(0,\infty)$ such that $f(1)=1$, and \\ \item[~]$\lim\limits_{t \to x}\dfrac{t^2f(x)-x^2f(t)}{t-x}=1$ for each $x > 0$.\item[~] \item[~] Then $f(x)$ is
\begin{itemize}
\begin{multicols}{2}
\item[(a)] $\dfrac{1}{3x}+\dfrac{2x^2}{3}$ \item[~]\item[(c)] $\dfrac{-1}{x}+\dfrac{2}{x^2}$ \item[(b)] $\dfrac{-1}{3x}+\dfrac{4x^2}{3}$ \item[~]\item[(d)] $\dfrac{1}{x}$
\end{multicols}
\end{itemize}

\item[~] \item $\lim\limits_{x \to \dfrac{\pi}{4}}\dfrac{\int\limits_2^{\sec x^2}f(t) dt}{x^2-\dfrac{\pi ^2}{16}}$ equals \\ 
\begin{itemize}
\begin{multicols}{2}
\item[(a)] $\dfrac{8}{\pi}f(2)$ \item[~]\item[(c)] $\dfrac{2}{\pi}f\left(\dfrac{1}{2}\right)$ \item[(b)] $\dfrac{2}{\pi}f(2)$ \item[~]\item[(d)] $4f(2)$
\end{multicols}
\end{itemize}

\item[~] \item Let $g(x)=\dfrac{(x-1)^n}{\log\cos^m(x-1)}$; $0<x<2$, \\ \\m and n are integers, $m \ne 0, n>0$, let $p$ be the left hand derivative of $|x-1|$ at x = 1. If $\lim\limits_{x \to 1}g(x)=p$, then
\begin{itemize}
\begin{multicols}{2}
\item[(a)] $n=1, m=1$ \item[~]\item[(c)] $n=2, m=2$ \item[(b)] $n=1, m=1$ \item[~]\item[(d)] $n>2, m=n$
\end{multicols}
\end{itemize} 

\item[~] \item If $\lim\limits_{x \to 0}\left[1+xln(1+b^2)\right]^{1/x}=2bsin^2\theta$, $b>0$ and $\theta \in (-\pi, \pi]$, then the value of $\theta$ is
\begin{itemize}
\begin{multicols}{4}
\item[(a)] $\pm\dfrac{\pi}{4}$ \item[(b)] $\pm\dfrac{\pi}{3}$ \item[(c)] $\pm\dfrac{\pi}{6}$ \item[(d)] $\pm\dfrac{\pi}{2}$
\end{multicols}
\end{itemize}

\item[~] \item If $\lim\limits_{x \to \infty}\left(\dfrac{x^2+x+1}{x+1}-ax-b\right)=4$, then
\begin{itemize}
\begin{multicols}{2}
\item[(a)] $a=1, b=4$ \item[(c)] $a=2, b=-3$ \item[(b)] $a=1, b=-4$ \item[(d)] $a=2, b=3$
\end{multicols}
\end{itemize} 

\item[~] \item Let $f(x)=\begin{cases}
x^2\left|\cos\dfrac{\pi}{x}\right|, & \text{$x\ne 0$}\\
0, & \text{$x=0$}
\end{cases}$, \\ \item[~]$x \in R$ then $f$ is
\begin{itemize}
\item[(a)] differentiable both at x = 0 and at x = 2
\item[(b)] differentiable at x= 0 but not differentiable at x = 2
\item[(c)] not differentiable at x = 0 but differentiable at x = 2
\item[(d)] differentiable neither at x = 0 nor at x = 2
\end{itemize}

\item Let $\alpha(a)$ and $\beta(a)$ be the roots of the equa\\ \\tion
$\left(\sqrt[3]{1+a}-1\right)x^2$+$\left(\sqrt{1+a}-1\right)x$+\\ \\$\left(\sqrt[6]{1+a}-1\right)$=0
 where a$>$-1. Then \\ \\$\lim\limits_{a \to 0^+}\alpha(a)$ and $\lim\limits_{x \to 0^+}\beta(a)$ are
\begin{itemize}
\begin{multicols}{2}
\item[(a)] $-\dfrac{5}{2}$ and 1 \item[~] \item[(c)] $-\dfrac{7}{2}$ and 2 \item[(b)] $-\dfrac{1}{2}$ and -1 \item[~]\item[(d)] $-\dfrac{9}{2}$ and 3
\end{multicols}
\end{itemize} 

\item If x+$|y|$ = 2y, then y as a function of x is
\begin{itemize}
\item[(a)] defined for all real x
\item[(b)] continuous at x = 0
\item[(c)] differentiable for all x
\item[(d)] such that $\dfrac{dy}{dx}=\dfrac{1}{3}$ for $x<0$\\
\end{itemize}

\item If $f(x)=x(\sqrt{x}-\sqrt{x+1})$, then
\begin{itemize}
\item[(a)] f(x) is continuous but not differentiable at x = 0
\item[(b)] f(x) is differentiable at x = 0
\item[(c)] f(x) is not differentiable at x = 0
\item[(d)] none of these
\end{itemize}

\item The function $f(x)=1+|\sin x|$ is
\begin{itemize}
\item[(a)] continuous nowhere
\item[(b)] continuous everywhere
\item[(c)] differentiable nowhere
\item[(d)] not differentiable at $x=0$
\item[(e)] not differentiable at infinite number of points
\end{itemize}

\item Let [x] denote the greatest integer less than or equal to x. If $f(x)=[x \sin\pi x]$, then $f(x)$ is
\begin{itemize}
\item[(a)] continuous at $x=0$ \item[(b)] continuous in (-1,0) \item[(c)] differentiable at $x=1$ \item[(d)] differentiable in (-1,1) \item[(e)] none of these
\end{itemize}

\item The set of all points where the function $f(x)=\dfrac{x}{(1+|x|)}$ is differentiable, is
\begin{itemize}
\begin{multicols}{2}
\item[(a)] $(-\infty,\infty)$ \item[(c)] $(-\infty,0)\cup(0,\infty)$ \item[(e)] None \item[(b)] $[0,\infty)$ \item[(d)] $(0,\infty)$
\end{multicols}
\end{itemize}

\item The function \\$f(x)=\begin{cases}
|x-3|, & \text{$x\geq 1$}\\
\dfrac{x^2}{4}-\dfrac{3x}{2}+\dfrac{13}{4}, & \text{$x < 1$}
\end{cases}$ is\\
\begin{itemize}
\item[(a)] continuous at $x=1$ \item[(b)] differentiable at $x=1$ \item[(c)] continuous at $x=3$ \item[(d)] differentiable at $x=3$
\end{itemize}

\item If $f(x)=\dfrac{1}{2}x-1$, then on the interval $[0,\pi]$
\begin{itemize}
\item[(a)] $\tan[f(x)]$ and $1/f(x)$ are both\\ continuous
\item[(b)] $\tan[f(x)]$ and $1/f(x)$ are both\\ discontinuous
\item[(c)] $\tan[f(x)]$ and $f^{-1}(x)$ are both\\ continuous
\item[(d)] $\tan[f(x)]$ is continuous but $1/f(x)$ is not
\end{itemize}

\item The value of $\lim\limits_{x \to 0}\dfrac{\sqrt{\dfrac{1}{2}(1-\cos 2x)}}{x}$
\begin{itemize}
\begin{multicols}{2}
\item[(a)] 1 \item[(c)] 0 \item[(b)] -1 \item[(d)] none of these
\end{multicols}
\end{itemize}

\item The following functions are continuous on $(0,\pi)$
\begin{itemize}
\item[(a)] $\tan x$ \\
\item[(b)] $\int\limits_0^xt\sin\dfrac{1}{t}dt$\\ \\
\item[(c)] $\begin{cases}
1, & \text{$0<x\leq\dfrac{3\pi}{4}$} \\
2\sin\dfrac{2}{9}x, & \text{$\dfrac{3\pi}{4}<x<\pi$}
\end{cases}$ \\ \\
\item[(d)] $\begin{cases}
x\sin x, & \text{$0<x\leq\dfrac{\pi}{2}$} \\
\dfrac{\pi}{2}\sin (\pi+x), & \text{$\dfrac{\pi}{2}<x<\pi$}
\end{cases}$
\end{itemize}

\item Let $f(x)=\begin{cases}
0, & \text{$x<0$}\\
x^2, & \text{$x\geq 0$}
\end{cases}$ then for all $x$
\begin{itemize}
\item[(a)] $f'$ is differentiable \item[(b)] $f$ is differentiable \item[(c)] $f'$ is continuous \item[(d)] $f$ is continuous
\end{itemize}

\item Let $g(x)=xf(x)$, where \\
\\$f(x)=\begin{cases}
x\sin\dfrac{1}{x}, & \text{$x\ne 0$}\\
0, & \text{$x=0$}
\end{cases}$. At $x=0$ \\
\begin{itemize}
\item[(a)] $g$ is differentiable but $g'$ is not contionuous
\item[(b)] $g$ is differentiable while $f$ is not
\item[(c)] both $f$ and $g$ are differentiable
\item[(d)] $g$ is differentiable and $g'$ is continuous \\
\end{itemize}

\item The function $f(x)$=max$\{(1-x),(1+x),2\}$, $x \in(-\infty,\infty)$ is
\begin{itemize}
\item[(a)] continuous at all points
\item[(b)] diffrentiable at all points
\item[(c)] diffrentiable at all points except at $x=1$ and $x=-1$
\item[(d)] continuous at all points except at $x=1$ and $x=-1$, where it is discontinuous\\
\end{itemize}

\item Let $h(x)$=min$\{x,x^2\}$, for every real number of $x$, then
\begin{itemize}
\item[(a)] $h$ is continuous for all $x$
\item[(b)] $h$ is differentiable for all $x$
\item[(c)] $h'(x)=1$, for all $x>1$
\item[(d)] $h$ is differentiable at two values of $x$
\end{itemize}

\item[~] \item $\lim\limits_{x \to 1}\dfrac{\sqrt{1-\cos 2(x-1)}}{x-1}$
\begin{itemize}
\item[(a)] exists and it equals $\sqrt{2}$
\item[(b)] exists and it equals $-\sqrt{2}$
\item[(c)] does not exist because $x-1 \to 0$
\item[(d)] does not exist because the left hand limit is not equal to the right hand limit\\
\end{itemize}

\item If $f(x)$=min$\{1,x^2,x^3\}$, then
\begin{itemize}
\item[(a)] $f(x)$ is continuous $\forall x \in R$
\item[(b)] $f(x)$ is continuous and differentiable everywhere
\item[(c)] $f(x)$ is not differentiable at two points
\item[(d)] $f(x)$ is not differentiable at one point
\end{itemize}

\item Let $L=\lim\limits_{x \to 0}\dfrac{a-\sqrt{a^2-x^2}-\dfrac{x^2}{4}}{x^4}, a>0$. If $l$ is finite, then
\begin{itemize}
\begin{multicols}{2}
\item[(a)] $a=2$ \item[~] \item[(c)] $L=\dfrac{1}{64}$ \item[(b)] $a=1$ \item[~] \item[(d)] $L=\dfrac{1}{32}$
\end{multicols}
\end{itemize}

\item Let $f:R\to R$ be a function such that $f(x+y)=f(x)+f(y), \forall x, y\in R$. If $f(x)$ is differentiable at $x=0$, then
\begin{itemize}
\item[(a)] $f(x)$ is differentiable only in a finite interval containing zero
\item[(b)] $f(x)$ is continuous $\forall x \in R$
\item[(c)] $f(x)$ is constant $\forall x \in R$
\item[(d)] $f(x)$ is differentiable except at finitely many points 
\end{itemize} 

\item[~]\item If $f(x)=\begin{cases}
-x-\dfrac{\pi}{2}, & \text{$x\leq -\dfrac{\pi}{2}$}\\
-\cos x, & \text{$-\dfrac{\pi}{2}<x\leq 0$}\\
x-1, & \text{$0<x\leq 1$}\\
lnx, & \text{$x>1$}
\end{cases}$, then
\begin{itemize}
\item[(a)] $f(x)$ is continuous at $x=-\dfrac{\pi}{2}$
\item[(b)] $f(x)$ is not differentiable at $x=0$
\item[(c)] $f(x)$ is differentiable at $x=1$
\item[(d)] $f(x)$ is differentiable at $x=-\dfrac{3}{2}$
\end{itemize}

\item For every integer $n$, let $a_n$ and $b_n$ be real numbers. Let function $f:IR \to IR$ be given by
$$f(x)=\begin{cases}
a_n+\sin\pi x, & \text{$x \in [2n, 2n+1]$}\\
b_n+\cos\pi x, & \text{$x \in (2n-1, 2n)$}
\end{cases}$$
for all integers $n$. If $f$ is continuous, then which of the fllowing hold(s) for all $n$ ?
\begin{itemize}
\begin{multicols}{2}
\item[(a)] $a_{n-1}-b_{n-1}=0$ \item[(c)] $a_n-b_{n+1}=1$ \item[(b)] $a_n-b_n=1$ \item[(d)] $a_{n-1}-b_n=-1$
\end{multicols}
\end{itemize}

\item For $a \in R$(the set of all real numbers), a$\neq$-1,\\
\resizebox{0.88\hsize}{!}{$\lim\limits_{x \to \infty}\dfrac{(1^a+2^a+..+n^a)}{(n+1)^{a-1}[(na+1)+(na+2)+..+(na+n)]}$}\\=$\dfrac{1}{60}$. Then $a=$
\begin{itemize}
\begin{multicols}{4}
\item[(a)] 5 \item[(b)] 7 \item[(c)] $\dfrac{-15}{2}$ \item[(d)] $\dfrac{-17}{2}$
\end{multicols}
\end{itemize}

\item Let $f:[a,b] \to [1,\infty)$ be a continuous function and let $g:R \to R$ be defined as\\
$g(x)= \begin{cases}
0, & \text{if $x<a$}\\
\int\limits_a^xf(t)dt, & \text{if $a\leq x\leq b$}\\
\int\limits_a^bf(t)dt, & \text{if $x>b$} 
\end{cases}$; then
\begin{itemize}
\item[(a)] $g(x)$ is continuous but not differentiable at a
\item[(b)] $g(x)$ is differentiable on $R$
\item[(c)] $g(x)$ is continuous but not differentiable at b
\item[(d)] $g(x)$ is continuous and differentiable at either (a) or (b) but not both
\end{itemize}

\item For every pair of continuous functions $f,g:[0,1] \to R$ such that max$\{f(x):x \in [0,1]\}$=max$\{g(x):x \in [0,1]\}$, the correct statement(s) is(are)
\begin{itemize}
\item[(a)] $(f(c))^2+3f(c)=(g(c))^2+3g(c)$ for some $c \in [0,1]$\\
\item[(b)] $(f(c))^2+f(c)=(g(c))^2+3g(c)$ for some $c \in [0,1]$\\
\item[(c)] $(f(c))^2+3f(c)=(g(c))^2+g(c)$ for some $c \in [0,1]$\\
\item[(d)] $(f(c))^2=(g(c))^2$ for some $c \in [0,1]$\\
\end{itemize}

\item Let $g:R \to R$ be a differentiable function with $g(0)=0, g'(0)=0$ and $g'(1)\neq 0$. Let \\ \\$f(x)=\begin{cases}
\dfrac{x}{|x|}g(x), & \text{$x\neq 0$}\\
0, & \text{$x=0$}
\end{cases}$ \\ and $h(x)=e^{|x|}$ for all $x \in R$. Let $(foh)(x)$ denote $f(h(x))$ and $(hof)(x)$ denote $h(f(x))$. Then which of the following is(are) true ?
\begin{itemize}
\item[(a)] $f$ is differentiable at $x=0$
\item[(b)] $h$ is differentiable at $x=0$
\item[(c)] $foh$ is differentiable at $x=0$
\item[(d)] $hof$ is differentiable at $x=0$
\end{itemize}

\item Let $a, b\in\mathbb{R}$ and $f:\mathbb{R}\to\mathbb{R}$ be defined by \\ $f(x)=a\cos (|x^3-x|)+b|x|\sin(|x^3+x|)$.\\ Then f is
\begin{itemize}
\item[(a)] differentiable at $x=0$ if $a=0$ and $b=1$
\item[(b)] differentiable at $x=1$ if $a=1$ and $b=0$
\item[(c)] NOT differentiable at $x=0$ if $a=1$ and $b=0$
\item[(d)] NOT differentiable at $x=1$ if $a=1$ and $b=1$\\
\end{itemize}

\item Let $f:\left[-\dfrac{1}{2},2\right] \to \mathbb{R}$ and $g:\left[-\dfrac{1}{2},2\right] \to \mathbb{R}$\\
\\ be functions defined by $f(x)=[x^2-3]$ and $g(x)=|x|f(x)+|4x-7|f(x)$, where [y] denotes the greatest integer less than or equal to y for y $\in R$. Then
\begin{itemize}
\item[(a)] $f$ is discontinuous exactly at three points in $\left[-\dfrac{1}{2},2\right]$\\
\item[(b)] $f$ is discontinuous exactly at four points in $\left[-\dfrac{1}{2},2\right]$\\
\item[(c)] $g$ is NOT differentiable excatly at four points in $\left(-\dfrac{1}{2},2\right)$\\
\item[(d)] $g$ is NOT differentiable exactly at five points in $\left(-\dfrac{1}{2},2\right)$\\
\end{itemize}

\item Let [x] be the greatest integer less than or equal to x. Then, at which of the following point(s) the function $f(x)$ = $x\cos(\pi(x+[x]))$ is discontinuous ?
\begin{itemize}
\begin{multicols}{2}
\item[(a)] $x=-1$ \item[(c)] $x=1$ \item[(b)] $x=0$ \item[(d)] $x=2$
\end{multicols}
\end{itemize}

\item Let \resizebox{.33 \textwidth}{!} 
{$f(x)=\dfrac{1-x(1+|1-x|)}{|1-x|}\cos\left(\dfrac{1}{1-x}\right)$} for $x\neq 1$. Then
\begin{itemize}
\item[(a)] $\lim\limits_{x \to 1^-}f(x)=0$\\
\item[(b)] $\lim\limits_{x \to 1^-}f(x)$ does not exist\\
\item[(c)] $\lim\limits_{x \to 1^+}f(x)=0$\\
\item[(d)] $\lim\limits_{x \to 1^+}f(x)$ does not exist\\
\end{itemize}

\item Let $f:\mathbb{R} \to \mathbb{R}$ and $g:\mathbb{R} \to \mathbb{R}$ be two non-constant differentiable functions. If\\
$f'(x)=\left(e^{(f(x)-g(x))}\right)g'(x)$ for all $x \in \mathbb{R}$,
\\ and $f(1)=g(2)=1$, then which of the following statement(s) is(are) TRUE ?
\begin{itemize}
\begin{multicols}{2}
\item[(a)] $f(2)<1-log_e2$ \item[~]\item[(c)] $g(1)>1-log_e2$ \item[(b)] $f(2)>1-log_e2$ \item[~]\item[(d)] $g(1)<1-log_e2$
\end{multicols}
\end{itemize}

\item Let $f:R \to R$ given by\\ $f(x)$=\resizebox{0.35\textwidth}{0.05 \textheight}{
$\begin{cases}
x^5+5x^4+10x^3+10x^2+3x+1, & \text{$x<0$}\\
x^2-x+1, & \text{$0\leq x<1$}\\
\dfrac{2}{3}x^3-4x^2+7x-\dfrac{8}{3}, &\text{$1\leq x<3$}\\
(x-2)\log_e(x-2)-x+\dfrac{10}{3}, & \text{$x\geq 3$}
\end{cases}$} \\ \\then which of the following options is/are correct ?
\begin{itemize}
\item[(a)] $f'$ has a local maximum at $x=1$
\item[(b)] $f$ is increasing on $(-\infty,0)$
\item[(c)] $f'$ is NOT differentiable at $x=1$
\item[(d)] $f$ is onto
\end{itemize}

\item Let $f:R \to R$ be a function. We say that $f$ has \\
\textbf{PROPERTY 1} if $\lim\limits_{h \to 0}\dfrac{f(h)-f(0)}{\sqrt{|h|}}$ exists and is finite, and \\
\textbf{PROPERTY 2} if $\lim\limits_{h \to 0}\dfrac{f(h)-f(0)}{h^2}$ exists and is finite\\
\begin{itemize}
\item[(a)] $f(x)=x^{2/3}$ has \textbf{PROPERTY 1}
\item[(b)] $f(x)=\sin x$ has \textbf{PROPERTY 2}
\item[(c)] $f(x)=|x|$ has \textbf{PROPERTY 1}
\item[(d)] $f(x)=x|x|$ has \textbf{PROPERTY 2}
\end{itemize} \item[~]

\item Evaluate $\lim\limits_{x \to a}\dfrac{\sqrt{a+2x}-\sqrt{3x}}{\sqrt{3a+x}-2\sqrt{x}}$, $(a\neq 0)$ \item[~] \item[~]

\item $f(x)$ is the integral of $\dfrac{2\sin x-\sin 2x}{x^3}, x\neq 0$, find $\lim\limits_{x \to 0}f'(x)$.\item[~]

\item Evaluate $\lim\limits_{h \to 0}\dfrac{(a+h)^2\sin(a+h)-a^2\sin a}{h}$ \item[~]

\item Let $f(x+y)=f(x)+f(y)$ for all $x$ and $y$. If the function $f(x)$ is continuous at $x=0$, then show that $f(x)$ is continuous at all $x$. \item[~]

\item Use the formula $\lim\limits_{x \to 0}\dfrac{a^x-1}{x}=lna$ to find $\lim\limits_{x \to 0}\dfrac{2^x-1}{(1+x)^{1/2}-1}$ \item[~]

\item Let $f(x)=\begin{cases}
1+x, & \text{$0\leq x\leq 2$}\\
3-x, & \text{$2\leq x\leq 3$}
\end{cases}$\\ 
\\Determine the form of $g(x)$ = $fIf(x)$ and hence find the points of discontinuity of $g$, if any.\\

\item Let $f(x)=\begin{cases}
\dfrac{x^2}{2}, & \text{$0\leq x<1$}\\
2x^2-3x+\dfrac{3}{2}, &\text{$1\leq x \leq 2$}
\end{cases}$ \\ 
\\Discuss the continuity of $f, f'$ and $f''$ on [0,2].\\

\item Let $f(x)=x^3-x^2+x+1$ and\\ $g(x)=\begin{cases}
$max$\{f(t); 0\leq t \leq x\}, &\text{$0\leq x\leq 1$}\\
3-x, &\text{$0\leq x \leq 2$}
\end{cases}$\\ \\Discuss the continuity and differentiability f the function $g(x)$ in the interval (0,2).\\

\item Let $f(x)$ be defined in the interval [-2,2] such that $f(x)=\begin{cases}
-1, &\text{$-2\leq x\leq 0$}\\
x-1, &\text{$0<x\leq 2$}
\end{cases}$ \\ \\and $g(x)=f(|x|)+|f(x)|$. Test the differentiability of $g(x)$ in (-2,2).\\

\item Let $f(x)$ be a continuous and $g(x)$ be a discontinuous function. Prove that $f(x)+g(x)$ is a discontinuous function.\\

\item Let $f(x)$ be a function satisfying the condition $f(-x)=f(x)$ for all real $x$. If $f'(0)$ exists, find its value.\\

\item Find the values of a and b so that the function\\ \\
$f(x)=\begin{cases}
x+a\sqrt{2}\sin x, &\text{$0\leq x\leq \pi/4$}\\
2x\cot x+b, &\text{$\pi/4\leq x\leq \pi/2$}\\
a\cos 2x-b \sin x, &\text{$\pi/2<x\leq \pi$}
\end{cases}$\\ \\is continuous for $0\leq x\leq \pi$.\\

\item Draw a graph of the function $y=[x]+|1-x|, -1\leq x\leq 3$. Determine the points, if any, where this function is not differentiable.
\item[~] 

\item Let $f(x)=\begin{cases}
\dfrac{1-\cos 4x}{x^2}, &\text{$x<0$}\\
a, &\text{$x=0$}\\
\dfrac{\sqrt{x}}{\sqrt{16+\sqrt{x}}-4}, &\text{$x>0$}
\end{cases}$\\ \\ Determine the value of $a$, if possible, so that the function is continuous at $x=0$.\\

\item A function $f:R \to R$ satisfies the equation $f(x+y)=f(x)f(y)$ for all $x,y$ in $R$ and $f(x)\neq 0$ for any $x$ in $R$. Let the function be differentiable at $x=0$ and $f'(0)=2$. Show that $f'(x)=2f(x)$ for all $x$ in $R$. hence, determine $f(x)$. \item[~]

\item Find $\lim\limits_{x \to 0}\{\tan(\pi/4+x)\}^{1/x}$. \item[~]

\item Let \\$f(x)=\begin{cases}
\{1+|\sin x|\}^{a/|\sin x|}; &\text{$\dfrac{\pi}{6}<x<0$}\\
b: &\text{$x=0$}\\
e^{\tan 2x/\tan 3x}; &\text{$0<x<\dfrac{\pi}{6}$}
\end{cases}$\\ \\Determine $a$ and $b$ such that $f(x)$ is continuous at $x=0$.\item[~]

\item Let $f\left(\dfrac{x+y}{2}\right)=\dfrac{f(x)+f(y)}{2}$ for all real \item[~] \item[~]$x$ and $y$. If $f'(0)$ exists and equal $-1$ and $f(0)=1$, find $f(2)$.\\
 
\item Determine the values of $x$ for which the following function fails to be continuous or differentiable : \\ \\
$f(x)=\begin{cases}
1-x, &\text{$x<1$}\\
(1-x)(2-x), &\text{$1\leq x\leq 2$}\\
3-x, &\text{$x>2$}
\end{cases}$ \\ \\Justify your answer.\\

\item Let $f(x),x\geq 0$, be non-negative continuous function, and let $F(x)=\int\limits_0^xf(t)dt, x\geq 0$. If for some $c>0, f(x)\leq cF(x)$ for all $x\geq 0$, then show that $f(x)=0$ for all $x\geq 0$. \item[~]

\item Let $\alpha \in R$. Prove that a function $f:R \to R$ is differentiable at $\alpha$ if and only if there is a function $g: R \to R$ which is continuous at $\alpha$ and satisfies $f(x)-f(\alpha)=g(x)(x-\alpha)$ for all $x \in R$. \item[~]

\item Let $f(x)=\begin{cases}
x+1, &\text{if $x<0$}\\
|x-1|, &\text{if $x\geq 0$}
\end{cases}$ and\\ $g(x)=\begin{cases}
x+1, &\text{if $x<0$}\\
(x-1)^2+b, &\text{if $x\geq 0$}
\end{cases}$ where $a$ and $b$ are non-negative numbers. Determine the composite function $g o f$. If $(g o f) (x)$ is continuous for all real $x$, determine the values of $a$ and $b$. Further, for these values of $a$ and $b$, is $g o f$ differentiable at $x=0$ ? Justify your answer. \item[~]

\item If a function $f:[-2a,2a] \to R$ is an odd function such that $f(x)=f(2a-x)$ for $x \in [a,2a]$ and the left hand derivative at $x=a$ is 0 then find the left hand derivative at $x=-a$.

\item $f'(0)=\lim\limits_{n \to \infty}nf\left(\dfrac{1}{n}\right)$ and $f(0)=0$. Using this find\\ \\ $\lim\limits_{n \to \infty}\left((n+1)\dfrac{2}{\pi}\cos^{-1}\left(\dfrac{1}{n}\right)-n\right)$, \\ \\$\left|cos^{-1}\dfrac{1}{n}\right| < \dfrac{\pi}{2}$

\item[~] \item if $|c|\leq\dfrac{1}{2}$ and $f(x)$ is a differentiable function at $x=0$ given by \\ \\
$f(x)=\begin{cases}
b\sin^{-1}\left(\dfrac{c+x}{2}\right), &\text{$-\dfrac{1}{2}<x<0$}\\
\dfrac{1}{2}, &\text{$x=0$}\\
\dfrac{e^{ax/2}-1}{x}, &\text{$0<x<\dfrac{1}{2}$}
\end{cases}$\\ \\
Find the value of 'a' and prove that $64b^2=4-c^2$.\\

\item If $f(x-y)=f(x)\dot{•}g(y)-f(y)\dot{•}g(x)$ and $g(x-y)=g(x)\dot{•}g(y)-f(x)\dot{•}f(y)$ for all $x, y \in R$. If right hand derivative at $x=0$ exists for $f(x)$. Find derivative of $g(x)$ at $x=0$.\\

\item Let $f:[1,\infty) \to [2,\infty)$ be a differentiable functions such that $f(1)=2$. If $6\int\limits_1^xf(t)dt=3xf(x)-x^3$ for all $x \geq 14$, then the value of $f(2)$ is ?\\

\item The largest value of non-negative integer a for which \\
$\lim\limits_{x \to 1}\left\{\dfrac{-ax+\sin(x-1)+a}{x+\sin(x-1)-1}\right\}^{\dfrac{1-x}{1-\sqrt{x}}}=\dfrac{1}{4}$ \\ \\is
\\

\item Let $f: R \to R$ and $g: R \to R$ be respectively given by $f(x)=|x|+1$ and $g(x)=x^2+1$. Define $h: R \to R$ by \\ \\
$h(x)=\begin{cases}
max \{f(x).g(x)\}, &\text{if $x\leq 0$}\\
min \{f(x).g(x)\}, &\text{if $x>0$}
\end{cases}$\\ \\
The number of points at which $h(x)$ is not differentiable is\\

\item Let $m$ and $n$ be two positive integers greater than 1. If $\lim\limits_{\alpha \to 0}\left(\dfrac{e^{\cos(\alpha^n)}-e}{\alpha^m}\right)=-\left(\dfrac{e}{2}\right)$ then \\
\\the value of $\dfrac{m}{n}$ is

\item Let $\alpha, \beta \in \mathbb{R}$ be such that $\lim\limits_{x \to 0}\dfrac{x^2\sin(\beta x)}{\alpha x-\sin x}=1$. Then $6(\alpha+\beta)$ equals \item[~]

\item $\lim\limits_{x \to 0}\dfrac{\sqrt{1-\cos 2x}}{\sqrt{2}x}$ is
\begin{itemize}
\begin{multicols}{2}
\item[(a)] 1 \item[(c)] zero \item[(b)] -1 \item[(d)] does not exist
\end{multicols}
\end{itemize}

\item $\lim\limits_{x \to \infty}\left(\dfrac{x^2+5x+3}{x^2+x+3}\right)^x$
\begin{itemize}
\begin{multicols}{4}
\item[(a)]$e^4$ \item[(b)] $e^2$ \item[(c)] $e^3$ \item[(d)] 1
\end{multicols}
\end{itemize}

\item Let $f(x)=4$ and $f'(x)=4$. Then\\ \\ $\lim\limits_{x \to 2}\dfrac{xf(2)-2f(x)}{x-2}$ is given by
\begin{itemize}
\begin{multicols}{4}
\item[(a)]2 \item[(b)] -2 \item[(c)] -4 \item[(d)] 3
\end{multicols}
\end{itemize}

\item$\lim\limits_{n \to \infty}\dfrac{1^p+2^p+3^p+...+n^p}{n^{p+1}}$
\begin{itemize}
\begin{multicols}{2}
\item[(a)] $\dfrac{1}{p+1}$ \item[~]\item[(c)] $\dfrac{1}{1-p}$ \item[(b)] $\dfrac{1}{p}-\dfrac{1}{p-1}$ \item[~] \item[(d)] $\dfrac{1}{p+2}$
\end{multicols}
\end{itemize}

\item[~] \item $\lim\limits_{x \to 0}\dfrac{\log x^n-[x]}{[x]}, n \in N$, ([x] denotes greatest integer less than or equal to x)
\begin{itemize}
\begin{multicols}{2}
\item[(a)] has value -1 \item[(c)] has value 1 \item[(b)] has value 0 \item[(d)] does not exist
\end{multicols}
\end{itemize}

\item If $f(1)=1, f'(1)=2$, then $\lim\limits_{x \to 1}\dfrac{\sqrt{f(x)}-1}{\sqrt{x}-1}$ is
\begin{itemize}
\begin{multicols}{4}
\item[(a)] 2 \item[(b)] 4 \item[(c)] 1 \item[(d)] 1/2
\end{multicols}
\end{itemize}

\item$f$ is defined in [-5,5] as\\
$f(x)=\begin{cases}
x, & \text{if $x$ is rational}\\
-x, & \text{if $x$ is irrational}
\end{cases}$. Then
\begin{itemize}
\item[(a)] $f(x)$ is continuous at every $x$, except $x=0$
\item[(b)] $f(x)$ is discontinuous at every $x$, except $x=0$
\item[(c)] $f(x)$ is continuous everywhere
\item[(d)] $f(x)$ is discontinuous everywhere\\
\end{itemize}

\item$f(x)$ and $g(x)$ are two differentiable functions on [0,2] such that $f''(x)-g''(x)=0, f'(1)=2g'(1)=4f(2)=3g(2)=9$ then $f(x)-g(x)$ at x = 3/2 is
\begin{itemize}
\begin{multicols}{4}
\item[(a)] 0 \item[(b)] 2 \item[(c)] 10 \item[(d)] 5
\end{multicols}
\end{itemize}

\item If $(x+y)=f(x).f(y) \forall x, y$ and $f(5)=2$, \\ \\$f'(0)=3$, then $f'(5)$ is
\begin{itemize}
\begin{multicols}{4}
\item[(a)] 0 \item[(b)] 1 \item[(c)] 6 \item[(d)] 2
\end{multicols}
\end{itemize}

\item \resizebox{.9\hsize}{.025\vsize}{$\lim\limits_{n \to \infty}\dfrac{1+2^4+3^4+...+n^4}{n^5}-\lim\limits_{n \to \infty}\dfrac{1+2^3+3^3+..+n^3}{n^5}$} \item[~] \item[~]
\begin{itemize}
\begin{multicols}{4}
\item[(a)] $\dfrac{1}{5}$ \item[(b)] $\dfrac{1}{30}$ \item[(c)] Zero \item[(d)] $\dfrac{1}{4}$
\end{multicols}
\end{itemize}\item[~]

\item If $\lim\limits_{x \to 0}\dfrac{\log(3+x)-\log(3-x)}{x}=k$, the value of $k$ is
\begin{itemize}
\begin{multicols}{4}
\item[(a)] $-\dfrac{2}{3}$ \item[(b)] 0 \item[(c)] $-\dfrac{1}{3}$ \item[(d)] $\dfrac{2}{3}$
\end{multicols}
\end{itemize} \item[~]

\item The value of $\lim\limits_{x \to 0}\dfrac{\int\limits_0^{x^2}\sec^2tdt}{x\sin x}$ is
\begin{itemize}
\begin{multicols}{4}
\item[(a)] 0 \item[(b)] 3 \item[(c)] 2 \item[(d)] 1
\end{multicols}
\end{itemize}

\item Let $f(a)=g(a)=k$ and their nth derivatives $f^n(a),g^n(a)$ exist and are not equal for some $n$. Further if \\ \\
$\lim\limits_{x \to a}\dfrac{f(a)g(x)-f(a)-g(a)f(x)+f(a)}{g(x)-f(x)}$=4 \\ \\ then the value of $k$ is
\begin{itemize}
\begin{multicols}{4}
\item[(a)] 0 \item[(b)] 4 \item[(c)] 2 \item[(d)] 1
\end{multicols}
\end{itemize}\item[~]

\item$\lim\limits_{x \to \dfrac{\pi}{2}}\dfrac{\left[1-\tan\left(\dfrac{x}{2}\right)\right][1-\sin x]}{\left[1+\tan\left(\dfrac{x}{2}\right)\right][\pi-2x]^3}$ is
\begin{itemize}
\begin{multicols}{4}
\item[(a)] $\infty$ \item[(b)] $\dfrac{1}{8}$ \item[(c)] 0 \item[(d)] $\dfrac{1}{32}$
\end{multicols}
\end{itemize} \item[~]

\item If $f(x)=\begin{cases}
xe^{-\left(\dfrac{1}{|x|}+\dfrac{1}{x}\right)}, &\text{$x\neq 0$}\\
0, &\text{$x=0$}
\end{cases}$ then $f(x)$ is
\begin{itemize}
\item[(a)] discontinuous everywhere
\item[(b)] continuous as well as differentiable for all $x$
\item[(c)] continuous for all $x$ but not differentiable at $x=0$
\item[(d)] neither differentiable nor continuous at $x=0$
\end{itemize}\item[~]

\item If $\lim\limits_{x \to \infty}\left(1+\dfrac{a}{x}+\dfrac{b}{x^2}\right)^{2x}=e^2$, then the \\ \\values of $a$ and $b$, are
\begin{itemize}
\begin{multicols}{2}
\item[(a)] $a=1$ and $b=2$ \item[(c)] $a \in R, b=2$ \item[(b)] $a=1, b\in R$ \item[(d)] $a\in R, b\in R$
\end{multicols}
\end{itemize}

\item Let $f(x)=\dfrac{1-\tan x}{4x-\pi}, x\neq \dfrac{\pi}{4}, x\in\left[0,\dfrac{\pi}{2}\right]$. \\ \\If $f(x)$ is continuous in $\left[0,\dfrac{\pi}{2}\right]$, then $f\left(\dfrac{\pi}{4}\right)$ is
\begin{itemize}
\begin{multicols}{4}
\item[(a)] -1 \item[(b)] $\dfrac{1}{2}$ \item[(c)] $-\dfrac{1}{2}$ \item[(d)] 1
\end{multicols}
\end{itemize}\item[~]

\item $\lim\limits_{n \to \infty}\left[\dfrac{1}{n^2}\sec^2\dfrac{1}{n^2}+\dfrac{2}{n^2}\sec^2\dfrac{4}{n^2}...+\dfrac{1}{n}\sec^21\right]$ \\ \\equals
\begin{itemize}
\begin{multicols}{2}
\item[(a)] $\dfrac{1}{2}\sec 1$ \item[~]\item[~]\item[(c)] $\tan 1$ \item[(b)] $\dfrac{1}{2}cosec 1$ \item[~]\item[~]\item[(d)] $\dfrac{1}{2}\tan 1$
\end{multicols}
\end{itemize}

\item Let $\alpha$ and $\beta$ be the distinct roots of $ax^2+bx+c=0$, then $\lim\limits_{x \to \alpha}\dfrac{1-\cos(ax^2+bx+c)}{(x-\alpha)^2}$ is equal to
\begin{itemize}
\begin{multicols}{2}
\item[(a)] $\dfrac{\alpha^2}{2}(\alpha-\beta)^2$\item[~]\item[(c)] $\dfrac{-\alpha^2}{2}(\alpha-\beta)^2$ \item[(b)] 0 \item[~]\item[(d)] $\dfrac{1}{2}(\alpha-\beta)^2$
\end{multicols}
\end{itemize}

\item Suppose $f(x)$ is differentiable at $x=1$ and $\lim\limits_{h \to 0}\dfrac{1}{h}f(1+h)=5$, then $f'(1)$ equals
\begin{itemize}
\begin{multicols}{4}
\item[(a)] 3 \item[(b)] 4 \item[(c)] 5 \item[(d)] 6
\end{multicols}
\end{itemize}

\item Let $f$ be differentiable for all $x$. If $f(1)=-2$ and $f'(x)\geq 2$ for $x \in [1,6]$, then
\begin{itemize}
\begin{multicols}{2}
\item[(a)] $f(6)\geq 8$ \item[(c)] $f(6)<5$ \item[(b)] $f(6)<8$ \item[(d)] $f(6)=5$
\end{multicols}
\end{itemize}

\item If $f$ is a real valued differentiable function satisfying $|f(x)-f(y)|\leq (x-y)^2, x, y \in R$ and $f(0)=0$, then $f(1)$ equals
\begin{itemize}
\begin{multicols}{4}
\item[(a)] -1 \item[(b)] 0 \item[(c)] 2 \item[(d)] 1
\end{multicols}
\end{itemize}

\item Let $f:R \to R$ be a function defined by$f(x)=$min$\{x+1,|x|+1\}$, Then which of the following is true ?
\begin{itemize}
\item[(a)] $f(x)$ is differentiable everywhere
\item[(b)] $f(x)$ is not differentiable at $x=0$
\item[(c)] $f(x)\geq 1$ for all $x \in R$
\item[(d)] $f(x)$ is not differentiable at $x=1$\\
\end{itemize}

\item The function $f:R/\{0\} \to R$ is given by\\
$f(x)=\dfrac{1}{x}-\dfrac{2}{e^{2x}-1}$ can be made continuous at $x=0$ by defininig $f(0)$ as
\begin{itemize}
\begin{multicols}{4}
\item[(a)] 0 \item[(b)] 1 \item[(c)] 2 \item[(d)] -1
\end{multicols}
\end{itemize} \item[~]

\item Let $f(x)=\begin{cases}
(x-1)\sin\dfrac{1}{x-1}, &\text{if $x\neq 1$}\\
0, &\text{if $x=1$}
\end{cases}$ \\ \\Then which of the following is true ?
\begin{itemize}
\item[(a)] $f$ is neither differentiable at $x=0$ nor at $x=1$
\item[(b)] $f$ is differentiable at $x=0$ and at $x=1$
\item[(c)] $f$ is differentiable at $x=0$ but not at $x=1$
\item[(d)] $f$ is differentiable at $x=1$ but not at $x=0$\\
\end{itemize}

\item Let $f:R \to R$ be a positive increasing function with $\lim\limits_{x \to \infty}\dfrac{f(3x)}{f(x)}=1$. then $\lim\limits_{x \to \infty}\dfrac{f(2x)}{f(x)}=$
\begin{itemize}
\begin{multicols}{4}
\item[(a)] $\dfrac{2}{3}$ \item[(b)] $\dfrac{3}{2}$ \item[(c)] 3 \item[(d)] 1
\end{multicols}
\end{itemize}

\item $\lim\limits_{x \to2}\left(\dfrac{\sqrt{1-\cos\{2(x-2)\}}}{x-2}\right)$
\begin{itemize}
\begin{multicols}{2}
\item[(a)] equals $\sqrt{2}$ \item[(c)] equals $\dfrac{1}{\sqrt{2}}$ \item[(b)] equals $-\sqrt{2}$ \item[(d)] does not exist
\end{multicols}
\end{itemize}

\item The values of $p$ and $q$ for which the function\\
\\
$f(x)=\begin{cases}
\dfrac{\sin(p+1)x+\sin x}{x}, &\text{$x<0$}\\
q, &\text{$x=0$}\\
\dfrac{\sqrt{x+x^2}-\sqrt{x}}{x^{3/2}}, & \text{$x>0$}
\end{cases}$ \\
\\is continuous for all $x$ in $R$, are
\begin{itemize}
\begin{multicols}{2}
\item[(a)] $p=\dfrac{5}{2}, q=\dfrac{1}{2}$ \item[~]\item[~]\item[(c)] $p=-\dfrac{3}{2}, q=\dfrac{1}{2}$ \item[(b)] $p=\dfrac{1}{2}, q=\dfrac{3}{2}$ \item[~]\item[~]\item[(d)] $p=\dfrac{1}{2}, q=-\dfrac{3}{2}$ \\
\end{multicols}
\end{itemize}

\item Let $f:R \to [0,\infty)$ be such that $\lim\limits_{x \to 5}f(x)$ exists and $\lim\limits_{x \to 5}\dfrac{(f(x))^2-9}{\sqrt{|x-5|}}=0$. Then $\lim\limits_{x \to 5}f(x)$ equals
\begin{itemize}
\begin{multicols}{4}
\item[(a)] 0 \item[(b)] 1 \item[(c)] 2 \item[(d)] 3
\end{multicols}
\end{itemize}

\item If $f:R \to R$ is a function defined by $f(x)=[x]\cos\left(\dfrac{2x-1}{2}\right)\pi$, where [x] denotes the greatest integer function, then $f$ is
\begin{itemize}
\item[(a)] continuous for every real $x$
\item[(b)] discontinuous only at $x=0$
\item[(c)] discontinuous only at non-zero integral values of $x$
\item[(d)] continuous only at $x=0$\\
\end{itemize}

\item Consider the function, $f(x)=|x-2|+|x-5|, x \in R$.\\
\textbf{Statement-1:}$f'(4)=0$\\
\textbf{Statement-2:}$f$ is continuous in [2,5], differentiable in (2,5) and $f(2)=f(5)$
\begin{itemize}
\item[(a)] Statement-1 is false, Statement-2 is true
\item[(b)] Statement-1 is true, Statement-2 is true; Statement-2 is a correct explanation for Statement-1
\item[(c)] Statement-1 is true, Statement-2 is true; Statement-2 is \textbf{not} a correct explanation for Statement-1
\item[(d)] Statement-1 is true, Statement-2 is false
\end{itemize} \item[~]

\item $\lim\limits_{x \to 0}\dfrac{(1-\cos 2x)(3+\cos x)}{x\tan 4x}$ is equal to
\begin{itemize}
\begin{multicols}{4}
\item[(a)] $-\dfrac{1}{4}$ \item[(b)] $\dfrac{1}{2}$ \item[(c)] 1 \item[(d)] 2
\end{multicols}
\end{itemize} \item[~]

\item $\lim\limits_{x \to 0}\dfrac{\sin(\pi\cos^2x)}{x^2}$ is equal to
\begin{itemize}
\begin{multicols}{4}
\item[(a)] $-\pi$ \item[(b)] $\pi$ \item[(c)] $\dfrac{\pi}{2}$ \item[(d)] 1
\end{multicols}
\end{itemize}

\item If the function,\\ \\
$g(x)=\begin{cases}
k\sqrt{x+1}, &\text{$0\leq x \leq 3$}\\
mx+2, &\text{$3<x\leq 5$}
\end{cases}$ is \\ \\differentiable, then the value of k + m is
\begin{itemize}
\begin{multicols}{4}
\item[(a)] $\dfrac{10}{3}$ \item[(b)] 4 \item[(c)] 2 \item[(d)] $\dfrac{16}{5}$
\end{multicols}
\end{itemize}

\item For $x \in R, f(x)=|\log 2-\sin x|$ and $g(x)=f(f(x))$, then
\begin{itemize}
\item[(a)] $g'(0)=-\cos(\log 2)$
\item[(b)] $g$ is differentiable at $x=0$ and $g'(0)=-\sin(\log 2)$
\item[(c)] $g$ is not differentiable at $x=0$
\item[(d)] $g'(0)=\cos(\log 2)$
\end{itemize}

\item $\lim\limits_{n \to \infty}\left(\dfrac{(n+1)(n+2)....3n}{n^{2n}}\right)^{\dfrac{1}{n}}$ is equal to
\begin{itemize}
\begin{multicols}{2}
\item[(a)] $\dfrac{9}{e^2}$ \item[~] \item[(c)] $\dfrac{18}{e^4}$ \item[(b)] $3\log 3-2$ \item[~]\item[(d)] $\dfrac{27}{e^2}$
\end{multicols}
\end{itemize}

\item Let $p=\lim\limits_{x \to 0^+}\left(1+\tan^2\sqrt{x}\right)^{\frac{1}{2x}}$ then $\log p$ is equal to
\begin{itemize}
\begin{multicols}{2}
\item[(a)] $\dfrac{1}{2}$ \item[~] \item[(c)] 2 \item[(b)] $\dfrac{1}{4}$ \item[~]\item[(d)] 1
\end{multicols}
\end{itemize}

\item $\lim\limits_{x \to \dfrac{\pi}{2}}\dfrac{\cot x-\cos x}{(\pi-2x)^3}$ equals
\begin{itemize}
\begin{multicols}{4}
\item[(a)] $\dfrac{1}{4}$ \item[(b)] $\dfrac{1}{24}$ \item[(c)] $\dfrac{1}{16}$ \item[(d)] $\dfrac{1}{8}$
\end{multicols}
\end{itemize}

\item For each $t \in R$, let [t] be the greatest integer less than or equal to t/. Then\\
\\$\lim\limits_{x \to 0^+}x\left(\left[\dfrac{1}{x}\right]+\left[\dfrac{2}{x}\right]+...+\left[\dfrac{15}{x}\right]\right)$
\begin{itemize}
\begin{multicols}{2}
\item[(a)] is equal to 15 \item[(c)] does not exist(in R) \item[(b)] is equal to 120 \item[(d)] is equal to 0
\end{multicols}
\end{itemize}

\item Let $S$ = $t \in R:f(x)=|x-\pi|(e^{|x|}-1)\sin |x|$ is not differentiable at t. Then the set S is equal to
\begin{itemize}
\begin{multicols}{2}
\item[(a)] \{0\} \item[(c)] $\{0,\pi\}$ \item[(b)] $\{\pi\}$ \item[(d)] $\phi$(an empty set)
\end{multicols}
\end{itemize}

\item $\lim\limits_{y \to 0}\dfrac{\sqrt{1+\sqrt{1+y^4}}-\sqrt{2}}{y^4}$
\begin{itemize}
\item[(a)] exists and equals $\dfrac{1}{4\sqrt{2}}$\\ \\
\item[(b)] exists and equals $\dfrac{1}{2\sqrt{2}(\sqrt{2}+1)}$\\ \\
\item[(c)] exists and equals $\dfrac{1}{2\sqrt{2}}$\\ 
\item[(d)] does not exist
\end{itemize}

\item Let $f: R \to R$ be a function defined as\\ \\
$f(x)=\begin{cases}
5, &\text{if $x\leq 1$}\\
a+bx, &\text{if $1<x<3$}\\
b+5x, &\text{if $3\leq x<5$}\\
30, &\text{if $x\geq 5$}
\end{cases}$ \\ \\Then f is
\begin{itemize}
\item[(a)] continuous if a = 5 and b = 5
\item[(b)] continuous if a = -5 and b = 10
\item[(c)] continuous if a = 0 and b = 5
\item[(d)] not continuous for any values of a and b
\end{itemize}

\item if the function f is defined on $\left(\dfrac{\pi}{6},\dfrac{\pi}{3}\right)$ by \\ \\
$f(x)=\begin{cases}
\dfrac{\sqrt{2}\cos x-1}{\cot x-1}, &\text{$x\neq \dfrac{\pi}{4}$}\\
k, &\text{$x=\dfrac{\pi}{4}$}
\end{cases}$\\ \\ is continuous, then k is equal to
\begin{itemize}
\begin{multicols}{4}
\item[(a)] 2 \item[(b)] $\dfrac{1}{2}$ \item[(c)] 1 \item[(d)] $\dfrac{1}{\sqrt{2}}$
\end{multicols}
\end{itemize}

\item Let $f(x)=15-|x-10|; x \in R$. Then the set of all values of x, at which the function, $g(x)=f(f(x))$ is not differentiable, is
\begin{itemize}
\begin{multicols}{2}
\item[(a)] \{5,10,15\} \item[(c)] \{5,10,15,20\} \item[(b)] \{10,15\} \item[(d)] \{10\}
\end{multicols}
\end{itemize}


\item In this questions there are entries in columns I and II. Each entry in column I is related to exactly one entry in column II.
\begin{align*}
\begin{tabular}{ll}
\textbf{Column I} &\textbf{Column II}\\
(A) $\sin(\pi[x])$ & (p) differentiable everywhere \\
(B) $\sin(\pi(x-[x]))$ & (q) nowhere differentiable\\
 &(r) not differentiable at 1 and -1
\end{tabular}
\end{align*}

\item In the following [x] denotes the greatest integer less than or equal to $x$.Match the functions in \textbf{Column I} with the properties in \textbf{Column II}
\begin{align*}
\begin{tabular}{ll}
\textbf{Column I} &\textbf{Column II}\\
(A) $x|x|$ & (p) continuous in (-1,1) \\
(B) $\sqrt{|x|}$ & (q) differentiable in (-1,1)\\
(C) $x+[x]$ &(r) strictly increasing in (-1,1)\\
(D) $|x-1|+|x+1|$ &(s) not differentiable at least at one point (-1,1)\\
\end{tabular}
\end{align*}

\item Let $f_1:R \to R, f_2:[0,\infty) \to R, f_3:R \to R$ and $f_4:R \to [0,\infty)$ be defined by $f_1(x)=\begin{cases}
|x|, &\text{if $x<0$}\\
e^x, &\text{if $x\geq 0$}
\end{cases}$ ;
$f_2(x)=x^2$; $f_3(x)=\begin{cases}
\sin x, &\text{if $x<0$}\\
x, &\text{if $x\geq 0$};
\end{cases}$ and 
\clearpage
\newpage$f_4(x)=\begin{cases}
f_2(f_1(x)), &\text{if $x<0$}\\
f_2(f_1(x))-1, &\text{if $x\geq 0$};
\end{cases}$
\begin{align*}
\begin{tabular}{ll}
\textbf{List-I} &\textbf{List-II}\\
(P) $f_4$ is & 1. onto but not one-one \\
(Q) $f_3$ is & 2. Neither continuous nor one-one\\
(R) $f_2of_1$ is &3. differentiable but not one-one\\
(S) $f_2$ is &4. continuous and one-one
\end{tabular}
\end{align*}
\begin{itemize}
\item[(a)] P $\to$ 3; Q $\to$ 1; R $\to$ 4; S $\to$ 2
\item[(b)] P $\to$ 1; Q $\to$ 3; R $\to$ 4; S $\to$ 2
\item[(c)] P $\to$ 3; Q $\to$ 1; R $\to$ 2; S $\to$ 4
\item[(d)] P $\to$ 1; Q $\to$ 3; R $\to$ 2; S $\to$ 4
\end{itemize} \item[~]

\item let $f_1:\mathbb{R} \to \mathbb{R}, f_2:\left(-\dfrac{\pi}{2},\dfrac{\pi}{2}\right) \to \mathbb{R}$\\ $f_3:\left(-1,e^{\dfrac{\pi}{2}-2}\right)$ and $f_4:\mathbb{R} \to \mathbb{R}$ be functions defined by
\begin{itemize}
\item[(i)] $f_1(x)=\sin\left(\sqrt{1-e^{-x^2}}\right)$
\end{itemize}
\begin{itemize}
\item[(ii)] $f_2(x)=\begin{cases}
\dfrac{|\sin x|}{tan^{-1}x}, &\text{if $x \neq 0$}\\
1, &\text{if $x=0$}
\end{cases}$, where the inverse trignometric function \\ \\$\tan^{-1}x$ assumes values in $\left(-\dfrac{\pi}{2},\dfrac{\pi}{2}\right)$\item[~]
\item[(iii)] $f_3(x)=[\sin(\log_e(x+2))]$, where for $t \in \mathbb{R}, [t]$ denotes the greatest integer less than or equal to $t$ \item[~]
\item[(iv)] $f_4(x)=\begin{cases}
x^2\sin\left(\dfrac{1}{x}\right), &\text{if $x \neq 0$}\\
0, &\text{if $x=0$}
\end{cases}$
\end{itemize}
\begin{align*}
\begin{tabular}{ll}
\textbf{List-I} &\textbf{List-II}\\
(P) The function $f_1$ is & 1. \textbf{NOT} continuous at $x=0$ \\
(Q) The function $f_2$ is & 2. Continuous at $x=0$ and \textbf{NOT} differentiable at $x=0$\\
(R) The function $f_3$ is &3. differentiable at $x=0$ and its derivative is \textbf{NOT} continuous at $x=0$\\
(S) The function $f_4$ is &4. differentiable at $x=0$ and its derivative is continuous at $x=0$
\end{tabular}
\end{align*}
\begin{itemize}
\item[(a)] P $\to$ 2; Q $\to$ 3; R $\to$ 1; S $\to$ 4
\item[(b)] P $\to$ 4; Q $\to$ 1; R $\to$ 2; S $\to$ 3
\item[(c)] P $\to$ 4; Q $\to$ 2; R $\to$ 1; S $\to$ 3
\item[(d)] P $\to$ 2; Q $\to$ 1; R $\to$ 4; S $\to$ 3
\end{itemize}
\end{enumerate}
 
\section{Definite integrals}
\renewcommand{\theequation}{\theenumi}
\begin{enumerate}[label=\arabic*.,ref=\thesubsection.\theenumi]
\numberwithin{equation}{enumi}

\item 
\begin{align*}
f(x) = \begin{vmatrix}
\sec x & \cos x & \sec^{2}x + \cot x \cosec x \\ \cos^{2}x & \cos^{2}x & \cosec^{2}x \\ 1 & \cos^{2}x & \cos^{2}x
\end{vmatrix}
\end{align*}
Then $\int_{0}^{\pi/2}f(x)dx$=......

\item The integral $\int_{0}^{1.5}[x^{2}]dx$, Where [ ] denotes the greatest integer function, equals.......

\item The value of 
\begin{align*}
\int_{-2}^{2}|1 - x^{2}|dx = 
\end{align*}

\item The value of 
\begin{align*}
\int_{\pi/4}^{3\pi/4}\frac{\phi}{1 + \sin\phi}d\phi = 
\end{align*}

\item The value of 
\begin{align*}
\int_{2}^{3}\frac{\sqrt{x}}{\sqrt{5 - x} + \sqrt{x}}dx = 
\end{align*}

\item If for non-zero x, 
\begin{align*}
af(x) + bf\left(\frac{1}{x}\right) = \frac{1}{x} - 5
\end{align*}
where $a \neq b$, then $\int_{1}^{2}f(x)dx$ = ...............

\item For $n > 0$, 
\begin{align*}
\int_{0}^{2\pi}\frac{x\sin^{2n}x}{\sin^{2n}x + \cos^{2n}x}dx = 
\end{align*}

\item The value of 
\begin{align*}
\int_{1}^{e^{37}}\frac{\pi \sin(\pi lnx)}{x}dx = 
\end{align*}

\item Let $\frac{d}{dx}F(x) = \frac{e^{\sin x}}{x}$, $x > 0$. If
\begin{align*}
\int_{1}^{4} \frac{2e^{\sin x^{2}}}{x} = F(k) - F(1)
\end{align*}
then one of the possible values of k is..........

\textbf{ True/False}

\item The value of the integral 
\begin{align*}
\int_{0}^{2a}\frac{f(x)}{f(x) + f(2a - x)}dx
\end{align*}
is equal to a.

\textbf{ MCQs with One Correct Answer}

\item The value of the definite integral $\int_{0}^{1}(1 + e^{-x^{2}})dx$ is
\begin{enumerate}
\item -1
\item 2
\item $1 + e^{-1}$
\item None of these
\end{enumerate} 

\item Let a, b, c be non-zero real numbers such that
\begin{align*}
\int_{0}^{1}(1+\cos^{8}x)(ax^2+bx+c)dx\\=\int_{0}^{2}(1+\cos^{8}x)(ax^2+bx+c)dx
\end{align*}
Then the quadratic equation $ax^2 + bx + c = 0$ has
\begin{enumerate}
\item no root in (0, 2)
\item at least one root in (0, 2)
\item a double root in (0, 2)
\item two imaginary roots
\end{enumerate}

\item The area bounded by the curves $y = f(x)$, the x-axis and the ordinates x = 1 and x = b is $(b - 1)\sin(3b + 4)$. Then $f(x)$ is
\begin{enumerate}
\item $(x - 1)\cos(3x + 4)$
\item $\sin(3x + 4)$
\item $\sin(3x + 4) + 3(x - 1)\cos(3x + 4)$
\item None of these
\end{enumerate}

\item The value of the integral
\begin{align*}
\int_{0}^{\pi/2}\frac{\sqrt{\cot x}}{\sqrt{\cot x} + \sqrt{\tan x}}dx = 
\end{align*}
\begin{enumerate}
\item $\pi/4$
\item $\pi/2$
\item $\pi$
\item None of these
\end{enumerate}

\item For any integer n the integral
\begin{align*}
\int_{0}^{\pi}e^{\cos^{2}x}\cos^{3}(2n + 1)dx
\end{align*}
has the value
\begin{enumerate}
\item $\pi$
\item 1
\item 0
\item None of these
\end{enumerate}

\item Let $f: R \to R$ and $g: R \to R$ be continuous functions. Then the value of the integral
\begin{align*}
\int_{-\pi/2}^{\pi/2}[f(x) + f(-x)][g(x) - g(-x)]dx = 
\end{align*}
\begin{enumerate}
\item $\pi$
\item 1
\item -1
\item 0
\end{enumerate}

\item The value of 
\begin{align*}
\int_{0}^{\pi/2}\frac{dx}{1 + \tan^{3}x} = 
\end{align*}
\begin{enumerate}
\item 0
\item 1
\item $\pi/2$
\item $\pi/4$
\end{enumerate}

\item If $f(x) = A\sin(\frac{\pi x}{2}) + B$, $f'(\frac{1}{2}) = \sqrt{2}$ and $\int_{0}^{1}f(x)dx = \frac{2A}{\pi}$, then constants A and B are
\begin{enumerate}
\item $\frac{\pi}{2}$ and $\frac{\pi}{2}$
\item $\frac{2}{\pi}$ and $\frac{3}{\pi}$
\item 0 and $\frac{-4}{\pi}$
\item $\frac{4}{\pi}$ and 0
\end{enumerate}

\item The value of 
$\int_{\pi}^{2\pi}[2\sin x]dx$
where [ ] represents the greatest integer funcion is
\begin{enumerate}
\item $\frac{-5\pi}{3}$
\item $-\pi$
\item $\frac{5\pi}{3}$
\item $-2\pi$
\end{enumerate}

\item If 
\begin{align*}
g(x) = \int_{0}^{x}\cos^{4}t dt
\end{align*}
then $g(x + \pi)$ equals
\begin{enumerate}
\item $g(x) + g(\pi)$
\item $g(x) - g(\pi)$
\item $g(x)g(\pi)$
\item $\frac{g(x)}{g(\pi)}$
\end{enumerate}

\item
\begin{align*}
\int_{\pi/4}^{3\pi/4}\frac{dx}{1 + \cos x} = 
\end{align*}
\begin{enumerate}
\item 2
\item -2
\item 1/2
\item -1/2
\end{enumerate}

\item If for a real number y, [y] is the greatest integer less than or equal to y, then the value of the integral
\begin{align*}
\int_{\pi/2}^{3\pi/2}[2\sin x]dx = 
\end{align*}
\begin{enumerate}
\item -$\pi$
\item 0
\item $\pi/2$
\item $\pi/2$
\end{enumerate}

\item Let
\begin{align*}
g(x) = \int_{0}^{x}f(t) dt
\end{align*}
where f is such that $\frac{1}{2} \leq f(t) \leq 1$ for $t \in [0, 1]$ and $0 \leq f(t) \leq \frac{1}{2}$, for $t \in [1, 2]$. Then g(2) satisfies the inequality
\begin{enumerate}
\item $\frac{3}{2} \leq g(2) < \frac{1}{2}$
\item $0 \leq g(2) < 2$
\item $\frac{3}{2} < g(2) \leq \frac{5}{2}$
\item $2 < g(2) < 4$
\end{enumerate}

\item If 
\begin{align*}
f(x) = 
\left\lbrace
\begin{array}{ll}
      e^{\cos x}\sin x & for |x| \leq 2 \\
      2 & otherwise \\
\end{array} 
\right\rbrace
\end{align*} 
then $\int_{-2}^{3}f(x)dx$ = 
\begin{enumerate}
\item 0
\item 1
\item 2
\item 3
\end{enumerate}

\item The value of the integral 
\begin{align*}
\int_{e^{-1}}^{e^{2}}\begin{vmatrix} \frac{log_ex}{x} \end{vmatrix}dx =
\end{align*}
\begin{enumerate}
\item 3/2
\item 5/2
\item 3
\item 5
\end{enumerate}

\item The value of 
\begin{align*}
\int_{-\pi}^{\pi}\frac{\cos^{2}x}{1 + a^{x}}dx = 
\end{align*}
\begin{enumerate}
\item $\pi$
\item $a\pi$
\item $\pi/2$
\item $2\pi$
\end{enumerate}

\item The area bounded by the curves $y = |x| - 1$ and $y = -|x| + 1$ is
\begin{enumerate}
\item 1
\item 2
\item $2\sqrt{2}$
\item 4
\end{enumerate}

\item Let
\begin{align*}
f(x) = \int_{1}^{x}\sqrt{2 - t^{2}}dt
\end{align*}
Then the real roots of the equation $x^2 - f'(x) = 0$ are
\begin{enumerate}
\item $\pm 1$
\item $\pm \frac{1}{\sqrt{2}}$
\item $\pm \frac{1}{2}$
\item 0 and 1
\end{enumerate}

\item Let  $T > 0$ be a fixed real number. Suppose f is a continuous function such that for all $x \in R$, $f(x + T) = f(x)$. If $I = \int_{0}^{T}f(x)dx$ then the value of $\int_{3}^{3 + 3T}f(2x)dx$ is
\begin{enumerate}
\item 3/2I
\item 2I
\item 3I
\item 6I
\end{enumerate}

\item The integral
\begin{align*}
\int_{-1/2}^{1/2}([x] + ln(\frac{1 + x}{1 - x}))dx = 
\end{align*}
\begin{enumerate}
\item $\frac{-1}{2}$
\item 0
\item 1
\item $2ln(\frac{1}{2})$
\end{enumerate}

\item If 
$l(m, n) = \int_{0}^{1}t^{m}(1 + t)^{n}dt$
then the expression for l(m, n) in terms of l(m + 1, n - 1) is
\begin{enumerate}
\item $\frac{2^n}{m + 1} - \frac{n}{m + 1}l(m + 1, n -1)$
\item $\frac{n}{m + 1}l(m + 1, n -1)$
\item $\frac{2^n}{m + 1} + \frac{n}{m + 1}l(m + 1, n -1)$
\item $\frac{m}{m + 1}l(m + 1, n -1)$
\end{enumerate}

\item If 
\begin{align*}
f(x) = \int_{x^{2}}^{x^{2} + 1}e^{-t^{2}}dt
\end{align*}
then, f(x) increases in
\begin{enumerate}
\item (-2, 2)
\item No value of x
\item $(0, \infty)$
\item $(-\infty, 0)$
\end{enumerate}

\item The area bounded by the curves $y = \sqrt{x}$, 2y + 3 = x and x-axis in the $1^{st}$ quadrant is
\begin{enumerate}
\item 9
\item 27/4
\item 36
\item 18
\end{enumerate}

\item If f(x) is differentiable and 
\begin{align*}
\int_{0}^{t^{2}}xf(x)dx = \frac{2}{5}t^{5}
\end{align*}
then $f(\frac{4}{25})$ equals
\begin{enumerate}
\item 2/5
\item -5/2
\item 1
\item 5/2
\end{enumerate}

\item The value of the integral
\begin{align*}
\int_{0}^{1}\sqrt{\frac{1 - x}{1 + x}}dx = 
\end{align*}
\begin{enumerate}
\item $\frac{\pi}{2} + 1$
\item $\frac{\pi}{2} - 1$
\item -1
\item 1
\end{enumerate}

\item The area enclosed between the curves $y = ax^{2}$  and $x = ay^{2}(a > 0)$ is 1 sq. unit, then the value of a is
\begin{enumerate}
\item $1/\sqrt{3}$
\item 1/2
\item 1
\item 1/3
\end{enumerate}

\item 
\begin{align*}
\int_{-2}^{0}\{x^3+3x^2+3x+3+(x+1)\cos(x+1)\}dx = 
\end{align*}
\begin{enumerate}
\item -4
\item 0
\item 4
\item 6
\end{enumerate}

\item The area bounded by the parabolas $y = (x + 1)^{2}$ and $y = (x - 1)^{2}$ and the line $y = 1/4$ is
\begin{enumerate}
\item 4 sq.units
\item 1/6 sq.units
\item 4/3 sq.units
\item 1/3 sq.units
\end{enumerate}

\item The area of the region between the curves $y = \sqrt{\frac{1 + \sin x}{\cos x}}$ and $y = \sqrt{\frac{1 - \sin x}{\cos x}}$ bounded by the lines x = 0 and $x = \frac{\pi}{4}$ is
\begin{enumerate}
\item $\int_{0}^{\sqrt{2} - 1}\frac{t}{(1 + t^2)\sqrt{1 - t^2}}dt$
\item $\int_{0}^{\sqrt{2} - 1}\frac{4t}{(1 + t^2)\sqrt{1 - t^2}}dt$
\item $\int_{0}^{\sqrt{2} + 1}\frac{4t}{(1 + t^2)\sqrt{1 - t^2}}dt$
\item $\int_{0}^{\sqrt{2} + 1}\frac{t}{(1 + t^2)\sqrt{1 - t^2}}dt$
\end{enumerate}

\item Let f be a non-negative function defined on the interval [0, 1]. If
\begin{align*}
\int_{0}^{x}\sqrt{1 - (f'(t))^2}dt = \int_{0}^{x}f(t)dt,
\end{align*}
$0 \leq x \leq 1$, and f(0) = 0, then
\begin{enumerate}
\item $f(\frac{1}{2}) < \frac{1}{2}$ and $f(\frac{1}{3}) > \frac{1}{3}$
\item $f(\frac{1}{2}) > \frac{1}{2}$ and $f(\frac{1}{3}) > \frac{1}{3}$
\item $f(\frac{1}{2}) < \frac{1}{2}$ and $f(\frac{1}{3}) < \frac{1}{3}$
\item $f(\frac{1}{2}) > \frac{1}{2}$ and $f(\frac{1}{3}) < \frac{1}{3}$
\end{enumerate}

\item The value of 
\begin{align*}
\lim_{x \to 0}\frac{1}{x^3}\int_{0}^{x}\frac{tln(1 + t)}{t^4 + 4}dt
\end{align*}
\begin{enumerate}
\item 0
\item $\frac{1}{12}$
\item $\frac{1}{24}$
\item $\frac{1}{64}$
\end{enumerate}

\item Let f be a real-valued function defined on the interval (-1, 1) such that 
\begin{align*}
e^{-x}f(x) = 2 + \int_{0}^{x}\sqrt{t^4 + 1}dt
\end{align*}
for all $x \in (-1, 1)$, and let $f^{-1}$ be the inverse function of f. Then $(f^{-1})'(2)$ is equal to
\begin{enumerate}
\item 1
\item 1/3
\item 1/2
\item 1/e
\end{enumerate}

\item The value of 
\begin{align*}
\int_{\sqrt\ln2}^{\sqrt{ln3}}\frac{x\sin x^2}{\sin x^2 + \sin(ln6 - x^2)}dx = 
\end{align*}
\begin{enumerate}
\item $\frac{1}{4}ln\frac{3}{2}$
\item $\frac{1}{2}ln\frac{3}{2}$
\item $ln\frac{3}{2}$
\item $\frac{1}{6}ln\frac{3}{2}$
\end{enumerate}

\item Let the straight line x = b divide the area enclosed by $y = (1 - x)^2$, y = 0 and x = 0 into two parts $R_1(0 \leq x \leq b)$ and $R_2(b \leq x \leq 1)$ such that $R_1 - R_2  = \frac{1}{4}$. Then b equlas
\begin{enumerate}
\item 3/4
\item 1/2
\item 1/3
\item 1/4
\end{enumerate}

\item Let $f: [-1, 2] \to [0, \infty)$ be a continuous function such that $f(x) = f(1 - x)$ for all $x \in [-1, 2]$. Let 
$R_1 = \int_{-1}^{2}xf(x)dx$, x = -1, x = 2 and the x-axis. Then
\begin{enumerate}
\item $R_1 = 2R_2$
\item $R_1 = 3R_2$
\item $2R_1 = R_2$
\item $3R_1 = R_2$
\end{enumerate}

\item The value of the integral
\begin{align*}
\int_{-\pi/2}^{\pi/2}(x^2 + ln\frac{\pi + x}{\pi - x})\cos x dx
\end{align*}
\begin{enumerate}
\item 0
\item $\frac{\pi^2}{2} - 4$
\item $\frac{\pi^2}{4} + 4$
\item $\frac{\pi^2}{2}$
\end{enumerate}

\item The area enclosed by the curves $y = \sin x + \cos x$ and $y = |\cos x - \sin x|$ over the interval $[0, \frac{\pi}{2}]$ is
\begin{enumerate}
\item $4(\sqrt{2} - 1)$
\item $2\sqrt{2}(\sqrt{2} - 1)$
\item $2(\sqrt{2} + 1)$
\item $2\sqrt{2}(\sqrt{2} + 1)$
\end{enumerate}

\item Let $f: [\frac{1}{2}, 1] \to R$(the set of all real number) be a positive, non-constant and diffrentiable function such that $f'(x) < 2f(x)$ and $f(\frac{1}{2}) = 1$. Then the value of $\int_{1/2}^{1}f(x)dx$ lies in the interval
\begin{enumerate}
\item (2e-1, 2e)
\item (e - 1, 2e - 1)
\item $(\frac{e - 1}{2}, e - 1)$
\item $(0, \frac{e - 1}{2})$
\end{enumerate}

\item The following integral
\begin{align*}
\int_{\pi/4}^{\pi/2}(2\cosec x)^{17}dx
\end{align*}
is equal to
\begin{enumerate}
\item $\int_{0}^{log(1 + \sqrt{2})}2(e^u + e^{-u})^{16}dx$
\item $\int_{0}^{log(1 + \sqrt{2})}(e^u + e^{-u})^{17}dx$
\item $\int_{0}^{log(1 + \sqrt{2})}2(e^u - e^{-u})^{17}dx$
\item $\int_{0}^{log(1 + \sqrt{2})}2(e^u - e^{-u})^{16}dx$
\end{enumerate}

\item The value of 
\begin{align*}
\int_{\pi/2}^{\pi/2}\frac{x^2 \cos x}{1 + e^x}dx
\end{align*}
is equal to
\begin{enumerate}
\item $\frac{\pi^2}{4} - 2$
\item $\frac{\pi^2}{4} + 2$
\item $\pi^2 - e^{\frac{\pi}{2}}$
\item $\pi^2 + e^{\frac{\pi}{2}}$
\end{enumerate}

\item Area of the region
\begin{align*}
\{(x, y) \in R^2: y \geq \sqrt{|x + 3|}, 5y \leq x + 9 \leq 15\}
\end{align*}
is equal to
\begin{enumerate}
\item $\frac{1}{6}$
\item $\frac{4}{3}$
\item $\frac{3}{2}$
\item $\frac{5}{3}$
\end{enumerate}

\item The area of the region
\begin{align*}
\{(x, y): xy \leq 8, 1 \leq y \leq x^2\} = 
\end{align*}
\begin{enumerate}
\item $8log_e2 - \frac{14}{3}$
\item $16log_e2 - \frac{14}{3}$
\item $8log_e2 - \frac{7}{3}$
\item $16log_e2 - 6$
\end{enumerate}

\textbf{MCQs with One or More than One Correct Answer}

\item If 
\begin{align*}
\int_{0}^{x}f(t)dt = x + \int_{x}^{t} tf(t)dt
\end{align*}
then the value of f(1) is
\begin{enumerate}
\item 1/2
\item 0
\item 1
\item -1/2
\end{enumerate}

\item Let $f(x) = x - [x]$, for every real number x, where [x] is the integral part of x. Then 
\begin{align*}
\int_{-1}^{1}f(x)dx = 
\end{align*}
\begin{enumerate}
\item 1
\item 2
\item 0
\item 1/2
\end{enumerate}

\item For which of the following values of m, is the area of the region bounded by the curve $y = x - x^{2}$ and the line 
$y = mx$ equals 9/2?
\begin{enumerate}
\item -4
\item -2
\item  2 
\item  4
\end{enumerate}

\item Let $f(x)$ be a non-constant twice diffrentiable function defined on $(-\infty, \infty)$ such that $f(x) = f(1 - x)$ and $f'(\frac{1}{4}) = 0$. Then,
\begin{enumerate}
\item $f''(x)$ vanishes at least twice on [0, 1]
\item $f'(\frac{1}{2}) = 0$
\item $\int_{-1/2}^{1/2}f(x + \frac{1}{2})\sin x dx = 0$
\item $\int_{0}^{1/2}f(t)e^{\sin \pi t}dt = \int_{1/2}^{1}f(1 - t)e^{\sin \pi t}dt$
\end{enumerate}

\item Area of the region bounded by the curve $y = e^x$ and lines x = 0 and y = e is
\begin{enumerate}
\item e - 1
\item $\int_{1}^{e}ln(e + 1 - y)dy$
\item $e - \int_{0}^{1}e^xdx$
\item $\int_{1}^{e}ln ydy$
\end{enumerate}

\item If 
\begin{align*}
I_n = \int_{-\pi}^{\pi}\frac{\sin nx}{(1 + \pi^{x})\sin x}dx
\end{align*}
where n = 0, 1 , 2..... then
\begin{enumerate}
\item $I_n = I_{n + 2}$
\item $\sum_{m = 1}^{10}I_{2m + 1} = 10\pi$
\item $\sum_{m = 1}^{10}I_{2m} = 0$
\item $I_n = I_{n + 1}$
\end{enumerate} 

\item The value(s) of
\begin{align*}
\int_{0}^{1}\frac{x^4(1 - x)^4}{1 + x^2}dx = 
\end{align*}
\begin{enumerate}
\item $\frac{22}{7} - \pi$
\item $\frac{2}{105}$
\item $0$
\item $\frac{71}{15} - \frac{3\pi}{2}$
\end{enumerate}

\item Let f be a real-valued function defined on the interval $(0, \infty)$ by 
\begin{align*}
f(x) = lnx + \int_{0}^{x}\sqrt{1 + \sin t}dt
\end{align*}
Then which of the following statement(s) is(are) true?
\begin{enumerate}
\item $f''(x)$ exists for all $x \in (0, \infty)$
\item $f'(x)$ exists for all $x \in (0, \infty)$ and $f'$ is continuous on $(0, \infty)$, but not differentiable on $(0, \infty)$
\item There exists $\alpha > 1$ such that $|f'(x)| < |f(x)|$ for all $x \in (\alpha, \infty)$
\item There exists $\beta > 1$ such that $|f'(x)| + |f(x)| \leq \beta$ for all $x \in (0, \infty)$
\end{enumerate}

\item Let S be the area of the region enclosed by $y = e^{x^{2}}$, y = 0, x = 0 and x = 1: then
\begin{enumerate}
\item $S \geq \frac{1}{e}$
\item $S \geq 1-\frac{1}{e}$
\item $S \leq \frac{1}{4}(1 + \frac{1}{\sqrt{e}})$
\item $S \leq \frac{1}{\sqrt{2}} + \frac{1}{\sqrt{e}}(1 - \frac{1}{\sqrt{2}})$
\end{enumerate}

\item The option(s) with the values of a and L that satisfy the following equation is(are)
\begin{align*}
\frac{\int_{0}^{4\pi}e^{t}(\sin^{6}at + \cos^{4}at)dt}{\int_{0}^{\pi}e^{t}(\sin^{6}at + \cos^{4}at)dt} = L?
\end{align*}
\begin{enumerate}
\item a = 2, $L = \frac{e^{4\pi} - 1}{e^{\pi} - 1}$
\item a = 2, $L = \frac{e^{4\pi} + 1}{e^{\pi} + 1}$
\item a = 4, $L = \frac{e^{4\pi} - 1}{e^{\pi} - 1}$
\item a = 4, $L = \frac{e^{4\pi} + 1}{e^{\pi} + 1}$
\end{enumerate}

\item Let 
\begin{align*}
f(x) = 7\tan^{8}x + 7\tan^{6}x - 3\tan^{4}x - 3\tan^{2}x
\end{align*}
for all $x \in (\frac{\pi}{2}, \frac{\pi}{2})$. Then the correct expression(s) is(are)
\begin{enumerate}
\item $\int_{0}^{\pi/4}xf(x)dx = \frac{1}{12}$
\item $\int_{0}^{\pi/4}f(x)dx = 0$
\item $\int_{0}^{\pi/4}xf(x)dx = \frac{1}{6}$
\item $\int_{0}^{\pi/4}f(x)dx = 1$
\end{enumerate}

\item Let $f'(x) = \frac{192x^{3}}{2 + \sin^{4}\pi x}$ for all $x \in R$ with $f(\frac{1}{2}) = 0$. If 
\begin{align*}
m < \leq \int_{1/2}^{1}f(x)dx \leq M
\end{align*}
then the possible values of m and M are
\begin{enumerate}
\item m = 13, M = 24
\item m = 0.25, M = 0.5
\item m = -11, M = 0
\item m = 1, M = 12
\end{enumerate}

\item Let
\begin{align*}
f(x) = \lim_{x \to \infty}(\frac{n^n(x+n)(x+\frac{n}{2}).....(x+\frac{n}{n})}{n!(x^2+n^2)(x^2+\frac{n^2}{4}).......(x^2+\frac{n^2}{n^2})})^{\frac{x}{n}} 
\end{align*}
for all $x > 0$. Then
\begin{enumerate}
\item $f(\frac{1}{2}) \geq f(1)$
\item $f(\frac{1}{3}) \leq f(\frac{2}{3})$
\item $f'(2) \leq 0$
\item $\frac{f'(3)}{f(3)} \geq \frac{f'(2)}{f(2)}$
\end{enumerate}

\item Let $f: R \to (0, 1)$ be a continuous function. Then, which of the following function(s) has(have) the value zero at some point in the interval (0, 1)?
\begin{enumerate}
\item $x^9 - f(x)$
\item $x - \int_{0}^{\frac{\pi}{2} - x}f(t)\cos t dt$
\item $e^{x} - \int_{0}^{x}f(t)\sin t dt$
\item $f(x) + \int_{0}^{\pi/2}f(t)\sin t dt$
\end{enumerate}

\item If 
\begin{align*}
g(x) = \int{\sin x}^{\sin (2x)}\sin^{-1}(t)dt
\end{align*}
then,
\begin{enumerate}
\item $g'(\frac{\pi}{2}) = -2\pi$
\item $g'(\frac{-\pi}{2}) = 2\pi$
\item $g'(\frac{\pi}{2}) = 2\pi$
\item $g'(\frac{-\pi}{2}) = -2\pi$
\end{enumerate}

\item If the line $sx = \alpha$ divides the area of the region
\begin{align*}
R = \{(x, y) \in R^{2}: x^3 \leq y \leq x, 0 \leq x \leq 1\}
\end{align*}
into two equal parts, then
\begin{enumerate}
\item $0 < \alpha \leq \frac{1}{2}$
\item $\frac{1}{2} < \alpha < 1$
\item $2\alpha^{4} - 4\alpha^{2} + 1 = 0$
\item $\alpha^{4} + 4\alpha^{2} - 1 = 0$
\end{enumerate}

\item If 
\begin{align*}
I = \sum_{k = 1}^{98}\int_{k}^{k + 1}\frac{k + 1}{x(x + 1)}dx, then
\end{align*}
\begin{enumerate}
\item $1 > log_e99$
\item $1 < log_e99$
\item $1 < \frac{49}{50}$
\item $1 > \frac{49}{50}$
\end{enumerate}

\item For, $a \in R$, $|a| > 1$, let
\begin{align*}
\lim_{x \to \infty}\frac{1 + 2^{3/2} + .......+ n^{3/2}}{n^{7/3}(\frac{1}{(an + 1)^2} + \frac{1}{(an + 2)^2}+.......+\frac{1}{(an + n)^2})} = 54
\end{align*}
Then the possible value(s) of a is/are
\begin{enumerate}
\item -9
\item 7
\item 6
\item 8
\end{enumerate}

\textbf{Subjective Problems}

\item Find the area bounded by the curve $x^2 = 4y$ and the straight line x = 4y - 2.

\item Show that:
\begin{align*}
\lim_{n \to \infty}(\frac{1}{n + 1} + \frac{1}{n + 2}+......+\frac{1}{6n}) = log 6
\end{align*}

\item Show that
\begin{align*}
\int_{0}^{\pi}xf(\sin x)dx = \frac{\pi}{2}\int_{0}^{\pi}f(\sin x)dx
\end{align*}

\item Find the value of
\begin{align*}
\int_{-1}^{3/2}|x \sin \pi x|dx
\end{align*}

\item For any real t, $x = \frac{e^t + e^{-t}}{2}$, $y = \frac{e^t - e^{-t}}{2}$ is a point on the hyperbola $x^2 - y^2 = 1$. Show that the area bounded by the this hyperbola and the line joining its centre to the points corresponding to $t_1$ and $-t_1$ is $t_1$.

\item Evaluate
\begin{align*}
\int_{0}^{\pi/4}\frac{\sin x + \cos x}{9 + 16\sin 2x}dx
\end{align*}

\item Find the area bounded by the x-axis, part of the curve $y = (1 + \frac{8}{x^2})$ and the ordinates at x = 2 and x = 4. If the ordinate at x = a divides the area into two equal parts, find a.

\item Evaluate the follwing
\begin{align*}
\int_{0}^{1/2}\frac{x\sin^{-1}x}{\sqrt{1 - x^2}}dx.
\end{align*}

\item Find the area of the region bounded by the x-axis and the curves defined by
\begin{align*}
y = \tan x, \frac{-\pi}{3} \leq x \leq \frac{\pi}{3}
\end{align*}
\begin{align*}
y = \cot x, \frac{\pi}{6} \leq x \leq \frac{3\pi}{2}
\end{align*}

\item Given a function $f(x)$ such that
\begin{enumerate}
\item it is integratable over every interval on the real line and
\item $f(t + x) = f(x)$, for every x and a real t, then Show that the integral $\int_{a}^{a + 1}f(x)dx$ is independent of a.
\end{enumerate}

\item Evaluate the following:
\begin{align*}
\int_{0}^{\pi/2}\frac{x\sin x \cos x}{\cos^{4}x + \sin^{4}x}dx
\end{align*}

\item Sketch the region bounded by the curves $y = \sqrt{5 - x^{2}}$ and $y = |x - 1|$ and find its area.

\item Evaluate
\begin{align*}
\int_{0}^{\pi}\frac{x dx}{1 + \cos \alpha \sin x}, 0 < \alpha < \pi
\end{align*}

\item Find the area bounded by the curves 
\begin{align}
x^3 + y^{2} = 25,
\end{align}
$4y = |4 - x^{2}|$ and x = 0 above the x-axis.

\item Find the area of the region bounded by the curves $C: y = \tan x$, tangent drawn to C at $x = \frac{\pi}{4}$ and the x-axis.

\item Evaluate
\begin{align*}
\int_{0}^{1}log[\sqrt{1 - x} + \sqrt{1 + x}]dx
\end{align*}

\item If $f$ and $g$ are continuous function on [0, a] satisfying $f(x) = f(a - x)$ and $g(x) + g(a - x) = 2$, then Show that 
\begin{align*}
\int_{0}^{a}f(x)g(x)dx = \int_{0}^{a}f(x)dx
\end{align*}

\item Show that
\begin{align*}
\int_{0}^{\pi/2}f(\sin 2x)\sin x dx = \sqrt{2}\int_{0}^{\pi/4}f(\cos 2x)\cos x dx.
\end{align*}

\item Prove that for any positive integer k,
\begin{align*}
\frac{\sin 2kx}{\sin x} = 2[\cos x + \cos 3x + .........+\cos(2k - 1)x]
\end{align*}
Hence prove that
\begin{align*}
\int_{0}^{\pi/2}\sin 2kx \cot x dx = \frac{\pi}{2}
\end{align*}

\item Compute the area of the region bounded by the curves $y = ex lnx$ and $y = \frac{lnx}{ex}$ where $lne = 1$.

\item Sketch the curves and identify the region bounded by the $x = \frac{1}{2}$, x = 2, $y = lnx$ and $y = 2^x$. Find the area of the region.  

\item Evaluate
\begin{align*}
\int_{0}^{\pi}\frac{x \sin 2x \sin(\frac{\pi}{2}\cos x)}{2x - \pi}dx
\end{align*}

\item Sketch the region bounded by the curves $y = x^2$ and $y = \frac{2}{1 + x^2}$. Find the area.

\item Determine a positive integer $n \leq 5$, such that
\begin{align*}
\int_{0}^{1}e^x(x - 1)^{n}dx = 16 - 6e
\end{align*}

\item Evaluate
\begin{align*}
\int_{2}^{3}\frac{2x^5 + x^4 - 2x^3 + 2x^2 + 1}{(x^2 + 1)(x^4 - 1)}dx
\end{align*}

\item Show that
\begin{align*}
\int_{0}^{n\pi + v}|\sin x|dx = 2n + 1 - \cos v
\end{align*}
where n is a positive integer and $0 \leq v < \pi$.

\item In what ratio does the x-axis divide the area of the region bounded by the parabolas $y = 4x - x^2$ and $y = x^2 - x$?

\item Let
\begin{align*}
I_m = \int_{0}^{\pi}\frac{1 - \cos mx}{1 - \cos x}dx
\end{align*}
Use the mathematical induction to prove that $I_m = m\pi$, m = 0, 1, 2........

\item Evaluate the definite integral
\begin{align*}
\int_{-1/\sqrt{3}}^{1/\sqrt{3}}\left(\frac{x^4}{1 - x^4}\right)\cos^{-1}\left(\frac{2x}{1 + x^2}\right) dx
\end{align*}

\item Consider a square with vertices at (1, 1), (-1, 1), (-1, -1) and (1, -1). Let S be the region consisting of all points inside the square which are near to the origin than to any edge. Sketch the region S and find its area.

\item Let $A_n$ be the area bounded by the curve $y = (\tan x)^n$ and the lines x = 0, y = 0 and $x = \frac{\pi}{4}$. Prove that $n > 2$,
\begin{align*}
A_n + A_{n-2} = \frac{1}{n - 1}
\end{align*}
and deduce
\begin{align*}
\frac{1}{2n + 2} < A_n < \frac{1}{2n - 2}
\end{align*}

\item Determine the value of
\begin{align*}
\int_{-\pi}^{\pi}\frac{2x(1 + \sin x)}{1 + \cos^{2}x}dx.
\end{align*}

\item Let 
\begin{align*}
f(x) = Maximum\{x^2, (1 - x^2), 2x(1 - x)\}
\end{align*}
where $0 \leq x \leq 1$. Determine the area of the region bounded by the curves $y = f(x)$ x-axis x = 0 and x = 1.

\item Prove that
\begin{align*}
\int_{0}^{1}\tan^{-1}\left(\frac{1}{1 - x + x^2}\right) dx = 2\int_{0}^{1}\tan^{-1}xdx
\end{align*}
Hence or otherwise, evaluate the integral
\begin{align*}
\int_{0}^{1}\tan^{-1}\left(1 - x + x^{2}\right) dx
\end{align*}

\item Integrate
\begin{align*}
\int_{0}^{\pi}\frac{e^{\cos x}}{e^{\cos x} + e^{-\cos x}}dx.
\end{align*}

\item Let $f(x)$ be a continuous function given by
\begin{align*}
f(x) = 
\left\lbrace
\begin{array}{ll}
      2x & |x| \leq 1 \\
      x^2 + ax + b & |x| > 1 \\
\end{array} 
\right\rbrace
\end{align*}
Find the area of the region in the third quadrant bounded by the curves $x = -2y^{2}$ and $y = f(x)$ lying on the left of the line $8x + 1 = 0$.

\item For $x > 0$, let 
\begin{align*}
f(x) = \int_{e}^{x}\frac{ln t}{1 + t}dt
\end{align*}
Find the function $f(x) + f(\frac{1}{x})$ and show that $f(e) + f(\frac{1}{e}) = \frac{1}{2}$.

\item Let $b \neq 0$ and for j = 0, 1, 2, ......n, let $S_j$ be the area of the region bounded by the y-axis and the curve $xe^{ay} = \sin by$, $\frac{jr}{b} \leq y \leq \frac{(j + 1)\pi}{b}$. Show that $S_0$, $S_1$, $S_2$,.......$S_n$ are in geometric progression. Also, find their sum for a = -1 and $b = \pi$.

\item Find the area of the region bounded by the curves $y = x^2$, $y = |2 - x^2|$ and y = 2, which lies to the right of the line x = 1.

\item If f is an even function then prove that
\begin{align*}
\int_{0}^{\pi/2}f(\cos 2x)\cos x dx = \sqrt{2}\int_{0}^{\pi/4}f(\sin 2x)\cos x dx
\end{align*}

\item Find the value of
\begin{align*}
\int_{-\pi/3}^{\pi/3}\frac{\pi + 4x^{3}}{2 - \cos \left(|x| + \frac{\pi}{3}\right)}dx
\end{align*}

\item If
\begin{align*}
y(x) = \int_{\pi^{2}/16}^{x^2}\frac{\cos x \cos\sqrt{\theta}}{1 + \sin^{2}\sqrt{\theta}}d\theta
\end{align*}
find $\frac{dy}{dx}$ at $x = \pi$

\item Evaluate
\begin{align*}
\int_{0}^{\pi}e^{|\cos x|}\left(2\sin\left(\frac{1}{2}\cos x\right) + 3\cos\left(\frac{1}{2}\cos x\right)\right)\sin x dx
\end{align*}

\item Find the area bounded by the curves $x^2 = y$, $x^2 = -y$ and $y^2 = 4x - 3$.

\item $f(x)$ is a differentiable function and $g(x)$ is a double differentiable function such that $|f(x)| \leq 1$ and
$f'(x) = g(x)$. If $f^2(0) + g^2(0) = 9$. Prove that there exists some $c \in (-3, 3)$ such that $g(c).g''(c) < 0$.

\item If
\begin{align*}
\begin{bmatrix}
4a^2 & 4a & 1 \\ 4b^2 & 4b & 1 \\ 4c^2 & 4c & 1
\end{bmatrix} \begin{bmatrix}
f(-1) \\ f(1) \\ f(2)
\end{bmatrix} = \begin{bmatrix}
3a^2 + 3a \\ 3b^2 + 3b \\ 3c^2 + 3c
\end{bmatrix}
\end{align*}
$f(x)$ is a quadrant function and its maximum value occurs at a point V. A is a point of intersection of $y = f(x)$ with x-axis and point B is such that chord AB subtends a right angle at V. Find the area enclosed by $f(x)$ and chord AB.

\item  Find the value of
\begin{align*}
5050\frac{\int_{0}^{1}(1 - x^{50})^{100}dx}{\int_{0}^{1}(1 - x^{50})^{101}dx}
\end{align*}

\item Let $f: R \to R$ be a function defined by 
\begin{align*}
f(x) = 
\left\lbrace
\begin{array}{ll}
      [x] & x \leq 2\\
      0 & x > 2\\
\end{array}
\right\rbrace
\end{align*}
where [x] is the greatest integer less than or equal to x, if 
\begin{align*}
I = \int_{-1}^{2}\frac{xf(x^2)}{2 + f(x + 1)}dx
\end{align*}
then the value of (4I - 1) is

\item Let 
\begin{align*}
F(x) = \int_{x}^{x^2 + \frac{\pi}{6}}2\cos^{2}t dt
\end{align*}
for all $x \in R$ and $f: [0, 0.5]$, if $F'(a) + 2$ is the area of the region bounded by x = 0, y = 0, y = f(x) and x = a, then f(0) is

\item If 
\begin{align*}
\alpha = \int_{0}^{1}(e^{9x + 3\tan^{-1}x})\left(\frac{12 + 9x^2}{1 + x^2}\right)dx
\end{align*}
where $\tan^{-1}x$ takes only polynomial values, then the value of $\left( log_e|1 + \alpha| - \frac{3\pi}{4}\right)$ is

\item Let $f: R \to R$ be a continuous odd function, which vanishes exactly at one point and $f(1) = \frac{1}{2}$. Suppose 
\begin{align*}
F(x) = \int_{-1}^{x}f(t)dt
\end{align*}
for all $x \in [-1, 2]$  and
\begin{align*}
G(x) = \int_{-1}^{x}t|f(f(t))|dt
\end{align*}
for all $x \in [-1, 2]$. If $\lim_{x \to 1}\frac{F(x)}{G(x)} = \frac{1}{14}$, then the value of $f(\frac{1}{2})$ is

\item The total number of distinct $x \in [0, 1]$ for which 
\begin{align*}
\int_{t^2}^{1 + t^4}dt = 2x - 1
\end{align*}
is

\item Let $f: R \to R$ be a differentiable function such that f(0)=0, $f(\frac{\pi}{2}) = 3$ and $f'(0) = 1$. If
\begin{align*}
g(x)  = \int_{x}^{\pi/2}[f'(t)\cosec t - \cot t \cosec t f(t)]dt
\end{align*}
for $x \in (0, \frac{\pi}{2}]$, then $\lim_{x \to 0}g(x)$ =

\item For positive integer n, let
\begin{align*}
y_n = \frac{1}{n}(n + 1)(n + 2).....(n + n)^{\frac{1}{n}}
\end{align*} 
For $x \in R$, let [x] be the greatest integer less than or equal to x. If $\lim_{n \to \infty}y_n = L$, then the value of $f(L)$ =

\item A farmer $F_1$ has a land in the shape of triangle with vertices at P(0, 0), Q(1, 1) and R(2, 0). From this land, a neighbouring farmer $F_2$ takes away the region which lies between the side PQ and a curve of the form $y = x^n(n > 1)$. If the area of the region taken away by the farmer $F_2$ is exactly 30 percentage of the area of $\Delta PQR$, then the value of n is...........

\clearpage

\textbf{Match the Following}

\item Match the following
\begin{table}[ht!]
\centering
\begin{tabular}{c c} 
 \textbf{Column I} & \textbf{Column II}\\ [0.5ex] 
 (A) $\int_{0}^{\pi/2}(\sin x)^{\cos x}$\\ $\left(\cos x\cot x
     - log(\sin x)^{\sin x} \right)dx$                                          &(p) 1\\ 
 (B) Area bounded by $-4y^2 = x$\\ and $x - 1 = -5y^2$                            &(q) 0\\
 (C) Cosine of angle of intersection\\ of curves
     $y = 3^{x-1}log x$\\ and $y = x^x - 1$ is                                      &(r) 6$ln2$\\                                                                     
 (D) Let $\frac{dy}{dx} = \frac{6}{x + y}$ where y(0) = 0\\
     then value of y when x + y = 6 is                                            &(s) $\frac{4}{3}$\\[1ex] 
\end{tabular}
\end{table}\\

\item Match the following
\begin{table}[ht!]
\centering
\begin{tabular}{c c} 
 \textbf{Column I} & \textbf{Column II}\\ [0.5ex] 
 (A) $\int_{-1}^{1}\frac{dx}{1 + x^2}$                      &(p) $\frac{1}{2}log\left(\frac{2}{3}\right)$\\ 
 (B) $\int_{0}^{1}\frac{dx}{\sqrt{1 - x^2}}$                &(q) $2log\left(\frac{2}{3}\right)$\\
 (C) $\int_{2}^{3}\frac{dx}{1 - x^2}$                       &(r) $\frac{\pi}{3}$\\                                                                     
 (D) $\int_{1}^{2}\frac{dx}{x\sqrt{x^2 - 1}}$               &(s) $\frac{\pi}{2}$\\[1ex] 
\end{tabular}
\end{table}\\

\item Match the following
\begin{table}[ht!]
\centering
\begin{tabular}{c c} 
 \textbf{Column I} & \textbf{Column II}\\ [0.5ex] 
 (A) The number of polynomials\\ $f(x)$ with non-negative integer\\
     coefficients of degree $\leq$ 2,\\ satisfying f(0) = 0 and\\
     $\int_{0}^{1}f(x)dx = 1$, is                                                 &(p) 8\\ 
 (B) The number of points in the\\ interval $[-\sqrt{13}, \sqrt{13}]$
     at which\\ $f(x) = \sin(x^2) + \cos(x^2)$ of\\ the maximum value is                  &(q) 2\\
 (C) $\int_{-2}^{2}\frac{3x^2}{(1 + e^x)}dx$ equals                               &(r) 4\\                                                                     
 (D) $\frac{\int_{-1/2}^{1/2}\cos 2xlog\left( \frac{1 + x}{1 - x}\right)dx}
     {\int_{0}^{1/2}\cos 2xlog\left( \frac{1 + x}{1 - x}\right)dx}$               &(s) 0\\[1ex] 
     
\textbf{codes:}
\begin{tabular}{ c c c c c}
      P & Q & R & S\\
  (a) 3 & 2 & 4 & 1\\
  (b) 2 & 3 & 4 & 1\\
  (c) 3 & 2 & 1 & 4\\
  (d) 2 & 3 & 1 & 4\\
\end{tabular}
\end{tabular}
\end{table}\\

\clearpage

\textbf{Comprehension Based Questions}

\textbf{PASSAGE-1}

Let the definite integral be defined by the fomula 
\begin{align*}
\int_{a}^{b}f(x)dx = \frac{b - a}{2}(f(a) + f(b))
\end{align*}
For more acurate results for $c \in (a, b)$ we can use
\begin{align*}
\int_{a}^{b}f(x)dx = \int_{a}^{c}f(x)dx + \int_{c}^{b}f(x)dx = F(c)
\end{align*}
so that for $c = \frac{a + b}{2}$, we get 
\begin{align*}
\int_{a}^{b}f(x)dx = \frac{b - a}{4}(f(a) + f(b) + 2f(c)).
\end{align*}

\item $\int_{0}^{\pi/2}\sin x dx$ = 
\begin{enumerate}
\item $\frac{\pi}{8}(1 + \sqrt{2})$
\item $\frac{\pi}{4}(1 + \sqrt{2})$
\item $\frac{\pi}{8\sqrt{2}}$
\item $\frac{\pi}{4\sqrt{2}}$
\end{enumerate}

\item If
\begin{align*}
\lim_{x \to a}\frac{\int_{a}^{x}f(x)dx - \left(\frac{x - a}{2}\right)(f(x) + f(a))}{(x - a)^3} = 0
\end{align*}
then $f(x)$ is of maximum degree
\begin{enumerate}
\item 4
\item 3
\item 2
\item 1
\end{enumerate}

\item If $f''(x) < 0$ for all $x \in (a, b)$ and c is a point such that $a < c < b$ and $(c, f(c))$ is the point lying on the curve for which $F(c)$ is maximum, then $f'(c)$ is equal to
\begin{enumerate}
\item $\frac{f(b) - f(a)}{b - a}$
\item $2\frac{f(b) - f(a)}{b - a}$
\item $2\frac{f(b) - f(a)}{2b - a}$
\item 0
\end{enumerate}

\textbf{PASSAGE-2}

Consider the functions defined implicity by the equation
\begin{align}
y^3 - 3y + x = 0
\end{align}
on various intervals in the real line. If $x \in (-\infty, -2) \cup (2, \infty)$, the equation implicity defines a unique real valued differentiable function $y = f(x)$. If $x \in (-2, 2)$, the equation implicity defines a unique real valued differentiable function $y = g(x)$ satisfying $g(0) = 0$.

\item If $f(-10\sqrt{2}) = 2\sqrt{2}$, then $f''(-10\sqrt{2})$ = 
\begin{enumerate}
\item $\frac{4\sqrt{2}}{7^{3}3^{2}}$
\item -$\frac{4\sqrt{2}}{7^{3}3^{2}}$
\item $\frac{4\sqrt{2}}{7^{3}3}$
\item -$\frac{4\sqrt{2}}{7^{3}3}$
\end{enumerate}

\item The area of the region bounded by the curve $y = f(x)$, the x-axis and the lines x = a and x = b, where $-\infty < a < b < -2$ is
\begin{enumerate}
\item $\int_{a}^{b}\frac{x}{3(f(x))^2 - 1}dx + bf(b) - af(a)$
\item -$\int_{a}^{b}\frac{x}{3(f(x))^2 - 1}dx + bf(b) - af(a)$
\item $\int_{a}^{b}\frac{x}{3(f(x))^2 - 1}dx - bf(b) + af(a)$
\item -$\int_{a}^{b}\frac{x}{3(f(x))^2 - 1}dx - bf(b) + af(a)$
\end{enumerate}

\item 
\begin{align*}
\int_{-1}^{1}g'(x)dx =
\end{align*} 
\begin{enumerate}
\item $2g(-1)$
\item 0
\item $-2g(1)$
\item $2g(1)$
\end{enumerate}

\textbf{PASAAGE-3}

Consider the function $f: (-\infty, \infty) \to (-\infty, \infty)$ defined by
\begin{align*}
f(x) = \frac{x^2 - ax + 1}{x^2 + ax + 1}, 0 < a < 2.
\end{align*}

\item Which of the following is True?
\begin{enumerate}
\item $(2 + a)^2 f''(1) + (2 - a)^2 f''(-1) = 0$
\item $(2 - a)^2 f''(1) - (2 + a)^2 f''(-1) = 0$
\item $f'(1)f'(-1) = (2 - a)^2$
\item $f'(1)f'(-1) = -(2 - a)^2$
\end{enumerate}

\item Which of the following is True?
\begin{enumerate}
\item $f(x)$ is decreasing on (-1, 1) and has a local minimum at x = 1
\item $f(x)$ is increasing on (-1, 1) and has a local minimum at x = 1
\item $f(x)$ is increasing on (-1, 1) but has a neither local maximum nor local minimum at x = 1
\item $f(x)$ is decreasing on (-1, 1) but has a neither local maximum nor local minimum at x = 1
\end{enumerate}

\item Let 
\begin{align*}
g(x) = \int_{0}^{e^x}\frac{f'(t)}{1 + t^2}dt
\end{align*}
Which of the following is True?
\begin{enumerate}
\item $g'(x)$ is positive on $(-\infty, 0)$ and negative on $(0, \infty)$
\item $g'(x)$ is negative on $(-\infty, 0)$ and positive on $(0, \infty)$
\item $g'(x)$ changes sign on both $(-\infty, 0)$ and $(0, \infty)$
\item $g'(x)$ does not change sign on $(-\infty, \infty)$
\end{enumerate}

\textbf{PASSAGE-4}

Consider the polynomial
\begin{align}
f(x) = 1 + 2x + 3x^2 + 4x^3
\end{align}
Let s be the sum of all distinct real roots of $f(x)$ and let $t = |s|$ 

\item The real numbers lies in the interval
\begin{enumerate}
\item $\left(-\frac{1}{4}, 0\right)$
\item $\left(-11, -\frac{3}{4}\right)$
\item $\left(-\frac{3}{4}, -\frac{1}{2}\right)$
\item $\left(0, \frac{1}{4}\right)$
\end{enumerate}

\item The area bounded by the curve $y = f(x)$ and the lines x = 0, y = 0 and x = t lies in the interval
\begin{enumerate}
\item $\left(\frac{3}{4}, 3\right)$
\item $\left(\frac{21}{64}, \frac{11}{16}\right)$
\item (9, 10)
\item $\left(0, \frac{21}{64}\right)$
\end{enumerate}

\item The function $f'(x)$ is
\begin{enumerate}
\item increasing in $\left(-t, -\frac{1}{4}\right)$ and decreasing in $\left(-\frac{1}{4}, t\right)$
\item decreasing in $\left(-t, -\frac{1}{4}\right)$ and increasing in $\left(-\frac{1}{4}, t\right)$
\item increasing in (-t, t)
\item decreasing in (-t, t)
\end{enumerate}

\textbf{PASSAGE-5}

Given that for each $a \in (0, 1)$,
\begin{align*}
\lim_{h \to 0^{+}}\int_{h}^{1 - h}t^{-a}(1 - t)^{a - 1}dt
\end{align*}
exists. Let this limit be $g(a)$. In addition, it is given that the function $g(a)$ is differentiable on (0, 1).

\item The value of $g\left(\frac{1}{2}\right)$ is
\begin{enumerate}
\item $\pi$
\item $2\pi$
\item $\frac{\pi}{2}$
\item $\frac{\pi}{4}$
\end{enumerate}

\item The value of $g'\left(\frac{1}{2}\right)$ is
\begin{enumerate}
\item $\frac{\pi}{2}$
\item $\pi$
\item $-\frac{\pi}{2}$
\item 0
\end{enumerate}

\textbf{PASSAGE-6}

Let $F: R \to R$ be a thrice differentiable function. Suppose that F(1) = 0, F(3) = -4 and $F(x) < 0$ for all $x \in \left(\frac{1}{2}, 3\right)$. Let $f(x) = xF(x)$ for all $x \in R$.

\item The correct statement(s) is(are)
\begin{enumerate}
\item $f'(1) < 0$
\item $f(2) < 0$
\item $f'(x) \neq 0$ for any $x \in (1, 3)$
\item $f'(x) = 0$ for any $x \in (1, 3)$
\end{enumerate}

\item If 
\begin{align*}
\int_{1}^{3}x^2F'(x)dx = -12
\end{align*}
and
\begin{align*}
\int_{1}^{3}x^3F''(x)dx = 40
\end{align*}
then the correct expression(s) is(are)
\begin{enumerate}
\item $9f'(3) + f'(1) - 32 = 0$
\item $\int_{1}^{3}f(x)dx = 12$
\item $9f'(3) - f'(1) + 32 = 0$
\item $\int_{1}^{3}f(x)dx = -12$
\end{enumerate}

\textbf{Integer Value Correct Type}

\item Let $f: R \to R$ be a continuous function which satisfies
\begin{align*}
f(x) = \int_{0}^{x}f(t)dt
\end{align*}
Then the value of $f(ln 5)$ is

\item For any real number x, let [x] denote the largest integer less than or equal to x. Let f be a real valued function defined on the interval [-10, 10] by
\begin{align*}
f(x) = 
\left\lbrace
\begin{array}{ll}
      x - [x] & if [x] is odd\\
      1 + [x] - x & if [x] is even\\
\end{array}
\right\rbrace
\end{align*}
Then the value of 
\begin{align*}
\frac{\pi^{2}}{10}\int_{-10}^{10}f(x)\cos \pi x dx = 
\end{align*}

\item The value of 
\begin{align*}
\int_{0}^{1}4x^3\left\lbrace\frac{d^2}{dx^2}(1 - x^2)^5\right\rbrace dx = 
\end{align*}

\item The value of the integral
\begin{align*}
\int_{0}^{1/2}\frac{1 + \sqrt{3}}{((x + 1)^2(1 - x)^6)^{1/4}}dx
\end{align*}
is

\item If 
\begin{align*}
I = \frac{2}{\pi}\int_{-\pi/4}^{\pi/4}\frac{dx}{(1 + e^{\sin x)})(2 - \cos 2x)} 
\end{align*}
then $27I^{2}$ equals........

\item The value of the integral
\begin{align*}
\int_{0}^{\pi/2}\frac{3\sqrt{\cos \theta}}{\left(\sqrt{\cos \theta} + \sqrt{\sin \theta}\right)^{5}}d\theta
\end{align*}
equals............

\textbf{Section - B}

\item $\int_{0}^{10\pi}|\sin x|dx$ is
\begin{enumerate}
\item 20
\item 8
\item 10
\item 18
\end{enumerate}

\item $I_n = \int_{0}^{\pi/4}\tan^{n}x dx$ then $\lim_{n \to \infty}n[I_n + I_{n + 2}]$ equals
\begin{enumerate}
\item 1/2
\item 1
\item $\infty$
\item 0
\end{enumerate}

\item $\int_{0}^{2}[x^2]dx$ is
\begin{enumerate}
\item $2 - \sqrt{2}$
\item $2 + \sqrt{2}$
\item $\sqrt{2} - 1$
\item $-\sqrt{2} - \sqrt{3} + 5$
\end{enumerate}

\item $\int_{-\pi}^{\pi}\frac{2x(1 + \sin x)}{1 + \cos^{2}x}dx$ is
\begin{enumerate}
\item $\frac{\pi^{2}}{4}$
\item $\pi^{2}$
\item 0
\item $\frac{\pi}{2}$
\end{enumerate}  

\item If $y = f(x)$ makes +ve intercept of 2 and 0 unit on x and y axes and encloses an area of 3/4 square unit with the axes then $\int_{0}^{2}xf'(x)dx$ is
\begin{enumerate}
\item 3/2
\item 1
\item 5/4
\item -3/4
\end{enumerate}

\item The area of bounded by the curves $y  = lnx $, $y = ln|x|$, $y = |ln x|$ and $y = |ln|x||$ is
\begin{enumerate}
\item 4 sq.units
\item 6 sq.units
\item 10 sq.units
\item None of these
\end{enumerate}

\item If $f(a + b + -x) = f(x)$, then 
\begin{align*}
\int_{a}^{b}xf(x)dx
\end{align*}
is equal to
\begin{enumerate}
\item $\frac{a + b}{2}\int_{a}^{b}f(a + b + x)dx$
\item $\frac{a + b}{2}\int_{a}^{b}f(b - x)dx$
\item $\frac{a + b}{2}\int_{a}^{b}f(x)dx$
\item $\frac{b - a}{2}\int_{a}^{b}f(x)dx$
\end{enumerate}

\item The area of the region bounded by the curves $y = |x - 1|$ and $y = 3 - |x|$ is
\begin{enumerate}
\item 6 sq.units
\item 2 sq.units
\item 3 sq.units
\item 4 sq.units
\end{enumerate}

\item Let $f(x)$ be a function satisfying $f'(x) = f(x)$ with $f(0) = 1$ and $g(x)$ be a function that satisfies $f(x) + g(x) = x^2$. Then the value of the integral
\begin{align*}
\int_{0}^{1}f(x)g(x)dx = 
\end{align*}
\begin{enumerate}
\item $e + \frac{e^2}{2} + \frac{5}{2}$
\item $e - \frac{e^2}{2} - \frac{5}{2}$
\item $e + \frac{e^2}{2} - \frac{3}{2}$
\item $e - \frac{e^2}{2} - \frac{5}{2}$
\end{enumerate}

\item The value of the integral
$I  = \int_{0}^{1}x(1 - x)^n dx$  
\begin{enumerate}
\item $\frac{1}{n + 1} + \frac{1}{n + 2}$
\item $\frac{1}{n + 1}$
\item $\frac{1}{n + 2}$
\item $\frac{1}{n + 1} - \frac{1}{n + 2}$
\end{enumerate}

\item $\lim_{n \to \infty}\sum_{r = 1}^{n}e^{\frac{r}{n}}$ = 
\begin{enumerate}
\item e + 1
\item e - 1
\item 1 - e
\item e
\end{enumerate}

\item The value of 
\begin{align*}
\int_{-2}^{3}|1 - x^2|dx = 
\end{align*}
\begin{enumerate}
\item $\frac{1}{3}$
\item $\frac{14}{3}$
\item $\frac{7}{3}$
\item $\frac{28}{3}$
\end{enumerate}

\item The value of
\begin{align*}
I = \int_{0}^{\pi/2}\frac{(\sin x + \cos x)^2}{\sqrt{1 + \sin 2x}}dx = 
\end{align*}
\begin{enumerate}
\item 3
\item 1
\item 2
\item 0
\end{enumerate}

\item If $f(x) = \frac{e^x}{1 + e^x}$,
\begin{align*}
I_1 = \int_{f(-a)}^{f(a)}xg\{x(1-x)\}dx
\end{align*}
and
\begin{align*}
I_2 = \int_{f(-a)}^{f(a)}g\{x(1-x)\}dx
\end{align*}
then the value of $\frac{I_2}{I_1}$ is
\begin{enumerate}
\item 1
\item -3
\item -1
\item 2
\end{enumerate}

\item The area of the region bounded by the curves $y = |x - 2|$, x = 1, x = 3 and the x-axis is
\begin{enumerate}
\item 4
\item 2
\item 3
\item 1
\end{enumerate}

\item If 
\begin{align*}
I_1 = \int_{0}^{1}2^{x^2}dx, I_2 = \int_{0}^{1}2^{x^3}dx
\end{align*}
\begin{align*}
I_3 = \int_{1}^{2}2^{x^2}dx, I_4 = \int_{1}^{2}2^{x^3}dx
\end{align*}
then
\begin{enumerate}
\item $I_2 > I_1$
\item $I_2 < I_1$
\item $I_3 = I_4$
\item $I_3 > I_4$
\end{enumerate}

\item The area enclosed between the curve $y = log_e(x + e)$ and the coordinate axes is
\begin{enumerate}
\item 1
\item 2
\item 3
\item 4
\end{enumerate}

\item The parabolas $y^2 = 4x$ and $x^2 = 4y$ divide the square region bounded by the lines x = 4, y = 4 and the coordinate axes. If $S_1$, $S_2$, $S_3$ are respectively the areas of these parts numbered from top to bottom; then 
$S_1:S_2:S_3$ is
\begin{enumerate}
\item 1:2:1
\item 1:2:3
\item 2:1:2
\item 1:1:1
\end{enumerate}

\item Let $f(x)$ be a non-negative continuous function such that the area bounded by the curve $y = f(x),$ x-axis and the ordinates $x = \frac{\pi}{4}$ and $x = \beta > \frac{\pi}{4}$ is $\left(\beta\sin \beta + \frac{\pi}{4}\cos \beta + \sqrt{2}\beta\right)$. Then $f(\frac{\pi}{2})$ is
\begin{enumerate}
\item $\left(\frac{\pi}{4} + \sqrt{2} -1\right)$
\item $\left(\frac{\pi}{4} - \sqrt{2} +1\right)$
\item $\left(1 - \frac{\pi}{4} - \sqrt{2}\right)$
\item $\left(1 - \frac{\pi}{4} + \sqrt{2}\right)$
\end{enumerate} 

\item The value of
\begin{align*}
\int_{-\pi}^{\pi}\frac{\cos^{2}x}{1 + a^x}dx = 
\end{align*}
\begin{enumerate}
\item $a\pi$
\item $\frac{\pi}{2}$
\item $\frac{\pi}{a}$
\item $2\pi$
\end{enumerate}

\item The value of integral
\begin{align*}
\int_{3}^{6}\frac{\sqrt{x}}{\sqrt{9 - x} + \sqrt{x}}dx = 
\end{align*}
\begin{enumerate}
\item 1/2
\item 3/2
\item 2
\item 1
\end{enumerate}

\item $\int_{0}^{\pi}xf(\sin x)dx$ = 
\begin{enumerate}
\item $\pi\int_{0}^{\pi}f(\cos x)dx$
\item $\pi\int_{0}^{\pi}f(\sin x)dx$
\item $\frac{\pi}{2}\int_{0}^{\pi/2}f(\sin x)dx$
\item $\pi\int_{0}^{\pi/2}f(\cos x)dx$
\end{enumerate}

\item 
\begin{align*}
\int_{-3\pi/2}^{-\pi/2}[(x + \pi^3) + \cos^{2}(x + 3\pi)]dx =
\end{align*} 
\begin{enumerate}
\item $\frac{\pi^{4}}{32}$
\item $\frac{\pi^{4}}{32} + \frac{\pi}{2}$
\item $\frac{\pi}{2}$
\item $\frac{\pi}{4} - 1$
\end{enumerate}

\item The value of
\begin{align*}
\int_{1}^{a}[x]f'(x)dx
\end{align*}
$a > 1$ where [x] denotes the greatest integer not exceeding x is
\begin{enumerate}
\item $af(a) - \{f(1) + f(2) +.........f([a])\}$
\item $[a]f(a) - \{f(1) + f(2) +.......f([a])\}$
\item $[a]f([a]) - \{f(1) + f(2) +........f(a)\}$
\item $af([a]) - \{f(1) + f(2) + ......f(a)\}$
\end{enumerate}

\item Let $F(x) = f(x) + f(\frac{1}{x})$, where
\begin{align*}
f(x) = \int_{l}^{t}\frac{logt}{1 + t}dt
\end{align*}
Then $F(e)$ equals
\begin{enumerate}
\item 1
\item 2
\item 1/2
\item 0
\end{enumerate}

\item The solution for x of the equation
\begin{align*}
\int_{\sqrt{2}}^{x}\frac{dt}{\sqrt{t^2 - 1}} = \frac{\pi}{2} = 
\end{align*}
\begin{enumerate}
\item $\frac{\sqrt{3}}{2}$
\item $2\sqrt{2}$
\item 2
\item None
\end{enumerate}

\item The area enclosed between the curves $y^2 = x$ and $y = |x|$ is
\begin{enumerate}
\item 1/6
\item 1/3
\item 2/3
\item 1
\end{enumerate}

\item Let
\begin{align*}
I = \int_{0}^{1}\frac{\sin x}{\sqrt{x}}dx, J = \int_{0}^{1}\frac{\cos x}{\sqrt{x}}dx 
\end{align*}
Then which one of the following is True?
\begin{enumerate}
\item $I > \frac{2}{3}$ and $J > 2$
\item $I < \frac{2}{3}$ and $J < 2$
\item $I < \frac{2}{3}$ and $J > 2$
\item $I > \frac{2}{3}$ and $J < 2$
\end{enumerate}

\item The area of the region bounded by the parabola $(y - 2)^2 = x - 1$, the tangent of the parabola at the point
(2, 3) and the x-axis is
\begin{enumerate}
\item 6
\item 9
\item 12
\item 3
\end{enumerate}

\item The area of the plane region bounded by the curves $x + 2y^2 = 0$ and $x + 3y^2 = 1$ is eqaul to
\begin{enumerate}
\item 5/3
\item 1/3
\item 2/3
\item 4/3
\end{enumerate}

\item $\int_{0}^{\pi}[\cot x]$, where [ ] denotes the greatest integer function is equal to
\begin{enumerate}
\item 1
\item -1
\item $\frac{\pi}{2}$
\item $\frac{\pi}{2}$
\end{enumerate}

\item The area of bounded by the curves $y = \cos x$ and $y = \sin x$ between the ordinates x = 0 and $x = \frac{3\pi}{2} $ is
\begin{enumerate}
\item $4\sqrt{2} + 2$
\item $4\sqrt{2} - 1$
\item $4\sqrt{2} + 1$
\item $4\sqrt{2} - 2$
\end{enumerate}

\item Let $p(x)$ be a function defined on $R$ such that $p'(x) = p'(1 - x)$, for all $x \in [0, 1]$, p(0) = 1 and p(1) = 41. Then
\begin{align*}
\int_{0}^{1}p(x)dx = 
\end{align*}
\begin{enumerate}
\item 21
\item 41
\item 42
\item $\sqrt{41}$
\end{enumerate}

\item The value of
\begin{align*}
\int_{0}^{1}\frac{8log(1 + x)}{1 + x^2}dx = 
\end{align*}
\begin{enumerate}
\item $\frac{\pi}{8}log2$
\item $\frac{\pi}{2}log2$
\item $log2$
\item $\pi log2$
\end{enumerate}

\item The area of the region enclosed by the curves y = x, x = e, $y = \frac{1}{x}$ and the positive x-axis is
\begin{enumerate}
\item 1 sq.units
\item $\frac{3}{3}$ sq.units
\item $\frac{5}{3}$ sq.units
\item $\frac{1}{2}$ sq.units
\end{enumerate}

\item The area between the parabolas $x^2 = \frac{y}{4}$ and $x^2 = 9y$ and the straight line y = 2 is
\begin{enumerate}
\item $20\sqrt{2}$
\item $\frac{10\sqrt{2}}{3}$
\item $\frac{20\sqrt{2}}{2}$
\item $10\sqrt{2}$
\end{enumerate}

\item If 
\begin{align*}
g(x)  = \int_{0}^{x}\cos 4t dt
\end{align*}
then $g(x + \pi)$ is equal to
\begin{enumerate}
\item $\frac{g(x)}{g(\pi)}$
\item $g(x) + g(\pi)$
\item $g(x) - g(\pi)$
\item $g(x).g(\pi)$
\end{enumerate}

\item 
\textbf{Statement-1:} The value of the integral
\begin{align*}
\int_{\pi/6}^{\pi/3}\frac{dx}{1 + \sqrt{\tan x}}
\end{align*}
is equal to $\pi/6$

\textbf{Statement-2:} 
\begin{align*}
\int_{a}^{b}f(x)dx = \int_{a}^{b}f(a + b -x)dx
\end{align*}
\begin{enumerate}
\item Statement-1 is true, Statement-2 is true, Statement-2 is a correct explanation for Statement-2
\item Statement-1 is true, Statement-2 is true, Statement-2 is not a correct explanation for Statement-2
\item Statement-1 is true, Statement-2 is false
\item Statement-1 is false, Statement-2 is true
\end{enumerate}

\item The area(in square units) bounded by the curves $y = \sqrt{x}$, 2y - x + 3 = 0, x-axis and lying in the first quadrant is
\begin{enumerate}
\item 9
\item 36
\item 18
\item 27/4
\end{enumerate}

\item The integral
\begin{align*}
\int_{0}^{\pi}\sqrt{1 + 4\sin^{2}\frac{x}{2} - 4\sin\frac{x}{2}}dx = 
\end{align*}
\begin{enumerate}
\item $4\sqrt{3} - 4$
\item $4\sqrt{3} - 4 - \frac{\pi}{3}$
\item $\pi - 4$
\item $\frac{2\pi}{3} - 4 - 4\sqrt{3}$
\end{enumerate}

\item The area of the region bounded by 
\begin{align*}
\{(x, y): y^2 \leq 2x, y \geq 4x - 1\} = 
\end{align*}
\begin{enumerate}
\item $\frac{15}{64}$
\item $\frac{9}{32}$
\item $\frac{7}{32}$
\item $\frac{5}{64}$
\end{enumerate}

\item The area of the region bounded by 
\begin{align*}
A = \{(x, y): x^2 + y^2 \leq 1, y^2 \leq 1 - x\} = 
\end{align*} 
\begin{enumerate}
\item $\frac{\pi}{2} - \frac{2}{3}$
\item $\frac{\pi}{2} + \frac{2}{3}$
\item $\frac{\pi}{2} + \frac{4}{3}$
\item $\frac{\pi}{2} - \frac{4}{3}$
\end{enumerate}

\item The integral
\begin{align*}
\int_{2}^{4}\frac{log x^2}{log x^2 + log(36 - 12x + x^2)}dx
\end{align*}
is equal to
\begin{enumerate}
\item 1
\item 6
\item 2
\item 4
\end{enumerate}

\item The area(in square units) of the region
\begin{align*}
\{(x, y): y^2 \geq 2x, x^2 + y^2 \leq 4x, x \geq 0, y \geq 0\} = 
\end{align*}
\begin{enumerate}
\item $\pi - \frac{4\sqrt{2}}{•3}$
\item $\frac{\pi}{2} - \frac{2\sqrt{2}}{3}$
\item $\pi - \frac{4}{3}$
\item $\pi - \frac{8}{3}$
\end{enumerate}

\item The area(in square units) of the region
\begin{align*}
\{(x, y): x \geq 0, x + y \leq 3, x^2 \leq 4y, y \geq 1 + \sqrt{x}\} = 
\end{align*}
\begin{enumerate}
\item 5/2
\item 59/12
\item 3/2
\item 7/3
\end{enumerate}

\item The integral
\begin{align*}
\int_{\pi/4}^{3\pi/4}\frac{dx}{1 + \cos x} = 
\end{align*}
\begin{enumerate}
\item -1
\item -2
\item 2
\item 4
\end{enumerate}

\item Let $g(x) = \cos^{2}x$, $f(x) = \sqrt{x}$ and $\alpha, \beta (\alpha < \beta)$ be the roots of the quadratic equation
\begin{align}
18x^2 - 9\pi x + \pi^2 = 0
\end{align} 
Then the area(in sq.units) bounded by the curve $y = (gof)(x)$ and he lines $x = \alpha$, $x = \beta$ and y = 0 is
\begin{enumerate}
\item $\frac{1}{2}(\sqrt{3} + 1)$
\item $\frac{1}{2}(\sqrt{3} - \sqrt{2})$
\item $\frac{1}{2}(\sqrt{2} - 1)$
\item $\frac{1}{2}(\sqrt{3} - 1)$
\end{enumerate}

\item The value of
\begin{align*}
\int_{-\pi/2}^{\pi/2}\frac{\sin^{2}x}{1 + 2^{x}}dx = 
\end{align*}
\begin{enumerate}
\item $\frac{\pi}{2}$
\item $4\pi$
\item $\frac{\pi}{4}$
\item $\frac{\pi}{8}$
\end{enumerate}

\item The value of
\begin{align*}
\int_{0}^{\pi}|\cos x|dx = 
\end{align*}
\begin{enumerate}
\item 0
\item 4/3
\item 2/3
\item -4/3
\end{enumerate}

\item The area(in sq.units) bounded by the parabola $y = x^2 - 1$, the tangent at the point (2, 3) to it and the y-axis is
\begin{enumerate}
\item 8/3
\item 32/3
\item 56/3
\item 14/3
\end{enumerate}

\item The value of
\begin{align*}
\int_{0}^{\pi/2}\frac{\sin^{3}x}{\sin x + \cos x}dx = 
\end{align*}
\begin{enumerate}
\item $\frac{\pi - 2}{8}$
\item $\frac{\pi - 1}{4}$
\item $\frac{\pi - 2}{4}$
\item $\frac{\pi - 1}{2}$
\end{enumerate}

\item The area(in sq.units) of the region
\begin{align*}
A = \{(x, y): x^2 \leq y \leq x + 2\} = 
\end{align*}
\begin{enumerate}
\item 10/3
\item 9/2
\item 31/6
\item 13/6
\end{enumerate}
\end{enumerate}








 
%\subsection{Exercises}
%\input{./calculus/calculus_exer.tex} 

\end{document}


