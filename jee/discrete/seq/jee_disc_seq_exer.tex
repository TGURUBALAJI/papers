\renewcommand{\theequation}{\theenumi}
\begin{enumerate}[label=\arabic*.,ref=\thesubsection.\theenumi]
\numberwithin{equation}{enumi}
    \item The sum of integers from 1 to 100 that are divisible by 2 or 5 is....................... 
     \item The solution of the equation $log_7 log_5(\sqrt{x+5}+\sqrt{x})=0$ is ....................
     \item The sum of the first n terms of the series $1^2+2.2^2+3^2+2.4^2+5^2+2.6^2+................$is $\frac{n(n+1)^2}{2}$, when n is even.When n is odd,the sum is ...............
     \item Let the harmonic mean and geometric mean of two positive numbers be the ratio 4:5.Then the two numbers are in the ratio...................
     \item For any odd integer n$\geq1$,$n^3-(n-1)^3+...+(-1)^{(n-1)} 1^3$=.............
     \item Let p and q be roots of the equation 
     \begin{align}
     x^2-2x+A=0
     \end{align} and let r and s be the roots of the equation 
     \begin{align}
     x^2-18x+B=0
     \end{align}. If $p<q<r<s$ are in arithmetic progression,then A =.......... and B=........

 \textbf{ MCQs with One Correct Answer}
 
 \item If x,y and z are $p^th$,$q^th$ and $r^th$ terms respectively of an A.P.and also of a G.P.,then $x^y-z y^z-x z^x-y$ is equal to:
 \begin{enumerate}
 \item xyz 
 \item 0 
 \item 1 
 \item None of these
 \end{enumerate}
\item The third term of a geometric progression is 4.The product of the first five terms is 
\begin{enumerate}
 \item $4^3$ 
 \item $4^5$
 \item $4^4$
 \item None of these
 \end{enumerate}
 \item The rational number,which equals the number $2.\overline{357}$ with recurring decimal is
 \begin{enumerate}
 \item $\frac{2355}{1001}$ 
\item  $\frac{2379}{997}$ 
 \item $\frac{2355}{999}$
 \item none of these
 \end{enumerate}
 \item If a,b,c are in G.P.,then the equations
  \begin{align}
  ax^2+2bx+c=0
  \end{align} and  
  \begin{align}
  dx^2+2ex+f=0
  \end{align} have a common root if $\frac{d}{a}, \frac{e}{b}, \frac{f}{c}$ are in......
  \begin{enumerate}
  \item A.P. 
  \item G.P. 
  \item H.P. 
  \item None of these
 \end{enumerate}
  \item Sum of the first n terms of the series $\frac{1}{2}$+$\frac{3}{4}$+$\frac{7}{8}$+$\frac{15}{16}$+.....is equal to 
   \begin{enumerate}
 \item  $2^n-n-1$ 
 \item $1-2^-n$
 \item  $n+2^-n-1$
 \item $2^n+1$
  \end{enumerate}
\item The number $log_2$7 is
   \begin{enumerate}
\item an integer 
\item a rational number 
\item an irrational number  
\item a prime number
   \end{enumerate}
  \item If $ln(a+c)$, $ln(a-c)$, $ln(a-2b+c)$ are in A.P.,then
 \begin{enumerate} 
\item a,b,c are in A.P. 
\item $a^2$, $b^2$, $c^2$ are in A.P.
\item a,b,c are in G.P. 
\item a,b,c are in H.P.
 \end{enumerate}
  \item Let $a_1,a_2,.....a_10$ be in A.P.and $h_1,h_2,.....h_{10}$ be in H.P.If $a_1=h_1=2$ and $a_{10}=h_{10}$=3, then $a_4 h_7$ is
 \begin{enumerate}
\item 2
\item 3 
\item 5 
\item 6
\end{enumerate}
  \item The harmonic mean of the roots of the equation 
  \begin{align}
  (5+\sqrt{2})x^2- (4+\sqrt{5})x+ 8 + 2\sqrt{5}=0
  \end{align} is
\begin{enumerate}
\item 2
\item 4 
\item 6 
\item 8
\end{enumerate}
\item Consider an infinite geometric series with first term a and common ratio r. If its sum is 4 and the second term is $\frac{3}{4}$, then
\begin{enumerate}
\item a=$\frac{4}{7}$,r=$\frac{3}{7}$ 
\item a=2,r=$\frac{3}{8}$
\item a=$\frac{3}{2}$,r=$\frac{1}{2}$
\item a=3,r=$\frac{1}{4}$
\end{enumerate}
\item Let $\alpha$, $\beta$ be the roots of 
\begin{align}
x^2-x+p=0
\end{align} and $\gamma$, $\delta$ be the roots of 
\begin{align}
x^2-4x+q=0
\end{align}. If $\alpha$, $\beta$, $\gamma$, $\delta$ are in G.P., then the integral values of p and q respectively, are
\begin{enumerate}
\item -2,-32 
\item -2,3 
\item -6,3 
\item -6,-32
\end{enumerate}
\item Let the positive numbers a,b,c,d be in A.P.Then abc,abd,acd,bcd are
\begin{enumerate}
\item NOT in A.P./G.P./H.P. 
\item in A.P.
\item in G.P.
\item in H.P.
\end{enumerate}
\item If the sum of the first 2n terms of the A.P.2,5,8,.....,is equal to the sum of the first n terms of the A.P.57,59,61,.....,then n equals
\begin{enumerate}
\item 10 
\item 12 
\item 11 
\item 13
\end{enumerate}
\item Suppose a,b,c are in A.P. and $a^2, b^2, c^2$ are in G.P.if $a<b<c$ and a+b+c=$\frac{3}{2}$, then the value of a is
\begin{enumerate}
\item $\frac{1}{2\sqrt{2}}$ 
\item $\frac{1}{2\sqrt{3}}$ 
\item $\frac{1}{2}$-$\frac{1}{\sqrt{3}}$ 
\item $\frac{1}{2}$-$\frac{1}{\sqrt{2}}$
\end{enumerate}
\item An infinite G.P. has first term 'x' and sum '5', then x belongs to
\begin{enumerate}
\item $x<-10$ 
\item $-10<x<0$ 
\item $0<x<10$ 
\item $x>10$
\end{enumerate}
\item In the quadratic equation 
\begin{align}
ax^2+bx+c=0,
\end{align} $\Delta =b^2-4ac$ and $\alpha+\beta$, $\alpha^2+\beta^2$, $\alpha^3+\beta^3$, are in G.P.where $\alpha, \beta$ are the root of $ax^2+bx+c=0$, then 
\begin{enumerate}
\item $\Delta\neq0$
\item $b\Delta=0$ 
\item $c\Delta0$ 
\item $\Delta=0$
\end{enumerate}
\item In the sum of first n terms of an A.P.is $cn^2$, then the sum of squares of these n terms is
\begin{enumerate}
\item $\frac{n(4n^2-1)c^2}{6}$ 
\item $\frac{n(4n^2+1)c^2}{3}$ 
\item $\frac{n(4n^2-1)c^2}{3}$ 
\item $\frac{n(4n^2+1)c^2}{6}$
\end{enumerate}
\item Let $a_1,a_2,a_3$,........be in harmonic progression with $a_1=5$ and $a_{20}=25$. The least positive integer n for which $a_n<0$ is
\begin{enumerate}
\item 22 
\item 23 
\item 24 
\item 25
\end{enumerate}
\item Let $b_1>1$ for $i=1,2,....,101.$Suppose $\log_e b_1,\log_e b_2,.....\log_e b_{101}$ are in Arithmetic progression(A.P) with the common difference $\log_e 2$. Suppose $a_1,......a_{101}$ are in A.P such that $a_1=b_1$ and $a_{51}=b_{51}$.If $t=b_1+b_2+......+b_{51}$ and $s=a_1+a_2+...a_{53},$ then 
\begin{enumerate}
\item $s>t$ and $a_{101}>b_{101}$
\item $s>t$ and $a_{101}<b_{101}$
\item $s<t$ and $a_{101}>b_{101}$
\item $s<t$ and $a_{101}<b_{101}$
\end{enumerate}

\textbf{MCQs with One or More than One Correct}

\item If the first and the $(2n-1)$st terms of an A.P., a G.P. and an H.P. are equal and their n-th terms are a,b and c respectively , then
\begin{enumerate}
\item a=b=c 
\item $a \geq b \geq c$   
\item a+c=b 
\item $ac-b^2=0$.
\end{enumerate}
\item For $0<\phi<\frac{\Pi}{2}$, if $x=\sum_{n=0}^{\infty}(\cos^{2n})\phi$, $y=\sum_{n=0}^{\infty}(\sin^{2n})\phi$, $z=\sum_{n=0}^{\infty}(\cos^{2n})\phi(\sin^{2n})\phi$ then:
\begin{enumerate}
\item xyz = xz+y 
\item xyz = xy+z 
\item xyz = x+y+z 
\item xyz = yz+x
\end{enumerate}
\item Let n be an odd integer.If $\sin n\theta=\sum_{r=0}^{n}(b)_r \sin^r\theta$, for every value of $\theta$,then 
\begin{enumerate}
\item $b_0=1, b_1=3$ 
\item $b_0=0, b_1=n$ 
\item $b_0=-1, b_1=n$ 
\item $b_0=0, b_1=n^2-3n+3$
\end{enumerate}
\item Let $T_r$ be the $r^{th}$ term of an A.P.,for r=1,2,3,..... If for some positive integers m,n we have $T_m=\frac{1}{n}$ and $T_n=\frac{1}{m}$, then $T_{mn}$ equals
\begin{enumerate}
\item $\frac{1}{mn}$ 
\item $\frac{1}{m}$+$\frac{1}{n}$ 
\item 1 
\item 0
\end{enumerate}
\item If $x>1, y>1, z>1$ are in G.P.,then $\frac{1}{1+ln x}$,$\frac{1}{1+ln y}$,$\frac{1}{1+ln z}$ are in 
\begin{enumerate}
\item A.P. 
\item H.P. 
\item G.P. 
\item None of these
\end{enumerate}
\item For a positive integer n, let $a(n)= 1+\frac{1}{2}+\frac{1}{3}+\frac{1}{4}+.....\frac{1}{(2^n)-1}$. Then
\begin{enumerate}
\item $a(100)\leq100$ 
\item $a(100)>100$ 
\item $a(200)\leq100$ 
\item $a(200)>100$
\end{enumerate}
\item A straight line through the vertex P of a triangle PQR intersects the side QR at the points 
$\vec{S}$ and the circumcircle of the triangle PQR at the point $\vec{T}$.If S is not the centre of the circumcircle, then
\begin{enumerate}
\item $\frac{1}{(PS}$+ $\frac{1}{(ST}$ $<\frac{2}{\sqrt{QS X SR}}$ 
\item $\frac{1}{(PS}$+ $\frac{1}{(ST}$ $>\frac{2}{\sqrt{QS X SR}}$  
\item $\frac{1}{(PS}$+ $\frac{1}{(ST}$ $<\frac{4}{QR}$ 
\item $\frac{1}{(PS}$+ $\frac{1}{(ST}$ $>\frac{2}{QR}$
\end{enumerate}
\item Let $S_n=\sum_{k=1}^{n} \frac{n}{n^2+kn+k^2}$ and $T_n=\sum_{k=0}^{n-1}\frac{n}{n^2+kn+k^2}$ for n=1,2,3......Then 
\item Let $S_n=\sum_{k=1}^{4n}(-1)^\frac{k(k+1)}{2}k^2$. Then $S_n$ can take value(S)
\begin{enumerate}
\item 1056 
\item 1088 
\item 1120 
\item 1332
\end{enumerate}
\item Let $\alpha$ and $\beta$ be the roots of $x^2-x-1=0$ with $\alpha > \beta$.For all positive integers n,define $a_n= \frac{\alpha^n - \beta^n}{\alpha-\beta},n\geq 1 b_1$ and $b_n=a_{n-1}+a_{n+1}, n\geq 2$ Then which of the following options is /are correct?
\begin{enumerate}
\item $\sum_{n=1}^{\infty} \frac{\alpha_n}{10^n}=\frac{10}{89}$
\item $b_n=\alpha^n+\beta^n$ for all $n\geq 1$
\item $a_1+a_2+a_3+....a_n=a_{n+2}-1$ for all $n\geq1$
\item $\sum_{n=1}^{\infty}\frac{b_n}{10^n}=\frac{8}{89}$
\end{enumerate} 

\textbf {Subjective Problems}

\item The harmonic mean of two numbers is 4. Their arithmetic mean A and the geometric mean G satisfy the relation, $2A+G^2=27$. Find the two numbers.
\item The interior angles of a polygon are in arithmetic progression. The smallest angle is 
$120\degree$, and the common difference is $5\degree$. Find the number of sides of the polygon.
\item Does there exists a geometric progression containing 27, 8 and 12 as three of its terms ? If it exits, how many such progressions are possible ?
\item Find three numbers a,b,c between 2 and 18 such that
\begin{enumerate}
\item their sum is 25 
\item the numbers 2,a,b are consecutive terms of an A.P. and  
\item the numbers b,c,18 are consecutive terms of a G.P.
\end{enumerate}
\item If $a>0,b>0$ and $c>0$, prove that $(a+b+c)(\frac{1}{a}+\frac{1}{b}+\frac{1}{c})\geq9$
\item If n is a natural number such that  $n= p_1^\alpha 1.p_2^\alpha 2.p_3^\alpha 3....p_k^\alpha k$ and $p_1,p_2,......p_k$ are distinct primes, then show that $ln n \geq k ln 2$
\item Find the sum of the series: $\sum_{r=0}^{n}(-1)^r n C_r[\frac{1}{2^r}+\frac{3^r}{2^{2r}}+\frac{7^r}{2^{3r}}+\frac{15^r}{2^{4r}}$......upto m terms]
\item Solve for x the following equation: $log_{2x+3}(6x^2+23x+21)=4-log_{3x+7}(4x^2+12x+9)$
\item If $log_3 2$, $log_3 (2^x-5)$, $log_3 (2^x-\frac{7}{2})$ are in arithmetic progression, determine the value of x.
\item Let p be the first of the n arithmetic means between two numbers and q the first of n harmonic means between the same numbers. Show that q does not lie between p and $[\frac{n+1}{n-1}]^2$p.
\item If $S_1$, $S_2$, $S_3$,......., $S_n$ are the sums of infinite geometric series whose first terms are 1, 2, 3,......, n and whose common ratios are $\frac{1}{2}$, $\frac{1}{3}$, $\frac{1}{4}$,...........$\frac{1}{n+1}$ respectively, then find the values of $S_1^2$+ $S_2^2$+ $S_3^2$,......., $S^2_{2n-1}$
\item The real numbers $x_1,x_2, x_3$ satisfying the equation $x^3-x^2+\beta x+\gamma=0$ are in A.P. Find the intervals in which $\beta$ and $\gamma$ lie.
\item Let a,b,c,d are the real numbers in G.P. If u,v,w, satisfy the system of equations $u+2v+3w=6$ $4u+5v+6w=12$ $6u+9v=4$ then show that the root of the equation $(\frac{1}{u}+\frac{1}{v}+\frac{1}{w})x^2+[(b-c)^2+(c-a)^2+(d-b)^2]x+u+v+w=0$ and $20x^2+10(a-d)^2x-9=0$ are reciprocals of the each other. 
\item The fouth power of the common difference of an arithmetic progression with integer entries is added to the product of any four consecutive terms of it. Prove that the resulting sum is the square of an integer.
\item Let $a_1$, $a_2$,.......,$a_n$ be positive real numbers in geometric progression. For each n, let $A_n$, $G_n$, $H_n$ be respectively, the arithmetic mean, geometric mean, and harmonic mean of $a_1$, $a_2$,.......,$a_n$. Find an expression for the geometric mean of $G_1$, $G_2$,......, $G_n$, in terms of $A_1$, $A_2$,.......,$A_n$,  $H_1$,$H_2$,......,$H_n$,
\item Let a, b be positive real numbers.If a, $A_1$, $A_2$, b are in arithmetic progression, a, $G_1$, $G_2$, b are in geometric progression and a, $H_1$, $H_2$, b are in harmonic progression, show that $\frac{G_1G_2}{H_1H_2}=\frac{A_1+A_2}{H_1+H_2}=\frac{(2a+b)(a+2b)}{9ab}$.
\item If a, b, c are in A.P., $a^2$, $b^2$, $c^2$ are in H.P., then prove that either a = b = c or a, b, $-\frac{c}{2}$ form a G.P.
\item If $a_n=\frac{3}{4}-[\frac{3}{4}]^2+[\frac{3}{4}]^3+......(-1)^{n-1} [\frac{3}{4}]^n$ and $b_n = 1- a_n$, then find the least natural number $n_0$ such that $b_n > a_n \forall  n \geq n_0$.

\textbf{Comprehension Based Questions}
                               
\textbf {{PASSAGE - 1}}

Let $V_r$ denote the sum of first r terms of an arithmetic progression(A.P.) whose first term is r and the common difference is (2r-1). Let $T_r=V_{r+1}-V_r-2$ and $Q_r=T_{r+1}-T_r$ for r=1,2,...

\item The sum $V_1+V_2+....+V_n$ is

\begin{enumerate}
\item $\frac{1}{12}n(n+1)(3n^2-n+1)$   
\item $\frac{1}{12}n(n+1)(3n^2+n+2)$  
\item $\frac{1}{2}n(2n^2-n+1)$   
\item $\frac{1}{3}(2n^3-2n+3)$
\end{enumerate}

\item $T_r$ is always 

\begin{enumerate}
\item an odd number          
\item an even number    
\item a prime number 
\item a composite number
\end{enumerate}

\item Which one of the following is a correct statement ?

\begin{enumerate}
\item $Q_1, Q_2, Q_3$,.... are in A.P. with common difference 5
\item $Q_1, Q_2, Q_3$,.... are in A.P. with common difference 6
\item $Q_1, Q_2, Q_3$,.... are in A.P. with common difference 11
\item $Q_1= Q_2=Q_3$=........
\end{enumerate}

\textbf {PASSAGE - 2}
               
Let $A_1, G_1, H_1$ denote the arithmetic, geometric and harmonic means respectively, of two distinct positive numbers. For $n\geq 2$, let $A_{n-1}$ and $H_{n-1}$  have  arithmetic, geometric and harmonic means as $A_n,G_n,H_n$ respectively.

\item Which one of the following statements is correct ?

\begin{enumerate}
\item $G_1 > G_2 > G_3 >..........$    
\item $G_1 < G_2 < G_3 <..........$      
\item $G_1 = G_2 = G_3 =..........$    
\item $G_1 < G_3 < G_5$ and $G_2 > G_4 > G_6 >..........$
\end{enumerate}

\item Which one of the following statements is correct ?

\begin{enumerate}
\item $A_1 > A_2 > A_3 >..........$    
\item $A_1 < A_2 < A_3 <..........$         
\item $A_1 > A_3 > A_5>....$ and $A_2 < A_4 < A_6 <..........$
\item $A_1 < A_3 < A_5<....$ and $A_2 > A_4 > A_6 >..........$
\end{enumerate}

\item Which one of the following statements is correct ?

\begin{enumerate}
\item $H_1 > H_2 > H_3 >..........$    
\item $H_1 < H_2 < H_3 <..........$        
\item $H_1 > H_3 > H_5>....$ and $H_2 < H_4 < H_6 <..........$
\item $H_1 < H_3 < H_6<....$ and $H_2 > H_4 > H_6 >..........$
\end{enumerate}
\textbf{Assertion Reson type quations} 
\item Suppose four distinct positive numbers $a_1$, $a_2$, $a_3$, $a_4$ are in G.P. Let $b_1 = a_1$, $b_2 = b_1 + a_2$, $b_3 = b_2 + a_3$ and $b_4 = b_3 + a_4$.

STATEMENT - 1: The numbers $b_1$, $b_2$, $b_3$, $b_4$ are neither in A.P. nor in G.P. and

STATEMENT - 2: The numbers $b_1$, $b_2$, $b_3$, $b_4$ are in H.P.
\begin{enumerate}
\item STATEMENT - 1 is True, STATEMENT - 2 is True; STATEMENT - 2 is a correct explanation for STATEMENT - 1
\item STATEMENT - 1 is True, STATEMENT - 2 is True; STATEMENT - 2 is a NOT a correct explanation for STATEMENT - 1
\item STATEMENT - 1 is True, STATEMENT - 2 is False
\item STATEMENT - 1 is False, STATEMENT - 2 is True
\end{enumerate}

\textbf{Integer Value Correct Type}

\item Let $S_k, k= 1, 2,......, 100$, denote the sum of the infinite geometric series whose first term is $\frac{k-1}{k!}$ and the common ratio is $\frac{1}{k}$. Then the value of $\frac{100^2}{100!}+\sum_{k=1}^{100} \abs{(k^2 - 3k + 1)S_k}$  is 
\item $a_1, a_2, a_3,.......,a_{11}$ be real numbers satisfying $a_1=15,27-2a_2>0$ and $a_k=2a_{k-1}-a_{k-2}$ for k= 3,4,.....,11, if $\frac{a_1^2 + a_2^2 +.....+a_{11}^2}{11}=90$,then the value of
$\frac{a_1 + a_2 +.....+a_{11}}{11}$ is equal to
\item Let $a_1$, $a_2$, $a_3$,.......,$a_{100}$ be an arithmetic progression with $a_1=3$ and $S_p=\sum_{i=1}^{p} a_i,1\leq p\leq 100$.For any integer n with $1\leq n\leq 20$, let m=5n .If $\frac{S_m}{S_n}$ does not depend on n, then $a_2$ is......
\item A pack contains n cards numbered from 1 to n. Two consecutive numbered cards are removed from the pack and the sum of the numbers on the remaining cards is 1224. If the smaller of the numbers on the removed cards is k, then k-20=..........
\item Let a ,b, c be positive integers such that $\frac{b}{a}$ is an integer. If a, b, c are in geometric progression and the arithmetic mean of a, b, c is b+2, then the value of 
$\frac{a^2+a-14}{a+1}$ is 
\item Suppose that all the terms of an arithmetic progression(A.P.)
are natural numbers. If the ratio of the sum of the first seven terms to the sum of the first eleven terms is 6:11 and the seventh term is lies in between 130 and 140, then the common difference of this A.P. is
\item The coefficient of $x^9$ in the expansion of $(1+x)(1+x^2)(1+x^3)......(1+x^{100})$ is
\item The sides of a right angled triangle are in arithmetic progression. If the triangle has area 24, then what is the length of its smallest side ?
\item Let X be the set consisting of the first 2018 terms of the arithmetic progression $1, 6, 11,....$, and Y be the set consisting of the first 2018 terms of the arithmetic progression $9, 16, 23,.....$ Then, the number of elements in the set $X\cup Y$ is ....
\item Let AP(a;d) denote the set of all the terms of an infinite arithmetic progression with first term a and common difference $d>0$. If AP(1;3) AP(2;5) AP(3;7)= AP(a;d) then a+d equals.....
{\textbf {Section-B JEE Main/AIEEE} }
\item If 1, $log_9(3^{1-x} +2),log_3(4.3^x - 1)$ are in A.P. then x equals
\begin{enumerate}
\item $log_3 4$  
\item $1- log_3 4$  
\item $1-log_4 3$  
\item $log_4 3$
\end{enumerate}
\item l, m, n are the $p^{th},q^{th}$ , and $r^{th}$ term of a G.P. all positive, then 
 $\myvec{\log l & p & 1 \\ \log m & q  & 1 \\ \log n n & r & 1}$  equals
\begin{enumerate}
\item -1  
\item 2   
\item 1  
\item 0
\end{enumerate}
\item The value of $2^\frac{1}{4}.4^\frac{1}{8}.8^\frac{1}{16}.........\infty$ is
\begin{enumerate}
\item 1  
\item 2   
\item $\frac{3}{2}$  
\item 4
\end{enumerate}
\item Fifth term of a G.P. is 2, then the product of its 9 terms is
\begin{enumerate}
\item 256  
\item 512   
\item 1024  
\item none of these
\end{enumerate}
\item Sum of infinite number of terms of GP is 20 and sum of their square is 100. The common ratio of GP is
\begin{enumerate}
\item 5  
\item $\frac{3}{5}$   
\item $\frac{8}{5}$  
\item $\frac{1}{5}$
\end{enumerate}
\item $1^3-2^3+3^3-4^3+....+9^3$=
\begin{enumerate}
\item 425  
\item -425   
\item 475  
\item -475
\end{enumerate}
\item The sum of the series $\frac{1}{1.2}-\frac{1}{2.3}+\frac{1}{3.4}.......$ upto $\infty$ is equal to 
\begin{enumerate}
\item $log_e\frac{4}{e}$
\item $2 log_e 2$   
\item $log_e 2-1$  
\item $log_e 2$
\end{enumerate}
\item If $S_n = \sum_{r=0}^{n}\frac{1}{nC_r}$ and $t_n = \sum_{r=0}^{n}\frac{r}{nC_r}$, then 
$\frac{t_n}{S_n}$ is equal to 
\begin{enumerate}
\item $\frac{2n-1}{2}$
\item $\frac{1}{2}$n-1   
\item  n-1 
\item $\frac{1}{2}$n
\end{enumerate}
\item Let $T_r$ be the $r^{th}$ term of an A.P.whose first term is a and common difference is d. If for some positive integers m, n, $m\neq n$, $T_m = \frac{1}{n}$ and $T_n = \frac{1}{m}$, then a-d equals
\begin{enumerate}
\item $\frac{1}{m}+\frac{1}{n}$
\item 1
\item $\frac{1}{mn}$
\item 0
\end{enumerate}
\item The sum of the first n terms of the series $1^2 + 2.2^2 + 3^2 + 2.4^2 + 5^2 + 2.6^2+.....$ is 
$\frac{n(n+1)^2}{2}$ when n is even. When n is odd the sum is 
\begin{enumerate}
\item $[\frac{n(n+1)}{2}]^2$
\item $\frac{n^2(n+1)}{2}$
\item $\frac{n(n+1)^2}{4}$
\item $\frac{3n(n+1)}{2}$
\end{enumerate}
\item The sum of series $\frac{1}{2!}+\frac{1}{4!}+\frac{1}{6!}.....$ is 
\begin{enumerate}
\item $\frac{(e^2 - 2)}{e}$
\item $\frac{(e-1)^2}{2e}$
\item $\frac{(e^2-1)}{2e}$
\item $\frac{(e^2-1)}{2}$
\end{enumerate}
\item If the coefficient of $r^{th}$,$(r+1)^th$ and $(r+2)^{th}$ terms in the binomial expansion of $(1+y)^m$ are in A.P., then m and r satisfy the equation
\begin{enumerate}
\item $m^2 - m(4r-1) + 4r^2 - 2 = 0$
\item $m^2 - m(4r+1) + 4r^2 + 2 = 0$
\item $m^2 - m(4r+1) + 4r^2 - 2 = 0$
\item $m^2 - m(4r-1) + 4r^2 + 2 = 0$
\end{enumerate}
\item If $x=\sum_{n=0}^{\infty}a^n$,  $y=\sum_{n=0}^{\infty}b^n$,  $z=\sum_{n=0}^{\infty}c^n$ where a, b, c are in A.P. and $\abs{a}<1,\abs{b}<1,\abs{c}<1$ then x, y, z are in
\begin{enumerate}
\item G.P.
\item A.P.
\item Arithmetic - Geometric Progression
\item H.P.
\end{enumerate}
\item The sum of series 1+$\frac{1}{4.2!}+\frac{1}{16.4!}+\frac{1}{64.6!}.....\infty$ is 
\begin{enumerate}
\item $\frac{(e - 1)}{\sqrt{e}}$
\item $\frac{(e + 1)}{\sqrt{e}}$
\item $\frac{(e - 1)}{2\sqrt{e}}$
\item $\frac{(e + 1)}{2\sqrt{e}}$
\end{enumerate}
\item Let $a_1$, $a_2$, $a_3$.....be terms in A.P. If $\frac{a_1+a_2+......+a_p}{a_1+a_2+......+a_q} = \frac{p^2}{q^2}$, $p\neq q$, then $\frac{a_6}{a_{21}}$ equals 
\begin{enumerate}
\item $\frac{41}{11}$
\item $\frac{7}{2}$.
\item $\frac{2}{7}$
\item $\frac{11}{41}$
\end{enumerate}
\item If $a_1$, $a_2$, $a_3$.....$a_n$ are in H.P., then the expression $a_1a_2+a_2a_3+.....+a_{n-1}a_n$ is equal to
\begin{enumerate}
\item $n(a_1 -a_n)$
\item $(n-1)(a_1 -a_n)$
\item $n(a_1a_n)$
\item $(n-1)(a_1a_n)$
\end{enumerate}
\item The sum of series $\frac{1}{2!}-\frac{1}{3!}+\frac{1}{4!}.....$ upto $\infty$ is 
\begin{enumerate}
\item $e^-\frac{1}{2}$
\item $e^+\frac{1}{2}$
\item $e^{-2}$
\item $e^{-1}$
\end{enumerate}
\item In a geometric progression consisting of positive terms, each term equals the sum of the next two terms. Then the common ratio of its progression is equals
\begin{enumerate}
\item $\sqrt{5}$
\item $\frac{1}{2}(\sqrt{5} - 1)$
\item $\frac{1}{2}(1-\sqrt{5})$
\item $\frac{1}{2}\sqrt{5}$
\end{enumerate}
\item The first two terms of a geometric progression add up to 12. the sum of the third and the fourth terms is 48. If the terms of the geometric progression are alternately positive and negative, then the first term is
\begin{enumerate}
\item -4
\item -12
\item 12
\item 4
\end{enumerate}
\item The sum to infinite term of the series $1+\frac{2}{3}+\frac{6}{3^2}+\frac{10}{3^3}+\frac{14}{3^4}+.....$ is 
\begin{enumerate}
\item 3
\item 4
\item 6
\item 2
\end{enumerate}
\item A person is to count 4500 currency notes. Let $a_n$ denote the number of notes he counts in the $n^{th}$ minute. If $a_1=a_2=.....=a_{10}=150$ and $a_{10},a_{11},.....$ are in A.P. with common difference -2, then the time taken by him to count all notes is
\begin{enumerate}
\item 34 minutes
\item 125 minutes
\item 135 minutes
\item 24 minutes
\end{enumerate}
\item A man saves 200 in each of the first three months of his service. In each of the subsequent months his saving increases by 40 more than the saving of immediately previous month. His total savings from the start of service will be 11040 after
\begin{enumerate}
\item 19 months
\item 20 months
\item 21 months
\item 18 months
\end{enumerate}
\item \textbf{Statement - 1:} The sum of the series $1+(1+2+4)+(4+6+9)+(9+12+16)+.....+(361+380+400)$ is $8000.$\\
	  \textbf{Statement - 2:} $\sum_{k=1}^{n}(k^3 -(k-1)^3)=n^3$, for any natural number n.

\begin{enumerate}
\item Statement - 1 is false, Statement - 2 is true.
\item Statement - 1 is true, Statement - 2 is true, Statement - 2 is a correct explanation for Statement - 1
\item Statement - 1 is true, Statement - 2 is true, Statement - 2 is a not a correct explanation for Statement - 1
\item Statement - 1 is true, Statement - 2 is false.
\end{enumerate}
\item The sum of the first 20 terms of sequence 0.7,0.77,0.777,....,is
\begin{enumerate}
\item $\frac{7}{81}(179-10^{-20})$
\item $\frac{7}{9}(99-10^{-20})$
\item $\frac{7}{81}(179+10^{-20})$
\item $\frac{7}{9}(99+10^{-20})$
\end{enumerate}
\item If $(10)^9+2(11)^1(10^8)+3(11)^2(10)^7+........+10(11)^9=k(10)^9,$ then k is equal to :
\begin{enumerate}
\item 100
\item 110
\item $\frac{122}{10}$
\item $\frac{441}{100}$
\end{enumerate}
\item Three positive numbers form an increasing G.P. If the middle term in this G.P. is doubled, the new numbers are in A.P. then the common ration of the G.P. is:
\begin{enumerate}
\item $2-\sqrt{3}$
\item $2+\sqrt{3}$
\item $\sqrt{2}+\sqrt{3}$
\item $3+\sqrt{2}$
\end{enumerate}
\item The sum of the first 9 terms of the series.$\frac{1^3}{1}+\frac{1^3+2^3}{1+3}+\frac{1^3+2^3+3^3}{1+3+5}$+.......
\begin{enumerate}
\item 142
\item 192
\item 71
\item 96
\end{enumerate}
\item If m is the A.M. of two distinct real numbers l and $n(l,n>1)$ and $G_1,G_2$ and $G_3$ are the three geometric means between l and n, then $G_1^4+2G_2^4+G_3^4$ equals:
\begin{enumerate}
\item $4lmn^2$
\item $4l^2m^2n^2$
\item $4l^2mn$
\item $4lm^2n$
\end{enumerate}
\item If the $2^{nd},5^{th}$ and $9^{th}$ terms of a non-constant A.P. are in G.P., then the common ratio of this G.P. is:
\begin{enumerate}
\item 1
\item $\frac{7}{4}$
\item $\frac{8}{5}$
\item $\frac{4}{3}$
\end{enumerate}
\item If the sum of the first ten terms of the series $(1\frac{3}{5})^2+(2\frac{2}{5})^2+(3\frac{1}{5})^2+4^2+(4\frac{4}{5})^2$+......, is $\frac{16}{5}$m then m is equal to :
\begin{enumerate}
\item 100
\item 99
\item 102
\item 101
\end{enumerate}
\item If, for a positive integer n, the quadratic equation, $x(x+1)+(x+1)(x+2)+.....+(x+\overline{n-1})(x+n)=10n$ has two consecutive integral solutions, then n is equal to :
\begin{enumerate}
\item 11
\item 12
\item 9
\item 10
\end{enumerate}
\item For any three positive real numbers a, b and c, 
\begin{align} 
9(25a^2+b^2)+25(c^2-3ac)=15b(3a+c)
\end{align} Then:
\begin{enumerate}
\item a, b and c are in G.P.
\item b, c and a are in G.P.
\item b, c and a are in A.P.
\item a, b and c are in A.P.
\end{enumerate}
\item Let a, b, c $\in$ R. If $f(x)=ax^2+bx+c$ is such that $a+b+c=3$   $f(x+y)=f(x)+f(y)+xy,\forall x, y \in R$, then $\sum_{n=1}^{10}f(n)$ is equal to :
\begin{enumerate}
\item 255
\item 330
\item 165
\item 190
\end{enumerate}
\item Let $a_1,a_2,a_3,......,a_{49}$ be in A.P. such that $\sum_{k=0}^{12}a_{4k+1}=416$ and $a_9+a_{43}=66$.If $a_1^2+a_2^2+.....+a_{17}^2=140m$, then m is equal to
\begin{enumerate}
\item 68
\item 34
\item 33
\item 66
\end{enumerate}
\item Let A be the sum of the first 20 terms and B be the sum of the first 40 terms of the series $1^2+2.2^2+3^2+2.4^2+5^2+2.6^2+.....$ If $B-2A=100\Lambda$, then $\Lambda$ is equal to :
\begin{enumerate}
\item 248
\item 464
\item 496
\item 232
\end{enumerate}
\item If a, b, and c be three distinct real numbers in G.P. and a+b+c=xb, then x cannot be :
\begin{enumerate}
\item -2
\item -3
\item 4
\item 2
\end{enumerate}
\item Let $a_1,a_2,a_3,......,a_{30}$ be in A.P., $S=\sum_{i=1}^{30}a_i$ and $T=\sum_{i=1}^{15}a_{2i-1}$. If $a_5=27$ and S-2T=75, Then $a_{10}$ is equal to :
\begin{enumerate}
\item 52
\item 57
\item 47
\item 42
\end{enumerate}
\item Three circles of radii a, b, c $(a < b < c)$ touch each other externally. If they have X-axis as a common tangent, then :
\begin{enumerate}
\item $\frac{1}{\sqrt{a}}=\frac{1}{\sqrt{b}}+\frac{1}{\sqrt{c}}$
\item $\frac{1}{\sqrt{b}}=\frac{1}{\sqrt{a}}+\frac{1}{\sqrt{c}}$
\item a, b, c are in A.P.
\item $\sqrt{a}, \sqrt{b} ,\sqrt{c}$ are in A.P.
\end{enumerate}
\item Let the sum of the first n terms of a non-constant A.P., $a_1,a_2,a_3,..........$ be 
$50n+\frac{n(n-7)}{2}A$, where A is a constant. If d is the common difference of this A.P.,then the ordered pair $(d,a_{10})$ is equal to:
\begin{enumerate}
\item (50, 50+46A)
\item (50, 50+45A)
\item (A, 50+45A)
\item (A, 50+46A)
\end{enumerate}
\end{enumerate} 
 
