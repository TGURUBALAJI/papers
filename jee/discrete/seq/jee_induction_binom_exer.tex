%\documentclass[journal,12pt,twocolumn]{IEEEtran}
%\usepackage{setspace}
%\usepackage{gensymb}
%\usepackage{caption}
%\usepackage{subfiles}
%%\usepackage{multirow}
%%\usepackage{multicolumn}
%%\usepackage{subcaption}
%%\doublespacing
%\singlespacing
%\usepackage{csvsimple}
%\usepackage{amsmath}
%\usepackage{multicol}
%%\usepackage{enumerate}
%\usepackage{amssymb}
%%\usepackage{graphicx}
%\usepackage{newfloat}
%%\usepackage{syntax}
%\usepackage{listings}
%%\usepackage{iithtlc}
%\usepackage{color}
%\usepackage{tikz}
%\usetikzlibrary{shapes,arrows}
%
%
%
%%\usepackage{graphicx}
%%\usepackage{amssymb}
%%\usepackage{relsize}
%%\usepackage[cmex10]{amsmath}
%%\usepackage{mathtools}
%%\usepackage{amsthm}
%%\interdisplaylinepenalty=2500
%%\savesymbol{iint}
%%\usepackage{txfonts}
%%\restoresymbol{TXF}{iint}
%%\usepackage{wasysym}
%\usepackage{amsthm}
%\usepackage{mathrsfs}
%\usepackage{txfonts}
%\usepackage{stfloats}
%\usepackage{cite}
%\usepackage{cases}
%\usepackage{mathtools}
%\usepackage{caption}
%\usepackage{enumerate}
%\usepackage{tfrupee}	
%\usepackage{enumitem}
%\usepackage{amsmath}
%%\usepackage{xtab}
%\usepackage{longtable}
%\usepackage{multirow}
%%\usepackage{algorithm}
%%\usepackage{algpseudocode}
%\usepackage{enumitem}
%\usepackage{mathtools}
%\usepackage{hyperref}
%%\usepackage[framemethod=tikz]{mdframed}
%\usepackage{listings}
%    %\usepackage[latin1]{inputenc}                                 %%
%    \usepackage{color}                                            %%
%    \usepackage{array}                                            %%
%    \usepackage{longtable}                                        %%
%    \usepackage{calc}                                             %%
%    \usepackage{multirow}                                         %%
%    \usepackage{hhline}                                           %%
%    \usepackage{ifthen}                                           %%
%  %optionally (for landscape tables embedded in another document): %%
%    \usepackage{lscape}     
%
%
%\usepackage{url}
%\def\UrlBreaks{\do\/\do-}
%
%
%%\usepackage{stmaryrd}
%
%
%%\usepackage{wasysym}
%%\newcounter{MYtempeqncnt}
%\DeclareMathOperator*{\Res}{Res}
%%\renewcommand{\baselinestretch}{2}
%\renewcommand\thesection{\arabic{section}}
%\renewcommand\thesubsection{\thesection.\arabic{subsection}}
%\renewcommand\thesubsubsection{\thesubsection.\arabic{subsubsection}}
%
%\renewcommand\thesectiondis{\arabic{section}}
%\renewcommand\thesubsectiondis{\thesectiondis.\arabic{subsection}}
%\renewcommand\thesubsubsectiondis{\thesubsectiondis.\arabic{subsubsection}}
%
%% correct bad hyphenation here
%\hyphenation{op-tical net-works semi-conduc-tor}
%
%%\lstset{
%%language=C,
%%frame=single, 
%%breaklines=true
%%}
%
%%\lstset{
%	%%basicstyle=\small\ttfamily\bfseries,
%	%%numberstyle=\small\ttfamily,
%	%language=Octave,
%	%backgroundcolor=\color{white},
%	%%frame=single,
%	%%keywordstyle=\bfseries,
%	%%breaklines=true,
%	%%showstringspaces=false,
%	%%xleftmargin=-10mm,
%	%%aboveskip=-1mm,
%	%%belowskip=0mm
%%}
%
%%\surroundwithmdframed[width=\columnwidth]{lstlisting}
%\def\inputGnumericTable{}                                 %%
%\lstset{
%%language=C,
%frame=single, 
%breaklines=true,
%columns=fullflexible
%}
% 
%
%\begin{document}
%%
%\tikzstyle{block} = [rectangle, draw,
%    text width=3em, text centered, minimum height=3em]
%\tikzstyle{sum} = [draw, circle, node distance=3cm]
%\tikzstyle{input} = [coordinate]
%\tikzstyle{output} = [coordinate]
%\tikzstyle{pinstyle} = [pin edge={to-,thin,black}]
%
%\theoremstyle{definition}
%\newtheorem{theorem}{Theorem}[section]
%\newtheorem{problem}{Problem}
%\newtheorem{proposition}{Proposition}[section]
%\newtheorem{lemma}{Lemma}[section]
%\newtheorem{corollary}[theorem]{Corollary}
%\newtheorem{example}{Example}[section]
%\newtheorem{definition}{Definition}[section]
%%\newtheorem{algorithm}{Algorithm}[section]
%%\newtheorem{cor}{Corollary}
%\newcommand{\BEQA}{\begin{eqnarray}}
%\newcommand{\EEQA}{\end{eqnarray}}
%\newcommand{\define}{\stackrel{\triangle}{=}}
%
%\bibliographystyle{IEEEtran}
%%\bibliographystyle{ieeetr}
%
%\providecommand{\nCr}[2]{\,^{#1}C_{#2}} % nCr
%\providecommand{\nPr}[2]{\,^{#1}P_{#2}} % nPr
%\providecommand{\mbf}{\mathbf}
%\providecommand{\pr}[1]{\ensuremath{\Pr\left(#1\right)}}
%\providecommand{\qfunc}[1]{\ensuremath{Q\left(#1\right)}}
%\providecommand{\sbrak}[1]{\ensuremath{{}\left[#1\right]}}
%\providecommand{\lsbrak}[1]{\ensuremath{{}\left[#1\right.}}
%\providecommand{\rsbrak}[1]{\ensuremath{{}\left.#1\right]}}
%\providecommand{\brak}[1]{\ensuremath{\left(#1\right)}}
%\providecommand{\lbrak}[1]{\ensuremath{\left(#1\right.}}
%\providecommand{\rbrak}[1]{\ensuremath{\left.#1\right)}}
%\providecommand{\cbrak}[1]{\ensuremath{\left\{#1\right\}}}
%\providecommand{\lcbrak}[1]{\ensuremath{\left\{#1\right.}}
%\providecommand{\rcbrak}[1]{\ensuremath{\left.#1\right\}}}
%\theoremstyle{remark}
%\newtheorem{rem}{Remark}
%\newcommand{\sgn}{\mathop{\mathrm{sgn}}}
%\providecommand{\abs}[1]{\left\vert#1\right\vert}
%\providecommand{\res}[1]{\Res\displaylimits_{#1}} 
%\providecommand{\norm}[1]{\left\Vert#1\right\Vert}
%\providecommand{\mtx}[1]{\mathbf{#1}}
%\providecommand{\mean}[1]{E\left[ #1 \right]}
%\providecommand{\fourier}{\overset{\mathcal{F}}{ \rightleftharpoons}}
%%\providecommand{\hilbert}{\overset{\mathcal{H}}{ \rightleftharpoons}}
%\providecommand{\system}{\overset{\mathcal{H}}{ \longleftrightarrow}}
%	%\newcommand{\solution}[2]{\textbf{Solution:}{#1}}
%\newcommand{\solution}{\noindent \textbf{Solution: }}
%\newcommand{\myvec}[1]{\ensuremath{\begin{pmatrix}#1\end{pmatrix}}}
%\providecommand{\dec}[2]{\ensuremath{\overset{#1}{\underset{#2}{\gtrless}}}}
%\DeclarePairedDelimiter{\ceil}{\lceil}{\rceil}
%%\numberwithin{equation}{section}
%%\numberwithin{problem}{subsection}
%%\numberwithin{definition}{subsection}
%\makeatletter
%\@addtoreset{figure}{section}
%\makeatother
%
%\let\StandardTheFigure\thefigure
%%\renewcommand{\thefigure}{\theproblem.\arabic{figure}}
%\renewcommand{\thefigure}{\thesection}
%
%
%%\numberwithin{figure}{subsection}
%
%%\numberwithin{equation}{subsection}
%%\numberwithin{equation}{section}
%%\numberwithin{equation}{problem}
%%\numberwithin{problem}{subsection}
%\numberwithin{problem}{section}
%%%\numberwithin{definition}{subsection}
%%\makeatletter
%%\@addtoreset{figure}{problem}
%%\makeatother
%\makeatletter
%\@addtoreset{table}{section}
%\makeatother
%
%\let\StandardTheFigure\thefigure
%\let\StandardTheTable\thetable
%\let\vec\mathbf
%%%\renewcommand{\thefigure}{\theproblem.\arabic{figure}}
%%\renewcommand{\thefigure}{\theproblem}
%
%%%\numberwithin{figure}{section}
%
%%%\numberwithin{figure}{subsection}
%
%
%
%\def\putbox#1#2#3{\makebox[0in][l]{\makebox[#1][l]{}\raisebox{\baselineskip}[0in][0in]{\raisebox{#2}[0in][0in]{#3}}}}
%     \def\rightbox#1{\makebox[0in][r]{#1}}
%     \def\centbox#1{\makebox[0in]{#1}}
%     \def\topbox#1{\raisebox{-\baselineskip}[0in][0in]{#1}}
%     \def\midbox#1{\raisebox{-0.5\baselineskip}[0in][0in]{#1}}
%
%\vspace{3cm}
%
%\title{ 
%%	\logo{
%Mathematical Induction and Binomial Theorem
%%	}
%}
%
%\author{ G V V Sharma$^{*}$% <-this % stops a space
%	\thanks{*The author is with the Department
%		of Electrical Engineering, Indian Institute of Technology, Hyderabad
%		502285 India e-mail:  gadepall@iith.ac.in. All content in this manual is released under GNU GPL.  Free and open source.}
%	
%}	
%
%\maketitle
%
%%\tableofcontents
%
%\bigskip
%
%\renewcommand{\thefigure}{\theenumi}
%\renewcommand{\thetable}{\theenumi}
%
%
%
%\begin{enumerate}[label=\arabic*]
%\numberwithin{equation}{enumi}
\renewcommand{\theequation}{\theenumi}
\begin{enumerate}[label=\arabic*.,ref=\thesubsection.\theenumi]
\numberwithin{equation}{enumi}

\item The large of $99^{50}+100^{50}$ and $101^{50}$ is........
\item The sum of the coefficients of the polynomial$(1+x-3x^2)^{2163}$ is....
\item If
\begin{align}
(1+ax)^n=1+8x+24x^2+.....
\end{align} then a=....and n=.....\\
\item let n be positive integer. If the coefficients of 2nd,3rd and 4th terms in the expansion of $(1+x)^n$ are in A.P.,then value of n is.....
\item the sum of the rational terms in the expansion of  $(\sqrt{2}+3^\frac{1}{5})^{10}$ is .......
\item Given Positive integers r$>1$,n$>1$ and that the coefficient of (3r)th and (r+2)th terms in the binomial expression of $(1+x)^{2n}$ are equal.Then
\begin{enumerate}
\item n=2r

\item n=3r

\item n=2r+1

\item none of these
\end{enumerate}

\item The Coefficient of $x^4$ in$ (\frac{x}{2}-\frac{3}{x^2})^{10} $ is
\begin{enumerate}
\item $\frac{405}{256}$

\item $\frac{504}{259}$

\item $\frac{450}{263}$

\item none of these
\end{enumerate}

\item The expression $(x+(x^3-1)^\frac{1}{2})^5+(x-(x^3-1)^\frac{1}{2})^5$ is a polynomial of degree 
\begin{enumerate}
\item 5

\item 6

\item 7

\item 8
\end{enumerate}
\item If in the expression of $(1+x)^m (1-x)^n$,the coefficient of x and $x^2$ are 3 and -6 respectively,then m is
\begin{enumerate}
\item 6

\item 9

\item 12

\item 24
\end{enumerate}
\item For $2 \leq r \leq n$,
$\myvec{n\\r}+2\myvec{n\\r-1}+\myvec{n\\r-2}=$
\begin{enumerate}
 \item $\myvec{n+1\\r-1}$

 \item $2\myvec{n+1\\r+1}$

 \item $2\myvec{n+2\\r}$

 \item $\myvec{n+2\\r}$ 
 \end{enumerate}
\item In the binomial expression of $(a-b)^n$,n$\geq$5,the sum of the 5th and 6th terms is Zero.Then $\frac{a}{b}$ equals
\begin{enumerate}
\item $\frac{n-5}{6}$

\item $\frac{n-4}{5}$

\item $\frac{5}{n-4}$

\item $\frac{6}{n-5}$
\end{enumerate}
\item The sum $$\sum_{i=0}^{m} \myvec{10\\i} \myvec{20\\m-i}, $$ (where \myvec{p\\q}=0 if p<q) is maximum when m is
\begin{enumerate}
\item 5

\item 10

\item 15

\item 20
\end{enumerate}
\item Coefficient of $t^{24}$ in $(1+t^2)^{12}(1+t^{12})(1+t^{24})$ is
\begin{enumerate}
\item $^{12}C_6+3$

\item $^{12}C_6+1$

\item $^{12}C_6$

\item $^{12}C_6+2$
\end{enumerate}
\item If $^{n-1}C_r=(k^2-3) ^{n}C_{r+1}$,Then k$\in$
\begin{enumerate}
\item $(-\infty,-2]$
\item $(2,-\infty,-2]$
\item $[-\sqrt{3},\sqrt{3}]$
\item $(\sqrt{3},-2]$
\end{enumerate}
\item The value of $\myvec{30\\0}\myvec{30\\10}-\myvec{30\\1}\myvec{30\\11}+\myvec{30\\2}\myvec{30\\12}....+\myvec{30\\20}\myvec{30\\30}$ is where $\myvec{n\\r}=^nC_r$
\begin{enumerate}
\item $\myvec{30\\10}$

\item $\myvec{30\\15}$

\item $\myvec{60\\30}$

\item $\myvec{31\\10}$
\end{enumerate}
 \item For r=0,1,....10,let $A_r,B_r and C_r$ denote respectively,the coefficient of $x^r$ in the expansions of $(1+x)^{10},(1+x)^{20} and (1+x)^{30}$.Then $$\sum_{r=1}^{10} A_r(B_{10}B_r-C_{10}A_r)$$ is equal to
 \begin{enumerate}
 \item $(B_{10}-C_{10})$
 \item $A_{10}(B^2_{10} C_{10} A_{10})$
 \item 0
 \item $(C_{10}-B_{10})$
 \end{enumerate}
 \item Coefficient of $x^{11}$ in the expansion of $(1+x^2)^4$ $(1+x^3)^7$ $(1+x^4)^{12}$ is
 \begin{enumerate}
     \item 1051
     \item 1106
     \item 1113
     \item 1120
 \end{enumerate}
\item If $C_r$ stands for $^{n}C_r$,then the sum of the series $\frac{2(\frac{n}{2})!(\frac{n}{2})!}{n!}[C^2_0-2C^2_1+3C^2_2-.....+(-1)^n(n+1)C^2_n]$,where n is an even positive integer,is equal to 
\begin{enumerate}
    \item 0
    \item $(-1)^\frac{n}{2}(n+1)$
    \item $(-1)^\frac{n}{2}(n+2)$
    \item $(-1)^n n$
    \item none of these
\end{enumerate}
\item if $a_n=\sum_{r=0}^{n} \frac{1}{^{n}C_r}$,then $\sum_{r=0}^{n} \frac{r}{^{n}C_r}$ equals
\begin{enumerate}
    \item $(n-1)a_n$
    \item $na_n$
    \item $\frac{1}{2}na_n$
    \item none of these 
\end{enumerate}
\item Given that 
\begin{align}
C_1+2C_2x+3C_3x^2+......+2nC_{2n}x^{2n-1}=2n(1+x)^{2n-1}
\end{align} where $C_r=\frac{(2n!)}{r!(2n-r)!}$  r=0,1,2,....,2n Prove that $C^2_1-2C^2_2+3C^2_3-......-2nC^2_2n=(-1)^{n}nC_n$.
\item Prove that $7^{2n}+(2^{3n-3})(3^{n-1})$ is divisible by 25 for any natural numbers n
\item If 
\begin{align}
(1+x)^n=C_0+C_1x+C_2x^2+......+C_nx^n
\end{align} then show that the sum of the products of the $C_i$'s taken at a time,represented by $\sum_{0\leq i<j \leq n}\sum C_i C_j$ is equal to $2^{2n-1}-\frac{(2n)!}{2(n!)^2}$
\item Use mathematical induction to prove : If n is any odd positive integer,then $n(n^2-1)$ is divisible by 24.
\item If p be a natural number then prove that $p^{n+1}+(p+1)^{2n-1}$ is divisible by $p^2+p+1$ for every position integer n.
\item Given $s_n=1+q+q^2+....+q^n$;
$s_n=1+\frac{q+1}{2}+(\frac{q+1}{2})^2+......+(\frac{q+1}{2})^n$, $q\neq 1$
prove that $^{n+1}C_1+^{n+1}C_2+ ^{n+1}C_3+.....+^{n+1}C_n s_n=2^n S_n$
\item Use method of mathematical induction that $2.7^n+3.5^n-5$ is divisible by 24 for all n$>$0
\item Prove by mathematical induction that $\frac{(2n)!}{2^{2n}(n!)^2}$ $\leq$ $\frac{1}{(3n+1)^\frac{1}{2}}$ for all positive integers n.
\item Let R=$(5\sqrt{5}+11)^{2n+1}$ and f=R-[R],where [] denotes the greater integer function.Prove that $Rf=4^{2n+4}$
\item Using mathematical induction prove that $^{m} C_0  ^{n} C_k+ ^{m} C_1  ^{n} C_{k-1}+.....+ ^{m} C_k  ^{n}C_0= ^{m+n} C_k$,where m,n,k are positive integers,and $^{p}C_q=0$ for p$<$q.
\item Prove that $C_0-2^2C_1+3^2C_2-......+(-1)^n (n+1)^2 C_n=0$,n$>$2,where $C_r=^{n}C_r$.
\item Prove that $\frac{n^7}{7}+\frac{n^5}{5}+\frac{2n^3}{3}-\frac{n}{105}$ is a integer for every positive integer n.
\item using induction or otherwise ,Prove that for any non-negative integers m,n,r and k,
$\sum_{m=0}^{k} (n-m)\frac{(r+m)!}{m!}=\frac{(r+k+1)!}{k!}[\frac{n}{r+1}-\frac{k}{r+2}]$
\item If 
\begin{align}
\sum_{r=0}^{2n}a_r(x-2)^r=\sum_{r=0}^{2n}b_r(x-3)^r
\end{align} and $a_k=1$ for all k $\geq$ n,then show that $b_n= ^{2n+1}C_{n+1}$
\item Let p$\geq$ 3 be an Integer and $\alpha,\beta$ be the roots of 
\begin{align}x^2-(p+1)x+1=0
\end{align} Using mathematical induction show that $\alpha^n+\beta^n$
(i) is an integer and (ii) is not divisible by p
\item Using mathematical induction Prove that \\
$\tan^{-1}\frac{1}{3}+\tan^{-1}\frac{1}{7}+....\tan^{-1}{1/(n^2+n+1)}=\tan^{-1}{n/(n+2)}$
\item Prove that $\sum_{r=1}^{k}(-3)^{r-1}\enspace {^{3n}C_{2r-1}}=0$, where $k=\frac{3n}{2}$ and n is an even positive integer.
\item If x is not an integral multiple of 2$\pi$ use mathematical induction to prove that:
$\cos x+\cos 2x+....+\cos nx=\cos \frac{n+1}{2}x\sin \frac{nx}{2}$ cosec$\frac{x}{2}$
\item let n be positive integer and \begin{align} (1+x+x^2)=a_0+a_1x+.....+a_{2n}x^{2n}\end{align} show that $a^2_0-a^1_2+a^2_2.....+a^2_{2n}=a_n$
\item Using mathematical induction prove that for every integer n$\geq$ 1,$(3^{2n}-1)$ is divisible by $2^{n+2}$ but not by $2^{n+3}$
\item Let $0<A_i<\pi$ for i=1,2,....,n.Use mathematical induction to prove that $\sin A_1+\sin A_2...+\sin A_n \leq n \sin(\frac{A_1+A_2+...+A_n}{n})$ where $\geq$ 1 is natural number.\\
(You may use the fact that $p\sin x+(1-p) \sin y\leq \sin [px+(1-p)y]$,where 0$\leq $p$\leq$1 and 0$\leq $x,y$\leq\pi$)
\item Let p be prime and m a positive integer.By mathematical induction on m or otherwise prove that whenever r is an integer such that p does not divide r,p divides $^{mp}C_r$,
[Hint: you may use the fact that $(1+x)^{(m+1)p}=(1+x)^p (1+x)^{mp}$]
\item Let n be any positive integer.prove that $\sum_{k=0}^{m} \frac{\myvec{2n-k\\k}}{\myvec{2n-k\\n}}.\frac{(2n-4k+1)}{(2n-2k+1)}2^{n-2k}=\frac{\myvec{n\\m}}{\myvec{2n-2m\\n-m}}2^{n-2m}$ for each non-be gatuve integer m $\leq $ n.\\(Here $\myvec{p\\q}=^{p}C_q$)
\item For any Positive integer m, n(with n $\geq$ m),let $\myvec{n\\m}= ^{n}C_m$. Prove that $\myvec{n\\m}+\myvec{n-1\\m}+\myvec{n-2\\m}+....+\myvec{m\\m}=\myvec{n+1\\m+2}$ Hence or otherwise, prove that$\myvec{n\\m}+2\myvec{n-1\\m}+3\myvec{n-2\\m}+....+(n-m+1)\myvec{m\\m}=\myvec{n+2\\m+2}$
\item for every positive integer n,Prove that\\
 $\sqrt{(4n+1)}<\sqrt{n}+\sqrt{n+1}<\sqrt{4n+2}.$ Hence or otherwise, prove that\\
  $[\sqrt{n}+\sqrt{(n+1)}]=[\sqrt{4n+1}]$ where [x] denotes the greater integers not exceeding x.
\item Let a,b,c be the positive real numbers such that $b^2-4ac>0$ and let $\alpha_1$=c prove by induction that$\alpha{n+1} = \frac{a\alpha^2_n}{(b^2-2a(\alpha_1+\alpha_2+...+\alpha_n))}$is well defined and 
$\alpha{n+1} < \frac{\alpha_n}{2}$ for all n = 1,2,.....(Here, 'well-defined' means that the denominator in the expresion for $\alpha_{n+1}$ is not zero.) 

\item Use the mathematical induction to show that $(25)^{n+1}-24n+5735$ is divisible by $(24)^2$ for all n=1,2,......
\item prove that $2^k\myvec{n\\0}\myvec{n\\k}-2^{k-1\myvec{n\\2}}\myvec{n\\1}\myvec{n-1\\k-1}+2^{k-2}\myvec{n-2\\k-2}-....(-1)^k \myvec{n\\k}\myvec{n-k\\0}=\myvec{n\\k}$
\item A coin has probability p of showing head when tossed.It is tossed n times.Let $p_n$ denote the probability that no two (or more) consecutive heads occur Prove that $p_1=1,p_2=1-p^2$ and $p_n=(1-p).p_(n-1)+p(1-p)p_{n-2}$ for all n $\geq$ 3.\\
Prove by induction on n, that $p_n=A\alpha^n+B\beta^n$ for all n $\geq$ 1,\\
where $\alpha$ and $\beta$ are the roots of quadratic equation 
\begin{align}
x^2-(1-p)x-p(1-p)=0
\end{align} and $A=\frac{p^2+\beta-1}{\alpha\beta-\alpha^2},B=\frac{p^2+\alpha-1}{\alpha\beta-\beta^2}.$

\item The coefficient of three consecutive terms of $(1+x)^{n+5}$ are in the ratio 5:10:14.Then n=
\item Let m be the smallest positive integer such that the Coefficient of $x^2$ in the expansion of \\$(1+x)^2+(1+x)^3+.....+(1+x)^{49}+(1+mx)^{50}$ is $(3n+1)^{51}C_3$ for some positive integers n.Then value of n is
\item Let X=$(^{10}C_1)^2+2(^{10}C_2)^2+3(^{10}C_3)^2+.....+(^{10}C_{10})^2$ where $^{10}C_r$,r$\in$ {1,2,3,.....,10} denote binomial coefficient then ,the value of $\frac{1}{1430}$X is.......
\item suppose det$\myvec{\sum_{k=0}^{n}k&\sum_{k=0}^{n} {n}C_k k^2\\\sum_{k=0}^{n} {n}C_k k&\sum_{k=0}^{n} {n}C_k 3^k}=0$ holds for some positive integer n.The $\sum_{k=0}^{n} \frac{^{n}C_k}{k+1}$ equals
\item the coefficient of $x^p$ and $x^q$ in the expansion of $(1+x)^{p+q}$ are
\begin{enumerate}
\item equal
\item equal with opposite signs
\item reciprocals of each other
\item none of these
\end {enumerate}
\item If sum of the coefficients in the expansion of $(a+b)^n$ is 4096, then the greatest coefficient in the expansion is 
\begin{enumerate}
\item 1594
\item 792
\item 924
\item 2924
\end{enumerate}
\item the positive integer just greater than \\$(1+0.0001)^{10000} $is 
\begin{enumerate}
\item 4
\item 5
\item 2
\item 3
\end{enumerate}
\item r and n are the positive integers r$>$1,n$>$2 and coefficient of $(r+2)^{th}$ term and $3r^{th}$ tern in the expansion of $(1+x)^{2n}$ are equal,then n equals
\begin{enumerate}
\item 3r
\item 3r+1
\item 2r
\item 2r+1
\end{enumerate}
\item If $\sqrt{7+\sqrt{7+\sqrt{7+.....}}}$ having n radical sign then by methods of mathematical induction which is true
\begin{enumerate}
    \item $a_n>7 \forall n\geq1$
    \item $a_n<7 \forall n\geq1$
    \item $a_n<4 \forall n\geq1$
    \item $a_n<3 \forall n\geq1$
\end{enumerate}
\item If x is positive,the first negative term in the expansion of $(1+x)^\frac{27}{5}$ is 
\begin{enumerate}
\item 6th term
\item 7th term 
\item 5th term 
\item 8th term
\end{enumerate}
\item The number of integral terms in the expansion of $(\sqrt{3}+\sqrt[8]{5})^{256}$ is 
\begin{enumerate}
    \item 35
    \item 32
    \item 33
    \item 34
\end{enumerate}
\item Let S(K)=1+3+5+....+(2K-1)=3+$K^2$ Then which of the following is true
\begin{enumerate}
    \item principle of mathematical induction can be used to prove the formula
    \item S(K)$\rightarrow$S(K+1)
    \item S(K)$\rightarrow$S(K+1)
    \item S(1) is correct
\end{enumerate}
\item The coefficient of middle term in the binomial expansion in powers of x of$(1+\alpha x)^4$ and of $(1-\alpha x)^6$ is the same if $\alpha$ equals to
\begin{enumerate}
    \item $\frac{3}{5}$
    \item $\frac{10}{3}$
    \item $\frac{-3}{10}$
    \item $\frac{-5}{3}$
\end{enumerate}
\item The coefficient of $x^n$ in the expansion of (1+x) $(1-x)^n$ is
\begin{enumerate}
    \item $(-1)^{n-1} n$
    \item $(-1)^n(1-n)$
    \item $(-1)^{n-1}(n-1)^2$
    \item n-1
\end{enumerate}
\item The value of $^{50}C_4+\sum_{r=1}^6$ $^{56-r}C_3$ is
\begin{enumerate}
    \item $^{55}C_4$
    \item $^{55}C_3$
    \item $^{56}C_3$
    \item $^{56}C_4$
\end{enumerate}
\item If A=\myvec{1&0\\1&1} and I=\myvec{1&0\\0&1}, then which one of the following holds for all n $\geq$  1, by the principle of mathematical induction
\begin{enumerate}
    \item $A^n=nA-(n-1)I$
    \item $A^n=2^{n-1}A-(n-1)I$
    \item $A^n=nA+(n-1)I$
    \item $A^n=2^{n-1}A+(n-1)I$
\end{enumerate}
\item If the coefficient of $x^7$ in $[ax^2+\frac{1}{bx}]^{11}$ equals the coefficient of $x^{-7}$ in $[ax-\frac{1}{bx^2}]^{11}$, then a and b satisfy the relation
\begin{enumerate}
    \item a-b=1
    \item a+b=1
    \item $\frac{a}{b}=1$
    \item ab=1
\end{enumerate}
\item If x is so small that $x^3$ and higher powers of x may be neglected, then $\frac{(1+x)^\frac{3}{2}-(1+\frac{1}{2}x)^3}{(1-x)^\frac{1}{2}}$ may be approximated as 
\begin{enumerate}
    \item $1-\frac{3}{8}x^2$
    \item $3x+\frac{3}{8}x^2$
    \item $-\frac{3}{8}x^2$
    \item $\frac{x}{2}-\frac{3}{8}x^2$
\end{enumerate}
\item If expansion in power of x of the function $\frac{1}{(1-ax)(1-bx)}$ is $a_0+a_1x+a_2 x^2+.....$ then $a_n$ is
\begin{enumerate}
    \item $\frac{b^n-a^n}{b-a}$
    \item $\frac{a^n-b^n}{b-a}$
    \item $\frac{a^{n+1}-b^{n+1}}{b-a}$
    \item $\frac{b^{n+1}-a^{n+1}}{b-a}$
\end{enumerate}
\item For natural numbers m,n if 
\begin{align} 
(1-y)^m (1+y)^n=1+a_1y+a_2y^2+.......
\end{align} and $a_1=a_2=10$, then $\myvec{m\\n}$ is
\begin{enumerate}
\item \myvec{20\\45}
\item \myvec{35\\20}
\item \myvec{45\\35}
\item \myvec{35\\45}
\end{enumerate}
\item In the binomial expansion of $(a-b)^n$, $n \geq 5$, the sum of 5th and 6th terms is zero then a/b equals
\begin{enumerate}
    \item $\frac{n-5}{6}$
    \item $\frac{n-4}{5}$
    \item $\frac{5}{n-4}$
    \item $\frac{6}{n-5}$
\end{enumerate}
\item The sum of the series\\
$^{20}C_0-^{20}C_1+^{20}C_2-^{20}C_3+...-.....+^{20}C_{10}$ is
\begin{enumerate}
    \item 0
    \item $^{20}C_{10}$
    \item $-^{20}C_{10}$
    \item $\frac{1}{2}^{20}C_{10}$
\end{enumerate}
\item \textbf{statement-1}:\\$\sum_{r=0}^{n}(r+1) ^nC_r=(n+2)2^{n-1}$\\
\textbf{statement-2}:\\$\sum_{r=0}^{n}(r+1) ^nC_r x^r=(1+x)^n+nx(1+x)^{n-1}$
\begin{enumerate}
    \item statement-1 is true,statement-2 is true
    \item statement-1 is true,statement-2 is true;statement-2 is correct explanation for statement-1
    \item statement-1 is true,statement-2 is true;statement-2 is not correct explanation for statement-1
    \item statement-1 is true,statement-2 is false
\end{enumerate}
\item The reminder left out when $8^{2n}-(62)^{2n+1}$ is divided by 9 is:
\begin{enumerate}
    \item 2
    \item 7
    \item 8
    \item 0
\end{enumerate}
\item $S_1=\sum_{j=1}^{10} j(j-1) ^{10}C_J,\\
S_2=\sum_{j=1} ^{10}$ $^{10}C_j$ and \\
$S_3=\sum_{j=1}^{10} j^2$ $^{10}C_j$\\
\textbf{statement-1}:$S_3=55\times2^9$\\
\textbf{statement-2}:$S_1=90\times2^8$ and $S_2=10\times2^8$
\begin{enumerate}
    \item statement-1 is true,statement-2 is true;statement-2 is not correct explanation for statement-1
    \item statement-1 is true,statement-2 is false
    \item statement-1 is false,statement-2 is true
    \item statement-1 is true,statement-2 is true;statement-2 is  correct explanation for statement-1
\end{enumerate}
\item The Coefficient of $x^7$ in the expansion of\\
 $(1-x-x^2+x^3)^6$ is
\begin{enumerate}
    \item -132
    \item -144
    \item 132
    \item 144
\end{enumerate}
\item If n is a positive integer, then\\
$(\sqrt{3}+1)^{2n}-(\sqrt{3}-1)^{2n}$ is:
\begin{enumerate}
    \item an irrational number 
    \item an odd positive integer
    \item an even positive integer
    \item a rational number other than positive integer
\end{enumerate}
\item The term independent of x in the expansion $(\frac{x+1}{x^\frac{2}{3}-x^\frac{1}{3}+1}-\frac{x-1}{x-x^\frac{1}{2}})^{10}$ is
\begin{enumerate}
    \item 4
    \item 120
    \item 210
    \item 310
\end{enumerate}
\item If the coefficient of $x^3$ and $x^4$ in the expansion of $(1+ax+bx^2)(1-2x)^{18}$ in powers of x both zero,then $\myvec{a\\b}$ is equal to:
\begin{enumerate}
    \item $\myvec{14\\\frac{272}{3}}$
    \item $\myvec{16\\\frac{272}{3}}$
    \item $\myvec{16\\\frac{251}{3}}$
    \item $\myvec{14\\\frac{251}{3}}$
\end{enumerate}
\item The sum of the coefficients of integral power of x in the binomial expansion \\
$(1-2\sqrt{x})^{50}$ is:
\begin{enumerate}
    \item $\frac{1}{2} (3^{50}-1)$
    \item $\frac{1}{2} (2^{50}+1)$
    \item $\frac{1}{2} (3^{50}+1)$
    \item $\frac{1}{2} (3^{50})$
\end{enumerate}
\item The number of terms in the expansion of \\$(1-\frac{2}{x}+\frac{4}{x^2})^n$, $x \neq 0$, is 28, Then the sum of the all the terms in the expansion is:
\begin{enumerate}
    \item 243
    \item 729
    \item 64
    \item 2187
\end{enumerate}
\item The value of $(^{21}C_1- ^{10}C_1)+(^{21}C_2- ^{10}C_2)+(^{21}C_3- ^{10}C_3)+(^{21}C_4- ^{10}C_4)+...+(^{21}C_{10}- ^{10}C_{10})$ is
\begin{enumerate}
    \item $2^{20}-2^{10}$
    \item $2^{21}-2^{11}$
    \item $2^{21}-2^{10}$
    \item $2^{20}-2^{9}$
\end{enumerate}
\item The sum of the all coefficients of all odd degree terms in the expansion of \\$(x+\sqrt{x^3-1})^5+(x-\sqrt{x^3-1})^5$, (x$>$1) is:
\begin{enumerate}
    \item 0
    \item 1
    \item 2
    \item -1
\end{enumerate}
\item If the forth term in the Binomial expansion of $(\frac{2}{x}+x^{log 8x})^6$ (x$>$0) is 20$\times$8$^7,$ then a value of x is:
\begin{enumerate}
    \item $8^3$
    \item $8^2$
    \item $8$
    \item $8^{-2}$
\end{enumerate}
\end{enumerate}
%\end{document}
    
