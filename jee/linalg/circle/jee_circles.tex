\renewcommand{\theequation}{\theenumi}
\begin{enumerate}[label=\arabic*.,ref=\thesubsection.\theenumi]
\numberwithin{equation}{enumi}
\item If A and B are points in the plane such that $\frac{PA}{PB} = k\text{(constant)}$ for all $\vec{P}$ on a given circle, then the value of k cannot be equal to?
 
\item Find the points of intersection of the line 
\begin{align} 
\myvec{4 & -3}\vec{x} - 10 = 0
\end{align} 
and the circle 
\begin{align}
\vec {x}^T\vec{x} + \myvec{-2 & 4}\vec{x} - 20 = 0.
\end{align}
are...................and................

\item The lines 
\begin{align}
\myvec{3 & -4}\vec{x} + 4 = 0 \\
\myvec{6 & -8}\vec{x} - 7 = 0
\end{align} 
are tangents to the same circle. The radius of the circle is.............

\item Let 
\begin{align}
\vec{x}^T \vec{x} + \myvec{-4 & -2}\vec{x} - 11 = 0
\end{align} 
be circle. A pair of tangents $\myvec{4 & 5}$ with a pair radii from quadrilateral of area..............

\item From the origin chords are drawn to the circle 
\begin{align}
\vec{x}^T\vec{x} + \myvec{-2 & 0}\vec{x} = 0
\end{align} 
The equation of the locus of the mid-points of these chords is............

\item The equation of the line passing through the points of intersection of the circles 
\begin{align}
3(\vec{x}^T\vec{x}) + \myvec{-2 & 12}\vec{x} - 9 = 0 \\
\vec{x}^T\vec{x} + \myvec{6 & 2}\vec{x} - 15 = 0
\end{align} 
is................

\item From the point A$\myvec{0 & 3}$ on the circle 
\begin{align}
\vec{x}^T\vec{x} + \myvec{4 & 6}\vec{x} + 9 = 0
\end{align} 
a chord AB is drawn and extended to a point $\vec{M}$ such that AM = 2AB. The equation of the locus of 
$\vec{M}$ is..............

\item The area of the triangle is formed by tangents from the point $\myvec{4 & 3}$ to the circle 
\begin{align}
\vec{x}^T\vec{x} = 9
\end{align} 
and the line joining their points of contact is

\item If the circle 
\begin{align}
C_1:\vec{x}^T\vec{x} = 16
\end{align} 
intersects another circle $C_2$ of radius 5 in such a manner that common chord is of maximum length and has a slope equal to $\frac{3}{4},$ then the coordinates of the centre $C_2$ are...........

\item The area of the triangle formed by the positive x-axis and the normal and the tangent to the circle 
\begin{align}
\vec{x}^T\vec{x} = 4
\end{align} 
at (1, $\sqrt{3}$)is?

\item If a circle passes through the points of intersection of the coordinate axes with the lines 
\begin{align}
\myvec{\lambda & -1}\vec{x} + 1 = 0\\
\myvec{1 & -2}\vec{x} + 3 = 0
\end{align} 
then the value of $\lambda$ = .............. 

\item The equation of the locus of the midpoints of the circle 
\begin{align}
4(\vec{x}^T\vec{x}) + \myvec{-12 & 4}\vec{x} + 1 = 0
\end{align} 
that subtend of a angle of $\frac{2\pi}{3}$  at its centre is..............

\item The intercept on the line
\begin{align} 
\myvec{1 & -1}\vec{x} = 0
\end{align}
by the circle
\begin{align}
C1:\vec{x}^T\vec{x} + \myvec{-2 & 0}\vec{x}=0
\end{align} 
is AB. Equation of the circle with AB as a diameter is...............

\item For each natural number k, let $C_k$ denote the circle with radius k centimetres and center at origin. On the circle $C_k$, $\alpha$-particle moves k centimetres in the counter clockwise direction. After completing its motion on $C_k$ the particle moves to $C_{k+1}$ in the radial direction. The motion of the particle is continues in the manner. The particle starts at $\myvec{1 & 0}$. If the particle crosses the positive direction of the x-axis for the first time on the circle $C_n$ then n=................

\item The chords of contact of the pair of tangents drawn from each point on the line 
\begin{align}
\myvec{2 & 1}\vec{x} = 4
\end{align} 
to circle 
\begin{align}
\vec{x}^T\vec{x} = 1
\end{align} 
pass through the point?

\item  A square is inscribed in the circle  
\begin{align}
\vec{x}^T\vec{x} + \myvec{-2 & 4}\vec{x} + 3 = 0.
\end{align} 
Its sides are parallel to the coordinate axes. The one vertex of the square is
\begin{enumerate}
\item $(1 + \sqrt{2}, -2)$
\item $(1 - \sqrt{2}, -2)$
\item $(1, -2+\sqrt{2})$
\item none of these
\end{enumerate}

\item Two circles 
\begin{align}
\vec{x}^T\vec{x} = 6 and \\
\vec{x}^T\vec{x} + \myvec{-6 & 8}\vec{x} + 3 = 0
\end{align} 
are given. Then the equation of the circle through their points of intersection and the point $\myvec{1 & 1}$ is
\begin{enumerate}
\item $\vec{x}^T\vec{x} + \myvec{-6 & 0}\vec{x} + 4 = 0$
\item $\vec{x}^T\vec{x} + \myvec{-3 & 0}\vec{x} + 1 = 0$
\item $\vec{x}^T\vec{x} + \myvec{0 & -4}\vec{x} + 2 = 0$
\item none of these
\end{enumerate}

\item The centre of the circle passing through the point $\myvec{0 & 1}$ and touching the curve 
\begin{align}
y = \vec{x}^T\myvec{1 & 0 \\ 0 & 0}\vec{x}
\end{align} 
at $\myvec{2 & 4}$ is
\begin{enumerate}
\item $(\frac{-16}{5}, \frac{27}{10})$
\item $(\frac{-16}{7}, \frac{53}{10})$
\item $(\frac{-16}{5}, \frac{53}{10})$
\item none of these
\end{enumerate}
    
\item The equation of the circle passing through the point $\myvec{1 & 1}$ and the points of intersection of 
\begin{align} 
\vec{x}^T\vec{x} + \myvec{13 & -3}\vec{x} = 0 \\
2(\vec{x}^T\vec{x}) + \myvec{4 & -7}\vec{x} - 25 = 0
\end{align} 
is 
\begin{enumerate}
\item $4(\vec x^T\vec x) + \myvec{-30 & -10}\vec{x} - 25 = 0$
\item $4(\vec x^T\vec x) + \myvec{30 & -13}\vec{x} - 25 = 0$
\item $4(\vec x^T\vec x) + \myvec{-17 & -10}\vec{x} + 25 = 0$
\item none of these
\end{enumerate}

\item The locus of the mid point of a chord of the circle
\begin{align}
\vec{x}^T\vec{x} = 4
\end{align} 
which subtends a right angle at the origin is
\begin{enumerate}
\item $\myvec{1 & 1}\vec{x} = 2$
\item $\vec{x}^T\vec{x} = 1$
\item $\vec{x}^T\vec{x} = 2$
\item $\myvec{1 & 1}\vec{x} = 1$
\end{enumerate}

\item If a circle passes through the point $\myvec{a & b}$ and cuts the circle 
\begin{align}
\vec{x}^T\vec{x} = k^{2}
\end{align} 
orthogonally, then the equation of the locus of its centre is  
\begin{enumerate}
\item $\myvec{2a & 2b}\vec{x} - (a^2 + b^2 + k^2) = 0$
\item $\myvec{2a & 2b}\vec{x} - (a^2 - b^2 + k^2) = 0$
\item $\vec{x}^T\vec{x} + \myvec{-3a, -4b}\vec{x} + (a^2 + b^2 - k^2) = 0$
\item $\vec{x}^T\vec{x} + \myvec{-3a, -4b}\vec{x} + (a^2 - b^2 - k^2) = 0$
\end{enumerate}

\item If the two circles 
\begin{align}
\vec{x}^T\vec{x} + \myvec{-2 &-2}\vec{x} + 2 = r^{2} \\
\vec{x}^T\vec{x} + \myvec{-8 & 2}\vec{x} + 8 = 0
\end{align} 
intersect in two distinct points, then
\begin{enumerate}
\item $2 < r < 8$
\item $r < 2$
\item r = 2
\item $r > 2$
\end{enumerate}

\item The lines 
\begin{align}
\myvec{2 & -3}\vec{x} = 5 \\
\myvec{3 & -4}\vec{x} = 7
\end{align}
are diameters of a circle of area 154 sq.units. Then the equation of this circle is
\begin{enumerate}
\item $ \vec{x}^T\vec{x} + \myvec{2 & -2}\vec{x} = 62$
\item $ \vec{x}^T\vec{x} + \myvec{2 & -2}\vec{x} = 47$
\item $ \vec{x}^T\vec{x} + \myvec{-2 & 2}\vec{x} = 47$
\item $ \vec{x}^T\vec{x} + \myvec{-2 & 2}\vec{x} = 62$
\end{enumerate}

\item Find the centre of a circle passing through the points $\myvec{0 & 0}$, $\myvec{1 & 0}$ and touching the circle
\begin{align}
\vec{x}^T\vec{x} = 9.
\end{align}
\begin{enumerate}
\item $(\frac{3}{2}, \frac{1}{2})$
\item $(\frac{1}{2}, \frac{3}{2})$
\item $(\frac{1}{2}, \frac{1}{2})$
\item $(\frac{1}{2}, (-2)^{\frac{1}{2}})$
\end{enumerate}
     
\item The locus of the centre of a circle, which touches externally the circle 
\begin{align}
\vec{x}^T\vec{x} + \myvec{-6 & -6}\vec{x} + 14 = 0
\end{align}
and also touches the y-axis, is given by the equation:
\begin{enumerate}
\item $\vec{x}^T \myvec{1 & 0 \\ 0 & 0}\vec{x} + \myvec{-6 & 10} \vec{x} + 14 = 0$
\item $\vec{x}^T \myvec{1 & 0 \\ 0 & 0}\vec{x} + \myvec{-10 & -6}\vec{x} + 14 = 0$
\item $\vec{x}^T \myvec{1 & 0 \\ 0 & 0}\vec{x} + \myvec{-6 & -10}\vec{x} + 14 = 0$
\item $\vec{x}^T \myvec{1 & 0 \\ 0 & 0}\vec{x} + \myvec{-10 & -6} \vec{x} + 14 = 0$
\end{enumerate}

\item The circles 
\begin{align}
\vec{x}^T\vec{x} + \myvec{-10 & 0}\vec{x} + 16 = 0 \\
\vec{x}^T\vec{x} = r^{2}
\end{align}
intersect each other in two distinct points if
\begin{enumerate}
\item $r < 2$
\item $r > 8$
\item $2 < r < 8$
\item $2 \leq r \leq  8$
\end{enumerate}
 
\item The angle between a pair of tangents drawn  from a point $\vec{P}$  to the circle 
\begin{align}
\vec{x}^T\vec{x} + \myvec{4 & -6}\vec{x} + 9\sin{^{2}(\alpha)} + 13\cos{^2(\alpha)} = 0
\end{align} 
is $2\pi$. The equation of the locus of the point $\vec{P}$ is 
\begin{enumerate}
\item $\vec{x}^T\vec{x} + \myvec{4 & -6}\vec{x} + 4 = 0$
\item $\vec{x}^T\vec{x} + \myvec{4 & -6}\vec{x} - 9 = 0$
\item $\vec{x}^T\vec{x} + \myvec{4 & -6}\vec{x} - 4 = 0$
\item $\vec{x}^T\vec{x} + \myvec{4 & -6}\vec{x} + 9 = 0$  
\end{enumerate}
    
\item If two distinct chords drawn from the point (p, q) on the circle 
\begin{align}
\vec{x}^T\vec{x} = \myvec{p & q}\vec{x}
\end{align}
(where pq $\neq$ 0) are bisected by the x-axis, then 
\begin{enumerate}
\item $p^2 = q^2$
\item $p^2 = 8q^2$
\item $p^2 < 8q^2$
\item $p^2 > 8q^2$   
\end{enumerate}
    
\item The triangle PQR is inscribed in the circle 
\begin{align}
\vec{x}^T\vec{x} = 25.
\end{align} 
If $\vec{Q}$ and $\vec{R}$ have co-ordinates \myvec{3 \\ 4} and \myvec{-4 \\ 3} respectively, then 
$\angle QPR$ is equal to
\begin{enumerate}
\item $\frac{\pi}{2}$
\item $\frac{\pi}{3}$
\item $\frac{\pi}{4}$
\item $\frac{\pi}{6}$
\end{enumerate}
    
\item If the circles 
\begin{align}
\vec{x}^T\vec{x}+\myvec{2 & 2k}\vec{x}+6=0 \\
\vec{x}^T\vec{x}+\myvec{0&2k}\vec{x}+k=2
\end{align} 
intersect orthogonally, then find k.

\item Let AB be chord of the circle 
\begin{align}
\vec{x}^T\vec{x}=r^2
\end{align} 
subtending a right angle at at the centre. Then the locus of the centroid of the triangle PAB as $\vec{P}$ moves on the circle is
\begin{enumerate}
\item a parabola
\item a circle
\item a ellipse
\item a pair of straight lines 
\end{enumerate}

\item Let PQ and RS be tangents at the extremities of the diameter PR of a circle of radius r. If PS and RQ intersect at a point on the circumference of the circle, then 2r equals
\begin{enumerate}
\item $\sqrt{PQ.RS}$
\item $\frac{(PQ+RS)}{2}$
\item $\frac{2PQ.RS}{(PQ+RS)}$
\item $\frac{\sqrt{(PQ^2+RS^2)}}{2}$ 
\end{enumerate}
    
\item If the tangent at the point $\vec{P}$ on the circle 
\begin{align}
\vec{x}^T\vec{x}+\myvec{6&6}\vec{x}-2=0
\end{align}
meets a straight line 
\begin{align}
\myvec{5&-2}\vec{x}+6=0
\end{align}
at a point $\vec{Q}$ on the y-axis then the length of PQ is
\begin{enumerate}
\item 4
\item 2$\sqrt{5}$
\item 5
\item 3$\sqrt{5}$ 
\end{enumerate}
    
\item Find the centre of the circle inscribed in square formed by the lines 
\begin{align}
\vec{x}^T\myvec{ 1 & 0 \\ 0 & 0}\vec{x}+\myvec{-8&0}\vec{x}+12=0 \\
\vec{x}^T\myvec{0 & 0 \\ 0 & 1}\vec{x}+\myvec{0&-14}\vec{x} +45=0
\end{align}
   
\item If one of the diameter of the circle 
\begin{align}
\vec{x}^T\vec{x}+\myvec{-2&-6}\vec{x}+6=0
\end{align} 
is a chord to the circle with centre \myvec{2 \\ 1}. then the radius of the circle is 
\begin{enumerate}
\item $\sqrt{3}$
\item $\sqrt{2}$
\item 3
\item 2
\end{enumerate}
    
\item A circle is given by
\begin{align}
\vec{x}^T\vec{x}+\myvec{0&-2}\vec{x}=0,
\end{align}
another circle C touches it externally and also the x-axis, then the locus of its centre is
\begin{enumerate}
\item $\{(x,y):\vec{x}^T\myvec{1&0\\0&0}\vec{x} +\myvec{0&-4} \vec{x}=0\} \cup \{(x,y):y \leq 0\}$
\item $\{(x,y): \vec{x}^T \vec{x} + \myvec{0&-2} \vec{x}=3\} \cup  \{(x,y):y \leq 0\}$
\item $\{(x,y): \vec{x}^T \myvec{1 & 0 \\ 0 & 0 }\vec{x} + \myvec{0 & -1} \vec{x}=0\} \cup \{(0,y):y \leq 0\}$
\item $\{(x,y): \vec{x}^T \myvec{1 & 0 \\ 0 & 0 } \vec{x} +\myvec {0,-4}\vec{x}= 0\} \cup  \{(0,y):y \leq 0\}$
\end{enumerate}
    
\item Tangents drawn from the point $\vec{P}=\myvec{1 \\ 8}$ to the circle 
\begin{align} 
\vec{x}^T\vec{x}+\myvec{-6&-4}\vec{x}-11=0
\end{align}
touch the circle at the points $\vec{A}$ and $\vec{B}.$ The equation of the circumcircle of the triangle PAB is
\begin{enumerate}
\item $\vec{x}^T\vec{x}+\myvec{4&-6}\vec{x}+19=0$
\item $\vec{x}^T\vec{x}+\myvec{-4&-10}\vec{x}+19=0$
\item $\vec{x}^T\vec{x}+\myvec{-2&6}\vec{x}-29=0$
\item $ \vec{x}^T\vec{x}+\myvec{-6&-4}\vec{x}+19=0$ 
\end{enumerate}
    
\item The circle passing through the point \myvec{-1\\0} and touching the y-axis at $\myvec{0 \\ 2}$ also passes through the point.
\begin{enumerate}
\item \myvec{-\frac{3}{2}\\ 0}
\item \myvec{-\frac{5}{2} \\2}
\item \myvec{-\frac{3}{2}\\ \frac{5}{2}}
\item \myvec{-4 \\ 0}
\end{enumerate}	    
    
\item The locus of the mid-point of the chord of contact of tangents drawn from points lying on the straight line 
\begin{align}
\myvec{4&-5}\vec{x}=20
\end{align}
to the circle 
\begin{align}
\vec{x}^T\vec{x}=9
\end{align} is
\begin{enumerate}
\item $20( \vec{x}^T\vec{x})+\myvec{-36&45}\vec{x}=0$
\item $20(\vec{x}^T\vec{x})+\myvec{36&-45}\vec{x}=0$
\item $36( \vec{x}^T\vec{x})+\myvec{-20&45}\vec{x}=0$
\item  $36( \vec{x}^T\vec{x})+\myvec{20&-45}\vec{x}=0$  
\end{enumerate}
    
\item A line 
\begin{align}
\myvec{-m & 1}\vec{x}=1
\end{align} intersects the circle 
\begin{align} 
\vec{x}^T\vec{x}+\myvec{-6&4}\vec{x}=12
\end{align}
at the points $\vec{P}$ and $\vec{Q}.$ If the mid point of the line segment PQ has x-coordinate $-\frac{3}{5},$ then which one of the following option is correct?
\begin{enumerate}
\item $  2 \leq m <  4 $
\item $ -3 \leq m < -1 $
\item $  4 \leq m <  6 $
\item $  6 \leq m <  8$
\end{enumerate}
 
\item The equations of the tangents drawn from the origin to the circle  
\begin{align}
\vec{x}^T\vec{x}+\myvec{-2r&-2h}\vec{x}+h^2=0
\end{align} 
are
\begin{enumerate}
\item \myvec{1 &0} $\vec x$=0
\item \myvec{0 & 1} $\vec x$=0
\item $\myvec{(h^2-r^2)&-2rh}\vec x=0$
\item $\myvec{(h^2-r^2)&2rh}\vec x=0$
\end{enumerate}
    
\item The number of common tangents to the circles
\begin{align}
\vec x^T\vec x=4 \\
\vec{x}^T\vec{x}+\myvec{-6&-8}\vec{x}=24
\end{align} 
is
\begin{enumerate}
\item 0
\item 1
\item 3
\item 4
\end{enumerate}
    
\item If the circle 
\begin{align}
\vec{x}^T\vec{x}=a^2
\end{align}
intersects the hyperbola 
\begin{align}
\vec{x}^T \myvec{ 0 & 1 \\ 0 & 0}\vec{x}=c^2
\end{align} 
in four points $\vec{P}=\myvec{x_1 \\ y_1}, \vec{Q}=\myvec{x_2 \\ y_2}, \vec{R}=\myvec{x_3 \\ y_3}, \vec{S}=\myvec{x_4 \\ y_4}$ then
\begin{enumerate}
\item $x_1+x_2+x_3+x_4=0$
\item $y_1+y_2+y_3+y_4=0$
\item $x_1 x_2 x_3x_4=c^4$
\item $y_1 y_2 y_3 y_4=c^4$
\end{enumerate}
    
\item Circles touching the x-axis at a distance 3 from the origin and having an intercept of length $2\sqrt{7}$ on y-axis are 
\begin{enumerate}
\item $\vec x^T\vec x+(-6,8)\vec x+9=0$
\item $\vec x^T\vec x+(-6,7)\vec x+9=0$
\item $\vec x^T\vec x+(-6,-8)\vec x+9=0$
\item $\vec x^T\vec x+(-6,-7)\vec x+9=0$
\end{enumerate}
    
\item A circle $\vec{S}$ passes through the points $\myvec{0 \\ 1}$ and is orthongonal to the circles 
\begin{align}
\vec{x}^T\vec{x}+\myvec{-2&0}\vec{x}=15 \\
\vec {x}^T\vec{x}=1.
\end{align}
Then  
\begin{enumerate}
\item radius of S is 8
\item radius of S is 7
\item radius of S is \myvec{-7\\1}
\item radius of S is \myvec{-8\\1}
\end{enumerate}
    
\item Let RS be the diameter of the circle 
\begin{align} 
\vec{x}^T\vec{x}=1
\end{align} 
where $\vec{S}$ is the point \myvec{1\\0}. Let $\vec{P}$ be a variable point on the circle and tangents to the circle at $\vec{S}$ and $\vec{P}$ meet at the point $\vec{Q}.$The normal to the circle at $\vec{P}$ intersects a line drawn through $\vec{Q}$ parallel to RS at point $\vec{E}.$ Then the locus of $\vec{E}$ passes through the points 
\begin{enumerate}
\item $\myvec{\frac{1}{3} \\ \frac{1}{\sqrt{3}}}$
\item $\myvec{\frac{1}{4} \\ \frac{1}{2}}$
\item $\myvec{\frac{1}{3} \\ -\frac{1}{\sqrt{3}}}$
\item $\myvec{\frac{1}{4} \\ -\frac{1}{2}}$
\end{enumerate}
    
\item Find the equation of the circle whose radius is 5 and which touches the circle 
\begin{align}
\vec{x}^T\vec{x}+\myvec{-2&-4}\vec{x}-20=0
\end{align} 
at the point $\myvec{5\\5}$.
   
\item Let $\vec{A}$ be the centre of the circle    
\begin{align}
\vec{x}^T\vec{x}+\myvec{-2&-4}\vec{x}-20=0.
\end{align} 
Suppose that the tangents at the points of $\vec{B}=\myvec{1 \\ 7}$ and $\vec{D} = \myvec{4 \\-2}$ on the circle meet at the point $\vec{C}.$ Find the area of the quadrilateral ABCD?
   
\item Find the equations of the circles passing through \myvec{-4 \\ 3} and the touching the lines 
\begin{align}
\myvec{1&1}\vec{x}=2 \\
\myvec{1&-1}\vec{x}=2
\end{align}
    
\item Through a fixed point $\myvec{h\\k}$ secants are drawn through the circles 
\begin{align}
\vec{x}^T\vec{x}=r^2.
\end{align} 
Show that the locus of the mid-point of the secants intercepted by the circle is 
\begin{align}
\vec{x}^T\vec{x}=\myvec{h&k}\vec{x}
\end{align}
    
\item The abscissa of the two points $\vec{A}$ and $\vec{B}$ are the roots of the equation 
\begin{align} 
\vec{x}^T \myvec{ 1 & 0 \\ 0 & 0} \vec{x}+\myvec{2a&0}\vec {x}-b^2=0
\end{align}
and their ordinates are the roots of the equation the two points $\vec{A}$ and $\vec{B}$ are the roots of the equation 
\begin{align}
\vec{x}^T \myvec{ 1 & 0 \\ 0 & 0} \vec{x}+\myvec{2p&0}\vec {x}-q^2=0.
\end{align} 
Find the equation and the radius of the circle with AB as diameter?

\item Lines 
\begin{align}
\myvec{5&12}\vec{x}-10=0 \\
\myvec{5&12}\vec{x}-40=0
\end{align} 
touch a circle $C_1$ of diameter 6. If the centre of $C_1$ lies in the first quadrant, Find the equation of the circle $C_2$ which is concentric with $C_1$ and cuts intercepts of length 8 on these lines.

\item Let a given line $L_1$ intersects the x and y axes at $\vec{P}$ and $\vec{Q}$ respectively. Let another line $L_2$, perpendicular to $L_1$, cut the x and y axes at $\vec{R}$ and $\vec{S}$ respectively. Show that locus of the point of intersection of the lines PS and QR is circle passing through the origin?
    
\item The circle 
\begin{align}
\vec{x}^T\vec{x}+\myvec{-4&-4} \myvec{1&0}\vec{x}+4=0
\end{align} 
is inscribed in a triangle which has two of its sides along the co-ordinate axes. The locus of the circumcentre of the triangle is 
\begin{align}
\myvec{1&1}\vec{x}-x^T\myvec{ 0 & 1 \\ 0 & 0 } \vec{x}+k(\vec{x}^T \vec{x})^{\frac{1}{2}}=0
\end{align} 
Find the k?
   
\item If ($m_i$, 1/$m_i$),$(m_i > 0)$, i=1,2,3,4, are 4 distinct points on circle, then show that $m_1 m_2m_3m_4=1$
 
\item A circle touches the line 
\begin{align}
\myvec{1&-1}\vec{x}=0
\end{align} 
at a point $\vec{P}$ such that OP=$4\sqrt{2}$,where $\vec{O}$ is the origin. The circle contains the point$\myvec{-10 \\ 2}$ in its interior and length of its chord on the line
\begin{align}
\myvec{1&1}\vec{x}=0
\end{align}
 is $6\sqrt{2}$. Determine the equation of the circle?

\item Two circles each of radius 5 units, touch each other at \myvec{1 \\ 2}. If the equation of their common tangent is 
\begin{align}
\myvec{4&3}\vec{x}=10
\end{align}
find the equation of the circles?
 
\item Let a circle be given by 
\begin{align}
2(\vec{x}^T\vec{x})+\myvec{-2a&-b}\vec{x}=0
\end{align}
($a\neq0,b\neq0$) Find the condition on a and b if two chords, each bisected by the x-axis can be drawn to the circle form $\myvec{a \\ \frac{b}{2}}$
    
\item Consider a family of circles passing through fixed point $\vec{A}=\myvec{3 \\ 7}$ and $\vec{B}=\myvec{6 \\ 5}.$ Such that the chords in which the circle 
\begin{align}
\vec{x}^T\vec{x}+\myvec{-4&-6}\vec{x}-3=0
\end{align}
cuts the members of the family are concurrent at a point.Find the coordinate of this point?
 
\item Find the coordinates the point at which the circles 
\begin{align}
\vec{x}^T\vec{x}+\myvec{-4&-2}\vec{x}+4=0 \\
\vec{x}^T\vec{x}+\myvec{-12&8}\vec{x}+36=0
\end{align}
touch each other. Also find the equations common tangents touching the circles in the distinct points.
 
\item Find the intervals of values of a for which the line y+x=0 bisects two chords drawn from a point  $\myvec{1+(\sqrt{2}a)/2 \\ (1-(\sqrt{2}a)/2 }$ to the circle 
\begin{align}
2(\vec{x}^T\vec{x})+\myvec {-(1+\sqrt{2}a)&-(1-\sqrt{2}a)}\vec{x}=0
\end{align}
    
\item A circle passes through three points $\vec{A},\vec{B}$ and $\vec{C}$ with the line segment AC as its diameter. A line passing through $\vec{A}$ intersects the chord BC at a point $\vec{D}$ inside the circle. If angles DAB and CAB are $\alpha$ and $\beta$ respectively and the distance between the point $\vec{A}$ and the mid point of the line segments DC is d, prove that the area of the circle is \\*
\begin{align}
\frac{\pi d^2 cos^2 \alpha}{cos^2\alpha + cos^2\beta +2 cos\alpha cos \beta cos(\beta-\alpha)}
\end{align}
    
\item Let C be any circle with centre \myvec{0 \\ \sqrt{2}}. Prove that at the most two rational points can be there on C.( A rational point is a point both of whose coordinates are rational numbers).

\item $C_1$ and $C_2$ are two concentric circles, the radius $C_2$ being twice that of $C_1$. From a point $\vec{P}$ on  $C_2$, tangent PA and PB are drawn to $C_1$. Prove that the centroid of the triangle PAB lies on $C_1$?
    
\item Let $T_1$, $T_2$ be two tangents drawn from $\myvec{-2 \\ 0}$ on to the circle C: 
\begin{align}
\vec{x}^T\vec{x}=1
\end{align} 
Determine the circles touching C and having $T_1$, $T_2$  as their pair of tangents. Further, find the equation of all possible common tangents to these circles, when take two at a time.

\item Let 
\begin{align}
\vec{x}^T \myvec{2 & -3 \\ 0 & 1} \vec{x}=0
\end{align}
be a equation of a pair of tangents drwan from the origin O to a circle of radius 3 with centre is the first coordinate. If $\vec{A}$ is the one of the points of contact, find the length of OA?
   
\item Let $C_1$ and $C_2$ be two circles with $C_2$ lying inside the $C_1$. A circle C lying inside $C_1$ touches $C_1$ internally and $C_2$ externally. Identify the locus of centre of C?
    
\item For the circle 
\begin{align}
\vec{x}^T\vec{x}=r^2
\end{align}
find the value of r for which the area enclosed by the tangents drawn from the point $\vec{P}=\myvec{6\\8}$ to the circle and the chord of contact is maximum?
    
\item Find the equation of the circle touching  the line 
\begin{align}
\myvec {2&3}\vec{x}+1=0
\end{align}
at $\myvec{1 \\-1}$ and cutting orthogonally the circle having line segment joining $\myvec{0 \\ 3}$ and $\myvec{-2 \\ -1}$ as diameter.
 
\item Circles with radii 3,4 and 5 touch each other externally. If $\vec{P}$ is the point of intersection of tangents to these circles at their  points of contact, find the distance of $\vec{P}$ from the points of contact?\\

\textbf{PASSAGE - 1}\\*
ABCD is square of side length 2 units. $C_1$ is the circle touching all the sides of the square ABCD and $C_2$ is the circumcircle of square ABCD.L is a fixed line in the same plane and $\vec{R}$ is a fixed point.\\*

\item If $\vec{P}$ is any point of $C_1$ and $\vec{Q}$ is another point on $C_2$ then 
\begin{align}
\frac{PA^2+PB^2+PC^2+PD^2}{QA^2+QB^2+QC^2+QD^2}
\end{align} 
is equal to
\begin{enumerate}
\item 0.75
\item 1.25
\item 1
\item 0.5
\end{enumerate}
    
\item If a circle is such that it touches the line L and the circle $C_1$ externally,such that both the circles are on the same side of the line, then the locus of centre of circle is
\begin{enumerate}
\item ellipse 
\item hyperbola
\item parabola
\item  pair of straight line
\end{enumerate}
    
\item A line $L^{'}$ through $\vec{A}$ is drawn parallel to BD. Point $\vec{S}$ moves such that its distance from the line BD and the vertex $\vec{A}$ arc equal. If locus of $\vec{S}$ cuts $L^{'}$ at $T_2$ and $T_3$ and AC at $T_1$, then area of  $\Delta T_1T_2T_3$ is
\begin{enumerate}
\item 1/2 sq.units
\item 2/3 sq.units
\item 1 sq.units
\item 2 sq.units
\end{enumerate} 

\textbf{PASSAGE-2}\\
A circle C of radius 1 is inscribed in an equilateral triangle PQR,The points of contact of C with the sides PQ,QR,RP are $\vec{D},\vec{E},\vec{F}$ respectively.The line PQ is given by the equation  $\myvec{\sqrt{3}&1}\vec{x}-6=0$  and point $\vec{D}$ is $\myvec{\frac{3\sqrt{3}}{2} \\ \frac{3}{2}}$ Further,it is given that the origin and the center of C are on the same side of the line PQ.\\

\item The equation of the circle C is 
\begin{enumerate}
\item $\vec{x}^T\vec{x}+\myvec{-4\sqrt{3}&-2}\vec{x}+13=0$
\item $\vec{x}^T\vec{x}+\myvec{-4\sqrt{3}&1}\vec{x}+13/4=0$
\item $\vec{x}^T\vec{x}+\myvec{-2\sqrt{3}&2}\vec{x}+3=0$
\item $\vec{x}^T\vec{x}+\myvec{-2\sqrt{3}&-2}\vec{x}+3=0$
\end{enumerate}
        
\item Equation of the sides QR,RP are
\begin{enumerate}
\item $(2/\sqrt{3},-1)\vec x+1=0,(-2/\sqrt{3},-1)\vec x-1=0$
\item $(1/\sqrt{3},-1)\vec x=0,(0,1)\vec x=0$
\item $(\sqrt{3}/2,-1)\vec x+1=0,(\sqrt{3}/2,-1)\vec x-1=0$
\item $(\sqrt{3},-1)\vec x=0,(0,1)\vec x=0$ 
\end{enumerate}

\textbf{PASSAGE-3}\\    
A tangent PT is drawn to the circle 
\begin{align}
\vec{x}^T\vec{x}=4
\end{align} 
at the point $\vec{P}=\myvec{\sqrt{3}\\1}$. A straight line L, perpendicular to PT is a tangent to the circle \begin{align}
\vec{x}^T\vec{x}+\myvec{-6&0}\vec{x}+8=0
\end{align}    
    
\item A possible equation of L is
\begin{enumerate}
\item $\myvec{1&-\sqrt{3}}\vec{x} =1$
\item $\myvec{1&\sqrt{3}}\vec{x}=1$
\item $\myvec{1&-\sqrt{3}}\vec{x}=-1$
\item $\myvec{1&\sqrt{3}}\vec{x}=5$
\end{enumerate}

\item A common tangent of the two circles is
\begin{enumerate}
\item $\myvec{1&0}\vec{x}=4$
\item $\myvec{0&1}\vec{x}=2$
\item $\myvec{1&\sqrt{3}}\vec{x} =4$
\item $\myvec{1&2\sqrt{2}}\vec{x} =6$    
\end{enumerate}    
 
\textbf{PASSAGE-4}\\  
Let S be the circle in the xy-plane defined by the equation  
\begin{align}
\vec{x}^T\vec{x}=4
\end{align}
\item Let $E_1$ $E_2$ and $F_1 F_2$ be the chords of S passing through the point $\vec{P_0}=\myvec{1\\1}$ and Parallel to the x-axis and the y-axis respectively.Let $G_1G_2$ be the chord of S passing through $P_0$ and having slope -1.Let the tangent to S at $F_1$ and $F_2$ meet at $F_3$ and the tangent to S at $G_1$ and $G_2$ meet at $G_3.$ Then,the points $E_3,F_3$ and $G_3$ lie on the curve
\begin{enumerate}
\item $\myvec{1&1}\vec{x}=4$
\item $\vec{x}^T\vec{x}+\myvec{-8&-8}\vec{x}+16=0$
\item $\vec{x}^T\myvec{ 0 & 1 \\ 0 & 0}\vec{x}+\myvec{-4&-4} \vec{x} = -12$
\item $\vec{x}^T \myvec{ 0 & 1 \\ 0 & 0}\vec{x}=4$  
\end{enumerate}
  
\item Let $\vec{P}$ be a point on the Circle S with both coordinates being positive.Let the tangent to S at $\vec{P}$ intersect the coordinate axes at the points $\vec{M}$ and $\vec{N}.$ Then,the mid-Point of the line segment MN must lie on the curve
\begin{enumerate}
\item $\vec x^T \vec x=3 x^T \begin{vmatrix} 0 & 1 \\ 0 & 0  \end{vmatrix}\vec x$
\item $x^{2/3}+y^{2/3}=2^{4/3}$
\item $\vec x^T \vec x=\vec x^T \myvec{0 & 2 \\ 0 & 0}  \vec x$
\item $\vec x^T \vec x=(x^T \myvec{0 & 1 \\ 0 & 0}\vec{x})2$
\end{enumerate}


\textbf{Assertion and Reason Type Questions}
\item Tangent are drawn from the point $\vec{P}=\myvec{17\\7}$ to the circle
\begin{align}
\vec x^T \vec x=169
\end{align}
\textbf{Statement-1}: the tangents are mutually perpendicular\\
\textbf{Statement-2}: The locus of the point from which mutually perpendicular tangents can be draw to the given circle
\begin{align}
\vec x^T\vec x=338
\end{align}
\begin{enumerate}
\item Statement-1 is True,Statement-2 is True;Statement-2 is correct explanation for Statement-1
\item Statement-1 is True,Statement-2 is False;Statement-2 is NOT correct explanation for Statement-1
\item Statement-1 is True,Statement-2 is False 
\item Statement-1 is False,Statement-2 is True
\end{enumerate}

\item Consider 
\begin{align}
L_1:\myvec{2&3}\vec x+p-3=0
\end{align} 
\begin{align}
L_2:\myvec{2&3}\vec x+p+3=0
\end{align} 
where P is a real number, and 
\begin{align}
C:\vec x^T\vec x+\myvec{6&-10}\vec x+30=0
\end{align}
\textbf{Statement-1}: If line $L_1$ is a chord of circle C,Then line $L_2$ is not always a diameter of circle C\\
\textbf{Statement-2}: If line $L_1$ is a diameter of circle C,Then line $L_2$ is not a chord of circle C
\begin{enumerate}     
\item Statement-1 is True, Statement-2 is True; Statement-2 is correct explanation for Statement-1
\item Statement-1 is True, Statement-2 is False; Statement-2 is NOT correct explanation for Statement-1
\item Statement-1 is True, Statement-2 is False 
\item Statement-1 is False, Statement-2 is True 
\end{enumerate}


\item The Center of two Circles $C_1$ and $C_2$ each of unit radius are at a distance of 6 units from each other. Let P be the mid point of the line segment joining the centers of $C_1$ and $C_2$ and C be the circle touching circle $C_1$ and $C_2$ externally. If a common tangent to $C_1$ and C passing through P is also a common tangent to $C_2$ and C, then the radius of circle C is

\item The straight line 
\begin{align}
\myvec{2&-3}\vec x=1
\end{align} 
divides the circular region 
\begin{align}\vec x^T\vec x \leq 6
\end{align} 
into two parts. If 
\begin{align}
S=\{(2, \frac{3}{4}), (\frac{5}{2}, \frac{3}{4}), (\frac{1}{4}, \frac{1}{4}), (\frac{1}{8}, \frac{1}{4})\}
\end{align} 
then number of points(s) in S lying inside the smaller part is
    
\item For how many values of p, the circle 
\begin{align}
\vec x^T\vec x+\myvec{2&4}\vec x-p=0
\end{align}
and the coordinate axes have exactly three common points?
 
\item Let point B be the reflection of the point $\vec{A}=\myvec{2\\3}$ with respect to the line 
\begin{align}
\myvec{8&-6}\vec x-23=0
\end{align}
Let $T_a$ and $T_b$ be circles of radii 2 and 1 with centers A and B respectively. Let T be a common tangent to a circle $T_a$ and $T_b$ such that both the circles are on the same side of T. If C is the point of intersection of T and the line passing through A and B, then the length of the line segment AC is

\item  If the chord 
\begin{align}
\myvec{-m&1}\vec x=1
\end{align} 
of the circle 
\begin{align}
\vec x^T\vec x=1
\end{align} 
subtends an angle of measure 45$\degree$ at the major segment of the circle then value of m is
\begin{enumerate}
\item $2\pm\sqrt{2}$
\item $-2\pm\sqrt{2}$
\item $-1\pm\sqrt{2}$
\item none of these
\end{enumerate}
     
\item The centers of set points of a circles,each of radius 3,lie on the circle 
\begin{align}
\vec x^T\vec x=25
\end{align} 
the locus of any point in the set is
\begin{enumerate}
\item $4\leq \vec x^T\vec x \leq 64$
\item $\vec x^T\vec x \leq 25$
\item $\vec x^T\vec x \geq 25$
\item $3\leq \vec x^T \vec x \leq 9$
\end{enumerate}
   
\item The center of the circle passing through (0,0) and (1,0) and touching the circle 
\begin{align}
\vec x^T\vec x=9
\end{align}
is
\begin{enumerate}
\item $(1/2,1/2)$
\item $(1/2,-\sqrt{2})$
\item $(3/2,1/2)$
\item $(1/2,3/2)$
\end{enumerate}   
 
\item The equation of a circle with origin as a center passing through equilateral triangle whose median is of length 3a is
\begin{enumerate}
\item $\vec x^T .\vec x=9a^2$
\item $\vec x^T .\vec x=16a^2$
\item $\vec x^T .\vec x=4a^2$
\item $\vec x^T .\vec x=a^2$
\end{enumerate}
    
\item If the two circles 
\begin{align}
\vec x^T\vec x+\myvec{-2&-6}\vec x+1=r^2
\end{align} 
and 
\begin{align}
\vec x^T\vec x+\myvec{-8&2}\vec x+8=0
\end{align} 
intersecting in two distinct point, then
\begin{enumerate}
\item $r > 2$
\item $2< r < 8$
\item $ r < 2$
\item r=2
\end{enumerate}
    
\item The lines 
\begin{align}
\vec x+\myvec{2&-3}
\vec x=05
\end{align} 
and 
\begin{align}
\vec x+\myvec{3&-4}\vec x=7
\end{align} 
are diameters of a circle having area as 154 sq.units. then the equation of the circle is
\begin{enumerate}
\item $\vec x^T\vec x+\myvec{-2&2}\vec x=62$
\item $\vec x^T\vec x+\myvec{2&-2}\vec x=62$
\item $\vec x^T\vec x+\myvec{2&-2}\vec x=47$
\item $\vec x^T\vec x+\myvec{-2&2}\vec x=47$
\end{enumerate}
    
\item If a circle passes through a point (a,b) and cuts the circle 
\begin{align}
\vec x^T\vec x=4
\end{align} 
orthogonally then the locus of its center is
\begin{enumerate}
\item $\myvec{2a&-2b}\vec x-(a^2+b^2+4)=0$
\item $\myvec{2a&2b}\vec x-(a^2+b^2+4)=0$
\item $\myvec{2a&-2b}\vec x+(a^2+b^2+4)=0$
\item $\myvec{2a&2b}\vec x+(a^2+b^2+4)=0$
\end{enumerate}
    
\item A variable circle passes through the fixed point A(p,q) and touches x-axis.the locus of the other end of the diameter through A is
\begin{enumerate}
\item $\vec  x^T \myvec{0 & 0 \\ 0 & 1}  \vec x+\myvec{-4p&-4q}\vec x+q^2=0$
\item $\vec x^T \myvec{1 & 0 \\ 0 & 0} \vec x+\myvec{-2q&2p}\vec x+q^2=0$
\item $\vec x^T \myvec{0 & 0 \\ 0 & 1}\vec x+\myvec{-4q&-2p}\vec x+q^2=0$
\item $\vec x^T \myvec{1 & 0 \\ 0 & 0}\vec x+\myvec {-2p&-4q}\vec x+q^2=0$
\end{enumerate}
      
\item If the lines 
\begin{align}
\myvec{2&3}\vec x+1=0
\end{align} 
and  
\begin{align}
\myvec{3&-1}\vec x-4=0
\end{align} 
is lie along diameter of a circle of circumference 10$\pi$,then the equation of the circle is
\begin{enumerate}
\item $\vec x^T\vec x+\myvec{2&-2}\vec x-23=0$
\item $\vec x^T\vec x+\myvec{-2&2}\vec x-23=0$ 
\item $\vec x^T\vec x+\myvec{2&2}\vec x-23=0$ 
\item $\vec x^T\vec x+\myvec{-2&2}\vec x-23=0$
\end{enumerate}
    
\item Intercept on the line 
\begin{align}
\myvec{1&-1}\vec x=0
\end{align} 
by the circle 
\begin{align}
\vec x^T\vec x + \myvec{2&0}\vec x=0
\end{align} 
is AB.Equation of the circle on AB as a diameter is
\begin{enumerate}
\item $\vec x^T\vec x + \myvec{1&-1}\vec x=0$ 
\item $\vec x^T\vec x + \myvec{-1&1}\vec x=0$ 
\item $\vec x^T\vec x + \myvec{1&1}\vec x=0$ 
\item $\vec x^T\vec x + \myvec{-1&-1}\vec x=0$
\end{enumerate} 
   
\item If the circle 
\begin{align}
\vec x^T\vec x + \myvec{2a&c}\vec x+a=0
\end{align} 
and 
\begin{align}
\vec x^T\vec x + \myvec{-3a&d}\vec x-1=0
\end{align} 
intersect in two distinct points P and Q then the lines 
\begin{align}
\myvec{5&-b}\vec x-a=0
\end{align} passes through P and Q for
\begin{enumerate}
\item exactly one value of a 
\item no value of a 
\item infinitely many values of a
\item exactly two values of a
\end{enumerate}
     
\item A circle touches the x-axis and also touches the circle with centre at (0,3) and radius 2.The locus of the centre of the circle is
\begin{enumerate}    
\item an ellipse
\item a circle 
\item a hyperbola
\item a parabola
\end{enumerate}
    
\item If a circle passes through the point (a,b) and cuts the circle  
\begin{align}
\vec x^T\vec x +p^2=0
\end{align} 
orthogonally, then the equation of the locus of its center is
\begin{enumerate}    
\item $\vec x^T\vec x + \myvec{-3a&-4b}\vec x+(a^2+b^2-p^2)=0$
\item $\myvec{2a&2b}\vec x-(a^2-b^2+p^2)=0$
\item $\vec x^T\vec x + \myvec{-2a&-3b}\vec x+(a^2-b^2-p^2)=0$
\item $\myvec{2a&2b}\vec x-(a^2+b^2+p^2)=0$
\end{enumerate}
    
\item If the lines 
\begin{align}
2(a+b)\vec x^T \myvec{0 & 1 \\ 0 & 0}  \vec x+\vec x^T \myvec{ a & 0 \\ 0 & b}  \vec x=0
\end{align} 
lie along diameters of a circle and divide the circle into four sectors such that the area of one of the sectors is thrice the area of another sector then
\begin{enumerate}
\item $3a^2-10ab+3b^2=0$
\item $3a^2-2ab+3b^2=0$
\item $3a^2+10ab+3b^2=0$
\item $3a^2+2ab+3b^2=0$
\end{enumerate}
     
\item If the lines 
\begin{align}
\myvec{3&-4}\vec x-7=0
\end{align}
and 
\begin{align}
\myvec{2&-3}\vec x-5=0
\end{align} 
are two diameters of a circle of area 49$\pi$ sq.units,the equation of the circle is
\begin{enumerate}    
\item $\vec x^T\vec x + \myvec{-2&-2}\vec x-47=0$ 
\item $\vec x^T\vec x + \myvec{2&-2}\vec x-62=0$ 
\item $\vec x^T\vec x + \myvec{-2&2}\vec x-62=0$ 
\item $\vec x^T\vec x + \myvec{-2&2}\vec x-47=0$
\end{enumerate}
     
\item Let C be the circle with center (0,0) and radius 3 units. The equation of the locus of the mid points of the chords of the circle C that subtend angle of 2$\pi$/3 at its center is
\begin{enumerate}    
\item $\vec x^T\vec x=3/2$
\item $\vec x^T\vec x=1$
\item $\vec x^T\vec x=27/4$
\item $\vec x^T\vec x=9/4$
\end{enumerate}
     
\item Consider a family of circles which are passing through the points (-1,1) and are tangent to x-axis. If (h,k) are the coordinates of the centre of the circles,then the set of values of k given by the interval
\begin{enumerate}
\item $-1/2\leq k\leq1/2$
\item $k\leq1/2$
\item $0\leq k\leq1/2$
\item $k\geq1/2$
\end{enumerate}
     
\item The point diametrically opposite to the point P(1,0) on the circle 
\begin{align}
\vec x^T\vec x+\myvec{2&4}\vec x-3=0
\end{align}
\begin{enumerate}
\item (3,-4)
\item (-3,4)
\item (-3,-4)
\item (3,4)
\end{enumerate}

\item The differential equation of family of circles with fixed radius 5 units and center on the line \begin{align}
\myvec{0&1}\vec x=2
\end{align} 
is
\begin{enumerate}
\item $(x-2)y^2=25-(y-2)^2$
\item $(y-2)y^2=25-(y-2)^2$
\item $(y-2)^2y^2=25-(y-2)^2$
\item $(x-2)^2y^2=25-(y-2)^2$
\end{enumerate}
    
\item If P and Q are the points of intersection of the circles 
\begin{align}
\vec x^T\vec x+\myvec{3&7}\vec x+2p-5=0
\end{align} 
and 
\begin{align}
\vec x^T\vec x+\myvec{2&2}\vec x-p^2=0
\end{align} 
then there is a circle passing through P,Q and (1,1) for
\begin{enumerate}    
\item all except one value of p
\item all except two values of p
\item exactly one value of p
\item all values of p
\end{enumerate}

\item The circle 
\begin{align}
\vec x^T\vec x+\myvec{-4&-8}\vec x=5
\end{align} 
intersects the line 
\begin{align}
\myvec{3&-4}\vec x=m
\end{align} 
at two distinct points if
\begin{enumerate}    
\item -35$<$m$<$15
\item 15$<$m$<$65
\item 35$<$m$<$85
\item -85$<$m$<$-35
\end{enumerate}

\item The two circles 
\begin{align}
\vec x^T\vec x-a\vec x^T\myvec{1 & 0 \\ 0 & 0}
\end{align} 
and 
\begin{align}
\vec x^T\vec x=c^2
\end{align} 
C$>$0 
touch each other if
\begin{enumerate}    
\item $\mid a\mid=c$
\item a=2c
\item $\mid a\mid=2c$
\item $2\mid a\mid=c$
\end{enumerate}
     
\item The length of the diameter of the circle which touches the x-axis at the point (1,0) and passes through the point (2,3) is
\begin{enumerate}    
\item 10/3
\item 3/5
\item 6/5
\item 5/3
\end{enumerate}

\item The circle passing through (1,-2)and touching the axis of x at (3,0) also passes through the point
\begin{enumerate}    
\item (-5,2)
\item (2,-5)
\item (5,-2)
\item (-2,5)
\end{enumerate}

\item Let C be the circle with center (1,1) and radius =1. If T is the circle centred at (0,y),passing through origin and touching the circle C externally, then the radius of T is equal to
\begin{enumerate}
\item 1/2
\item 1/4
\item $\sqrt{3}/\sqrt{2}$
\item $\sqrt{3}/2$
\end{enumerate}
     
\item Locus of the image of the point (2,3) in the line 		  	
\begin{align}
[\myvec{2&-3}\vec{x}+4]+k(1,-2)\vec{x}+3=0
\end{align} 
k$\in$ R is a
\begin{enumerate}
\item circle of radius $\sqrt{2}$
\item circle of radius $\sqrt{3}$
\item straight line parallel to x-axis
\item straight line parallel to y-axis
\end{enumerate}

\item The number of common tangents to the circles   			   	
\begin{align}
\vec{x}^T\vec{x}+\myvec{-4&-6}\vec{x}-12=0\\
\vec{x}^T\vec{x}+\myvec{6&18}\vec{x}+26=0
\end{align} 
is
\begin{enumerate}    
\item 3
\item 4
\item 1
\item 2
\end{enumerate}

\item The centers of those circles which touches the circle 			
\begin{align}
\vec x^T\vec x+\myvec{-8&-8}\vec x-4=0
\end{align} 
externally and also touches the x-axis lie on
\begin{enumerate}
\item an parabola
\item a circle 
\item a hyperbola
\item a ellipse which is not a circle
\end{enumerate}

\item If one of the diameter of the circle given by the equation 	
\begin{align}
\vec x^T\vec x+\myvec{-4&6}\vec x-12=0
\end{align}  
is a chord of a circle S, whose centre is at (-3,2), then the radius of S is
\begin{enumerate}    
\item  5
\item 10
\item 5$\sqrt{2}$
\item 5$\sqrt{3}$
\end{enumerate}

\item If the tangent to the circle 
\begin{align}
\vec x^T\vec x=1
\end{align} 
intersects the coordinates axes at distinct points P and Q, then the locus of the mid-point of PQ is
\end{enumerate}

















